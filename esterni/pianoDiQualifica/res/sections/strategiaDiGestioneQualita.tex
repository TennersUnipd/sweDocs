\subsection{Strategia di gestione della qualità nel dettaglio}
\subsubsection{Risorse}
Le risorse utilizzate per effettuare il controllo di qualità sono sia umane che tecnologiche.

I ruoli principali nel team che hanno un forte impatto nella qualità del prodotto sono il Responsabile e il Verificatore. Tali ruoli sono stati descritti nelle \textit{Norme di progetto 1.0.0}\docs. Il compito del primo è di monitorare la qualità del prodotto, mentre quello del Verificatore è collegato alla qualità di processo.

Le risorse tecnologiche comprendono tutti gli strumenti software e hardware usati per effettuare le procedure di verifica. 
\subsubsection{Misure e metriche}
\`{E} importante per raggiungere degli obiettivi di qualità è far si che il processo di verifica produca dei risultati quantificabili che si possono confrontare con obbiettivi fissati a priori. Per questo vengono scelti le metriche ed i valori di sufficienza minimi necessari, che indicano se i livelli qualitativi di processo e di prodotto sono conformi agli obiettivi prefissati.

Ogni metrica considerata verrà presentata in tre elementi principali:
\begin{itemize}
	\item \textbf{Nome e descrizione}: presenta una breve descrizione della metrica utilizzata e indica cosa rappresenta;
	\item \textbf{Valore minimo}: indica il valore minimo da raggiungere in modo che il processo o il prodotto siano considerati accettabili;
	\item \textbf{Valore ottimale}: indica il valore nel quale si può collocare il risultato della verifica.
\end{itemize}

\paragraph{Metriche per il processo}
\paragraph{Metriche per i documenti}
Per verificare la validità dei documenti, dato che viene usata la lingua italiana, abbiamo deciso di usare l'indice Gulplease. 
\subparagraph*{Indice Gulplease}
Tale indice è stato definito nel 1988 dal GULP (Gruppo Universitario Linguistico Pedagogico) presso l'Istituto Filosofia dell'Università degli studi di Roma "La Sapienza".

Il vantaggio di tale metodo è dato dal fatto che utilizza la lunghezza delle parole e non sulle sillabe, semplificando di molto il calcolo. L'indice si basa su due variabili linguistiche: la lunghezza della parola e la lunghezza della frase rispetta al numero di lettere.

La formula è la seguente:
\begin{center}
$89+\dfrac{300\times(numerodellefrasi)-10\times(numero delle lettere)}{numero di parole}$
\end{center}
Il valore che si ottiene è un numero compreso tra 0 e 100 ed è possibile suddividerlo in tre indici distinti:
\begin{itemize}
	\item inferiore a 80: indica che il testo risulta difficile da leggere per tutti coloro che hanno una licenza elementare;
	\item inferiore a 60: indica che il testo risulta difficile da leggere per tutti coloro che hanno una licenza media;
	\item inferiore a 40: indica che il testo risulta difficile da leggere per tutti coloro che hanno un diploma superiore.
\end{itemize}
\subparagraph{Valori}
\begin{itemize}
	\item \textbf{Valore minimo}: $\geq$50;
	\item \textbf{Valore ottimale}: 100.
\end{itemize}
\paragraph{Metriche per il software}
\subparagraph*{Code coverage}
%\subparagraph*{Valori}
\subparagraph*{Halstead complexity measures}
%\subparagraph{Valori}
%indice dipendenza
