\section{Analisi dei rischi}
Il gruppo di lavoro ha effettuato un'attenta analisi dei rischi relativi al progetto in questione. L'identificazione e l'analisi dei principali fattori di rischio permette di evitare tali circostanze oppure, nel caso in cui ci si trova in una situazione difficoltosa, sapere come procedere.
\subsection{Metodo di lavoro}
Per l'analisi dei rischi viene utilizzata la seguente procedura di identificazione e risoluzione:
\begin{itemize}
	\item \textbf{Identificazione:} il team riunito identifica i fattori problematici che possono rallentare o bloccare il completamento del progetto;
	\item \textbf{Analisi:} per ogni rischio identificato viene calcolata la probabilità di occorrenza, la gravità del rischio e le possibile conseguenze;
	\item \textbf{Pianificazione:} viene identificato come evitare i rischi individuati e, nel caso questi avvengano, come mitigarne le conseguenze;
	\item \textbf{Controllo:} definizione dei processi di monitoraggio dei rischi e definizione delle procedure di intervento nel caso di occorrenza di un rischio.
\end{itemize}
\subsection{Tipologie di rischio}
\subsubsection{Categoria}
Per la classificazione dei rischi identificati verrà utilizzata la seguente categorizzazione:
\begin{itemize}
	\item \textbf{RT:} Rischi tecnologici;
	\item \textbf{RO:} Rischi organizzativi;
	\item \textbf{RI:} Rischi interpersonali;
	\item \textbf{RA:} Rischi analitici.
\end{itemize}
\subsubsection{Grado di rischio}
Il grado di rischio viene identificato dalla Probabilità di Occorrenza (PO) (\textbf{Alta, Media, Bassa}) e dalla Pericolosità (PR) (\textbf{Alta, Media, Bassa}).
\subsubsection{Elenco rischi}
%\begin{landscape}
%\rowcolors{2}{pari}{dispari}
%\begin{table}[h]
%\caption{Tabella dei rischi}  
%\begin{center}
%\begin{tabular}{p{0.6cm}|p{2.2cm}|p{3.2cm}|p{2cm}|p{4cm}|p{0.9cm}|p{0.8cm}|p{4cm}}
%\rowcolor{header}
%Cod. & Nome & Descrizione & Conseguenza  & Rilevamento & PO & PR & Piano di contingenza \\
%
%RO1  & Calcolo Tempistiche & La molteplicità di attività nuove per il team può portare a una stima errata delle tempistiche & Sforamento delle tempistiche & Per ogni singola attività verrà prevista una tempistica; l'incaricato o il responsabile dovrà notificare se questa non viene rispettata & Alta  & Alta  & Tempistiche e risorse assegnate ad ogni task ben calcolate; all’insorgere della problematica, il responsabile dovrà rivedere le tempistiche e risorse assegnate all’attività in questione \\
%RO2 & Calcolo Costi& Un errato calcolo di tempistiche comporta una variazione dei costi & Sforamento dei costi inizialmente preventivati & Confronto tra le ore inizialmente previste e quelle effettive & Alta  & Alta  &  La nuova previsione sui costi dovrà essere comunicata tempestivamente al committente \\
%RO3 & Impegni Accademici & I membri del team sono impegnati in un percorso scolastico in contemporanea al progetto & Periodi di assenza o non disponibilità dei membri & Ogni membro del team dovrà comunicare tempestivamente i suoi impegni accademici specificando periodo e disponibilità & Alta  & Bassa & Assegnazione di attività e scadenze in base agli impegni accademici e riassegnazione dei task in caso di imprevisti \\
%RO4 & Impegni Personali  & Periodo di non disponibilità di un membro a causa di impegni personali & Periodi di assenza o non disponibilità dei membri & Ogni membro del team dovrà comunicare tempestivamente i suoi impegni personali specificando periodo e disponibilità  & Media & Bassa & Assegnazione di attività e scadenze in base agli impegni personali e riassegnazione dei task in caso di imprevisti \\
%\end{tabular}
%\end{center}
%\end{table}
%\rowcolors{2}{pari}{dispari}
%\begin{table}[h]
%	\caption{Tabella dei rischi}  
%	\begin{center}
%		\begin{tabular}{p{0.6cm}|p{2.2cm}|p{3.2cm}|p{2cm}|p{4cm}|p{0.9cm}|p{0.8cm}|p{4cm}}
%			\rowcolor{header}
%			Cod. & Nome & Descrizione & Conseguenza  & Rilevamento & PO & PR & Piano di contingenza \\
%RO5 & Coronavirus  & Periodo di non disponibilità di un membro a causa del coronavirus & Periodi di assenza o non disponibilità dei membri & Ogni membro del team dovrà comunicare lo stato di salute in caso di sintomi & Media & Bassa & Il membro dovrà evitare di avere interazioni fisiche con gli altri membri \\
%RT1 & Inesperienza tecnologica & La maggior parte delle tecnologie necessarie per la realizzazione del prodotto sono sconosciute ai membri del team & Sforamento delle tempistiche & Vengono identificate e monitorate le lacune del team di lavoro. I membri del progetto dovranno comunicare tempestivamente eventuali difficoltà & Alta & Alta & È previsto un periodo di studio delle nuove tecnologie per tutti i membri del team. I task che richiedono maggiori conoscenze verranno assegnate a più membri in modo tale da favorire l’aiuto reciproco e la collaborazione \\
%RT2 & Problemi di connessione & Problemi legati alla connessione Internet, che durante gli incontri possono impedire le comunicazioni & La comunicazione è inefficace e uno o più membri non riescono a partecipare alle video chiamate & Ogni membro che riscontra tale problema dovrà comunicarlo al gruppo attraverso un altro mezzo di comunicazione & Media & Bassa & Sono stati predisposti altri canali di comunicazioni in modo tale da tener aggiornato ciascun componente del gruppo \\
%RT3 & Guasto tecnico & Possibile guasto dei computer, anche a causa del frequente utilizzo e l'installazione di nuovi software & Impossibilità del componente che riscontrasse tale problema di lavorare & Ogni membro del gruppo dovrà monitorare il corrette funzionamento del proprio pc & Bassa& Media& A seconda della gravità del guasto si provvede alla reinstallare i software, del sistema operativo o, se possibile, alla sostituire della propria macchina\\
%\end{tabular}
%\end{center}
%\end{table}
%\rowcolors{2}{pari}{dispari}
%\begin{table}[h]
%	\caption{Tabella dei rischi}  
%	\begin{center}
%		\begin{tabular}{p{0.6cm}|p{2.2cm}|p{4cm}|p{2cm}|p{4cm}|p{0.9cm}|p{0.8cm}|p{4cm}}
%			\rowcolor{header}
%			Cod. & Nome & Descrizione & Conseguenza  & Rilevamento & PO & PR & Piano di contingenza \\
%RI1 & Comunicazione interna & Il team non condivide uno spazio di lavoro condiviso. Le comunicazioni via messaggistica e/o telefonicamente non sono efficaci quanto la comunicazione verbale diretta & Non tutti sono sempre reperibili, quindi la comunicazione è inefficace & I membri sono tenuti ad avvertire il gruppo quando saranno irreperibili, e a chiarire qualsiasi dubbio per evitare fraintendimenti & Media & Media & Sono stati predisposti diversi canali di comunicazione interna. Vengono organizzate riunioni di persona per discutere gli argomenti più importanti\\
%RI2 & Comunicazione esterna & Il proponente esterno ha sede all’estero, ciò potrebbe rendere più difficili eventuali incontri e riunioni esterne & Le comunicazioni risultano più difficili; & Comunicazioni e incontri con il proponente verranno pianificati preventivamente & Bassa & Media & Sono stati predisposti diversi canali di comunicazione con il proponente. Vengono organizzate videoconferenze per discutere gli argomenti più importanti \\
%RI3 & Contrasti interni     & Possono insorgere contrasti e tensioni tra i membri                                                                                                                  & Lavoro inefficace                                                      & I membri coinvolti devono comunicare l’incomprensione a tutto il gruppo                                                            & Bassa & Media & Il gruppo al completo discute e cerca di risolvere i problemi  \\
%RA1 & Interpretazione della richiesta del committente & Il committente richiede e si aspetta un certo comportamento del prodotto, ma il team lo sviluppa diversamente a causa della scarsa o imprecisa comunicazione & Sforamento tempistiche e aumento costi & A intervalli prefissati si controlla assieme al committente se il prodotto è congruo a quanto richiesto & Media & Alta & Nel caso questo avvenga, si cerca di risolvere subito l’inconveniente in modo da non continuare a costruire su una base invalida \\
%RA2 & Inesperienza gestione progetto & La documentazione o organizzazione del progetto risulta errata rispetto a quanto richiesto & Sforamento tempistiche e aumento costi e inefficienza & Verifica accurata della documentazione prodotta e confronto con altre sorgenti di informazioni per assicurare la loro correttezza & Alta & Alta & Risoluzione immediata dei errori in modo tale da limitarne le conseguenze negative \\                                                         
%\end{tabular}
%\end{center}
%\end{table}
%\end{landscape}

\begin{landscape}
	\rowcolors{2}{pari}{dispari}
	\begin{longtable}{p{0.6cm}|p{2.2cm}|p{3.2cm}|p{2cm}|p{4cm}|p{0.9cm}|p{0.8cm}|p{4cm}}
		\arrayrulecolor{white}
		\caption{Tabella dei rischi}\\ 
		\hline
		\rowcolor{header}
		\textbf{Cod.} & \textbf{Nome} & \textbf{Descrizione} & \textbf{Conseguenza}  & \textbf{Rilevamento} & \textbf{PO} & \textbf{PR} & \textbf{Piano di contingenza} \\
		\hline
		
		RO1  & Calcolo Tempistiche & La molteplicità di attività nuove per il team può portare a una stima errata delle tempistiche & Sforamento delle tempistiche & Per ogni singola attività verrà prevista una tempistica; l'incaricato o il responsabile dovrà notificare se questa non viene rispettata & Alta  & Alta  & Tempistiche e risorse assegnate ad ogni task ben calcolate; all’insorgere della problematica, il responsabile dovrà rivedere le tempistiche e risorse assegnate all’attività in questione \\
		RO2 & Calcolo Costi& Un errato calcolo di tempistiche comporta una variazione dei costi & Sforamento dei costi inizialmente preventivati & Confronto tra le ore inizialmente previste e quelle effettive & Alta  & Alta  &  La nuova previsione sui costi dovrà essere comunicata tempestivamente al committente \\
		RO3 & Impegni Accademici & I membri del team sono impegnati in un percorso scolastico in contemporanea al progetto & Periodi di assenza o non disponibilità dei membri & Ogni membro del team dovrà comunicare tempestivamente i suoi impegni accademici specificando periodo e disponibilità & Alta  & Bassa & Assegnazione di attività e scadenze in base agli impegni accademici e riassegnazione dei task in caso di imprevisti \\
		RO4 & Impegni Personali  & Periodo di non disponibilità di un membro a causa di impegni personali & Periodi di assenza o non disponibilità dei membri & Ogni membro del team dovrà comunicare tempestivamente i suoi impegni personali specificando periodo e disponibilità  & Media & Bassa & Assegnazione di attività e scadenze in base agli impegni personali e riassegnazione dei task in caso di imprevisti \\
		RO5 & Coronavirus  & Periodo di non disponibilità di un membro a causa del coronavirus & Periodi di assenza o non disponibilità dei membri & Ogni membro del team dovrà comunicare lo stato di salute in caso di sintomi & Media & Bassa & Il membro dovrà evitare di avere interazioni fisiche con gli altri membri \\
		RT1 & Inesperienza tecnologica & La maggior parte delle tecnologie necessarie per la realizzazione del prodotto sono sconosciute ai membri del team & Sforamento delle tempistiche & Vengono identificate e monitorate le lacune del team di lavoro. I membri del progetto dovranno comunicare tempestivamente eventuali difficoltà & Alta & Alta & È previsto un periodo di studio delle nuove tecnologie per tutti i membri del team. I task che richiedono maggiori conoscenze verranno assegnate a più membri in modo tale da favorire l’aiuto reciproco e la collaborazione \\
		RT2 & Problemi di connessione & Problemi legati alla connessione Internet, che durante gli incontri possono impedire le comunicazioni & La comunicazione è inefficace e uno o più membri non riescono a partecipare alle video chiamate & Ogni membro che riscontra tale problema dovrà comunicarlo al gruppo attraverso un altro mezzo di comunicazione & Media & Bassa & Sono stati predisposti altri canali di comunicazioni in modo tale da tener aggiornato ciascun componente del gruppo \\
		RT3 & Guasto tecnico & Possibile guasto dei computer, anche a causa del frequente utilizzo e l'installazione di nuovi software & Impossibilità del componente che riscontrasse tale problema di lavorare & Ogni membro del gruppo dovrà monitorare il corrette funzionamento del proprio pc & Bassa& Media& A seconda della gravità del guasto si provvede alla reinstallare i software, del sistema operativo o, se possibile, alla sostituire della propria macchina\\
		RI1 & Comunicazione interna & Il team non condivide uno spazio di lavoro condiviso. Le comunicazioni via messaggistica e/o telefonicamente non sono efficaci quanto la comunicazione verbale diretta & Non tutti sono sempre reperibili, quindi la comunicazione è inefficace & I membri sono tenuti ad avvertire il gruppo quando saranno irreperibili, e a chiarire qualsiasi dubbio per evitare fraintendimenti & Media & Media & Sono stati predisposti diversi canali di comunicazione interna. Vengono organizzate riunioni di persona per discutere gli argomenti più importanti\\
		RI2 & Comunicazione esterna & Il proponente esterno ha sede all’estero, ciò potrebbe rendere più difficili eventuali incontri e riunioni esterne & Le comunicazioni risultano più difficili; & Comunicazioni e incontri con il proponente verranno pianificati preventivamente & Bassa & Media & Sono stati predisposti diversi canali di comunicazione con il proponente. Vengono organizzate videoconferenze per discutere gli argomenti più importanti \\
		RI3 & Contrasti interni     & Possono insorgere contrasti e tensioni tra i membri                                                                                                                  & Lavoro inefficace                                                      & I membri coinvolti devono comunicare l’incomprensione a tutto il gruppo                                                            & Bassa & Media & Il gruppo al completo discute e cerca di risolvere i problemi  \\
		RA1 & Interpretazione della richiesta del committente & Il committente richiede e si aspetta un certo comportamento del prodotto, ma il team lo sviluppa diversamente a causa della scarsa o imprecisa comunicazione & Sforamento tempistiche e aumento costi & A intervalli prefissati si controlla assieme al committente se il prodotto è congruo a quanto richiesto & Media & Alta & Nel caso questo avvenga, si cerca di risolvere subito l’inconveniente in modo da non continuare a costruire su una base invalida \\
		RA2 & Inesperienza gestione progetto & La documentazione o organizzazione del progetto risulta errata rispetto a quanto richiesto & Sforamento tempistiche e aumento costi e inefficienza & Verifica accurata della documentazione prodotta e confronto con altre sorgenti di informazioni per assicurare la loro correttezza & Alta & Alta & Risoluzione immediata dei errori in modo tale da limitarne le conseguenze negative \\   
	\end{longtable}
\end{landscape}
