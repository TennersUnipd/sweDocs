\section{Qualità di prodotto}
Per quanto riguarda la qualità del prodotto, si è deciso di adoperare le parti di interesse dello \textit{standard ISO/IEC 9126\glos} ritenute necessarie all'ottenimento della qualità.
\subsection{Funzionalità}
Rappresenta la capacità del prodotto di soddisfare tutti i requisiti che il sistema deve avere.
%\subsubsection{Obiettivi}
%\begin{itemize}
%	\item \textbf{Accuratezza:} il prodotto deve fornire i risultati attesi nel modo più preciso possibile;
%	\item \textbf{Appropriatezza:} il prodotto deve fornire il corretto numero di funzionalità per le specifiche concordate;
%	\item \textbf{Interoperabilità:} il prodotto deve interagire e operare nella maniera corretta con altri sistemi.
%\end{itemize}
\subsubsection{Metriche}
\begin{itemize}
	\item \textbf{Copertura funzionale}
\end{itemize}

\renewcommand{\arraystretch}{2.2}
\rowcolors{2}{pari}{dispari}
\begin{longtable}{|C{4.5cm}|C{2.25cm}|C{2.25cm}|C{3cm}|}
	\arrayrulecolor{white}

	\caption{Metriche per la funzionalità del prodotto}\\
	\hline
	\rowcolor{header}

	\textbf{Metrica} & \textbf{Intervallo preferibile}  & \textbf{Intervallo ottimale} & \textbf{Unità di misura}
	\tabularnewline
	\endfirsthead

	Copertura funzionale &  100 & 100 & Percentuale \\

\end{longtable}
%\begin{itemize}
%	\item \textbf{Requisiti obbligatori soddisfatti:} indica la percentuale di requisiti obbligatori soddisfatti. Il calcolo avviene secondo la formula:\\\\
%	\centerline{
%		\begin{math}
%		PERC_{OS}=100*\frac{R_{OS}}{R_O}
%		\end{math}
%	}
%	\\\\Dove:
%	\begin{itemize}
%		\item \textbf{R\textsubscript{OS}:} numero di requisiti obbligatori soddisfatti;
%		\item \textbf{R\textsubscript{O}:} numero di requisiti obbligatori
%	\end{itemize}
%	\item \textbf{Requsiiti desiderabili soddisfatti:} indica la percentuale di requisiti desiderabili soddisfatti. Il calcolo avviene secondo la formula:\\\\
%		\centerline{
%		\begin{math}
%		PERC_{DS}=100*\frac{R_{DS}}{R_D}
%		\end{math}
%	}
%	\\\\Dove:
%	\begin{itemize}
%		\item \textbf{R\textsubscript{DS}:} numero di requisiti obbligatori soddisfatti;
%		\item \textbf{R\textsubscript{D}:} numero di requisiti obbligatori
%	\end{itemize}
%\end{itemize}
%
%\renewcommand{\arraystretch}{2.2}
%\rowcolors{2}{pari}{dispari}
%\begin{longtable}{|C{4.5cm}|C{2.25cm}|C{2.25cm}|C{3cm}|}
%	\arrayrulecolor{white}
%
%	\caption{Metriche per la funzionalità del prodotto}\\
%	\hline
%	\rowcolor{header}
%
%	\textbf{Metrica} & \textbf{Intervallo di accettazione}  & \textbf{Intervallo ottimale} & \textbf{Unità di misura}
%	\tabularnewline
%	\endfirsthead
%
%	Requisiti obbligatori soddisfatti &  100 & 100 & Percentuale \\
%	Requisiti desiderabili soddisfatti &  > 50 & 100 & Percentuale \\
%\end{longtable}


\subsection{Usabilità}
Rappresenta la capacità del prodotto di essere comprensibile ed utilizzabile dall'utente.
%\subsubsection{Obiettivi}
%\begin{itemize}
%	\item \textbf{Comprensibilità:} il prodotto deve essere facile da comprendere nelle funzionalità e nei metodi di utilizzo;
%	\item \textbf{Apprendibilità:} il prodotto deve permettere l'apprendimento delle funzionalità da parte dell'utente in poco tempo e in maniera corretta;
%	\item \textbf{Operabilità:} il prodotto deve consentire all'utente un utilizzo autonomo, per i propri scopi.
%\end{itemize}
\subsubsection{Metriche}
\begin{itemize}
	\item \textbf{Numero comandi falliti}
	\item \textbf{Percentuale fallimenti in una sessione}
\end{itemize}

\renewcommand{\arraystretch}{2.2}
\rowcolors{2}{pari}{dispari}
\begin{longtable}{|C{4.5cm}|C{2.25cm}|C{2.25cm}|C{3cm}|}
	\arrayrulecolor{white}

	\caption{Metriche per l'usabilità del prodotto}\\
	\hline
	\rowcolor{header}

	\textbf{Metrica} & \textbf{Intervallo preferibile}  & \textbf{Intervallo ottimale} & \textbf{Unità di misura}
	\tabularnewline
	\endfirsthead

	Numero comandi falliti &  <= 3 & 0 & Valore numerico intero positivo \\
	Percentuale fallimenti in una sessione &  > 70 & 100 & Percentuale \\
\end{longtable}



\subsection{Manutenibilità}
Rappresenta la capacità del prodotto nel subire modifiche o miglioramenti a seguito di cambiamenti dell'ambiente, delle specifiche o dei requisiti del sistema.
%\subsubsection{Obiettivi}
%\begin{itemize}
%	\item \textbf{Analizzabilità:} il prodotto, per come è realizzato, deve permette di rilevare velocemente le cause di possibili errori o malfunzionamenti;
%	\item \textbf{Modificabilità:} una modifica correttiva del prodotto deve causare cambiamenti minimi al resto del sistema;
%	\item \textbf{Stabilità:} apportando una modifica al prodotto, devono essere ridotti al minimo effetti inaspettati.
%\end{itemize}
\subsubsection{Metriche}
\begin{itemize}
	\item \textbf{Indice di manutenibilità}
	\item \textbf{Percentuale commenti}
\end{itemize}

\renewcommand{\arraystretch}{2.2}
\rowcolors{2}{pari}{dispari}
\begin{longtable}{|C{3cm}|C{3cm}|C{3cm}|C{3cm}|}
	\arrayrulecolor{white}

	\caption{Metriche per la manutenibilità del prodotto}\\
	\hline
	\rowcolor{header}

	\textbf{Metrica} & \textbf{Intervallo preferibile}  & \textbf{Intervallo ottimale} & \textbf{Unità di misura}
	\tabularnewline
	\endfirsthead

	Indice di manutenibilità &  >= 10 e <= 100 & >= 20 e <= 100 & Valore numerico positivo (da 0 a 100) \\
	Percentuale commenti &  >= 20 e <= 50 & >= 20 e <= 50 & Percentuale \\
\end{longtable}
