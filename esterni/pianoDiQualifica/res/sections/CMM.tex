\section{Miglioramento continuo e tracciamento}

\subsection{Scopo}
In questa sezione del documento ci si prefigge di illustrare la metologia utilizzata
per migliorare il nostro way of working in maniera incrementale.

\subsection{Introduzione}
Il way of working illustrato nel documento \textit{Norme di progetto} è un approccio
ben definito, ripetibile, misurabile e sotto continuo miglioramento.
Il miglioramento continuo è necesario per ottenere un way of working maturo.

\subsection{Miglioramento continuo}
Il way of working è necessariamente in continua evoluzione per migliorare l'efficacia
e l'efficienza degli sforzi di sviluppo.
Durante ogni attività del gruppo ogni membro è teneuto a prendere nota delle modifiche
che vuole apportare al way of working e a proporle alla riunione che si tiene alla
fine dell'attività in corso.
In riunione si discuterà quindi se approvare o meno le modifiche proposte.
\subsection{Strumenti e tracciamento}
Il way of working viene riportato nelle \textit{Norme di Progetto}%\doc
e per tenere traccia dei cambiamenti ad esso si usa il numero di versionamento semantico
che permette agli utilizzatori di tale documento di capire quanto le modifiche abbiano
impattato sul contenuto.
Per tracciare e presentare le richieste di modifica al way of working biosgna aprire
una issue sul repository di riferimento della documentazione utilizzando il template
per una \textit{feature request}\glo relativa al documento \textit{Norme di Progetto}.
Qualora le richieste dovessero essere approvate le issue verrano riporate sulla
board di Progetto, verranno assegnate, implementate, verificate e approvate.
In caso di rifiuto le issue verranno chiuse inserendo in commento il motivo per cui
si è deciso di non accettarle.

\subsection{Integrazione}
Per ogni attività ripetitiva ci si prefigge di automatizzarla per evitare che la
ripetività del'operazione porti ad errori.
Verranno automattizzate tutte le attività di test e di integrazione tra i vali
elementi del progetto.
