\subsection{Accertamento della qualità}
\subsubsection{Obiettivi}
La Qualità di un prodotto software indica la capacità di soddisfare le aspettative di tutti gli \textit{stakeholders\glo} coinvolti in un progetto.
La Qualità di un prodotto software dipende dalla maturità dell'organizzazione e dei processi utilizzati da chi lo produce, dall'utilizzo di metriche, strumenti e metodi come supporto.
Questo processo si prefigge di:
\begin{itemize}
	\item garantire la qualità dei processi adottati nel ciclo di vita del prodotto;
	\item ridurre il ciclo di sviluppo rendendolo maggiormente efficiente;
	\item garantire la qualità del prodotto finale;
	\item fornire metriche oggettive che permettano di rilevare il livello della qualità raggiunta;
	\item soddisfare le esigenze del cliente in maniera efficiente in termini di risorse usate e costi;
	\item permettere a tutti gli \textit{stakeholders\glo} di visualizzare in maniera chiara i risultati qualitativi raggiunti.
\end{itemize}

Verranno in seguito elencati gli obiettivi qualitativi che il gruppo si prefigge di raggiungere per quanto riguarda la qualità del prodotto e dei processi coinvolti.

\paragraph{Processo di sviluppo}
\subparagraph*{Analisi dei requisiti}
\begin{itemize}
	\item L'analisi deve essere formare requisiti chiari e non contraddittori;
	\item Con l'avanzamento del progetto, i requisiti possono esser raffinati;
	\item I requisiti devono essere completi e approvati da ogni \textit{stakeholder\glos} coinvolto nel progetto.
\end{itemize}
\subparagraph*{Progettazione}
\begin{itemize}
	\item utilizzo delle best practice derivanti da design pattern strutturali noti e consolidati;
	\item in caso di modifiche dei requisiti, l'architettura deve permettere la modifica delle sole componenti interessate;
	\item l'architettura realizzata deve ridurre la complessità di realizzazione del sistema.
\end{itemize}

\paragraph{Processo di documentazione}
\begin{itemize}
	\item semplicità e immediata comprensione da parte di tutti gli \textit{stakeholders\glo};
	\item correttezza grammaticale e ortografica;
	\item contenuti completi e coerenti per l'attività documentata.
\end{itemize}

\paragraph{Processo di verifica}
\begin{itemize}
	\item provare la correttezza rispetto a norme e metriche;
	\item correggere errori dei processi implicati nel ciclo di vita software.
\end{itemize}

\paragraph{Processo di gestione}
\subparagraph*{Pianificazione}
\begin{itemize}
	\item definizione di un modello di sviluppo opportuno per lo scopo e la complessità del progetto;
	\item definizione degli obiettivi da perseguire nel tempo;
	\item allocazione delle risorse disponibili in base al budget a disposizione e alle attività da portare a termine.
\end{itemize}

\paragraph{Funzionalità del prodotto}
Rappresenta la capacità del prodotto di soddisfare tutti i requisiti che il sistema deve avere. Identificano la funzionalità del prodotto le seguenti proprietà:
\begin{itemize}
	\item \textbf{Accuratezza:} il prodotto deve fornire i risultati attesi nel modo più preciso possibile;
	\item \textbf{Appropriatezza:} il prodotto deve fornire il corretto numero di funzionalità per le specifiche concordate;
	\item \textbf{Interoperabilità:} il prodotto deve interagire e operare nella maniera corretta con altri sistemi.
\end{itemize}

\paragraph{Usabilità del prodotto}
Rappresenta la capacità del prodotto di essere comprensibile ed utilizzabile dall'utente. Identificano l'usabilità del prodotto le seguenti proprietà:
\begin{itemize}
	\item \textbf{Comprensibilità:} il prodotto deve essere facile da comprendere nelle funzionalità e nei metodi di utilizzo;
	\item \textbf{Apprendibilità:} il prodotto deve permettere l'apprendimento delle funzionalità da parte dell'utente in poco tempo e in maniera corretta;
	\item \textbf{Operabilità:} il prodotto deve consentire all'utente un utilizzo autonomo, per i propri scopi.
\end{itemize}

\paragraph{Manutenibilità del prodotto}
Rappresenta la capacità del prodotto nel subire modifiche o miglioramenti a seguito di cambiamenti dell'ambiente, delle specifiche o dei requisiti del sistema. Identificano la manutenibilità del prodotto le seguenti proprietà:
\begin{itemize}
	\item \textbf{Analizzabilità:} il prodotto, per come è realizzato, deve permette di rilevare velocemente le cause di possibili errori o malfunzionamenti;
	\item \textbf{Modificabilità:} una modifica correttiva del prodotto deve causare cambiamenti minimi al resto del sistema;
	\item \textbf{Stabilità:} apportando una modifica al prodotto, devono essere ridotti al minimo effetti inaspettati.
\end{itemize}


\subsubsection{Attività}
%Il controllo dei processi e la loro maturità sono essenziali nel processo di accertamento della qualità (Software Quality Assurance - SQA).
%Perseguire e migliorare la Qualità di un prodotto, ha come principali obiettivi:
%\begin{itemize}
%	\item un aumento della qualità finale del prodotto;
%	\item una riduzione del ciclo di sviluppo;
%	\item efficienza in termini di risorse usate e costi.
%\end{itemize}
%Il raggiungimento di questi scopi è determinato da:
%\begin{itemize}
%	\item una pianificazione corretta degli obiettivi di qualità da perseguire;
%	\item il controllo del livello di Qualità raggiunto in base agli obiettivi prefissati;
%	\item il miglioramento continuo del processo di Qualità.
%\end{itemize}
Le attività coinvolte in questo processo fanno riferimento al modello \textit{PDCA\glo} (anche chiamato "Ciclo di Deming"), un approccio concreto utilizzato per il controllo e il miglioramento continuo dei processi produttivi.\\
In seguito verranno stabilite le metriche specifiche per ciascun processo e per il prodotto software in generale mentre, al \textit{Piano di Qualifica 2.1.1\doc} sarà affidato il compito di descrivere per ciascuna di essa gli obiettivi qualitativi definendo intervalli e valori ottenuti. Ogni metrica è stata stabilita con l'obiettivo di garantire oggettività e quantificabilità.
\subsection{Qualità di Processo}
Il gruppo ha deciso di adeguare i processi descritti nell'ISO/IEC 12207 secondo esigenze e stabilire per ciascuno degli obiettivi e delle metriche per la qualità. I 5 livelli di maturità di un processo descritti nel CMMI (descritti in appendice alla sezione \textsection B.2), aiutano a comprendere quali metriche e obiettivi porre per la misurazione della qualità e il suo miglioramento continuo.
\subsubsection{Sviluppo}
Lo Sviluppo è un processo che raggruppa le attività relative alla realizzazione effettiva del prodotto software. Garantendo la qualità delle attività che lo compongono mediante le apposite metriche, il Processo di Sviluppo può considerarsi a sua volta qualitativamente definito. Si procederà dunque con la stesura di un Piano per la Qualità per alcune di esse:
\begin{itemize}
	\item Analisi dei Requisiti;
	\item Progettazione.
\end{itemize}
\paragraph{Analisi dei Requisiti}
\subparagraph{Obiettivi}
\begin{itemize}
	\item l'analisi deve essere formare requisiti chiari e non contraddittori;
	\item con l'avanzamento del progetto, i requisiti possono esser raffinati;
	\item i requisiti devono essere completi e approvati da ogni \textit{stakeholder\glos} coinvolto nel progetto.
\end{itemize}
\subparagraph{Metriche}
\begin{itemize}
	\item \textbf{Requisiti obbligatori soddisfatti:} indica la percentuale di requisiti obbligatori soddisfatti. Il calcolo avviene secondo la formula:\\\\
	\centerline{
		\begin{math}
		PERC_{OS}=100*\frac{R_{OS}}{R_O}
		\end{math}
	}
	\\\\Dove:
	\begin{itemize}
		\item \textbf{R\textsubscript{OS}:} numero di requisiti obbligatori soddisfatti;
		\item \textbf{R\textsubscript{O}:} numero di requisiti obbligatori.
	\end{itemize}
	\item \textbf{Requisiti desiderabili soddisfatti:} indica la percentuale di requisiti desiderabili soddisfatti. Il calcolo avviene secondo la formula:\\\\
		\centerline{
		\begin{math}
		PERC_{DS}=100*\frac{R_{DS}}{R_D}
		\end{math}
	}
	\\\\Dove:
	\begin{itemize}
		\item \textbf{R\textsubscript{DS}:} numero di requisiti obbligatori soddisfatti;
		\item \textbf{R\textsubscript{D}:} numero di requisiti obbligatori.
	\end{itemize}
\end{itemize}

\paragraph{Progettazione}
\subparagraph{Obiettivi}
\begin{itemize}
	\item utilizzo delle best practice derivanti da design pattern strutturali noti e consolidati;
	\item in caso di modifiche dei requisiti, l'architettura deve permettere la modifica delle sole componenti interessate;
	\item l'architettura realizzata deve ridurre la complessità di realizzazione del sistema.
\end{itemize}
\subparagraph{Metriche}
\begin{itemize}
	\item \textbf{Structural Fan-out (SFOUT):} indica il numero di moduli che dipendono dal modulo corrente;
	\item \textbf{Structural Fan-in (SFIN):} indica il numero di moduli dai quali dipende il modulo corrente. Indica dunque il grado di dipendenza di un modulo.
\end{itemize}

\subsubsection{Documentazione}
La Documentazione è un processo che supporta le attività svolte durante il progetto software. Il compito di questo processo è documentare e gestire in forma scritta le informazioni necessarie per le diverse fasi del ciclo di vita di un prodotto.
\paragraph{Obiettivi}
\begin{itemize}
	\item semplicità e immediata comprensione da parte di tutti gli \textit{stakeholders\glo};
	\item correttezza grammaticale e ortografica;
	\item contenuti completi e coerenti per l'attività documentata.
\end{itemize}
\paragraph{Metriche}
\begin{itemize}
	\item \textbf{indice Gulpease:} indice di leggibilità per testi in lingua italiana. La formula per il calcolo dell'indice è:\\\\
	\centerline{
		\begin{math}
		I_{GULP}=89+(\frac{300*N_F-10*N_L}{N_P})
		\end{math}
	}
	\\\\Dove:
	\begin{itemize}
		\item \textbf{N\textsubscript{F}:} numero di frasi del testo;
		\item \textbf{N\textsubscript{L}:} numero di lettere del testo;
		\item \textbf{N\textsubscript{P}:} numero di parole del testo.
	\end{itemize}
\end{itemize}

\subsubsection{Verifica}
Il processo di Verifica si attua mediante controlli basati su norme e metriche che consentono l'accertamento della la correttezza del prodotto software.
\paragraph{Obiettivi}
\begin{itemize}
	\item provare la correttezza rispetto a norme e metriche;
	\item correggere errori dei processi implicati nel ciclo di vita software.
\end{itemize}
\paragraph{Metriche}
\begin{itemize}
	\item \textbf{Code Coverage (CC):} indica la percentuale di codice sorgente ispezionata e coperta dai test.
\end{itemize}

\subsubsection{Gestione}
\paragraph{Pianificazione}
La pianificazione è un'attività del processo di Gestione che si occupa dell'allocazione delle risorse e delle attività disponibili nel tempo, rientrando nei costi prestabiliti.
\subparagraph{Obiettivi}
\begin{itemize}
	\item definizione di un modello di sviluppo opportuno per lo scopo e la complessità del progetto;
	\item definizione degli obiettivi da perseguire nel tempo;
	\item allocazione delle risorse disponibili in base al budget a disposizione e alle attività da portare a termine.
\end{itemize}
\subparagraph{Metriche}
\begin{itemize}
	\item \textbf{Planned Value (PV):} indica il costo stimato, alla data corrente, per la realizzazione di tutte le attività del progetto;
	\item \textbf{Actual cost (AC):} indica il costo effettivo sostenuto sino alla data corrente;
	\item \textbf{Budget at Completion (BAC):} indica il costo totale di progetto previsto al termine della pianificazione;
	\item \textbf{Earned Value (EV):} indica il valore del prodotto alla data corrente. Il calcolo avviene secondo la formula:\\\\
	\centerline{
		\begin{math}
		EV=\% ProdottoRealizzato*BAC
		\end{math}
	}
	\item \textbf{Estimated at Completion (EAC):} indica una revisione del costo finale del progetto in base all'andamento sostenuto sino a quel momento. Il calcolo avviene secondo la formula:\\\\
	\centerline{
		\begin{math}
		EAC=\frac{AC}{\% ProdottoRealizzato}
		\end{math}
	}
	\item \textbf{Estimate to Complete (ETC):} indica il costo per realizzare le attività rimanenti, basato sull'andamento attuale del progetto. Il calcolo avviene secondo la formula:\\\\
	\centerline{
		\begin{math}
		ETC=EAC-AC
		\end{math}
	}
	\item \textbf{Cost Variance	(CV):} indica quanto il costo effettivo per realizzare il progetto sia minore o uguale al costo totale previsto. Il calcolo avviene secondo la formula:\\\\
	\centerline{
		\begin{math}
		CV=EV-AC
		\end{math}
	}
	\item \textbf{Schedule Variance	(SV):} indica se si è in linea, in anticipo o in ritardo rispetto alla schedulazione delle attività di progetto pianificate nella baseline. Il calcolo avviene secondo la formula:\\\\
	\centerline{
		\begin{math}
		SV=EV-PV
		\end{math}
	}
\end{itemize}

\section{Qualità di prodotto}
Per quanto riguarda la qualità del prodotto, si è deciso di adoperare le parti di interesse dello \textit{standard ISO/IEC 9126\glos} ritenute necessarie all'ottenimento della qualità.
\subsection{Funzionalità}
Rappresenta la capacità del prodotto di soddisfare tutti i requisiti che il sistema deve avere.
%\subsubsection{Obiettivi}
%\begin{itemize}
%	\item \textbf{Accuratezza:} il prodotto deve fornire i risultati attesi nel modo più preciso possibile;
%	\item \textbf{Appropriatezza:} il prodotto deve fornire il corretto numero di funzionalità per le specifiche concordate;
%	\item \textbf{Interoperabilità:} il prodotto deve interagire e operare nella maniera corretta con altri sistemi.
%\end{itemize}
\subsubsection{Metriche}
\begin{itemize}
	\item \textbf{Copertura funzionale}
\end{itemize}

\renewcommand{\arraystretch}{2.2}
\rowcolors{2}{pari}{dispari}
\begin{longtable}{|C{4.5cm}|C{2.25cm}|C{2.25cm}|C{3cm}|}
	\arrayrulecolor{white}

	\caption{Metriche per la funzionalità del prodotto}\\
	\hline
	\rowcolor{header}

	\textbf{Metrica} & \textbf{Intervallo preferibile}  & \textbf{Intervallo ottimale} & \textbf{Unità di misura}
	\tabularnewline
	\endfirsthead

	Copertura funzionale &  100 & 100 & Percentuale \\

\end{longtable}
%\begin{itemize}
%	\item \textbf{Requisiti obbligatori soddisfatti:} indica la percentuale di requisiti obbligatori soddisfatti. Il calcolo avviene secondo la formula:\\\\
%	\centerline{
%		\begin{math}
%		PERC_{OS}=100*\frac{R_{OS}}{R_O}
%		\end{math}
%	}
%	\\\\Dove:
%	\begin{itemize}
%		\item \textbf{R\textsubscript{OS}:} numero di requisiti obbligatori soddisfatti;
%		\item \textbf{R\textsubscript{O}:} numero di requisiti obbligatori
%	\end{itemize}
%	\item \textbf{Requsiiti desiderabili soddisfatti:} indica la percentuale di requisiti desiderabili soddisfatti. Il calcolo avviene secondo la formula:\\\\
%		\centerline{
%		\begin{math}
%		PERC_{DS}=100*\frac{R_{DS}}{R_D}
%		\end{math}
%	}
%	\\\\Dove:
%	\begin{itemize}
%		\item \textbf{R\textsubscript{DS}:} numero di requisiti obbligatori soddisfatti;
%		\item \textbf{R\textsubscript{D}:} numero di requisiti obbligatori
%	\end{itemize}
%\end{itemize}
%
%\renewcommand{\arraystretch}{2.2}
%\rowcolors{2}{pari}{dispari}
%\begin{longtable}{|C{4.5cm}|C{2.25cm}|C{2.25cm}|C{3cm}|}
%	\arrayrulecolor{white}
%
%	\caption{Metriche per la funzionalità del prodotto}\\
%	\hline
%	\rowcolor{header}
%
%	\textbf{Metrica} & \textbf{Intervallo di accettazione}  & \textbf{Intervallo ottimale} & \textbf{Unità di misura}
%	\tabularnewline
%	\endfirsthead
%
%	Requisiti obbligatori soddisfatti &  100 & 100 & Percentuale \\
%	Requisiti desiderabili soddisfatti &  > 50 & 100 & Percentuale \\
%\end{longtable}


\subsection{Usabilità}
Rappresenta la capacità del prodotto di essere comprensibile ed utilizzabile dall'utente.
%\subsubsection{Obiettivi}
%\begin{itemize}
%	\item \textbf{Comprensibilità:} il prodotto deve essere facile da comprendere nelle funzionalità e nei metodi di utilizzo;
%	\item \textbf{Apprendibilità:} il prodotto deve permettere l'apprendimento delle funzionalità da parte dell'utente in poco tempo e in maniera corretta;
%	\item \textbf{Operabilità:} il prodotto deve consentire all'utente un utilizzo autonomo, per i propri scopi.
%\end{itemize}
\subsubsection{Metriche}
\begin{itemize}
	\item \textbf{Numero comandi falliti}
	\item \textbf{Percentuale fallimenti in una sessione}
\end{itemize}

\renewcommand{\arraystretch}{2.2}
\rowcolors{2}{pari}{dispari}
\begin{longtable}{|C{4.5cm}|C{2.25cm}|C{2.25cm}|C{3cm}|}
	\arrayrulecolor{white}

	\caption{Metriche per l'usabilità del prodotto}\\
	\hline
	\rowcolor{header}

	\textbf{Metrica} & \textbf{Intervallo preferibile}  & \textbf{Intervallo ottimale} & \textbf{Unità di misura}
	\tabularnewline
	\endfirsthead

	Numero comandi falliti &  <= 3 & 0 & Valore numerico intero positivo \\
	Percentuale fallimenti in una sessione &  > 70 & 100 & Percentuale \\
\end{longtable}



\subsection{Manutenibilità}
Rappresenta la capacità del prodotto nel subire modifiche o miglioramenti a seguito di cambiamenti dell'ambiente, delle specifiche o dei requisiti del sistema.
%\subsubsection{Obiettivi}
%\begin{itemize}
%	\item \textbf{Analizzabilità:} il prodotto, per come è realizzato, deve permette di rilevare velocemente le cause di possibili errori o malfunzionamenti;
%	\item \textbf{Modificabilità:} una modifica correttiva del prodotto deve causare cambiamenti minimi al resto del sistema;
%	\item \textbf{Stabilità:} apportando una modifica al prodotto, devono essere ridotti al minimo effetti inaspettati.
%\end{itemize}
\subsubsection{Metriche}
\begin{itemize}
	\item \textbf{Complessità Ciclomatica}
	\item \textbf{Percentuale commenti}
\end{itemize}

\renewcommand{\arraystretch}{2.2}
\rowcolors{2}{pari}{dispari}
\begin{longtable}{|C{3cm}|C{3cm}|C{3cm}|C{3cm}|}
	\arrayrulecolor{white}

	\caption{Metriche per la manutenibilità del prodotto}\\
	\hline
	\rowcolor{header}

	\textbf{Metrica} & \textbf{Intervallo preferibile}  & \textbf{Intervallo ottimale} & \textbf{Unità di misura}
	\tabularnewline
	\endfirsthead

	Complessità Ciclomatica &  >= 0 & <= 10 & Valore numerico positivo \\
	Percentuale commenti &  >= 20 e <= 50 & >= 20 e <= 50 & Percentuale \\
\end{longtable}


\subsubsection{Procedure}
\paragraph{Rilevazione misurazioni}
Per attuare adottare delle misurazioni rispetto alle metriche esposte nel paragrafo precedente, il gruppo si è avvalso dell'utilizzo di alcuni strumenti automatici che permettessero le rilevazioni in maniera rapida e precisa. In particolar modo, l'iter per tenere traccia delle qualità, è il seguente:
\begin{enumerate}
	\item adoperare la misurazione relativa al prodotto o al processo coinvolto. Essa può avvenire in due modi:
	\begin{itemize}
		\item mediante operazione di merge sul branch principale, vengono scatenati alcuni tipi di verifiche automatiche;
		\item mediante l'utilizzo di software locali.
	\item ottenuta la misura, il verificatore procederà con l'inserire l'esito all'interno dell'apposito file.csv contenuto nella folder del PdQ;
	\item controllare che il corrispettivo grafico, generato dal file.csv, sia riportato all'interno del PdQ in maniera corretta;
	\item inserire la misura ottenuta, in caso di vicinanza alla revisione di avanzamento, anche all'interno della relativa tabella nel PdQ;
	\item inserire a seguito del corrispettivo grafico in appendice del PdQ, una valutazione sull'andamento della qualità monitorata.  
	\end{itemize}
\end{enumerate}

\subsubsection{Strumenti}
\paragraph{Code Metrics}
Tool utilizzato per il calcolo della complessità ciclomatica durante lo sviluppo. Scelto per la facile integrazione con l'ambiente di sviluppo, in particolare per la veloce installazione come plug-in di VSCode.

\paragraph{CLOC}
Pacchetto installabile mediante \textit{npm\glo} scritto in linguaggio Perl che consente il calcolo in percentuale delle linee di commento rispetto alle linee di codice effettive.

\paragraph{Gulpease Calculator}
Software utilizzato per il calcolo dell'indice di Gulpease, automatizzato mediante le GitHub Actions ogni qualvolta viene effettuato il merge del branch con il ramo principale del repository.

\paragraph{Excel}
Sono stati utilizzati i fogli Excel per il calcolo delle metriche relative alla pianificazione. Questa modalità ha reso automatico e semplice il calcolo delle misurazioni, con l'inserimento dei soli valori iniziali in tabella. 


\subsubsection{Riferimenti}
Per la valutazione della qualità del prodotto e dei processi coinvolti nel progetto, il gruppo ha deciso di istanziare i seguenti standard e modelli:
\begin{itemize}
	\item \textbf{ISO/IEC 9126:} per la qualità del prodotto;
	\item \textbf{CMMI:} per la qualità del prodotto.
\end{itemize}
