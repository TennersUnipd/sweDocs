\section{Introduzione}

\subsection{Scopo del documento}
Il presente documento descrive approfonditamente le specifiche e i requisiti tecnici del prodotto finale. Tali informazioni sono raccolte analizzando i documenti messi a disposizione dal proponente (capitolato\glos), interagendo con il cliente e comprendendo il suo dominio mediante discussioni collaborative.\\
Il cliente in questione è l'azienda \textit{Red Babel} e propone la realizzazione del software \textit{Etherless}, secondo le modalità descritte nel capitolato\glo C2.

\subsection{Obiettivo del prodotto}
Lo scopo di \textit{Etherless} è quello di permettere a sviluppatori JavaScript\glo di rendere disponibili ad altri utenti del servizio (utilizzatori) le loro funzioni. Gli utilizzatori possono richiamare le funzionalità rese pubbliche pagando un compenso economico che verrà distribuito tra l'autore della funzione e il servizio stesso.
	
\subsection{Glossario}
Come supporto alla documentazione, viene fornito un \textit{Glossario v.1.0.0}, contenente delle definizioni per termini specifici che possono richiedere chiarimento. Ognuno di questi verrà contrassegnato con un pedice \glo nel documento e la sua spiegazione verrà riportata sotto la corrispondente lettera del glossario. Ciò consentirà di avere un linguaggio comune ed evitare ambiguità. 
	
\subsection{Riferimenti}
\subsubsection{Normativi}
	\begin{itemize}
		\item \textbf{Norme di Progetto}: \textit{Norme di Progetto v1.0.0};
		\item \textbf{Capitolato\glo d'appalto 2}: Etherless\\ 
			\url{https://www.math.unipd.it/\textasciitilde tullio/IS-1/2019/Progetto/C2.pdf}
		\item \textbf{VERBALE ESTERNI TODO}
	\end{itemize}
\subsubsection{Informativi}
\begin{itemize}
	\item \textbf{Studio di fattibilità}: \textit{Studio di Fattibilità v1.0.0}
    \item \textbf{Capitolato\glo d'appalto 2}: Etherless\\ 
			\url{https://www.math.unipd.it/\textasciitilde tullio/IS-1/2019/Progetto/C2.pdf}
	\item \textbf{Truffle}:  Ambiente di sviluppo e testing in ambiente Ethereum\glo. Materiale informativo su Smart contract\glo e rete Ethereum\glo
		\\ \url{https://www.trufflesuite.com/}
	\item \textbf{Serverless}:  Framework di sviluppo per applicativi serverless		\\ \url{https://serverless.com/}
	\item \textbf{AWS}: Servizi computazionali in cloud \\ \url{https://aws.amazon.com/}
	\item \textbf{Sito ufficiale Ethereum\glo}: Descrizione dei concetti base, funzionamento e procedure per la realizzazione di applicativi Ethereum\glo.
\\ \url {https://www.ethereum.org/}
\end{itemize}

	
	
	