\subsection*{\textbf{\hfill \Huge{T} \hfill}} 
\addcontentsline{toc}{subsection}{T}
\subsubsection*{Time series}
\index*{Time series}
La Time Serie (serie temporale) si definisce come un insieme di variabili casuali ordinate rispetto al tempo, ed esprime la dinamica di un certo fenomeno nel tempo. Le serie temporali vengono studiate sia per interpretare un fenomeno, individuando componenti di trend, di ciclicità, di stagionalità e/o di accidentalità, sia per prevedere il suo andamento futuro. 
\subsubsection*{Training set}
\index*{Training set}
Un training set (o insieme di addestramento) è un insieme di dati che vengono utilizzati per addestrare un sistema supervisionato (come una rete neurale o un classificatore probabilistico).
