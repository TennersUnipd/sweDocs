\section{Application functionality}
\subsection{Access}
To have access to the application, use the following  command to view the list of all commands with a short description:
\\
\\
\centerline{\code{etherless init}}

\subsubsection{Registration}
In order to use the functionality of the Etherless application, the user must be registered into the platform.\\
To do it, you have to use the next command and replace <password>  with a password of your choice.
\\
\centerline{\code{signup <password>}}\\
\\\\
The system proceeds to register an account and an \textit{Ethereum\glos} wallet  and then returns two values:
\begin{itemize}
	\item An \textbf{address} associated to the \textit{Ethereum\glos} account;
	\item A \textbf{private-key\glos} associated to the \textit{Ethereum\glos} account.
\end{itemize}
aggiungere screen del comando


\subsubsection{Login}
The developer can decide to use an account already existing typing the \textit{private-key\glos} of your own \textit{Ethereum\glos} account already existing and the password chosen.\\\\
\centerline{\code{login <privateKey> <password>}}\\
\\aggiungere screen comando

\subsubsection{Logout}
The user can also disconnect his \textit{Ethereum\glo} account from the \textit{Etherless} application using the command "logout".\\\\
\centerline{\code{logout}}\\
\\aggiungere screen comando


\subsection{Commands guide}
\subsubsection{List}
Use the command "list" to view all the available functions deployed\glo to \textit{etherless-server}.\\\\
\centerline{\code{list}}

%aggiungere screen comando


%\subsubsection{Find}
%The user can search informations about a %specific function already deployed\glo by %another developer, typing the function %name.\\\\
%\centerline{\code{find <function name>}}
%\\
%\\aggiungere screen comando

%\subsubsection{Log}
%The user can retrieve all the information about the latest transactions through the command "log".\\\\
%\centerline{\code{log}}
%aggiungere screen comando

\subsubsection{Run}
This command runs myFunction, on \textit{etherless-server}, passing param1, ...paramN as parameters. The parameter order follows the order of the function signature. Run is a synchronous command. Once launched it returns the result of the function or, in case the functions has some execution problems, an exception.\\\\
\centerline{\code{run <functionName> <password> [parameters...]}}\\
\\
\\
aggiungere screen comando

\subsubsection{Deploy\glo}
The command allows the user to deploy\glo to etherless-server the JavaScript function exported in \textit{file.js} under the name "myFunction".\\
"myFunction" is also the handler of which the function will be available to the user once deployed. Deploy\glo is a synchronous command. Once launched, it returns only when the deploy\glo is successful or an exception if it encounters some problems.\\
\\
\centerline{\code{deploy <name> <description> <prototype> <cost> <file> <password>}}\\
\\
aggiungere screen comando
