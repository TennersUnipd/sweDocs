\section{Qualità di prodotto}
Per quanto riguarda la qualità del prodotto, si è deciso di adoperare le parti di interesse dello \textit{standard ISO/IEC 9126\glos} ritenute necessarie all'ottenimento della qualità.
\subsection{Funzionalità}
Rappresenta la capacità del prodotto di soddisfare tutti i requisiti che il sistema deve avere.
%\subsubsection{Obiettivi}
%\begin{itemize}
%	\item \textbf{Accuratezza:} il prodotto deve fornire i risultati attesi nel modo più preciso possibile;
%	\item \textbf{Appropriatezza:} il prodotto deve fornire il corretto numero di funzionalità per le specifiche concordate;
%	\item \textbf{Interoperabilità:} il prodotto deve interagire e operare nella maniera corretta con altri sistemi.
%\end{itemize}
\subsubsection{Metriche}
\begin{itemize}
	\item \textbf{Copertura funzionale}
\end{itemize}

\renewcommand{\arraystretch}{2.2}
\rowcolors{2}{pari}{dispari}
\begin{longtable}{|C{4.5cm}|C{2.25cm}|C{2.25cm}|C{3cm}|}
	\arrayrulecolor{white}

	\caption{Metriche per la funzionalità del prodotto}\\
	\hline
	\rowcolor{header}

	\textbf{Metrica} & \textbf{Intervallo sufficiente}  & \textbf{Intervallo ottimale} & \textbf{Unità di misura}
	\tabularnewline
	\endfirsthead

	Copertura funzionale &  >=90 & 100 & Percentuale \\

\end{longtable}

\subsection{Manutenibilità}
Rappresenta la capacità del prodotto nel subire modifiche o miglioramenti a seguito di cambiamenti dell'ambiente, delle specifiche o dei requisiti del sistema.
%\subsubsection{Obiettivi}
%\begin{itemize}
%	\item \textbf{Analizzabilità:} il prodotto, per come è realizzato, deve permette di rilevare velocemente le cause di possibili errori o malfunzionamenti;
%	\item \textbf{Modificabilità:} una modifica correttiva del prodotto deve causare cambiamenti minimi al resto del sistema;
%	\item \textbf{Stabilità:} apportando una modifica al prodotto, devono essere ridotti al minimo effetti inaspettati.
%\end{itemize}
\subsubsection{Metriche}
\begin{itemize}
	\item \textbf{Complessità Ciclomatica}
	\item \textbf{Percentuale commenti}
\end{itemize}

\renewcommand{\arraystretch}{2.2}
\rowcolors{2}{pari}{dispari}
\begin{longtable}{|C{3cm}|C{3cm}|C{3cm}|C{3cm}|}
	\arrayrulecolor{white}

	\caption{Metriche per la manutenibilità del prodotto}\\
	\hline
	\rowcolor{header}

	\textbf{Metrica} & \textbf{Intervallo sufficiente}  & \textbf{Intervallo ottimale} & \textbf{Unità di misura}
	\tabularnewline
	\endfirsthead

	Complessità Ciclomatica &  <= 20 & <= 10 & Valore numerico positivo \\
	Percentuale commenti &  >= 20 e <= 50 & >= 20 e <= 50 & Percentuale \\
\end{longtable}
