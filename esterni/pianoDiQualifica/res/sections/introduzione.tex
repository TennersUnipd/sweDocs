\section{Introduzione}

\subsection{Scopo del Documento}
Questo documento si prefigge di dichiarare le strategie atte a garantire la qualit\'a \glo, definendo quindi i processi di verifica \glo e validazione \glo. Saranno presentate qui non solo le politiche trasversali all'intera organizzazione Tenners, ma anche gli obiettivi da perseguire sul progetto intero. Sar\'a necessario quindi rendere note le procedure e gli strumenti di controllo attraverso le quali ci prefissiamo di raggiungere la qualit'a \glo.
\'E necessario specificare che il documento non \'e statico, ma cambier\'a durante tutta la durata del progetto. Infatti, molte delle politiche o dei processi selezionati inizialmente potranno rivelarsi insufficienti o inadeguati, costringendoci a modificare o aggiungerli. Il piano di qualifica \glo sar\'a prodotto in maniera incrementale e i cambiamenti saranno tracciati nella sezione del registro delle modifiche.

\subsection{Scopo del prodotto}
Il capitolato C2 ha come obbiettivo una piattaforma per svilippatori che metta
in comunicazione un fornitore di funzionalit\`a e chi \`e interessato ad utilizzarle
all'interno del proprio progetto.
La piattaforma prender\`a nome di etherless permettendo a chi sviluppa software e
lo condivide sulla piattaforma di essere pagato, per ogni esecuzione, attraverso
la criptovaluta etherium.


\subsection{Glossario}
Come supporto alla documentazione, viene fornito un \textit{Glossario v.1.0.0},
contenente le definizioni di termini specifici che necessitano di un chiarimento.
Ognuno di questi verr\`a contrassegnato con un pedice \glo nel documento e la sua
spiegazione verr\`a riportata sotto la corrispondente lettera del glossario. Ci\`o
consentir\`a di avere un linguaggio comune ed evitare ambiguit\`a.

\subsection{Rifermenti}
\subsubsection{Rifermenti normativi}
\begin{enumerate}
  \item Piano di qualifica in particola la sezione legata al capitolato C2.
\end{enumerate}
\subsubsection{Rifermenti informativi}
\begin{enumerate}
  \item ISO Standards
	\url{https://www.praxiom.com/}
\end{enumerate}
