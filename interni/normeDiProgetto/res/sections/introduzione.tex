\section{Introduzione}

\subsection{Scopo del Documento}
Questo atto si propone di essere una raccolta delle linee di condotta per lo
sviluppo di software e documentazione a cui tutti i componenti del team Tenners
dovranno attenersi.
Questo documento sar\`a aggiornato in maniera incrementale in quanto i processi
di produzione sono in costante miglioramento.

\subsection{Scopo del Prodotto}
Il capitolato C2 ha come obbiettivo una piattaforma per svilippatori che metta
in comunicazione un fornitore di funzionalit\`a e chi \`e interessato ad utilizzarle
all'interno del proprio progetto.
La piattaforma prender\`a nome di \textbf{Etherless} permettendo a chi sviluppa software e
lo condivide sulla piattaforma di essere pagato, per ogni esecuzione, attraverso
la criptovaluta etherium.


\subsection{Glossario}
Come supporto alla documentazione, viene fornito un \textit{Glossario}\docs,
contenente le definizioni di termini specifici che necessitano di un chiarimento.
Ognuno di questi verr\`a contrassegnato con un pedice \glo nel documento e la sua
spiegazione verr\`a riportata sotto la corrispondente lettera del glossario. Ci\`o
consentir\`a di avere un linguaggio comune ed evitare ambiguit\`a.

\subsection{Rifermenti}
\subsubsection{Rifermenti normativi}
\begin{itemize}
  \item \textbf{ISO 90001:2015}: \href{https://www.praxiom.com/iso-9001.htm}{questa trasposizione};
  \item \textbf{ISO IEC 90003:2014}: \href{https://www.praxiom.com/iso-90003.htm}{questa trasposizione};
  \item \textbf{Capitolato C2 - Etherless}: \href{https://www.math.unipd.it/~tullio/IS-1/2019/Progetto/C2.pdf}{il capitolato d'appalto}.
\end{itemize}

\subsubsection{Rifermenti informativi}
\begin{itemize}
  \item \textbf{SEMAT}: In particolare i diagrammi mostrati a lezione
  \href{https://www.math.unipd.it/~tullio/IS-1/2013/Materiale/SEMAT_Cards_A7.pdf}{1}
  \href{https://www.math.unipd.it/~tullio/IS-1/2013/Materiale/SEMAT_Cards_A8.pdf}{2};
  \item \textbf{Scrum}: Con riferimento a \href{https://www.scrumguides.org/docs/scrumguide/v2017/2017-Scrum-Guide-US.pdf}{\textit{The Scrum Guide}}
  come base di partenza al nostro modello di sviluppo;
  \item \textbf{The Application of ISO 9001 to Agile Software Development}: \href{https://www.researchgate.net/profile/Tor_Stalhane/publication/221219267_The_Application_of_ISO_9001_to_Agile_Software_Development/links/00b49536d4f058a949000000/The-Application-of-ISO-9001-to-Agile-Software-Development.pdf?origin=publication_detail&page=2}{Questo Documento};
  \item \textbf{\textit{Studio di Fattibilità\docs}}: in particolare la sezione legata al capitolato C2.

\end{itemize}
