\subsection*{\textbf{\hfill \Huge{S} \hfill}} 
\addcontentsline{toc}{subsection}{S}
\subsubsection*{Serverless}
Aggettivo usato per descrivere un’architettura software che non ha in carico la gestione di alcun server.
\subsubsection*{Sistema di Versionamento}
Un sistema di controllo di versione (VCS) consente di gestire le modifiche apportate ai file di un progetto (non necessariamente software) su cui tipicamente lavorano più persone. Scopo di un VCS è quello di realizzare una corretta gestione delle modifiche garantendone la reversibilità, la concorrenza ed una annotazione (ad esempio il perché di una modifica).
\subsubsection*{Skype}
Skype è un software proprietario freeware di messaggistica istantanea e VoIP.
\subsubsection*{Slack}
Slack è un software che rientra nella categoria degli strumenti di collaborazione aziendale utilizzato per inviare messaggi in modo istantaneo ai membri del team.
\subsubsection*{Smart-contract}
Gli smart contract sono protocolli informatici che facilitano, verificano, o fanno rispettare, la negoziazione o l’esecuzione di un contratto, permettendo talvolta la parziale o la totale esclusione di una clausola contrattuale. Uno Smart contract è un applicativo che, dopo essere stato inserito sulla Blockchain G , risulta visibile (ma non modificabile) da chiunque ne conosca l’indirizzo.
\subsubsection*{Software house}
La software house è un'azienda specializzata principalmente nella produzione di software e applicazioni.
\subsubsection*{Stakeholder}
Nel campo della progettazione software e sviluppo software il front end è la parte di un sistema che gestisce l’interazione con l’utente o con sistemi esterni che producono dati di ingresso.
\subsubsection*{Stand-alone}
L'espressione stand-alone indica che un oggetto o un software è capace di funzionare da solo o in maniera indipendente da altri oggetti o software.
\subsubsection*{Stub}
Stub è una porzione di codice utilizzata in sostituzione di altre funzionalità software. Può simulare il comportamento di codice ancora da sviluppare.
