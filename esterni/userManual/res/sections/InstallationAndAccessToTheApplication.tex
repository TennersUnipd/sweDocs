\section{Installation and setup}
\subsection{General requirements}
\subsubsection{Hardware requirements}
The only hardware required is to own a CPU newer than a Pentium 4 with SSE2 instruction set. 
\subsubsection{Software requirements }
This application can run on GNU/Linux on any distribution newer than Ubuntu 14.04, MacOs from 10.10 onward and Windows from 7 and onward; Other than an operative system this application needs: 
\begin{enumerate}
	\item \textbf{git}: a working version of git cli installed;
	\item \textbf{Node.js}: version 12.16.2 (or higher) need to be installed on your system. To install it, the user can refer to the official web page nodejs.org for installation we recommend the LTS version;
	\item \textbf{npm}: Node Package Manager (npm) version 6.9.0 (or higher) is the package manager that comes with Node.Js; 
	\item \textbf{TypeScript}: install the npm package TypeScript\glo globally, minimum version required is the 3.6.5; \newline\newline \centerline{\code{npm install typescript -g}}\\
	\item \textbf{ts-node}: install the ts-node package globally, minimum version required is the 8.6.0. \newline\newline \centerline{\code{npm install ts-node -g}}\\
%	\item Having an \textit{Ethereum\glo} account on the network Ropsten with an amount of credit inside.
\end{enumerate}
For simplicity both packages can be installed with one command after nodejs and npm are set up\\\\
\centerline{\code{npm install typescript ts-node -g}}\\
\subsubsection{Application requirements}
The client component interacts with the Ethereum\glo network directly, but in order to be able to do that an Etherem account and a Internet connection is needed. The account also needs to have a positive balance so that gas and execution costs are covered. Ethereum\glo test network Ropsten\glo is used.\\\\
\textbf{Attention!} If you don't have any account on the Ropsten network with a positive balance, please use the following account to login:\\\\
	\code{PRIVATE KEY: 0x42941c8eda235c0102c94793f671b2d6b68080a2ccadd5d5248dafa5b72b94be}\\
	\code{ADDRESS: 0x6883F8f5efA1fE30f6903A11398b19C25c5E324d}\\
\subsection{Etherless-cli}
Assuming a UNIX-based operating system is used.
\subsection{Software download}
A copy of Etherless cli can be retrieved by cloning our github.com repo. For a complete setup, proceed with the following instructions:
\begin{enumerate}
	\item Clone etherless-cli repository \\\\\centerline{\code{git clone https://github.com/TennersUnipd/etherless-cli.git}}\\
	\item Move working directory to downloaded folder \\\\\centerline{\code{cd etherless-cli-master}}\\
	\item Install all missing dependencies \\\\\centerline{\code{npm install}}\\
\end{enumerate}
\subsubsection{Usage}
Since etherless-cli is a command line application, you will make use of specific commands to make use of its features. The commands syntax used is the following:
\\\\\centerline{\code{ts-node src/index.ts <command> [parameters...]}}\\\\
\textbf{Attention!} While performing this commands, your working directory needs to match the downloaded repo.\\

The following commands are currently available:
\begin{itemize}
	\item login <privateKey> <password>
	\item logout
	\item signup <password>
	\item list
	\item create <functionName> <description> <prototype> <cost> <file> <password>
	\item run <functionName> <password> [parameters...]
	\item find <functionName>
\end{itemize}


%To download and install the application it is necessary to open the terminal/shell of your operating system and write the command \\
%\centerline{\code{npm install etherless-cli -g}}\\
%\begin{figure}
%	\centering
%	\includegraphics[width=\textwidth]{res/img/Screenshot_etherless_install.jpg}
%	\caption{command list}
%\end{figure}
%aggiungere uno screen
%Once installed the user must be able to perform the following commands by CLI\glo: 
%\begin{itemize}
%	\item init;
%	\item signup;
%	\item login;
%	\item logout;
%	\item list;
%	\item find;
%	\item create;
%	\item run;
%	\item log;
%	\item update;
%	\item delete.
%\end{itemize}