\section{Processi di supporto}
\subsection{Documentazione}

\subsubsection{Scopo}
Questa sezione si occupa di illustrare i processi necessari alla stesura di un
documento; poiché ogni processo e attivit\`a di sviluppo significativa necessita
di documentazione, in particolare si occupa di descrivere il processo di stesura e
di gestione di testi validi.
I documenti prodotti verranno rilasciati nel repository\glo.
\href{https://github.com/Jatus93/sweDocs}{link temporaneo}

\subsubsection{Aspettative}
Le aspettative sono la definizione di norme con le quali
si gestiscono i documenti dalla creazione all'approvazione.

\subsubsection{Descrizione}
Questa parte del documento contiene le regole che sono state definite per una
migliore gestione dei documenti a cui tutti si devono attenere per produrre un
documento valido e ufficiale, senza eccezioni.

\subsubsection{Ciclo di vita del documento}
Ogni documento ha diversi stadi nel suo ciclo di vita:
\begin{itemize}
  \item \textbf{Preparazione alla creazione di un documento}: come da norme di
  sviluppo interne, per ogni nuova componente del progetto \`e necessaro creare un
  feature branch\glo con il nome del componente in snake case\glo dopodich\'e
  si potr\`a creare il documento;

  \item \textbf{Creazione del documento}: il documento deve essere creato in un
  percorso che ne rifletta la destinazione d'uso, interno o esterno, e
  all'interno di una cartella che indichi il nome del del documento in
  camel case\glo che conterr\`a, al suo interno, il file main.tex;

  \item \textbf{Preparazione al primo push\glo e push in repository\glo}: prima del primo
   push bisogna modificare il file .filesToCompile con il path\glo del documento,
   questo \`e necessario alla github action\glo per testare in maniera continuativa
   la stesura del documento e fare una verifica della validit\`a del codice
   \LaTeX \space utilizzato per scriverlo;

  \item \textbf{Realizzazione}: il documento viene scritto in maniera incrementale
  considerando le osservazioni e le necessit\`a che si evidenziano nel corso del
  progetto.

  \item \textbf{Verifica automatica in un feature branch}: una volta fatto il push
  sul repository del progetto, se i file si trovano in un branch di sviluppo
  sotto feature, verranno compilati dai tool automatici segnalando eventuali errori
  nella sintassi del linguaggio \LaTeX \space utilizzato;

  \item \textbf{Richiesta di verifica manuale}: una volta terminata la verifica
  automatica il redattore segnala ai verificatori che le modifiche richieste
  sono state eseguite facendo una pull request sul branch develop;

  \item \textbf{Seconda verifica automatica}: eseguita la pull request sul branch
  develop il sistema attiva la github action di controllo relativa.

  \item \textbf{Revisione delle modifiche}: ricevuta la pull request i verificatori
  prendono in considerazione le modifiche fatte e se ritenute adeguate, faranno a
  loro volta una pull request sul branch master.

  \item \textbf{Verifica automatica e produzione automatica del documento nel
  branch develop}: Una volta nel branch develop i tool automatici produrrano il
  documento in formato pdf pronto per essere approvato del Responsabile di progetto;

  \item \textbf{Approvazione}: il Responsabile di progetto, dopo aver ricevuto la notifica
  di pull request su master, si preoccuper\`a di leggere le modifiche apportate al documento
  ed eventualmente di approvarle accettando la pull request sul branch master;

  \item \textbf{Generazione del documento}: Dopo che il documento ha raggiunto il
  branch master il sistema automatico si preoccuper\`a di creare la release del
  documento con i relativi changelog;
\end{itemize}

\subsubsection{Template}
\`E stato creato un template, che velocizza il processo di stesura verifica e
approvazione dei documenti, unificando carattere, logo e stile di formattazione
di un documento \LaTeX \space valido.

\subsubsection{Struttura dei documenti}
Una directory valida per la creazione di un documento deve essere così configurata
\begin{itemize}
  \item /
  \begin{itemize}
    \item main.tex
    \item res che contiene immagini e sezioni del documento
    \begin{itemize}
      \item img che contiene le immagini
      \item sections che contiene le sezioni del documento
    \end{itemize}
    \item config
    \begin{itemize}
      \item config.tex contiene tutte le configurazioni relative al documento
    \end{itemize}
  \end{itemize}
\end{itemize}
