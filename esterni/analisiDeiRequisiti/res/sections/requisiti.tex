\section{Requisiti}
I requisiti del sistema \textit{Etherless} sono classificabili secondo due macro categorie:
\begin{itemize}
	\item \textbf{Requisiti funzionali:} riguardano i servizi e funzionalità che il sistema deve offrire all'utente in modo che vengano soddisfatti tutti i bisogni, impliciti ed espliciti;
	\item \textbf{Requisiti non funzionali:} si riferiscono ai vincoli sui servizi che il sistema offre. Essi non sono inerenti alle funzionalità specifiche del prodotto bensì a come esse devono essere realizzate;
\end{itemize}

\subsection{Requisiti funzionali}

\renewcommand{\arraystretch}{2.2}

\rowcolors{2}{pari}{dispari}
\begin{longtable}{|c|p{8cm}|C{2cm}|}
	\arrayrulecolor{white}

	\caption{Tabella riassuntiva dei requisiti funzionali}\\

	\rowcolor{header}

	\textbf{Requisito} & \centering{\textbf{Descrizione}} & \textbf{Fonti}\\

	\endfirsthead

	R1F1 & Un utente ha la possibilità di visualizzare una guida introduttiva contenente la lista dei comandi eseguibili e il loro funzionamento. &  \centering Interna \\ UC1 \tabularnewline

	R1F2 & Un utente ha la possibilità di autenticarsi a \textit{Etherless} mediante un utenza \textit{Ethereum\glos}. & Capitolato \\

	R1F2.1 & L'utente si può autenticare con il login manuale mediante l'inserimento del comando "login" seguito dalle credenziali che lo identificano univocamente all'interno della rete \textit{Ethereum\glos}.  & \centering Interna \\ UC5 \tabularnewline

	R2F2.2 & L'utente si può autenticare mediante login automatico. & \centering Interna \\ UC6 \tabularnewline

%	R1F2.1.1 &  L'utente ha la possibilità di inserire il proprio address per l'autenticazione. & \centering Interna \\ UC5.1 \tabularnewline

	R1F2.1.1 &  L'utente ha la possibilità di inserire la propria \textit{private key\glo} per l'autenticazione. & \centering Interna \\ UC5.1 \tabularnewline

	R1F2.1.2 &  Il sistema dovrà restituire un errore in caso di inserimento di credenziali errate. & \centering Interna \\ UC5.2 \tabularnewline

	R1F2.1.3 &  Il sistema dovrà restituire un messaggio di autenticazione riuscita nel caso in cui esista nel network \textit{Ethereum\glo} un utente con le credenziali inserite. & \centering Interna \\ UC5 \tabularnewline

	R2F2.2.1 &  L'utente può eseguire automaticamente l'accesso tramite un file salvato sul suo dispositivo durante la prima autenticazione o registrazione mediante \textit{Etherless}.  & \centering Interna \\ UC6 \tabularnewline

	R1F3 &  L'utente ha la possibilità di registrarsi su \textit{Etherless}.  & Capitolato \\

	R1F3.1 &  La registrazione su \textit{Etherless} avverrà richiedendo una coppia univoca di address - \textit{private key\glo} alla rete \textit{Ethereum\glos}.  & Capitolato \\

	R1F3.1.1 &  A seguito dell'avvenuta richiesta di registrazione al network \textit{Etherium\glo} verrà visualizzato l'address tramite \textit{CLI\glos}.  & \centering Interna \\ UC4.1 \tabularnewline

	R1F3.1.2 &  A seguito dell'avvenuta richiesta di registrazione al network \textit{Etherium\glo} verrà visualizzata la \textit{private key\glo} tramite \textit{CLI\glos}.  & \centering Interna \\ UC4.2 \tabularnewline

	R2F4 &  Ogni qualvolta l'utente effettuerà l'accesso manuale o la registrazione al sistema, verrà salvato un file contenente le credenziali di accesso per permettere l'autenticazione automatica durante le successive esecuzioni della applicazione.  & Interna \\

	R2F5 &  Un utente autenticato può eseguire il logout richiedendo al sistema di eliminare il file di autenticazione salvato sulla propria macchina.  & \centering Interna \\ UC8 \tabularnewline

	R2F5.1 &  Dovrà essere restituito all'utente un messaggio in caso di logout eseguito con successo.  & \centering Interna \\ UC8 \tabularnewline

	R1F6 &  Un utente autenticato può ricercare una funzione caricata da un altro utente sviluppatore su \textit{Etherless}.  & Interna \\

	R1F6.1 &  La ricerca della funzione viene effettuata inserendo il comando "find" seguito dal nome della funzione stessa. & \centering Interna \\ UC9.1 \tabularnewline

	R1F6.2 &  La ricerca della funzione permetterà di visualizzare sull'\textit{Etherless-cli} alcune informazioni riguardanti la stessa.  & \centering Interna \\ UC9.2 \tabularnewline

	R1F6.2.1 &  La ricerca permetterà di visualizzare sull'\textit{Etherless-cli} il costo della funzione.  & \centering Interna \\ UC9.2.2 \tabularnewline
%	 il nome della funzione, la sua firma, una descrizione riassuntiva delle sue funzionalità, il costo e l'autore

	R2F6.2.2 &  La ricerca permetterà di visualizzare sull'\textit{Etherless-cli} la descrizione della funzione.  & \centering Interna \\ UC9.2.3 \tabularnewline

	R2F6.2.3 &  La ricerca permetterà di visualizzare sull'\textit{Etherless-cli} l'autore della funzione.  & \centering Interna \\ UC9.2.1 \tabularnewline

	R1F6.2.4 &  La ricerca permetterà di visualizzare sull'\textit{Etherless-cli} la firma della funzione.  & \centering Interna \\ UC9.2.4 \tabularnewline

	R1F6.3 &  Verrà segnalato un errore su \textit{Etherless-cli} se il nome della funzione cercata non fosse disponibile su \textit{Etherless-server}.  & \centering Interna \\ UC9.3 \tabularnewline

	R1F7 &  Un utente autenticato può eseguire una funzione caricata da uno sviluppatore sull'\textit{Etherless-server} mediante il comando "run".  & Capitolato \\

	R1F7.1 &  Per eseguire una funzione l'utente deve conoscere e inserire il nome con la quale è caricata su \textit{Etherless}.  & \centering Interna \\ UC10.1 \tabularnewline

	R1F7.2 &  Per eseguire una funzione l'utente deve inserire il valore dei parametri richiesti.  & \centering Interna \\ UC10.2 \tabularnewline
	
	R1F
	
	7.3 & Per eseguire una funzione l'utente deve inserire la propria password. & \centering Interna \\ UC10.5 \tabularnewline

	R1F7.1.1 &  Il sistema deve restituire un messaggio di errore nel caso in cui il nome inserito per il comando "run" non corrisponda ad alcuna funzione presente su \textit{Etherless}.  & \centering Interna \\ UC10.3 \tabularnewline

	R1F7.2.1 &  Il sistema deve restituire un messaggio di errore nel caso in cui il numero dei valori dei parametri inseriti per il comando "run" non corrispondano con quelli richiesti dalla funzione.  & \centering Interna \\ UC10.4 \tabularnewline

	R1F7.4 &  L'esecuzione di una funzione può avvenire solo previo pagamento al suo sviluppatore e alla piattaforma \textit{Etherless}. & Capitolato \\

	R1F7.4.1 &  Il pagamento dovrà avvenire con valuta \textit{Ether\glos}. & Capitolato \\

	R1F7.4.2 &  La transazione avverrà in maniera diretta. & Interna \\

	R3F7.4.3 &  La transazione avverrà tramite \textit{escrow\glos}. & Capitolato \\

	R1F7.4.4 &  Il sistema deve restituire un messaggio di errore all'utente nel caso non avesse fondi  \textit{Ether\glos} sufficienti per coprire il costo dell'esecuzione. & \centering Interna \\ UC13 \tabularnewline

	R1F7.4.5 &  Il sistema deve restituire un messaggio di errore all'utente nel caso lo sviluppatore della funzione non dovesse averne ancora associato un costo. & \centering Interna \\ UC12 \tabularnewline

	R2F8 &  Un utente autenticato può richiedere un log di tutte le funzioni eseguite tramite il comando "log". & \centering Interna \\ UC14.1 \tabularnewline

	R2F8.1 &  Il sistema dovrà restituire tramite \textit{CLI\glo} un log delle funzioni. & \centering Interna \\ UC14.2 \tabularnewline

	R2F8.1.1 &  Il sistema dovrà restituire l'orario e la data delle funzioni eseguite dall'utente. & \centering Interna \\ UC14.2.1.1 \tabularnewline

	R2F8.1.2 &  Il sistema dovrà restituire il nome della funzione eseguita dall'utente. & \centering Interna \\ UC14.2.1.2 \tabularnewline

	R2F8.1.3 &  Il sistema dovrà restituire la firma della funzione eseguita dall'utente. & \centering Interna \\ UC14.2.1.3 \tabularnewline

	R2F8.1.4 &  Il sistema dovrà restituire il costo della funzione eseguita dall'utente. & \centering Interna \\ UC14.2.1.4 \tabularnewline

	R2F8.2 &  Il sistema dovrà restituire messaggio di errore in caso di log vuoto (nel caso l'utente non abbia eseguito alcuna funzione) & \centering Interna \\ UC14.3 \tabularnewline

	R1F9 &  Un utente autenticato può visualizzare un elenco di tutte le funzioni disponibili su \textit{Etherless} tramite il comando "list". & Capitolato \\

	R1F9.1 &  Il sistema dovrà restituire su \textit{CLI\glo} elenco delle funzioni con le relative informazioni per ciascuna di esse. & \centering Interna \\ UC15.2 \tabularnewline

	R1F9.1.1 &  Il sistema dovrà restituire su \textit{CLI\glo} il nome di ciascuna funzione presente su \textit{Etherless}. & \centering Interna \\ UC15.2.1.1 \tabularnewline

	R1F9.1.2 &  Il sistema dovrà restituire su \textit{CLI\glo} il costo di ciascuna funzione presente su \textit{Etherless}. & \centering Interna \\ UC15.2.1.2 \tabularnewline

	R1F9.1.3 &  Il sistema dovrà restituire su \textit{CLI\glo} la firma di ciascuna funzione presente su \textit{Etherless}. & \centering Interna \\ UC15.2.1.3 \tabularnewline

	R2F9.1.4 &  Il sistema dovrà restituire su \textit{CLI\glo} la descrizione di ciascuna funzione presente su \textit{Etherless}. & \centering Interna \\ UC15.2.1.4 \tabularnewline

	R1F10 &  Un utente autenticato ha la possibilità di creare una funzione JavaScript con il comando \textit{"deploy\glos"} e condividerla su \textit{Etherless-server}, diventando un utente sviluppatore. La funzione dovrà essere disponibile al resto degli utenti di \textit{Etherless}. & Capitolato \\

	R1F10.1 &  L'utente eseguirà il \textit{deploy\glo} della sua funzione inserendo il percorso del file contenente la funzione stessa. & \centering Interna \\ UC17.1 \tabularnewline

	R1F10.2 &  L'utente eseguirà il \textit{deploy\glo} della sua funzione inserendo il nome con il quale la si vuole rendere disponibile a tutti gli utenti. & \centering Interna \\ UC17.3 \tabularnewline

	R1F10.1.1 &  Il sistema dovrà restituire un messaggio di errore nel caso in cui il nome della funzione che si sta tentando di condividere sia già in uso. & \centering Interna \\ UC17.2 \tabularnewline

	R1F10.1.2 &  Il sistema dovrà restituire un messaggio di errore nel caso in cui il percorso specificato non fosse corretto (il file di riferimento dovrà avere un'estensione .js). & \centering Interna \\ UC17.4 \tabularnewline

	R1F10.1.3 &  Il sistema dovrà, in mancanza di errori nell'esecuzione del comando \textit{"deploy\glos"}, restituire un messaggio di avvenuta condivisione. & \centering Interna \\ UC17 \tabularnewline

	R1F11 &  Un utente sviluppatore ha la possibilità di modificare delle informazioni riguardanti una sua funzione con il comando "set". & \centering Interna \\ UC18 \tabularnewline

	R1F11.1 &  È possibile modificare il costo di esecuzione. & \centering Interna \\ UC18.1 \tabularnewline

	R3F11.2 &  È possibile definire una descrizione sommaria che ne riepiloghi le funzionalità. & \centering Interna \\ UC18.3 \tabularnewline

	R1F11.3 &  È possibile inserire la firma della funzione da rendere disponibile agli utenti. & \centering Interna \\ UC18.5 \tabularnewline

	R1F11.4 &  Il sistema dovrà, in mancanza di errori nell'esecuzione del comando "set", restituire un messaggio di avvenuta modifica dei parametri richiesti. & \centering Interna \\ UC18 \tabularnewline

	R1F11.1.1 &  Nel caso in cui il costo della funzione inserito sia minore o uguale a 0, è necessario ritornare all'utente un errore e non procedere con la modifica. & \centering Interna \\ UC18.2 \tabularnewline

	R3F11.2.1 &  Nel caso in cui la descrizione della funzione superi il numero massimo di caratteri consentiti, è necessario ritornare all'utente un errore e non procedere con la modifica. & \centering Interna \\ UC18.4 \tabularnewline

	R1F12 &  Un utente sviluppatore ha la possibilità di eliminare una sua funzione. & Capitolato \\

	R1F12.1 & Il sistema dovrà, in mancanza di errori nell'esecuzione del comando "delete", restituire un messaggio di avvenuta rimozione della funzione. & \centering Interna \\ UC19 \tabularnewline

	R1F13 &  Un utente sviluppatore ha la possibilità di aggiornare il codice di una sua funzione. & \centering Interna \\ UC20 \tabularnewline

	R1F13.1 &  L'aggiornamento del codice deve avvenire inserendo il percorso del nuovo file.js contenente la funzione. & \centering Interna \\ UC20.1 \tabularnewline

	R1F13.1.1 & Il sistema dovrà restituire un errore nel caso in cui il nuovo percorso del file inserito per il comando "update" sia errato. & \centering Interna \\ UC20.2 \tabularnewline

	R1F13.1.2 & Il sistema dovrà, in mancanza di errori nell'esecuzione del comando "update", restituire un messaggio di avvenuta aggiornamento del codice. & \centering Interna \\ UC20 \tabularnewline

	%R1F14 &  Per le operazioni di aggiornamento del codice, rimozione di una funzione, modifica delle informazioni aggiuntive, sarà necessario che l'utente abbia condiviso su \textit{Etherless} almeno una funzione. & \centering Interna \\ UC18 \tabularnewline

	%R1F14.1 &  In ognuno dei comandi "set", "update", "delete", dovrà essere specificata la funzione di riferimento. & \centering Interna \\ UC18.1 \tabularnewline

	R1F14 &  Dovrà essere restituito un errore nel caso in cui la funzione di riferimento non sia presente tra quelle rese disponibili su \textit{Etherless} dallo sviluppatore. & \centering Interna \\ UC21 \tabularnewline

	R1F15 & Gli \textit{smart-contract\glo} devono essere upgradable. & Capitolato \\
	\hline

\end{longtable}

\newpage
\subsection{Requisiti non funzionali}
\subsubsection{Requisiti di qualità}
\renewcommand{\arraystretch}{2.2}
  
  \rowcolors{2}{pari}{dispari}
  \begin{longtable}{|c|p{8cm}|c|}
  	\arrayrulecolor{white}
  	
  	\caption{Tabella riassuntiva dei requisiti di vincolo}\\
  	
    \rowcolor{header}
    
    \textbf{Requisito} & \centering{\textbf{Descrizione}} & \textbf{Fonte}\\
    
    \endhead
    
    R1Q1 & Gli \textit{smart-contract\glo} devono essere upgradable. & Capitolato \\
	
 	R1Q2 & Nella codifica devono essere utilizzate le \textit{promise\glo} ed un approccio a \textit{chiamate asincrone\glo} & Capitolato \\
 	
 	R1Q3 & Dovrà essere utilizzato ESLint come software per l'analisi statica del codice & Capitolato \\
 	
 	R1Q4 & Devono essere rispettate le regole descritte nelle \textit{Norme di Progetto v.X.X.X} & Capitolato \\
   
   	R1Q5 & Deve essere rispetto ciò che è descritto nel \textit{Piano di Qualifica v.X.X.X} & Capitolato \\
    \hline
  \end{longtable}
\newpage
\subsubsection{Requisiti di vincolo}
\renewcommand{\arraystretch}{2.2}
  
  \rowcolors{2}{pari}{dispari}
  \begin{longtable}{|c|p{8cm}|c|}
  	\arrayrulecolor{white}
  	
  	\caption{Tabella riassuntiva dei requisiti di vincolo}\\
  	
    \rowcolor{header}
    
    \textbf{Requisito} & \centering{\textbf{Descrizione}} & \textbf{Fonte}\\
    
    \endfirsthead
    
    R1V1 & \textit{Etherless} dev'essere sviluppato utilizzando il linguaggio Typescript 3.6. & Capitolato \\

    R1V2 & \textit{Etherless-server} dev'essere implementato utilizzando il framework {Serverless\glos}. & Capitolato \\
    
    R1V3 & \textit{Etherless} deve utilizzare le tecnologie AWS. & Capitolato \\

    R1V3.1 & Per realizzare l'infrastruttura \textit{Serverless\glo} è richiesto l'utilizzo di AWS Lambda.  & Capitolato \\
    
    R1V4 & Gli \textit{smart-contract\glo} devono essere sviluppati utilizzando il linguaggio Solidity. & Capitolato \\
    
    R1V5 & Gli \textit{smart-contract\glo} devono essere sviluppati utilizzando il framework Truffle. & Capitolato \\
    
    R1V6 & Il progetto deve essere realizzato utilizzando alcuni ambienti specifici. & Capitolato \\
    
    R1V6.1 & Devono essere utilizzati gli ambienti di sviluppo locale, di testing e di staging. & Capitolato \\
    
    R2V6.2 & Il prodotto software può essere testato in ambiente di produzione. & Capitolato \\
    
    R1V6.1.1 & Per gli ambienti di sviluppo locale, testing e staging è desiderabile l'utilizzo di: \textit{Ethereum\glo} testrpc network e un web server locale per lo sviluppo locale e in ambiente di testing; l'\textit{Ethereum\glo} network Ropsten per lo staging. & Capitolato \\
    
    R1V6.2.1 & Per l'ambiente di produzione è necessario l'utilizzo della \textit{MainNet\glo} di \textit{Ethereum\glo} e della valuta \textit{Ether\glos}. & Capitolato \\
    
    R1V7 & Il proponente Red Babel deve essere menzionato tra i crediti all'interno del file README.md. & Capitolato \\
    
    \hline
  \end{longtable}

\newpage
\subsection{Tracciamento}
\subsubsection{Tracciamento Fonti - Requisiti}

\renewcommand{\arraystretch}{2.2}

\rowcolors{2}{pari}{dispari}
\begin{longtable}{|C{6.5cm}|C{6.5cm}|}
	\arrayrulecolor{white}
	
	\caption{Tabella per il tracciamento fonti-requisiti}\\
	\hline
	\rowcolor{header}
	
	\textbf{Fonte} & \textbf{Requisito} 
	\tabularnewline
	\endfirsthead
	
	Capitolato &  
	\centering
	R1F2\\
	R1F3 \\
	R1F3.1\\
	R1F7\\
	R1F7.3\\
	R1F7.3.1\\
	R3F7.3.3\\
	R1F9\\
	R1F10\\
	R1F12\\
	R1Q1\\
	R1Q2\\
	R1Q3\\
	R1Q4\\
	R1Q5\\
	R1Q6\\
	R1Q7\\
	R1V1\\
	R1V2\\
	R1V3\\
	R1V3.1\\
	R1V4\\
	R1V5\\
	R1V6\\
	R1V6.1\\
	R2V6.2\\
	R1V6.1.1\\
	R1V6.2.1\\
	R1V7\\
	
	\tabularnewline
	
	Interna & 
	\centering
	R1F1\\
	R1F2.1 \\
	R2F2.2\\
	R1F2.1.1\\
	R1F2.1.2\
	R1F2.1.3\\
	R1F2.1.4\\
	R2F2.2.1\\
	R2F4\\
	R2F5\\
	R2F5.1\\
	R1F6\\
	R1F6.1\\
	R1F6.2\\
	R1F6.3\\
	R1F7.1\\
	R1F7.2\\
	R1F7.1.1\\
	R1F7.2.1\\
	R1F7.3.2\\
	R1F7.3.4\\
	R1F7.3.5\\
	R2F8\\
	R2F8.1\\
	R1F9.1\\
	R1F10.1\\
	R1F10.2\\
	R1F10.1.1\\
	R1F10.1.2\\
	R1F10.1.3\\
	R1F11\\
	R1F11.1\\
	R3F11.2\\
	R1F11.3\\
	R1F11.1.1\\
	R3F11.2.1\\
	R1F11.4\\
	R1F12.1\\	
	R1F13\\
	R1F13.1\\
	R1F13.1.1\\
	R1F13.1.2\\
	R1F14\\
	R1F14.1\\
	R1F14.1.1\\
	R1F15\\
	
	\tabularnewline
	
	UC1 & R1F1 \\
	UC5 & \centering R1F2.1 \\ R1F2.1.4 \tabularnewline
	UC6 & \centering R2F2.2 \\ R2F2.2.1 \tabularnewline
	UC5.1 & R1F2.1.1 \\
	UC5.2 & R1F2.1.2 \\
	UC5.3 & R1F2.1.3 \\
	UC7 & \centering R2F5 \\ R2F5.1 \tabularnewline
	UC8 & R1F6.1 \\
	UC8.1 & R1F6.1 \\
	UC8.2 & R1F6.3 \\
	UC9.1 & R1F7.1 \\
	UC9.2 & R1F7.2 \\
	UC9.3 & R1F7.1.1 \\
	UC9.4 & R1F7.2.1 \\
	UC12 & R1F7.3.4 \\
	UC11 & R1F7.3.5 \\
	UC13 & \centering R2F8 \\ R2F8.1 \tabularnewline
	UC14 & R1F9.1 \\
	UC15.1 & R1F10.1 \\
	UC15.3 & R1F10.2 \\
	UC15.2 & R1F10.1.1 \\
	UC15.4 & R1F10.1.2 \\
	UC15 & R1F10.1.3 \\
	UC16 & \centering R1F11 \\ R1F11.4 \tabularnewline
	UC16.1 & R1F11.1 \\
	UC16.3 & R3F11.2 \\
	UC16.5 & R1F11.3 \\
	UC16.2 & R1F11.1.1 \\
	UC16.4 & R3F11.2.1 \\
	UC17 & R1F12.1 \\
	UC19 & \centering R1F13 \\ R1F13.1.2 \tabularnewline
	UC19.1 & R1F13.1 \\
	UC19.2 & R1F13.1.1 \\
	UC18 & R1F14 \\
	UC18.1 & R1F14.1 \\
	UC18.2 & R1F14.1.1 \\
	
\end{longtable}
\newpage
\subsubsection{Tracciamento requisiti - fonti}

\renewcommand{\arraystretch}{2.2}

\rowcolors{2}{pari}{dispari}
\begin{longtable}{|C{6.5cm}|C{6.5cm}|}
	\arrayrulecolor{white}
	\caption{Tabella per il tracciamento requisiti-fonti}\\
	\rowcolor{header}
	\textbf{Requisito} & \textbf{Fonti}
	\endfirsthead

	\hline
	\rowcolor{header}

	\textbf{Requisito} & \textbf{Fonti}
	\tabularnewline
	\endhead

	R1F1 & \centering Interna \\ UC1 \tabularnewline

	R1F2 & Capitolato \\

	R1F2.1 & \centering Interna \\ UC5 \tabularnewline

	R2F2.2 & \centering Interna \\ UC6 \tabularnewline

	%R1F2.1.1 & \centering Interna \\ UC5.1 \tabularnewline

	R1F2.1.1 &   \centering Interna \\ UC5.1 \tabularnewline

	R1F2.1.2 &  \centering Interna \\ UC5.2 \tabularnewline

	R1F2.1.3 &  \centering Interna \\ UC22 \tabularnewline
	
	R1F2.1.4 &   \centering Interna \\ UC5 \tabularnewline

	R2F2.2.1 &  \centering Interna \\ UC6 \tabularnewline

	R1F3 & Capitolato \\

	R1F3.1 & Capitolato \\

	R1F3.1.1 &  \centering Interna \\ UC4.1 \tabularnewline

	R1F3.1.2 &  \centering Interna \\ UC4.2 \tabularnewline

	R2F4 &  Interna \\

	R2F5 &  \centering Interna \\ UC8 \tabularnewline

	R2F5.1 &  \centering Interna \\ UC8 \tabularnewline

	R1F6 & Interna \\

	R1F6.1 & \centering Interna \\ UC9.1 \tabularnewline

	R1F6.2 & \centering Interna \\ UC9.2 \tabularnewline

	R1F6.2.1 & \centering Interna \\ UC9.2.2 \tabularnewline

	R2F6.2.2 & \centering Interna \\ UC9.2.3 \tabularnewline

	R2F6.2.3 & \centering Interna \\ UC9.2.1 \tabularnewline

	R1F6.2.4 & \centering Interna \\ UC9.2.4 \tabularnewline

	R1F6.3 & \centering Interna \\ UC9.3 \tabularnewline

	R1F7 & Capitolato \\

	R1F7.1 & \centering Interna \\ UC10.1 \tabularnewline

	R1F7.2 &  \centering Interna \\ UC10.2 \tabularnewline

	R1F7.1.1 & \centering Interna \\ UC10.3 \tabularnewline

	R1F7.2.1 &   \centering Interna \\ UC10.4 \tabularnewline
	
	R1F7.3 & \centering Interna \\ UC10.5 \tabularnewline

	R1F7.4 &   Capitolato \\

	R1F7.4.1 &  Capitolato \\

	R1F7.4.2 &  Interna \\

	R3F7.4.3 &  Capitolato \\

	R1F7.4.4 &  \centering Interna \\ UC13 \tabularnewline

	R1F7.4.5 & \centering Interna \\ UC12 \tabularnewline

	R2F8.1 & \centering Interna \\ UC14 \tabularnewline

	R2F8.1 & \centering Interna \\ UC14.2 \tabularnewline

	R2F8.1.1 & \centering Interna \\ UC14.2.1.1 \tabularnewline

	R2F8.1.2 & \centering Interna \\ UC14.2.1.2 \tabularnewline

	R2F8.1.3 & \centering Interna \\ UC14.2.1.3 \tabularnewline

	R2F8.1.4 & \centering Interna \\ UC14.2.1.4 \tabularnewline

	R2F8.2 & \centering Interna \\ UC14.3 \tabularnewline

	R1F9 &  Capitolato \\

	R1F9.1 & \centering Interna \\ UC15.2 \tabularnewline

	R1F9.1.1 & \centering Interna \\ UC15.2.1.1 \tabularnewline

	R1F9.1.2 & \centering Interna \\ UC15.2.1.2 \tabularnewline

	R1F9.1.3 & \centering Interna \\ UC15.2.1.3 \tabularnewline

	R2F9.1.4 & \centering Interna \\ UC15.2.1.4 \tabularnewline

	R1F10 & Capitolato \\

	R1F10.1 &  \centering Interna \\ UC17.1 \tabularnewline

	R1F10.2 & \centering Interna \\ UC17.3 \tabularnewline

	R1F10.1.1 &  \centering Interna \\ UC17.2 \tabularnewline

	R1F10.2.1 &  \centering Interna \\ UC17.4 \tabularnewline

	R1F10.3 &  \centering Interna \\ UC17 \tabularnewline

	R1F11 & \centering Interna \\ UC18 \tabularnewline

	R1F11.1 &  \centering Interna \\ UC18.1 \tabularnewline

	R3F11.2 & \centering Interna \\ UC18.3 \tabularnewline

	R1F11.3 & \centering Interna \\ UC18.5 \tabularnewline

	R1F11.4 & \centering Interna \\ UC18 \tabularnewline

	R1F11.1.1 & \centering Interna \\ UC18.2 \tabularnewline

	R3F11.2.1 & \centering Interna \\ UC18.4 \tabularnewline

	R1F12 &   Capitolato \\

	R1F12.1 & \centering Interna \\ UC19 \tabularnewline

	R1F13 & \centering Interna \\ UC20 \tabularnewline

	R1F13.1 & \centering Interna \\ UC20.1 \tabularnewline

	R1F13.1.1 & \centering Interna \\ UC20.2 \tabularnewline

	R1F13.1.2 & \centering Interna \\ UC20 \tabularnewline

	R1F14 & \centering Interna \\ UC21 \tabularnewline

	R1F15 & Capitolato \\

	R1Q1 & Capitolato \\

	R1Q2 & Capitolato \\

	R1Q3 & Capitolato \\

	R1Q4 & Capitolato \\

	R1Q5 & Capitolato \\

	R1Q6 & Capitolato \\

	R1V1 & Capitolato \\

	R1V2 & Capitolato \\

	R1V3 & Capitolato \\

	R1V3.1 & Capitolato \\

	R1V4 & Capitolato \\

	R1V5 & Capitolato \\

	R1V6 & Capitolato \\

	R1V6.1 & Capitolato \\

	R1V6.1.1 & Capitolato \\

	R1V6.2 & Capitolato \\

	R1V7 & Capitolato \\
	R1V8 & Capitolato \\

	\hline

\end{longtable}


