\section{Processi Primari}
\subsection{Processi di Fornitura}
\subsubsection{Scopo}
Lo scopo del processo di Fornitura è quello di delineare le risorse e le procedure da utilizzare nello svolgimento del progetto. Si propone, quindi di normare tutti i comportamenti che il gruppo Tenners deve tenere con i proponenti e i committenti al fine di diventare quindi fornitori degli stessi.
\subsubsection{Aspettative}
I componenti del gruppo sperano di intrattenere una comunicazione stretta con i proponenti, in modo tale da assodare subito eventuali dubbi e avere consigli su come effettivamente procedere.
\subsubsection{Attività}
\paragraph{Studio di Fattibilità}
Lo Studio di Fattibilità si prefigge di raccogliere le impressioni del gruppo relativi ai capitolati presentati. Il documento è stato redatto cercando di raggiungere un compromesso su tutte le opinioni dei componenti di Tenners. Per ogni capitolato viene indicato:
\begin{itemize}
  \item \textbf{Descrizione Generale}: viene esposto in sintesi il contenuto del capitolato, indicando nome del progetto e il proponente.
  \item \textbf{Finalità del Progetto}: viene presentato il capitolato più in dettaglio specificando lo scopo del Progetto.
  \item \textbf{Tecnologie Interessate}:vengono elencate le possibili tecnologie interessate nel progetto.
  \item \textbf{Aspetti positivi}: viene spiegato perché secondo il gruppo sia favorevole scegliere questo capitolato.
  \item \textbf{Criticità}: vengono esposte le eventuali problematiche nello sviluppo di un progetto su questo capitolato.
  \item \textbf{Valutazione Conclusiva}: viene riportata la discussione del gruppo e le conclusioni a cui è giunto.
\end{itemize}
\paragraph{Piano di Progetto}
Quali attività svolgere e dove allocare le risorse è indicato nel Piano del progetto. I responsabili e gli amministratori del progetto si occuperanno di distribuirle nel periodo del progetto, in modo da rendere il Piano di Progetto il più efficiente possibile. Questo documento deve contenere le seguenti sezioni:
\begin{itemize}
  \item \textbf{Analisi dei Rischi}: vengono individuati i possibili rischi e errori che possono palesarsi durante il progetto. Inoltre, vengono esposti eventuali soluzioni o azioni preventive per evitarli.
  \item \textbf{Modello di Sviluppo}: definizione del ciclo di vita scelto e utilizzato per il progetto.
  \item \textbf{Pianificazione}: vengono definite le scadenze temporali per ciascuna attività e i ruoli ai quali assegnare le stesse.
  \item \textbf{Preventivo e Consuntivo}: viene stimato il tempo necessario a ciascuna attività e di conseguenza i presunto costo totale del progetto. Viene anche stilato un documento detto Consuntivo, per quanto riguarda l'andamento dello sviluppo del progetto e la differenza con ciò che é stato preventivato.
\end{itemize}
\paragraph{Piano di Qualifica}
Lo scopo del piano di Qualifica è quello di definire le tecniche di verifica e validazione. In particolare chi è stato assegnato a questo compito, i verificatori, saranno tenuti a garantire:
\begin{itemize}
  \item \textbf{Qualità di Prodotto}:
\end{itemize}
