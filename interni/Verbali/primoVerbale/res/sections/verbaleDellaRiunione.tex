\section{Verbale della riunione}
		Al primo incontro si sono presentati tutti i membri del gruppo, dopo la conoscenza reciproca dei vari membri si è discusso riguardo al nome e il logo del gruppo. Successivamente si è discusso della organizzazione interna del team e delle varie preferenze riguardo ai capitolati\glos. 
	\begin{itemize}
		\item \textbf {Nome del gruppo}: tutti i membri hanno espresso le varie idee riguardo al nome del gruppo, è stato scelto "Tenners" in quanto le prime tre lettere del nome (dal inglese Ten) rappresenta il numero 10, che è anche il gruppo con il quale siamo identificati il giorno della formazione dei gruppi;
		\item \textbf {Creazione email}: è stato deciso di utilizzare Gmail come mail principale per le varie comunicazioni tra gli stakeholders\glos;
		\item \textbf {Logo}: ci sono state vari proposte riguardo i possibili loghi;
		\item \textbf {Canali di comunicazione}: per le comunicazioni tra i membri del gruppo è stato scelto il software Slack\glos;
		\item \textbf {Analisi preliminare dei Capitolati\glos}: il gruppo ha discusso gli aspetti positivi e negativi in seguito alla presentazione dei Capitolati\glo fatta dalle aziende proponenti in data 2019-11-15;
		\item \textbf {Strumenti per la stesura dei documenti}: per la scrittura della documentazione di progetto è stato scelto \LaTeX  in quanto oltre a permette la suddivisione delle sezioni dei documenti, cosi sono sempre bene ordinati, permette anche la trasformazione diretta dei file in formato pdf;
		\item \textbf {Sistema di versionamento}: come software di versionamento è stato scelto Git utilizzato attraverso l'interfaccia Github;
		\item \textbf {Project board Github}: per le organizzazioni interne dei vari compiti da fare con le rispettive scadenze il gruppo ha ritenuto opportuno utilizzare la project board di Github.
	\end{itemize}

