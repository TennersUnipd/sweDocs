\section{Capitolato C1 - Autonomous Highlights Platform}

\subsection{Descrizione generale}
Il \textit{capitolato\glo} proposto dall'azienda Zero12 ha lo scopo di creare una piattaforma web che riesca ad estrarre i momenti salienti, per una durata massima di 5 minuti, a partire dal video di un evento sportivo.

\subsection{Finalità del progetto}
La piattaforma web dovrà essere realizzata perseguendo determinate caratteristiche:
\begin{itemize}
	\item l'utente avrà la possibilità di caricare il proprio video dell'evento sportivo da\textit{ CLI\glos};
	\item una volta eseguito l'\textit{upload\glos}, avverrà l'identificazione e l'estrazione dei momenti salienti per creare successivamente il video finale;
	\item l'utente potrà interfacciarsi alla piattaforma web per seguire le operazioni precedentemente descritte e visualizzare il risultato finale.
\end{itemize}

\subsection{Tecnologie interessate}
\begin{itemize}
	\item \textbf{AWS (Amazon Web Services)}: piattaforma che si occupa di fornire servizi di \textit{cloud computing\glos};
	\item \textbf{Amazon ECS (Elastic Container Service)}: servizio per l'orchestrazione di contenitori altamente dimensionabile ad elevate	prestazioni;
	\item \textbf{Amazon DynamoDB}:  database non relazionale per applicazioni ad elevate performance su qualsiasi scala;
	\item \textbf{Amazon Elastic Transcoder}: per la conversione dei diversi formati video;
	\item \textbf{Amazon SageMaker}: utile per creare e allenare modelli di \textit{apprendimento automatico\glos};
	\item \textbf{Amazon Rekognition video}: per il riconoscimento di soggetti, volti, oggetti nei fotogrammi all'interno di un video;
	\item \textbf{HTML5, CSS3 e JavaScript}: linguaggi per lo sviluppo della piattaforma web;
	\item \textbf{Bootstrap}: \textit{framework\glo} per realizzare il \textit{front-end\glo} della piattaforma web;
	\item \textbf{NodeJS}: per lo sviluppo di \textit{API\glo} a supporto dell'applicativo;
	\item \textbf{Python\glos}: per lo sviluppo delle componenti dedicate all'\textit{apprendimento automatico}\glos.
\end{itemize}

\subsection{Aspetti positivi}
\begin{itemize}
	\item Possibilità di apprendere e utilizzare le tecnologie AWS;
	\item Progetto basato su concetti di \textit{apprendimento automatico}\glos, mai affrontati durante il percorso di studi e ritenuti interessanti da molti componenti del gruppo.
\end{itemize}

\subsection{Criticità}
\begin{itemize}
	\item Seppur stimolante, l'apprendimento e il corretto utilizzo della piattaforma AWS e dei suoi servizi può essere molto dispendioso in termini di tempo.
	
\end{itemize}

\subsection{Valutazione conclusiva}
Il gruppo fin da subito ha ritenuto molto accattivante il \textit{capitolato\glo}. Tuttavia, dopo un'attenta analisi dei rischi e vantaggi, si è deciso di focalizzare le attenzioni verso il C2. Sebbene siano consigliati linguaggi per lo sviluppo della \textit{UI\glo} per lo più già noti, addestrare il modello per identificare i momenti significativi dell'evento sportivo e apprendere le nuove tecnologie AWS mai utilizzate sino ad ora, hanno demotivato il team nell'intraprendere questo progetto.