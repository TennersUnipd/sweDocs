\subsection*{\textbf{\hfill \Huge{S} \hfill}} 
\addcontentsline{toc}{subsection}{S}
\subsubsection*{Serverless}
\index*{Serverless}
Aggettivo usato per descrivere un’architettura software che non ha in carico la gestione di alcun server.
\subsubsection*{Sistema di Versionamento}
\index*{Sistema di Versionamento}
Un sistema di controllo di versione (VCS) consente di gestire le modifiche apportate ai file di un progetto (non necessariamente software) su cui tipicamente lavorano più persone. Scopo di un VCS è quello di realizzare una corretta gestione delle modifiche garantendone la reversibilità, la concorrenza ed una annotazione (ad esempio il perché di una modifica).
\subsubsection*{Skype}
\index*{Skype}

\subsubsection*{Slack}
\index*{Slack}

\subsubsection*{Smart-contract}
\index*{Smart-contract}
Gli smart contract sono protocolli informatici che facilitano, verificano, o fanno rispettare, la negoziazione o l’esecuzione di un contratto, permettendo talvolta la parziale o la totale esclusione di una clausola contrattuale. Uno Smart contract è un applicativo che, dopo essere stato inserito sulla Blockchain G , risulta visibile (ma non modificabile) da chiunque ne conosca l’indirizzo.
\subsubsection*{Stakeholder}
\index*{Stakeholder}
Nel campo della progettazione software e sviluppo software il front end è la parte di un sistema che gestisce l’interazione con l’utente o con sistemi esterni che producono dati di ingresso.
\subsubsection*{Stub}
\index*{Stub}

