\section{Introduzione}

\subsection{Scopo del documento}
Il presente documento ha lo scopo di definire delle linee guida riguardo il metodo di lavoro che il gruppo Tenners adotta per la durata del progetto. Stabilire un metodo di lavoro comune a tutti i componenti consente il raggiungimento di un'omogeneità e aiuta il processo di verifica. Questo documento sar\`a aggiornato in maniera incrementale perché soggetto a costante miglioramento.

\subsection{Scopo del prodotto}
Il \textit{Capitolato\glo} C2 ha come obiettivo la realizzazione della piattaforma \textit{Etherless}. Essa ha lo scopo di mettere in comunicazione sviluppatori che intendono condividere funzioni proprie scritte in linguaggio JavaScript con altri utenti che desiderano avere accesso a tali funzionalità. Un utente, corrisposto un pagamento in valuta \textit{Ether\glo} allo sviluppatore e alla piattaforma, avrà la possibilità di eseguire una tra le funzioni messe a disposizione e visualizzarne l'output.

\subsection{Glossario}
Come supporto alla documentazione, viene fornito un \textit{Glossario}\docs,
contenente le definizioni di termini specifici che necessitano di un chiarimento.
Ognuno di questi è contrassegnato con un pedice \glo nel documento e la sua
spiegazione viene riportata sotto la corrispondente lettera del glossario. Ciò
consentir\`a di avere un linguaggio comune ed evitare ambiguità.

\subsection{Rifermenti}
\subsubsection{Rifermenti normativi}
\begin{itemize}
  \item \textbf{Capitolato\glo C2 - Etherless}: \url{https://www.math.unipd.it/~tullio/IS-1/2019/Progetto/C2.pdf};
  \item \textbf{ISO 8601:1988 (Data and Time Formats)}: \url {https://www.w3.org/TR/NOTE-datetime}.
\end{itemize}

\subsubsection{Rifermenti informativi}
\begin{itemize}
  \item \textbf{ISO/IEC 12207:1995 (Sezioni: 5.1, 5.2, 5.3, 6.1, 6.2, 6.3, 6.4, 6.5, 6.8,  7.1,  7.4)}: \url{https://www.math.unipd.it/~tullio/IS-1/2009/Approfondimenti/ISO_12207-1995.pdf}
  \item \textbf{SWEBOK v.3.0 (Capitolo 1: Software Requirements; Capitolo 4: Software Testing)}: \url{https://cs.fit.edu/~kgallagher/Schtick/Serious/SWEBOKv3.pdf};
  \item \textbf{Slides del corso di Ingegneria Del Software (2019-2020) - Processi software:} \url{https://www.math.unipd.it/~tullio/IS-1/2019/Dispense/L03.pdf};
  \item \textbf{Slides del corso di Ingegneria Del Software (2019-2020) - Amministrazione di progetto:} \url{https://www.math.unipd.it/~tullio/IS-1/2019/Dispense/FC01.pdf};
  \item \textbf{Slides del corso di Ingegneria Del Software (2019-2020) - Verifica e validazione:} \url{https://www.math.unipd.it/~tullio/IS-1/2019/Dispense/L14.pdf};
  \item \textbf{Slides del corso di Ingegneria Del Software (2019-2020) - Verifica analisi statiche:} \url{https://www.math.unipd.it/~tullio/IS-1/2019/Dispense/L15.pdf};
  \item \textbf{Slides del corso di Ingegneria Del Software (2019-2020) - Progettazione software:}
  \url{  https://www.math.unipd.it/~tullio/IS-1/2019/Dispense/L10.pdf};
  \item \textbf{Slides del corso di Tecnologie Open-Source - GIT:}   \url{  https://elearning.unipd.it/math/pluginfile.php/53623/mod_resource/content/1/5-GIT.pdf};
  \item \textbf{Versionamento semantico}: \url{https://semver.org/lang/it/;}
  \item \textbf{Studio di Fattibilità}: \textit{Studio di Fattibilità 3.0.0\docs};
  \item \textbf{Piano di Qualifica}: \textit{Piano di Qualifica 3.0.0\docs};
  \item \textbf{Piano di Progetto}: \textit{Piano di Progetto 3.0.0\docs}.
\end{itemize}
