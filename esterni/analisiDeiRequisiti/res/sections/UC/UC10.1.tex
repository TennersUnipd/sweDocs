\subsubsection{UC10.1 - Inserimento nome funzione}
\begin{itemize}
	\item \textbf{Attori primari:} Utente autenticato;
	\item \textbf{Descrizione:} l'utente potrà inserire tramite \textit{CLI\glo} il nome della funzione presente su \textit{Etherless} da eseguire; 
	\item \textbf{Pre-condizioni:} Il sistema fornisce un comando "run" seguito dal nome della funzione di interesse;
	\item \textbf{Post-condizioni:} l'utente ha inserito, a seguito del comando "run", il nome della funzione da eseguire;
	\item \textbf{Scenario principale:} l'utente inserisce il nome della funzione che vuole eseguire;
	\item \textbf{Estensioni:} 
	\begin{itemize}
		\item \textbf{UC10.3}: se non è presente in \textit{Etherless} alcuna funzione con il nome inserito, verrà visualizzato un messaggio di errore.
	\end{itemize}
\end{itemize}