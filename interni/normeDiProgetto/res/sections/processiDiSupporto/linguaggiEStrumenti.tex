\subsection{Linguaggi e Strumenti}
\subsection{Scopo}
Questa sezione del documento si prefigge di spiegare quali \textbf{Linguaggi} e
\textbf{Strumenti} sono stati selezionati per semplificare lo sviluppo di documenti
e di codice anche in relazione alle richieste del proponente.
\subsubsection{Linguaggi}
\paragraph{Linguaggi per la stesura dei documenti}:
Per la stesura dei documenti si è deciso di utilizzare principalmente due Linguaggi:
  \begin{itemize}
    \item \textbf{\LaTeX}:Il linguaggio scelto per la stesura dei documenti è \LaTeX \space in quanto permette
    la suddivisione di un documento in parti diverse e quindi permette a più persone
    di lavorare in contemporanea ad un documento.
    Questo linguaggio ha anche il vantaggio di avere un numero cospicuo di plug-in
    che permettono una facile estensione delle funzionalità.
    \item \textbf{CSV}:Il linguaggio CSV è necessario per la creazione del glossario e viene utilizzato
    per alleggerire la stesura del documento automatizzandola.
  \end{itemize}

\paragraph{Linguaggi di codifica del software di supporto}
Per il supporto dello sviluppo software e di documenti sono emersi diversi linguaggi
di programmazione e Scripting:
 \begin{itemize}
   \item \textbf{\textit{JavaScript\glos}}: verrà utilizzato per scrivere gli unit test per l'applicazione \textit{CLI}\glos;
   \item \textbf{\textit{Solidity\glos}}: verrà utilizzato per scrivere gli unit test per la componente \textit{smart contract}\glos;
   \item \textbf{\textit{Bash Script\glos}}:questo linguaggio è stato utilizzato per la creazione del compilatore automatico
   dei documenti \LaTeX e per fare il controllo sulla digitazione.
   \item \textbf{\textit{Python\glos}}:Il linguaggio Python è stato utilizzato per creare uno strumento per la generazione
   automatica del glossario partendo da un CSV e utilizzando i commit relativi
   al repository per creare una tabella dei cambiamenti;
   \item \textbf{\textit{php\glos}}:questo linguaggio viene utilizzato per  fare
   un \textit{webhook}\glo che mette in comunicazione Automate.io con google drive;
   \item \textbf{\textit{YAML\glos}}: questo linguaggio viene utilizzato per la configurazione dello strumento \textit{GitHub Actions\glos}.
 \end{itemize}

\subsubsection{Strumenti}
Per supportare lo sviluppo e la stesura dei documenti sono stati utilizzati diversi
strumenti.
gli strumenti usati fin ora sono i seguenti:
\begin{itemize}
  \item \textbf{\textit{GitHub Actions\glos}}:
  GitHub Actions è un ambiente per l'esecuzione di script e programmi quando viene
  fatto un'operazione su i repository del progetto ed automatizza i alcuni processi;
  \item \textbf{\textit{Compilatore \LaTeX\glos}}:\LaTeX \space è un linguaggio
  di markup da compilare per ottenere un documento leggibile, si è scelto quindi l'uso di un
  \href{https://GitHub.com/dante-ev/docker-texlive}{compilatore} in ambiente Docker\glos,
  per assicurarci di utilizzare tutti la stessa versione con gli stessi plug-in,
  è stato utilizzato per creare una GitHub Action utile
  alla verifica e creazione dei documenti, tale Action è stata resa pubblica sotto licenza MIT
  ed è possibile visionare \href{https://GitHub.com/Jatus93/Latex-multicompiler}qui lo script;
  \item \textbf{Spell checker}:
  Per facilitare e velocizzare le operazioni di verifica e validazione dei documenti è
  stato scritto, ed inserito in un docker container apposito, uno script per lo
  spell checking, ovvero il controllo di errori di digitazione.
  Anch'esso è stato rilasciato pubblicamente ed è reperibile \href{https://GitHub.com/Jatus93/spellCheck}qui;
  \item \textbf{Script Generazione di glossario}: è stato scritto un script in python per velocizzare la stesura
  del glossario, esso prende in input un file di estensione csv che usa come
  separatore il carattere ',', e ne restituisce un documento che contiene tutti i
  termini in ordine alfabetico e il registro delle modifiche realizzato automaticamente
  in base ai commit del sistema di versionamento utilizzato;
  \item \textbf{Google Sheet}: Questo strumento viene utilizzato per la stesura del glossario
  in congiunzione con gli scritp di automatzione e il webhook;
  \item \textbf{\textit{Truffle\glos}}: come è emerso da \textit{verbaleEsterno\_2019-12-20\_01} è il framework che
  utilizzeremo per lo sviluppo e gestione delle applicazioni in \textit{Ethereum\glos}.
\end{itemize}
