\subsection{Gestione della configurazione}
\subsubsection{Scopo}
Questa sezione si prefiggi di descrivere come e dove verranno conservati le configurazioni
per gli strumenti di supporto.
\subsubsection{Repository e Versionamento}
Le configurazioni del software e relativi rifermenti verranno conservati nel repository
apposito sotto TennersUnipd/Env-Configs.
Qui sarà possibile trovare il template base di ogni documenti e la relativa configurazione
per l'automatizzazione dei processi di creazione.

\subsubsection{Numero di versione}
Per tenere traccia dei progressi si utilizza un sistema di numerazione delle versioni
semantico(\href{https://semver.org/lang/it/}{semver.org}) con un'aggiunta di un TAG.
Generalmente il TAG successivo al numero di versione viene utilizzato per indicare una
pre realease e conseguentemente indica che ciò che è stato prodotto è da testare,
verificare e approvare.
Il numero di versione è nel formato X.Y.Z-TAG e l'incremento dei numeri o cambio
di TAG è così gestito:
\begin{itemize}
  \item \textbf{l'incremento in X}: il primo incremento in X avviene quando vengono
  soddisfatti tutti gli obbiettivi primari dell'attività, un successivo incremento
  avviene se le modifiche del componente, soggetto all'attività, lo stravolgono
  completamente, modificandone il contenuto e/o l'utilizzo;
  \item \textbf{Incremento in Y}: l'incremento in Y avviene quando c'è un'aggiunta
  di una funzionalità o, in caso di un documento, l'aggiunta di un capitolo;
  \item \textbf{Incremento in Z}: l'incremento in Z avviene per modifiche minori,
  ad esempio \textit{code refactoring\glos}, correzioni di forme verbali o errori di digitazione.
  \item \textbf{Cambio di TAG}: i TAG possono essere TBR (To Be Reviewed) e TBA (To Be Approved).
  In caso di versione approvata il TAG viene rimosso.
\end{itemize}

\subsubsection{Code Review}
Come è stabilito nei processi organizzativi ogni nuova feature ha bisogno di un
proprio branch di riferimento.
Per facilitare le operazioni di test e verifica vi sono sistemi automatici di verifica.
Per assicurarsi che ogni nuova modifica venga verificata e approvata sono state attivate
sulla piattaforma GitHub delle \textit{policy}\glo di protezione dei \textit{branch}\glo
di sviluppo.
In questo modo quando una nuova modifica viene completata e viene richiesto l'aggiunta
al ramo di sviluppo principale, verrà richiesto che le modifiche superino i controlli automatici
e viene imposto che venga revisionato ciò che è stato fatto.
