\section{Pianificazione}

La pianificazione del progetto viene suddivisa nelle seguenti fasi:
\begin{itemize}
	\item \textbf{Analisi;}
	\item \textbf{Consolidamento dei requisiti;}
	\item \textbf{Progettazione architetturale;}
	\item \textbf{Progettazione di dettaglio e codifica;}
	\item \textbf{Validazione e collaudo;}
\end{itemize}
Ogni fase ha la sua corrispettiva scadenza (vedi 1.5) e viene suddivisa in sotto-attività prima di iniziare la sua implementazione.
\subsection{Analisi}
Questa fase è stata suddivisa nelle sotto-attività:
\begin{itemize}
	\item \textbf{Individuazione degli strumenti:} questa attività consiste nel determinare quali strumenti il gruppo deve utilizzare per la comunicazione, per la stesura dei documenti e per il versionamento, lo sviluppo e la verifica del software;
	\item \textbf{Norme di Progetto:} sono definite tutte le regole utili per lo svolgimento del progetto, relative al prodotto da realizzare e ai processi da adottare. Il documento \textit{Norme di Progetto} viene redatto dall’Amministratore per conto del Responsabile di progetto;
	\item \textbf{Studio di fattibilità:} in questa attività gli analisti effettuano uno studio sommario dei capitolati in modo da determinare quale di essi verrà scelto. Questa attività è da considerarsi bloccante per l’attività di Analisi dei Requisiti;
	\item \textbf{Analisi dei Requisiti:} durante questa attività vengono identificati ed analizzati i requisiti del capitolato scelto nell’attività di studio di fattibilità e il relativo documento viene composto dagli Analisti;
	\item \textbf{Piano di Qualifica:} in questa attività si individuano le metodologie attraverso le quali si garantisce la qualità del prodotto. A supporto di ciò viene redatto il documento Piano di Qualifica da parte dell’Amministratore e per la parte programmatica dal Progettista;
	\item \textbf{Glossario:} tutti i termini che possono risultare ambigui vengono individuati e definiti nel documento Glossario, che viene redatto durante tutta la fase di analisi dei requisiti;
\end{itemize}

