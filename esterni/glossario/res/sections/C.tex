\subsection*{\textbf{\hfill \Huge{C} \hfill}} 
\addcontentsline{toc}{subsection}{C}
\subsubsection*{Capitolato}
\index{Capitolato}
Documento tecnico, generalmente parte di un contratto d'appalto, che descrive in maniera dettagliata le specifiche tecniche dei prodotti da realizzare. Stabilisce obblighi e diritti tra le parti che stipulano il contratto e descrivono tecniche e materiali da utilizzare per la produzione dell'opera. 

\subsubsection*{CLI}
\index{CLI}
Una interfaccia a riga di comando (dall'inglese Command Line Interface, acronimo CLI) è un tipo di interfaccia utente caratterizzata da un'interazione testuale tra utente ed elaboratore. L'utente impartisce comandi testuali in input mediante tastiera alfanumerica e riceve risposte testuali in output dall'elaboratore mediante display.

\subsubsection*{Cloud}
\index{Cloud}
Abbreviazione di cloud computing\glos.

\subsubsection*{Cloud computing}
\index{Cloud computing}
Indica un paradigma di erogazione di servizi offerti on demand da un fornitore ad un cliente finale attraverso la rete Internet, a partire da un insieme di risorse preesistenti, configurabili e disponibili in remoto sotto forma di architettura distribuita.

\subsubsection*{Cluster}
\index{Cluster}
Insieme di computer connessi tramite una rete telematica.

\subsubsection*{Container}
\index{Container}
Forma di server virtualizzato a livello del sistema operativo, ció permette di dover simulare il solo spazio utente, invece che un'intera macchina, rendendo l'avvio del sistema più veloce ed una condivisione più facile, dovuta a dimensioni ridotte e maggior facilità di importazione. 

\subsubsection*{Criptovalute}
\index{Criptovalute}
Valuta virtuale che può essere scambiata in modalità peer-to-peer (ovvero tra due dispositivi direttamente, senza necessità di intermediari) per acquistare beni e servizi. È protetta mediante un sistema di criptazione che la rende visibile/utilizzabile solo tramite chiavi di accesso pubbliche e private. 

\subsubsection*{Criteri di accettazione}
Criteri secondo cui una \textit{user story}\glo può definirsi completata o meno.
\subsubsection*{Cron job}
Cron è uno schedulatore di lavoro basato sul tempo nei sistemi operativi per computer Unix o Unix. È possibile utilizzare Cron per pianificare i lavori, ad esempio per eseguire comandi o script di shell in determinati orari, date o intervalli. Ciò consente, ad esempio, di automatizzare la manutenzione o la gestione del sistema, di scaricare file da Internet o di inviare e-mail su base regolare.
