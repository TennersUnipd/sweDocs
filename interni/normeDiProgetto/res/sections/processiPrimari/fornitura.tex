\section{Processi Primari}
\subsection{Processi di Fornitura}
\subsubsection{Scopo}
Lo scopo del processo di fornitura è quello di delineare tutte le risorse e le procedure che riguardano l'erogazione del prodotto software che si dovrà realizzare all'interno del progetto. In particolar modo, il processo di fornitura coinvolge e riguarda le seguenti fasi:
\begin{itemize}
	\item \textbf{Avvio:} il fornitore conduce un analisi preliminare sulla fattibilità del progetto e decide se candidarsi o meno come fornitore del prodotto;
	\item \textbf{Proposta e negoziazione del contratto:} a seguito di una richiesta di una prima analisi, il fornitore presenta la proposta formale al cliente. In quanto vincolante, questa contratto dovrà essere opportunamente negoziato o modificato dalle parti in causa;
	\item \textbf{Pianificazione:} dovranno essere stabilite le procedure e le risorse necessarie alla realizzazione del prodotto finale. Il fornitore avrà dunque la necessità di elaborare e presentare un \textit{Piano di Progetto 1.0.0\doc} e un \textit{Piano di Qualifica 1.0.0\doc} che ne assicurino la buona riuscita;
	\item \textbf{Esecuzione e controllo:} il sistema dovrà essere opportunamente implementato, eseguito e controllato secondo i parametri stabiliti nell'attività di pianificazione;
	\item \textbf{Revisione e valutazione:} il fornitore dovrà verificare e validare il prodotto insieme al cliente, in modo da assicurare e garantire il soddisfacimento di tutti i requisiti stabiliti;
	\item \textbf{Consegna:} il prodotto dovrà essere infine consegnato e il fornitore dovrà garantire assistenza al cliente come previsto dai termini contrattuali.
\end{itemize} 
I componenti del gruppo si impegnano nel mantenere una comunicazione stretta con il proponente \textit{Red Babel}, al fine di comprendere al meglio i bisogni e le necessità. Il confronto continuo con il proponente permetterà inoltre una migliore realizzazione del prodotto finale.

\subsubsection{Studio di Fattibilità}
Lo \textit{Studio di Fattibilità 1.0.0\doc} ha lo scopo di analizzare nel dettaglio i capitolati presentati dalle aziende. Il documento,  redatto dagli analisti, è realizzato dopo attenta analisi e discussione da parte dei membri del gruppo nelle apposite riunioni. Per ogni \textit{capitolato\glo} viene indicato:
\begin{itemize}
  \item \textbf{Descrizione Generale}: viene esposto in sintesi il contenuto del \textit{capitolato\glos}, indicando nome del progetto e il proponente;
  \item \textbf{Finalità del Progetto}: viene presentato il \textit{capitolato\glo} dettagliatamente specificando lo scopo del Progetto;
  \item \textbf{Tecnologie Interessate}: vengono elencate le tecnologie coinvolte nel progetto;
  \item \textbf{Aspetti positivi/Criticità}: sono elencati i fattori positivi e di rischio di ciascun \textit{capitolato\glos};
  \item \textbf{Valutazione Conclusiva}: vengono riportate le conclusioni a cui è giunto il team, motivando la scelta o meno del \textit{capitolato\glos}.
\end{itemize}
Il \textit{capitolato\glo} scelto dal gruppo viene evidenziato nella parte introduttiva del documento, nella sezione "Scopo del Prodotto".
\subsubsection{Piano di Progetto}
Quali attività svolgere e dove allocare le risorse a disposizione è indicato nel piano del progetto. Il responsabile e gli amministratori di progetto si occuperanno della loro assegnazione, in modo da rendere il \textit{Piano di Progetto 1.0.0\doc} più efficiente possibile. Questo documento deve contenere le seguenti sezioni:
\begin{itemize}
  \item \textbf{Analisi dei Rischi}: vengono individuati i possibili rischi e difficoltà che possono palesarsi durante lo svolgimento del progetto. Inoltre, vengono esposti eventuali soluzioni o azioni preventive per evitarli;
  \item \textbf{Modello di Sviluppo}: viene definito il metodo di sviluppo scelto e utilizzato per il progetto;
  \item \textbf{Pianificazione}: vengono definite le scadenze temporali per ciascuna attività e i ruoli ai quali assegnare le stesse;
  \item \textbf{Preventivo - Consuntivo}: viene stimato il tempo necessario per ciascuna fase del progetto e, di conseguenza, anche il costo totale del progetto. Viene anche redatto un consuntivo, che riguarda l'andamento effettivo dello stato del progetto e la differenza con ciò che é stato preventivato.
\end{itemize}
\subsubsection{Piano di Qualifica}
Documento redatto dai verificatori con il compito di delineare le strategie per garantire la qualità del prodotto e dei processi coinvolti durante nel suo ciclo di vita. Il documento sarà così strutturato:
\begin{itemize}
  \item \textbf{Qualità di Processo:} sono descritti gli obiettivi e le metriche per garantire la qualità dei processi coinvolti durante il progetto;
  \item \textbf{Qualità di Prodotto:} vengono descritte le caratteristiche di qualità principali del sistema. Per ciascuna verranno descritte gli obiettivi e le metriche per misurarle;
  \item \textbf{Test:} sono elencati i tipi di test da eseguire sul sistema. Vengono suddivisi per categoria e sono caratterizzati da un codice univoco, da una descrizione e dal relativo esito;
  \item \textbf{Risultati processo di Verifica:} vengono riportati i risultati delle verifiche eseguite mediante l'utilizzo delle apposite metriche e il relativo esito, per ciascuna revisione di avanzamento;
  \item \textbf{Standard e modelli di riferimento:} vengono descritti gli standard e i modelli selezionati per la qualità.
\end{itemize}
