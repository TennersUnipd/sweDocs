\section{Introduzione}

\subsection{Scopo del Documento}
Le \textit{Norme di Progetto 1.3.1\doc} hanno lo scopo di definire delle linee guida riguardo il metodo di lavoro che il gruppo Tenners adopererà per la durata del progetto. Stabilire un metodo di lavoro comune a tutti i componenti permetterà il raggiungimento di un'omogeneità e aiuterà il processo di verifica. Questo documento sar\`a aggiornato in maniera incrementale perchè soggetto a costante miglioramento.

\subsection{Scopo del Prodotto}
Il \textit{Capitolato\glo} C2 ha come obiettivo la realizzazione della piattaforma \textit{Etherless}. Essa ha lo scopo di mettere in comunicazione sviluppatori che intendono condividere funzioni proprie scritte in linguaggio JavaScript con altri utenti che desiderano avere accesso a tali funzionalità. Un utente, corrisposto un pagamento in valuta \textit{Ether\glos} allo sviluppatore e alla piattaforma, avrà la possibilità di eseguire una tra le funzioni messe a disposizione e visualizzarne l'output.

%%\subsection{Documenti}
%% TODO Cosa dovrei mettere? Non basta Rifermenti?

\subsection{Glossario}
Come supporto alla documentazione, viene fornito un \textit{Glossario}\docs,
contenente le definizioni di termini specifici che necessitano di un chiarimento.
Ognuno di questi verr\`a contrassegnato con un pedice \glo nel documento e la sua
spiegazione verr\`a riportata sotto la corrispondente lettera del glossario. Ci\`o
consentir\`a di avere un linguaggio comune ed evitare ambiguit\`a.

\subsection{Rifermenti}
\subsubsection{Rifermenti normativi}
\begin{itemize}
  \item \textbf{ISO/IEC 12207:1995 :} \url{https://www.math.unipd.it/~tullio/IS-1/2009/Approfondimenti/ISO_12207-1995.pdf};
  \item \textbf{ISO 8601:1988 (Data and Time Formats)}: \url {https://www.w3.org/TR/NOTE-datetime};
  \item \textbf{Capitolato\glo C2 - Etherless}: \url{https://www.math.unipd.it/~tullio/IS-1/2019/Progetto/C2.pdf}.
\end{itemize}

\subsubsection{Rifermenti informativi}
\begin{itemize}
  \item \textbf{SWEBOK - 2004}: \url{http://www.math.unipd.it/~tullio/IS-1/2007/Approfondimenti/SWEBOK.pdf};
  \item \textbf{Slides del corso di Ingegneria Del Software - Processi software:} \url{https://www.math.unipd.it/~tullio/IS-1/2019/Dispense/L03.pdf};
  \item \textbf{Slides del corso di Ingegneria Del Software - Amministrazione di progetto:} \url{https://www.math.unipd.it/~tullio/IS-1/2019/Dispense/FC01.pdf};
  \item \textbf{Slides del corso di Ingegneria Del Software - Verifica e validazione:} \url{https://www.math.unipd.it/~tullio/IS-1/2019/Dispense/L14.pdf};
  \item \textbf{Slides del corso di Ingegneria Del Software - Verifica analisi statiche:} \url{https://www.math.unipd.it/~tullio/IS-1/2019/Dispense/L15.pdf};
  \item \textbf{Versionamento semantico}: \url{https://semver.org/lang/it/};
  \item \textbf{\textit{Studio di Fattibilità}}: \textit{Studio di Fattibilità 1.0.0\docs};
  \item \textbf{\textit{Piano di Qualifica}}: \textit{Piano di Qualifica 1.0.0\docs};
  \item \textbf{\textit{Piano di Progetto}}: \textit{Piano di Progetto 1.0.0\docs}.

\end{itemize}
