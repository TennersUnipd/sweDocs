\subsubsection{UC18.1 - Inserimento nome funzione}
\begin{itemize}
	\item \textbf{Attori primari:} Utente sviluppatore;
	\item \textbf{Descrizione:} l'utente potrà inserire tramite \textit{CLI}, a seguito di un comando di "set", "update" o "delete", il nome di una su funzione presente su \textit{Etherless}; 
	\item \textbf{Pre-condizioni:} l'utente ha inserito un comando tra "set", "update" e "delete";
	\item \textbf{Post-condizioni:} l'utente ha inserito il nome della funzione da lui caricata alla quale applicare il comando scelto;
	\item \textbf{Scenario principale:} l'utente inserisce il nome della funzione su cui compiere una rimozione, una modifica delle impostazioni o un aggiornamento del codice;
	\item \textbf{Estensioni:} 
	\begin{itemize}
		\item \textbf{UC18.2}: se non è presente in \textit{Etherless} alcuna funzione con il nome inserito, verrà visualizzato un messaggio di errore.
	\end{itemize}
\end{itemize}