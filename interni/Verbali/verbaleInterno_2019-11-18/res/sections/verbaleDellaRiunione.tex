\section{Verbale della riunione}
		Al primo incontro si sono presentati tutti i membri del gruppo, dopo la conoscenza reciproca dei vari membri si è discusso riguardo al nome e il logo del gruppo. Successivamente si è discusso della organizzazione interna del team e delle varie preferenze riguardo ai \textit{capitolati\glos}. 
	\begin{itemize}
		\item \textbf {Nome del gruppo}: tutti i membri hanno espresso le varie idee riguardo al nome del gruppo, è stato scelto "Tenners" in quanto le prime tre lettere del nome (dal inglese Ten) rappresenta il 10, numero con il quale siamo stati identificati il giorno della formazione dei gruppi;
		\item \textbf {Creazione email}: è stato deciso di utilizzare Gmail come servizio mail principale per le varie comunicazioni tra gli \textit{stakeholders\glos};
		\item \textbf {Logo}: sono state proposte varie idee per il logo del gruppo, tra le quali è stata scelta la più convincente;
		\item \textbf {Canali di comunicazione}: per le comunicazioni tra i membri del gruppo è stato scelto il software \textit{Slack\glos};
		\item \textbf {Analisi preliminare dei Capitolati\glos}: il gruppo ha discusso gli aspetti positivi e negativi in seguito alla presentazione dei \textit{Capitolati\glo} fatta dalle aziende proponenti in data 2019-11-15;
		\item \textbf {Strumenti per la stesura dei documenti}: per la scrittura della documentazione di progetto è stato scelto \LaTeX  in quanto oltre a permette una suddivisione adeguata delle sezioni dei documenti e una facile compilazione generando il file in formato pdf;
		\item \textbf {Sistema di versionamento}: come sistema di versionamento è stato scelto \textit{Git\glo} attraverso il servizio \textit{GitHub\glos};
		\item \textbf {Project board Github\glos}: per le organizzazioni interne dei vari compiti da fare con le rispettive scadenze il gruppo ha ritenuto opportuno utilizzare la project board di \textit{GitHub\glos}.
	\end{itemize}

