\section{Test}
In questa sezione vengono ricapitolati i test da eseguire sul prodotto con relativi codici, descrizioni e esiti, se presenti (altrimenti contrassegnati dalla sigla "ND"). I test seguono la logica descritta dal Modello V, di cui menzione e spiegazione è fatta nelle \textit{Norme di Progetto 3.0.0\docs}. La sezione sarà dunque progressivamente aggiornata con l'avanzamento del progetto, aggiungendo nuovi test e relativi esiti.

Ogni test viene contrassegnato nei seguenti modi:
\begin{itemize}
	\item \textbf{ND:} se il test non è ancora stato implementato;
	\item \textbf{I:} se il test è stato implementato.
\end{itemize}
Inoltre ciascun test, se implementato, riporta l'esito effettivo nei seguenti modi:
\begin{itemize}
	\item \textbf{TP}: test implementato con successo;
	\item \textbf{TN}: test implementato senza successo.
\end{itemize}
\subsection{Test di accettazione}
I test di accettazione vengono effettuati per verificare il comportamento del prodotto software rispetto ai suoi requisiti. Essi verranno indicati nel seguente modo:\\\\
\centerline{\textbf{TA[Priorità][Tipologia][Codice]}}\\\\
Dove:
\begin{itemize}
	\item \textbf{Priorità:} indica il livello di importanza dei requisiti di riferimento. Esso può essere:
	\begin{itemize}
		\item \textbf{1:} se i requisiti sono contrassegnati come obbligatori;
		\item \textbf{2:} se i requisiti sono contrassegnati come desiderabili;
		\item \textbf{3:} se i requisiti sono contrassegnati come opzionali.
	\end{itemize}
	\item \textbf{Tipologia:} indica la tipologia dei requisiti di riferimento. Esso può essere:
	\begin{itemize}
		\item \textbf{F:} se i requisiti sono funzionali;
		\item \textbf{P:} se i requisiti sono prestazionali;
		\item \textbf{Q:} se i requisiti sono di qualità;
		\item \textbf{V:} se i requisiti sono di vincolo.
	\end{itemize}
	\item \textbf{Codice:} indica il codice identificativo dei requisiti.
\end{itemize}

\setlength{\tabcolsep}{0.5em}
\renewcommand{\arraystretch}{1.5}
\rowcolors{2}{pari}{dispari}
\begin{longtable}{|C{1.5cm}|C{8cm}|C{3cm}|C{1cm}|}
	\arrayrulecolor{white}

	\caption{Test di accettazione }\\
	\hline
	\rowcolor{header}

	\textbf{Codice} & \textbf{Descrizione test}  & \textbf{Implementazione}& \textbf{Esito}
	\tabularnewline
	\endfirsthead

	TA1F1 &
	L’utente deve poter accedere alla guida dell’applicazione. All’utente viene chiesto di verificare che venga visualizzata la guida e che riporti le informazioni riguardanti i comandi utilizzabili.  &
	ND \\

	TA1F2  &
	L’utente deve poter autenticarsi nel sistema attraverso un'utenza \textit{Ethereum\glos}.
	All’utente è chiesto di verificare l’avvenuta autenticazione. &
	ND \\

	TA1F3 &
	L’utente deve poter autenticarsi nel sistema attraverso un'utenza \textit{Ethereum\glos}. All’utente viene chiesto di verificare che sia possibile eseguire il comando "login" inserendo le proprie credenziali e che sia stato salvato sul dispositivo il file con le credenziali. &
	ND \\

	TA1F3.1 &
	L’utente deve poter autenticarsi nel sistema attraverso un'utenza \textit{Ethereum\glos}. All’utente viene chiesto di verificare sia possibile inserire il proprio address e la propria \textit{private key\glos}. &
	ND \\

	TA1F3.2 &
	L’utente deve poter visualizzare un messaggio di errore nel caso in cui le credenziali non fossero	presenti nella rete \textit{Ethereum\glo}. &
	ND \\

	TA1F4 &
	L’utente deve poter autenticarsi nel sistema attraverso un'utenza \textit{Ethereum}. All’utente viene chiesto di verificare che sia possibile utilizzare tutti i comandi presentati nella guida introduttiva. &
	ND \\

	TA1F5 &
	L’utente non ancora autenticato deve poter effettuare la registrazione alla rete. All’utente viene chiesto di verificare:
	\begin{itemize}
		\item sia possibile eseguire il comando "signup" per effettuare la registrazione;
		\item sia stato salvato sul proprio dispositivo il file con le credenziali;
		\item sia possibile autenticarsi tramite login automatico.
	\end{itemize} &
	ND \\ [-5ex]

	TA2F6 &
	L'utente che effettua l'accesso sul dispositivo per la prima volta, o che vuole registrarsi alla rete, provocherà la creazione del file di configurazione sul dispositivo. Verificare che nel file sia presente un address e la relativa \textit{private key\glo}. &
	ND \\

	TA1F7  &
	L’utente deve potersi disconnettere dalla rete. All’utente viene chiesto di:
	\begin{itemize}
		\item autenticarsi correttamente nel sistema;
		\item verificare sia possibile eseguire il comando "logout" per disconnettersi e che la disconnessione sia avvenuta con successo;
		\item verificare l'avvenuta eliminazione del file.
	\end{itemize} &
	ND \\[-5ex]

	TA1F8  &
	L’utente autenticato deve poter ricercare una funzione. All’utente viene chiesto di verificare sia possibile:
	\begin{itemize}
		\item eseguire il comando "find" per effettuare la ricerca;
		\item inserire il nome della funzione.
	\end{itemize} &
	ND \\[-5ex]

	TA1F8.1  &
	L’utente deve poter visualizzare un messaggio di errore nel caso in cui il nome della funzione non dovesse corrispondere ad alcuna funzione. &
	ND \\

	TA1F9  &
	L’utente autenticato deve poter eseguire una funzione. All’utente viene chiesto di verificare:
	\begin{itemize}
		\item sia possibile eseguire il comando "run" per eseguire una funzione;
		\item sia possibile inserire il nome della funzione;
		\item l'avvenuta esecuzione della funzione con relativo pagamento.
	\end{itemize} &
	ND \\[-5ex]

	TA1F9.1  &
	L’utente deve poter eseguire una funzione attraverso un'utenza \textit{Ethereum\glos}. All’utente viene chiesto di verificare sia possibile inserire il nome della funzione. &
	ND \\

	TA1F9.2  &
	L’utente deve poter passare dei parametri ad una funzione \textit{Ethereum\glos}. All’utente viene chiesto di verificare sia possibile inserire i parametri della funzione. &
	ND \\

	TA1F9.3  &
	L’utente deve poter visualizzare un messaggio di errore nel caso in cui il nome della funzione non dovesse corrispondere ad alcuna funzione. &
	ND \\

	TA1F9.4  &
	L’utente deve poter visualizzare un messaggio di errore nel caso in cui i parametri passati alla funzione non dovessero corrispondere a quelli attesi. &
	ND \\

	TA1F10  &
	L’utente deve pagare per l'esecuzione della funzione \textit{Ethereum\glos} eseguita.
	All’utente viene chiesto di verificare l'avvenuto pagamento del costo della funzione eseguita. &
	ND \\

	TA1F11  &
	L’utente deve poter visualizzare un messaggio di errore nel caso in cui lo sviluppatore della funzione eseguita non ne abbia ancora definito il costo. &
	ND \\

	TA1F12  &
	L’utente deve poter visualizzare un messaggio di errore nel caso in cui il costo della funzione fosse maggiore della sua effettiva capacità di spesa. &
	ND \\

	TA2F13  &
	L’utente autenticato deve poter visualizzare un report delle funzioni da lui eseguite. All’utente viene chiesto di verificare:
	\begin{itemize}
		\item sia possibile eseguire il comando "log";
		\item la presenza nel log dell'elenco delle funzioni eseguite comprendente la data e l'ora di esecuzione di ciascuna;
	\end{itemize} &
	ND \\[-5ex]

	TA1F14  &
	L’utente autenticato deve poter visualizzare l'elenco delle funzioni per lui disponibili. All’utente viene chiesto di verificare:
	\begin{itemize}
		\item sia possibile eseguire il comando "list";
		\item la presenza dell'elenco delle funzioni disponibili;
		\item la presenza, per ogni funzione, del nome, della firma, del costo e della descrizione;
	\end{itemize} &
	ND \\[-5ex]

	TA1F15  &
	L'utente autenticato deve essere in grado di fare il deploy della propria funzione.
	All’utente viene chiesto di verificare sia possibile eseguire il comando \textit{"deploy\glos"}. &
	ND \\

	TA1F15.1  &
	L'utente autenticato deve essere in grado di fare il deploy della propria funzione.All’utente viene chiesto di verificare sia possibile:
	\begin{itemize}
		\item eseguire il comando \textit{"deploy"\glos};
		\item inserire il percorso del file contente la funzione.
	\end{itemize} &
	ND \\[-5ex]

	TA1F15.2  &
	L’utente deve poter visualizzare un messaggio di errore nel caso in cui il percorso specificato non corrisponde ad alcun file con estensione .js. &
	ND \\

	TA1F15.3  &
	L'utente autenticato deve essere in grado di fare il \textit{deploy\glo} della propria funzione. All’utente viene chiesto di inserire il nome della funzione di cui vuole effettuare il \textit{deploy\glos}. &
	ND \\

	TA1F15.4  &
	L’utente deve poter visualizzare un messaggio di errore nel caso in cui il nome specificato appartenga già ad una funzione presente in \textit{Etherium\glo}. &
	ND \\

	TA1F16  &
	L'utente autenticato deve essere in grado di definire alcune informazioni necessarie all'esecuzione della funzione. All’utente viene chiesto di verificare sia possibile eseguire il comando "set". &
	ND \\

	TA1F16.1  &
	L'utente autenticato deve essere in grado di definire il costo della propria funzione. All’utente viene chiesto di verificare sia possibile:
	\begin{itemize}
		\item eseguire il comando "set";
		\item inserire il costo della funzione;
		\item inserire la descrizione della funzione;
		\item inserire la firma della funzione.
	\end{itemize} &
	ND \\[-5ex]

	TA1F16.2  &
	L’utente deve poter visualizzare un messaggio di errore nel caso in cui il nome specificato non dovesse corrispondere ad alcuna funzione di cui l'utente ha eseguito il \textit{deploy\glos}. &
	ND \\

	TA1F16.3  &
	L’utente deve poter visualizzare un messaggio di errore nel caso in cui il costo specificato sia minore o uguale a 0. &
	ND \\

	TA1F16.4  &
	L’utente deve poter visualizzare un messaggio di errore nel caso in cui la descrizione specificata supera il numero massimo consentito di caratteri. &
	ND \\

	TA1F17  &
	L'utente autenticato deve essere in grado di rimuovere da \textit{Etehrless} una delle funzioni caricate. All’utente viene chiesto di verificare sia possibile eseguire il comando "delete". &
	ND \\

	TA1F17.1  &
	L'utente autenticato deve essere in grado di definire il costo della propria funzione. All’utente viene chiesto di verificare sia possibile:
	\begin{itemize}
		\item eseguire il comando "delete";
		\item inserire il nome della funzione.
	\end{itemize} &
	ND \\[-5ex]

	TA1F17.2  &
	L’utente deve poter visualizzare un messaggio di errore nel caso in cui il nome specificato durante la rimozione non dovesse corrispondere ad alcuna funzione di cui l'utente ha eseguito il \textit{deploy\glos}. &
	ND \\

	TA1F18  &
	L'utente autenticato deve essere in grado di aggiornare il codice di una propria funzione. All’utente viene chiesto di verificare sia possibile eseguire il comando "update" di una funzione di cui l'utente ha già eseguito il \textit{deploy\glos}. &
	ND \\

	TA1F18.1  &
	L'utente autenticato deve essere in grado di aggiornare il codice di una propria funzione. All’utente viene chiesto di verificare sia possibile:
	\begin{itemize}
		\item eseguire il comando "update";
		\item inserire il nome della funzione;
		\item inserire il percorso del file contente la funzione.
	\end{itemize} &
	ND \\[-5ex]

	TA1F18.2  &
	L’utente deve poter visualizzare un messaggio di errore nel caso in cui il percorso specificato non corrisponde ad alcun file con estensione .js. &
	ND \\
\end{longtable}


\newpage
\subsection{Test di integrazione}

I test di integrazione vengono effettuati per verificare se l'interazione tra le componenti genera l'effetto desiderato. Essi vengono indicati nel seguente modo:\\\\
\centerline{\textbf{TI[Codice]}}\\\\
Dove:
\begin{itemize}
	\item \textbf{Codice:} indica il codice identificativo dei requisiti.
\end{itemize}
\renewcommand{\arraystretch}{1.5}
\rowcolors{2}{pari}{dispari}
\begin{longtable}{|C{1.5cm}|C{8cm}|C{3cm}|C{1cm}|}
	\arrayrulecolor{white}

	\caption{Test di integrazione}\\
	\hline
	\rowcolor{header}

	\textbf{Codice} & \textbf{Descrizione test}  & \textbf{Implementazione}& \textbf{Esito}
	\tabularnewline
	\endfirsthead
	TI1 &
	Viene verificata l'effettivo logout e disconnessione dell'utente, a livello di interfaccia di rete.  &
	ND \\

	TI2 &
	Viene verificata la disponibilità della funzionalità registrazione di una nuova utenza \textit{Ethereum\glo} a livello di interfaccia di rete.  &
	I \\

	TI3 &
	Viene verificata la disponibilità della funzionalità di accesso con credenziali già esistenti a livello di interfaccia di rete.  &
	I \\

	TI4 &
	Viene verificata la corretta restituzione di tutte le funzioni caricate dagli utenti sul contratto a livello di interfaccia di rete.  &
	ND \\

	TI5 &
	Viene verificata la corretta restituzione del costo di una funzione caricata da un utente sul contratto, a livello di interfaccia di rete.  &
	ND \\

	TI6 &
	Viene verificata la corretta esecuzione di una funzione remota, a livello di interfaccia di rete.  &
	ND \\

	TI7 &
	Viene verificato il corretto caricamento di una funzione sul contratto, a livello di interfaccia di rete.  &
	ND \\

	TI8 &
	Viene verificata la corretta eliminazione di una funzione sul contratto, a livello di interfaccia di rete.  &
	ND \\

	TI9 &
	Viene verificato il corretto aggiornamento di una funzione sul contratto, a livello di interfaccia di rete.  &
	ND \\

	TI10 &
	Viene verificato la corretta restituzione della cronologia delle operazioni effettuate sul contratto, a livello di interfaccia di rete.  &
	ND \\

	TI11 &
	Viene verificato la corretta disconnessione di una sessione, a livello di interfaccia di rete.  &
	ND \\

	TI12 &
	Viene verificato la corretta disconnessione di una sessione, a livello di interfaccia di rete.  &
	ND \\

	TI12 &
	Viene verificato la corretta disconnessione di una sessione, a livello di interfaccia di rete.  &
	ND \\

	TI13 &
	Viene verificata la corretta creazione di una funzione.  &
	I \\

	TI14 &
	Viene verificata la corretta restituzione di tutte le funzioni aggiunte al sistema.  &
	I \\

	TI15 &
	Viene verificata la corretta restituzione di tutti i dettagli di una specifica funzione del sistema.  &
	I \\

	TI16 &
	Viene verificato il corretto rilevamento di una funzione non esistente .  &
	I \\

	TI17 &
	Viene verificata la capacità di eseguire una funzione del sistema.  &
	I \\

	TI18 &
	Viene verificata la restituzione corretta del costo (netto + costo di servizio) di esecuzione di una funzione.  &
	I \\

	TI19 &
	Viene verificata la corretta transazione di credito in seguito all'esecuzione di una funzione.  &
	I \\

	TI20 &
	Viene verificata la corretta transazione di credito in seguito all'esecuzione di una funzione.  &
	I \\
\end{longtable}

\newpage
\subsection{Test di unità}
I test di unità vengono effettuati per la rilevazione dei problemi sulle singole unità che compongono il prodotto software. Essi vengono indicati nel seguente modo:\\\\
\centerline{\textbf{TU[Codice]}}\\\\
Dove:
\begin{itemize}
	\item \textbf{Codice:} indica il codice identificativo dei requisiti.
\end{itemize}


\renewcommand{\arraystretch}{1.5}
\rowcolors{2}{pari}{dispari}
\begin{longtable}{|C{1.5cm}|C{8cm}|C{3cm}|C{1cm}|}
	\arrayrulecolor{white}

	\caption{Test di unità client}\\
	\hline
	\rowcolor{header}

	\textbf{Codice} & \textbf{Descrizione test}  & \textbf{Implemetazione}& \textbf{Esito}
	\tabularnewline
	\endfirsthead

	TU1 &
	Viene verificata la disponibilità della funzionalità registrazione di una nuova utenza \textit{Ethereum\glo} a livello di sessione.  &
	I \\

	TU2 &
	Viene verificata la disponibilità della funzionalità di accesso con credenziali già esistenti a livello di sessione.  &
	I \\

	TU3 &
	Viene verificata la disponibilità della funzionalità "logout" a livello di sessione.  &
	I \\

	TU4 &
	Viene verificata la disponibilità della funzionalità di firma di una transazione a livello di sessione.  &
	I \\

	TU5 &
	Viene verificata l'effettiva autenticazione dell'utente a livello di sessione.  &
	I \\

	TU6 &
	Viene verificato la corretta restituzione dell'indirizzo pubblico dell'account \textit{Ethereum\glo} dell'utente a livello di sessione.  &
	I \\

	TU7 &
	Viene verificata la corretta restituzione del bilancio dell'account \textit{Ethereum\glos}.  &
	I \\

	TU8 &
	Viene verificata la corretta restituzione dell'indirizzo pubblico e della \textit{private-key\glos}.  &
	ND \\

	TU9 &
	Viene verificata la corretta restituzione della lista delle funzionalità disponibili sul contratto.  &
	I \\

	TU10 &
	Viene verificata la corretta restituzione degli argomenti della funzionalità richiesta del contratto.  &
	I \\

	TU11 &
	Viene verificato se la funzionalità del contratto richiede un pagamento da parte dell'utente.  &
	I \\

	TU12 &
	Viene verificata la stima del costo di esecuzione di una funzione sulla rete \textit{Ethereum\glos}.  &
	I \\

	TU13 &
	Viene verificata la restituzione di una transazione necessaria per l'esecuzione di una funzione sulla rete \textit{Ethereum\glo}.  &
	I \\

	TU14 &
	Viene verificata la restituzione di una funzione eseguibile sul contratto disponibile sulla rete \textit{Ethereum\glo} senza necessità di una transazione.  &
	ND \\

	TU15 &
	Viene verificata la restituzione della cronologia delle operazioni effettuate sul contratto.  &
	ND \\

	TU16 &
	Viene verificata la cattura dei segnali provenienti dal contratto sulla rete \textit{Ethereum\glos}.  &
	ND \\

	TU17 &
	Viene verificata la corretta disconnessione dalla rete \textit{Ethereum\glos}.  &
	I \\

	TU18 &
	Viene verificata l'invio di una transazione firmata al contratto sulla rete \textit{Ethereum\glos}.  &
	I \\

	TU19 &
	Viene verificata la richiesta di esecuzione di una funzionalità sul contratto senza necessità di una transazione.  &
	I \\

	TU20 &
	Viene verificata la richiesta di esecuzione di una funzionalità sul contratto senza necessità di una transazione.  &
	I \\

	TU21 &
	Viene verificata l'effettivo logout e disconnessione dell'utente.  &
	ND \\

	TU22 &
	Viene verificata la disponibilità della funzionalità registrazione di una nuova utenza \textit{Ethereum\glo} a livello di interfaccia di rete.  &
	I \\

	TU23 &
	Viene verificata la disponibilità della funzionalità di accesso con credenziali già esistenti a livello di interfaccia di rete.  &
	I \\

	TU24 &
	Viene verificata la corretta restituzione di tutte le funzioni caricate dagli utenti sul contratto a livello di interfaccia di rete.  &
	I \\

	TU25 &
	Viene verificata la corretta restituzione del costo di una funzione caricata da un utente sul contratto, a livello di interfaccia di rete.  &
	I \\

	TU26 &
	Viene verificata la corretta esecuzione di una funzione remota, a livello di interfaccia di rete.  &
	ND \\

	TU27 &
	Viene verificato il corretto caricamento di una funzione sul contratto, a livello di interfaccia di rete.  &
	ND \\

	TU28 &
	Viene verificata la corretta eliminazione di una funzione sul contratto, a livello di interfaccia di rete.  &
	ND \\

	TU29 &
	Viene verificato il corretto aggiornamento di una funzione sul contratto, a livello di interfaccia di rete.  &
	ND \\

	TU30 &
	Viene verificato la corretta restituzione della cronologia delle operazioni effettuate sul contratto, a livello di interfaccia di rete.  &
	ND \\

	TU31 &
	Viene verificato la corretta disconnessione di una sessione, a livello di interfaccia di rete.  &
	ND \\

	TU32 &
	Viene verificata la corretta esecuzione del comando "create", a livello di comando.  &
	I \\

	TU33 &
	Viene verificata la corretta restituzione della descrizione del comando "create", a livello di comando.  &
	I \\

	TU34 &
	Viene verificata la corretta restituzione della alias del comando "create", a livello di comando.  &
	I \\

	TU33 &
	Viene verificata la corretta restituzione del nome del comando "create", a livello di comando.  &
	I \\

	TU34 &
	Viene verificata la corretta esecuzione del comando "find", a livello di comando.  &
	I \\

	TU35 &
	Viene verificata la corretta restituzione della descrizione del comando "find", a livello di comando.  &
	I \\

	TU36 &
	Viene verificata la corretta restituzione della alias del comando "find", a livello di comando.  &
	I \\

	TU37 &
	Viene verificata la corretta restituzione del nome del comando "find", a livello di comando.  &
	I \\

	TU38 &
	Viene verificata la corretta esecuzione del comando "list", a livello di comando.  &
	I \\

	TU39 &
	Viene verificata la corretta restituzione della descrizione del comando "list", a livello di comando.  &
	I \\

	TU40 &
	Viene verificata la corretta restituzione della alias del comando "list", a livello di comando.  &
	I \\

	TU41 &
	Viene verificata la corretta restituzione del nome del comando "list", a livello di comando.  &
	I \\

	TU42 &
	Viene verificata la corretta esecuzione del comando "login", a livello di comando.  &
	I \\

	TU43 &
	Viene verificata la corretta restituzione della descrizione del comando "login", a livello di comando.  &
	I \\

	TU44 &
	Viene verificata la corretta restituzione della alias del comando "login", a livello di comando.  &
	I \\

	TU45 &
	Viene verificata la corretta restituzione del nome del comando "login", a livello di comando.  &
	I \\

	TU46 &
	Viene verificata la corretta esecuzione del comando "logout", a livello di comando.  &
	I \\

	TU47 &
	Viene verificata la corretta restituzione della descrizione del comando "logout", a livello di comando.  &
	I \\

	TU48 &
	Viene verificata la corretta restituzione della alias del comando "logout", a livello di comando.  &
	I \\

	TU49 &
	Viene verificata la corretta restituzione del nome del comando "logout", a livello di comando.  &
	I \\

	TU50 &
	Viene verificata la corretta esecuzione del comando "run", a livello di comando.  &
	I \\

	TU51 &
	Viene verificata la corretta restituzione della descrizione del comando "run", a livello di comando.  &
	I \\

	TU52 &
	Viene verificata la corretta restituzione della alias del comando "run", a livello di comando.  &
	I \\

	TU53 &
	Viene verificata la corretta restituzione del nome del comando "run", a livello di comando.  &
	I \\

	TU54 &
	Viene verificata la corretta esecuzione del comando "signup", a livello di comando.  &
	I \\

	TU55 &
	Viene verificata la corretta restituzione della descrizione del comando "signup", a livello di comando.  &
	I \\

	TU56 &
	Viene verificata la corretta restituzione della alias del comando "signup", a livello di comando.  &
	I \\

	TU57 &
	Viene verificata la corretta restituzione del nome del comando "signup", a livello di comando.  &
	I \\

	TU58 &
	Viene verificata la corretta esecuzione del comando "delete", a livello di comando.  &
	ND \\

	TU59 &
	Viene verificata la corretta restituzione della descrizione del comando "delete", a livello di comando.  &
	ND \\

	TU60 &
	Viene verificata la corretta restituzione della alias del comando "delete", a livello di comando.  &
	ND \\

	TU61 &
	Viene verificata la corretta restituzione del nome del comando "delete", a livello di comando.  &
	ND \\

	TU62 &
	Viene verificata la corretta esecuzione del comando "update", a livello di comando.  &
	ND \\

	TU63 &
	Viene verificata la corretta restituzione della descrizione del comando "update", a livello di comando.  &
	ND \\

	TU64 &
	Viene verificata la corretta restituzione della alias del comando "update", a livello di comando.  &
	ND \\

	TU65 &
	Viene verificata la corretta restituzione del nome del comando "update", a livello di comando.  &
	ND \\
\end{longtable}

\renewcommand{\arraystretch}{1.5}
\rowcolors{2}{pari}{dispari}
\begin{longtable}{|C{1.5cm}|C{8cm}|C{3cm}|C{1cm}|}
	\arrayrulecolor{white}

	\caption{Test di unità smart-contract}\\
	\hline
	\rowcolor{header}

	\textbf{Codice} & \textbf{Descrizione test}  &\textbf{Implementazione}& \textbf{Esito}
	\tabularnewline
	\endfirsthead

	TU66 &
	Viene verificata la presenza di almeno una funzione nel contratto.  &
	I \\

	TU67 &
	Viene verificato il credito del contratto quando ne viene eseguito il deploy.  &
	I \\

	TU68 &
	Viene verificata la corretta creazione di una funzione.  &
	I \\

	TU69 &
	Viene verificata la corretta restituzione di tutte le funzioni aggiunte al sistema.  &
	I \\

	TU70 &
	Viene verificata la corretta restituzione di tutti i dettagli di una specifica funzione del sistema.  &
	I \\

	TU71 &
	Viene verificato il corretto rilevamento di una funzione non esistente .  &
	I \\

%	TU72 &
%	Viene verificata la capacità di eseguire una funzione del sistema.  &
%	I \\

	TU72 &
	Viene verificata la restituzione corretta del costo  di esecuzione di una funzione.  &
	I \\

%	TU74 &
%	Viene verificata la corretta transazione di credito in seguito all'esecuzione di una funzione.  &
%	I \\

%	TU75 &
%	Viene verificata la corretta transazione di credito in seguito all'esecuzione di una funzione.  &
%	I \\

	TU73 &
	Viene verificata la corretta transazione di credito dal contratto ad un utente.  &
	ND \\

	TU74 &
	Viene verificata la corretta eliminazione di una funzione.  &
	ND \\

	TU75 &
	Viene verificata il corretto aggiornamento di una funzione.  &
	ND \\

	TU76 &
	Viene verificata il corretto confronto tra due stringhe, a livello di smart contract.  &
	I \\

	TU77 &
	Viene verificata la corretta concatenazione tra due stringhe, a livello di smart contract.  &
	I \\

	TU78 &
	Viene verificata la corretta conversione di un uint256 in stringa, a livello di smart contract.  &
	I \\

	TU79 &
	Viene verificata l'assenza di funzioni appena effettuato il deploy del contratto.  &
	I \\

	TU80 &
	Viene verificata l'impossibilità di aggiungere una funzione se già esistente.  &
	I \\

	TU81 &
	Viene verificata la presenza di una data funzione nel sistema.  &
	I \\

	TU81 &
	Viene verificata la corretta creazione della richiesta di salvataggio su AWS Lambda della funzione.  &
	I \\

	TU82 &
	Viene verificato il corretto invio della funzione su AWS Lambda.  &
	I \\


\end{longtable}
