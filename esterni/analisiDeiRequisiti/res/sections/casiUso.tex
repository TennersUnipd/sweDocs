\section{Casi d'uso}

\subsection{Attori}
In un diagramma dei casi d'uso, gli attori rappresentano le entità esterne che interagiscono con il prodotto \textit{Etherless}. Essi si possono distinguere in due categorie:
\begin{itemize}
	\item \textbf{Attori primari:} coinvolti nell'esecuzione dei casi d'uso, interagiscono con il servizio per soddisfare i propri bisogni;
	\item \textbf{Attori secondari:} forniscono servizio o supporto al sistema. 
\end{itemize}

\subsubsection{Attori primari}
\begin{figure}[h]
	\centering
	\includegraphics[width=9.7cm]{res/img/gerarchiaAttoriPrimari.jpg}
	\caption{Gerarchia attori primari}
\end{figure}

Sono state identificate quattro diverse tipologie di attori primari relazionati tra loro in maniera gerarchica:

\begin{itemize}
	\item \textbf{Utente generico:} utente che ha eseguito il comando per l'avvio dell'applicativo tramite \textit{Etherlesss-cli};
	\item \textbf{Utente non autenticato:} utente che non ha ancora eseguito l'accesso o la registrazione al network \textit{Ethereum\glo} e che dunque non potrà usufruire delle funzionalità dell'applicazione;
	\item \textbf{Utente autenticato:} utente che ha eseguito l'accesso al network \textit{Ethereum\glo} e che potrà eseguire i comandi messi a disposizione del servizio per gli utenti utilizzatori;
	\item \textbf{Utente sviluppatore:} utente che ha la possibilità di eseguire il \textit{deploy\glo} di funzioni JavaScript proprie, oltre che eseguire le altre funzioni messe a disposizione dagli altri utenti del servizio.
\end{itemize}


\subsubsection{Attori secondari}
Dall'analisi sui requisiti del sistema è emersa la presenza di un unico attore secondario:
\begin{itemize}
	\item \textbf{Ethereum\glo Network:} l'autenticazione a \textit{Etherless} coinciderà con l'accesso alla rete \textit{Ethereum\glos}. Oltre a ciò, l'\textit{Ethereum\glo} Network sarà coinvolto nei pagamenti tra gli utenti del sistema.
\end{itemize}


\subsection{Elenco dei casi d'uso}

\begin{figure}[h]
	\centering
	\includegraphics[width=12.3cm]{res/img/useCaseDiagram.jpg}
	\caption{Casi d'uso principali}
\end{figure}
\newpage
\subsubsection{UC1 - Guida introduttiva}
\begin{itemize}
	\item \textbf{Attori primari:} Utente generico;
	\item \textbf{Descrizione:} l'utente generico, appena entrato nell'applicazione, visualizza una guida dei comandi utilizzabili, mediante il comando "init"; 
	\item \textbf{Pre-condizioni:} il sistema è raggiungibile e l'applicazione è stata avviata;
	\item \textbf{Post-condizioni:} nella \textit{CLI\glo} vengono visualizzati i comandi utilizzabili dall'utente ed una loro descrizione;
	\item \textbf{Scenario principale:} l'utente, mediante il comando "init", visualizza la guida introduttiva.
\end{itemize}
\subsubsection{UC2 - Login}
\begin{itemize}
	\item \textbf{Attori primari:} Utente non autenticato
	\item \textbf{Descrizione:} l'utente ha la possibilità di autenticarsi al network \textit{Ethereum\glo} mediante l'inserimento dell'address e di una private key\glos; 
	\item \textbf{Pre-condizioni:} l'utente ha visualizzato la guida introduttiva e vuole eseguire l'accesso al network \textit{Ethereum} tramite l'apposito comando;
	\item \textbf{Post-condizioni:} il sistema avrà autenticato o meno l'utente a seconda dei valori di accesso forniti;
\end{itemize}
\newpage
\subsubsection{UC5 - Login Manuale}
\begin{figure}[h]
	\centering
	\includegraphics[width=\linewidth]{res/img/UC5.jpg}
	\caption{Diagramma UC5 - Login manuale}
\end{figure}
\begin{itemize}
	\item \textbf{Attori primari:} Utente non autenticato;
	\item \textbf{Attori secondari:} \textit{Ethereum\glo} Network;
	\item \textbf{Descrizione:} l'utente, mediante il comando "login" ha la possibilità di autenticarsi al network \textit{Ethereum\glo} inserendo manualmente da \textit{CLI\glo} il proprio address e \textit{private key\glos} e, contestualmente, salvare in automatico le credenziali di accesso su un file sul proprio dispositivo; 
	\item \textbf{Pre-condizioni:} l'utente ha visualizzato la guida introduttiva e desidera autenticarsi manualmente;
	\item \textbf{Post-condizioni:} il sistema avrà autenticato o meno l'utente a seconda dei valori di accesso inseriti dalla \textit{CLI\glos};
	\item \textbf{Scenario principale:} 
	\begin{enumerate}
		\item L'utente scriverà un comando da \textit{CLI\glos}, composto nel seguente modo:
		\begin{itemize}
			\item nome del comando "login";
			\item address;
			\item \textit{private key\glos}.
		\end{itemize} 
		\item Verrà visualizzato a schermo un messaggio relativo all'esito dell'autenticazione.
	\end{enumerate}	
	\item \textbf{Inclusioni:}
		\begin{itemize}
		\item\textbf{UC3}: ogni qualvolta viene eseguito il login manuale da parte dell'utente, viene salvato sul dispositivo un file contenente le credenziali.
	\end{itemize}  
\end{itemize}

\subsubsection{UC5.1 - Inserimento address}
\begin{itemize}
	\item \textbf{Attori primari:} Utente non autenticato;
	\item \textbf{Descrizione:} l'utente potrà inserire tramite \textit{CLI\glo} l'address per l'autenticazione al suo account \textit{Ethereum\glos}; 
	\item \textbf{Pre-condizioni:} l'utente ha inserito il comando "login";
	\item \textbf{Post-condizioni:} l'utente ha inserito, a seguito del comando "login", il proprio address;
	\item \textbf{Scenario principale:} l'utente compila il comando per la richiesta di autenticazione inserendo il proprio address;
\end{itemize}
\subsubsection{UC5.2 - Inserimento private key\glo}
\begin{itemize}
	\item \textbf{Attori primari:} Utente non autenticato;
	\item \textbf{Descrizione:} l'utente potrà inserire tramite \textit{CLI\glo} la \textit{private key\glo} per l' autenticazione al suo account \textit{Ethereum\glos}; 
	\item \textbf{Pre-condizioni:} l'utente ha inserito il comando "login" seguito dal proprio address;
	\item \textbf{Post-condizioni:} l'utente ha inserito, a seguito dell'address, la propria \textit{private key\glos};
	\item \textbf{Scenario principale:} l'utente compila il comando per la richiesta di autenticazione inserendo la propria \textit{private key\glos};
\end{itemize}
\subsubsection{UC5.2 - Inserimento password}
\begin{itemize}
	\item \textbf{Attori primari:} Utente non autenticato;
	\item \textbf{Descrizione:} l'utente potrà inserire tramite \textit{CLI\glo} la password per criptare in locale la coppia address-\textit{private key\glos}; 
	\item \textbf{Pre-condizioni:} l'utente ha inserito il comando "login";
	\item \textbf{Post-condizioni:} l'utente ha inserito, a seguito della propria \textit{private-key\glo}, una password a sua scelta;
	\item \textbf{Scenario principale:} l'utente compila il comando per la richiesta di autenticazione inserendo una password a scelta per criptare i dati in locale.
\end{itemize}
%\subsubsection{UC5.3 -Visualizzazione errore "Credenziali inserite non valide"}
%\begin{itemize}
%	\item \textbf{Attori primari:} Utente non autenticato;
%    \item \textbf{Attori secondari:} \textit{Ethereum\glo} Network;
%	\item \textbf{Descrizione:} il sistema, a seguito dell'inserimento delle credenziali per l'accesso ad \textit{Ethereum\glo} da parte dell'utente, restituisce un errore per il fallimento dell'autenticazione;
%	\item \textbf{Pre-condizioni:} l'utente invia il comando "login" seguito da address e \textit{private key\glos};
%	\item \textbf{Post-condizioni:} il sistema restituisce un errore sulla \textit{CLI\glo} in seguito al fallimento dell'autenticazione;
%	\item \textbf{Scenario principale:} Il sistema notifica all'utente un errore avvenuto nella autenticazione al sistema tramite i valori da lui inseriti non corrispondenti ad alcuna utenza esistente. L'utente non sarà dunque autenticato al sistema e non potrà eseguire le funzionalità messe a disposizione da \textit{Etherless}.
%\end{itemize}

\subsubsection{UC6 - Login Automatico}
\begin{itemize}
	\item \textbf{Attori primari:} Utente non autenticato;
	\item \textbf{Attori secondari:} \textit{Ethereum\glo} Network;
	\item \textbf{Descrizione:} l'utente, se avrà già effettuato almeno una volta l'accesso alla rete \textit{Ethereum\glo} mediante l'applicazione \textit{Etherless}, potrà accedere automaticamente senza l'inserimento manuale delle credenziali; 
	\item \textbf{Pre-condizioni:} l'utente si è registrato oppure ha eseguito l'accesso almeno una volta mediante \textit{Ethereum\glos};
	\item \textbf{Post-condizioni:} il sistema autenticherà l'utente a seconda dei valori di accesso inseriti nel file;
	\item \textbf{Scenario principale:} se il file è presente nel proprio dispositivo, l'utente avrà la possibilità di utilizzare tutti i comandi disponibili poiché automaticamente autenticato dal sistema. In caso il file risulti assente o corrotto, verrà richiesto il login manuale.
\end{itemize}
\subsubsection{UC4 - Registrazione}
\begin{itemize}
	\item \textbf{Attori primari:} Utente non autenticato;
	\item \textbf{Attori secondari:} \textit{Ethereum\glo} Network;
	\item \textbf{Descrizione:} l'utente, se non possiede un account \textit{Ethereum\glos}, potrà richiederne uno mediante il comando "signup"; 
	\item \textbf{Pre-condizioni:} l'utente ha visualizzato la guida introduttiva e vuole eseguire la registrazione al network \textit{Ethereum\glos};
	\item \textbf{Post-condizioni:} il sistema registrerà l'utente al network \textit{Ethereum\glo} e lo autenticherà al sistema;
	\item \textbf{Scenario principale:} 
	\begin{enumerate}
		\item L'utente inserisce il comando "signin" per la registrazione;
		\item Il sistema registrerà l'utenza sulla rete \textit{Ethereum\glos};
		\item L'utente risulterà autenticato al sistema;
		\item L'utente potrà vedere le proprie credenziali sulla \textit{CLI\glos};
		\item Sarà salvato un file sul dispositivo contenente le credenziali di accesso.
	\end{enumerate}
	\item \textbf{Inclusioni:}
	\begin{itemize}
		\item\textbf{UC3}: ogni qualvolta viene eseguito il login manuale da parte dell'utente, viene salvato sul dispositivo un file contenente le credenziali.
	\end{itemize} 
\end{itemize}
\subsubsection{UC3 - Login Manuale}
\begin{itemize}
	\item \textbf{Attori primari:} Utente non autenticato;
	\item \textbf{Descrizione:} l'utente ha la possibilità di autenticarsi al network \textit{Ethereum\glo} inserendo da \textit{CLI\glo} il proprio address e \textit{private key\glos}; 
	\item \textbf{Pre-condizioni:} l'utente ha visualizzato la guida introduttiva e vuole eseguire l'accesso manuale al network \textit{Ethereum} tramite l'apposito comando;
	\item \textbf{Post-condizioni:} il sistema avrà autenticato o meno l'utente a seconda dei valori di accesso inseriti dall'utente dalla \textit{CLI\glos};
	\item \textbf{Scenario principale:} l'utente effettua l'autenticazione.
\end{itemize}
\subsubsection{UC7 - Logout}
\begin{itemize}
	\item \textbf{Attori primari:} Utente autenticato;
	\item \textbf{Descrizione:} l'utente autenticato vuole eseguire il logout dal sistema eliminando il file contenente le credenziali di accesso; 
	\item \textbf{Pre-condizioni:} l'utente ha effettuato l'accesso ad \textit{Etherless};
	\item \textbf{Post-condizioni:} verrà eseguito il logout dell'utente autenticato;
	\item \textbf{Scenario principale:} 
	\begin{enumerate}
		\item L'utente inserisce il comando "logout";
		\item Il sistema eseguirà il logout dell'utenza eliminando il file per l'accesso automatico.
	\end{enumerate}
\end{itemize}
\subsubsection{UC8 - Ricerca funzione}
\begin{itemize}
	\item \textbf{Attori primari:} Utente autenticato;
	\item \textbf{Descrizione:} l'utente autenticato ricerca per nome i dettagli di una specifica funzione caricata da un utente sviluppatore sulla rete \textit{Etherless}; 
	\item \textbf{Pre-condizioni:} l'utente ha effettuato l'accesso ad \textit{Etherless} e vuole ricercare una funzione specifica mediante l'apposito comando;
	\item \textbf{Post-condizioni:} il sistema visualizzerà a schermo i dettagli della funzione ricercata;
	\item \textbf{Scenario principale:} 
	\begin{itemize}
		\item L'utente inserisce il comando per la ricerca;
		\item Il sistema mostrerà sulla \textit{CLI\glo} nome della funzione, autore, costo e descrizione (se associati) della funzione richiesta.
	\end{itemize}
\end{itemize}
\subsubsection{UC8.1 - Inserimento nome funzione da cercare}
\begin{itemize}
	\item \textbf{Attori primari:} Utente autenticato;
	\item \textbf{Descrizione:} l'utente potrà inserire tramite \textit{CLI\glo} il nome della funzione presente su \textit{Etherless};
	\item \textbf{Pre-condizioni:} l'utente ha inserito il comando "find";
	\item \textbf{Post-condizioni:} l'utente ha inserito, a seguito del comando "find", il nome della funzione da cercare;
	\item \textbf{Scenario principale:} l'utente inserisce il nome della funzione di cui visualizzare i dettagli;
    \item \textbf{Inclusioni:}
    \begin{itemize}
		\item \textbf{UC8.2}: l'inserimento del comando find provoca la visualizzazione dei risulati.
	\end{itemize}
	\item \textbf{Estensioni:}
	\begin{itemize}
		\item \textbf{UC8.3}: se non è presente in \textit{Etherless} alcuna funzione con il nome inserito, verrà visualizzato un messaggio di errore.
	\end{itemize}
\end{itemize}

\subsubsection{UC8.2 - Visualizzazione risultati ricerca}
\begin{itemize}
	\item \textbf{Attori primari:} Utente autenticato;
	\item \textbf{Descrizione:} l'utente visualizzerà sul \textit{CLI\glo} il risultato della ricerca della funzione della quale potrà vedere i dettagli;
	\item \textbf{Pre-condizioni:} l'utente ha eseguito il comando "find";
	\item \textbf{Post-condizioni:} \textit{CLI\glo} visualizza il risultato;
	\item \textbf{Scenario principale:} Il sistema mostrerà sulla \textit{CLI\glo} le seguenti informazioni della funzione:
    		\begin{itemize}
    			\item autore;
    			\item costo;
    			\item descrizione;
    			\item firma della funzione.
    		\end{itemize}
\end{itemize}

\subsubsection{UC9 - Esecuzione funzione}
\begin{itemize}
	\item \textbf{Attori primari:} Utente autenticato;
	\item \textbf{Descrizione:} l'utente autenticato esegue una delle funzioni disponibili sulla piattaforma \textit{Etherless} con l'apposito comando; 
	\item \textbf{Pre-condizioni:} l'utente ha effettuato l'accesso ad \textit{Etherless} e vuole eseguire una tra le funzioni messe a disposizione dagli utenti sviluppatori pagando la somma richiesta;
	\item \textbf{Post-condizioni:} il sistema visualizzerà a schermo gli output della funzione eseguita dall'utente;
	\item \textbf{Scenario principale:} 
	\begin{enumerate}
		\item L'utente inserisce il comando per l'esecuzione della funzione;
		\item Il sistema mostrerà sulla \textit{CLI\glo} l'output della funzione richiesta in base al valore dei parametri inseriti.
	\end{enumerate}
	\item \textbf{Inclusioni:} 
	\begin{itemize}
		\item \textbf{UC10:} Ogni qualvolta l'utente esegua una funzione, deve corrispondere il pagamento della cifra imposta dall'utente sviluppatore.
	\end{itemize}
\end{itemize}
\subsubsection{UC9.1 - Inserimento nome funzione da cercare}
\begin{itemize}
	\item \textbf{Attori primari:} Utente autenticato;
	\item \textbf{Descrizione:} l'utente potrà inserire tramite \textit{CLI\glo} il nome della funzione presente su \textit{Etherless};
	\item \textbf{Pre-condizioni:} l'utente ha inserito il comando "find";
	\item \textbf{Post-condizioni:} l'utente ha inserito, a seguito del comando "find", il nome della funzione da cercare;
	\item \textbf{Scenario principale:} l'utente inserisce il nome della funzione di cui visualizzare i dettagli;
	\item \textbf{Estensioni:}
	\begin{itemize}
		\item \textbf{UC9.2}: l'inserimento del comando find provoca la visualizzazione dei risultati;
		\item \textbf{UC9.3}: se non è presente in \textit{Etherless} alcuna funzione con il nome inserito, verrà visualizzato un messaggio di errore.
	\end{itemize}
\end{itemize}

\subsubsection{UC9.2 - Visualizzazione risultati ricerca}
\begin{figure}[h]
	\centering
	\includegraphics[width=0.7\linewidth]{res/img/UC8.2.jpg}
	\caption{Diagramma UC9.2 - Visualizzazione risultati ricerca}
\end{figure}
\begin{itemize}
	\item \textbf{Attori primari:} Utente autenticato;
	\item \textbf{Descrizione:} l'utente visualizzerà sul \textit{CLI\glo} il risultato della ricerca della funzione della quale potrà vedere i dettagli;
	\item \textbf{Pre-condizioni:} l'utente ha eseguito il comando "find";
	\item \textbf{Post-condizioni:} \textit{CLI\glo} visualizza le informazioni relative alla funzione;
	\item \textbf{Scenario principale:} Il sistema mostrerà sulla \textit{CLI\glo} le informazioni relative alla funzione ricercata.
%    		\begin{itemize}
%    			\item autore;
%    			\item costo;
%    			\item descrizione;
%    			\item firma della funzione.
%    		\end{itemize}
\end{itemize}

\subsubsection{UC9.3 - Visualizzazione errore "Nome della funzione non valido"}
\begin{itemize}
	\item \textbf{Attori primari:} Utente autenticato;
	\item \textbf{Descrizione:} il sistema, a seguito dell'inserimento di un nome di funzione non esistente su \textit{Etherless} e dell'invio del comando, restituirà un errore; 
	\item \textbf{Pre-condizioni:} l'utente invia il comando "run" seguito dal nome della funzione da ricercare e i valori dei parametri;
	\item \textbf{Post-condizioni:} il sistema restituisce un errore sulla \textit{CLI\glo} dopo non aver trovato la funzione richiesta tra le disponibili;
	\item \textbf{Scenario principale:} il sistema notifica all'utente l'assenza della funzione che si è tentato di eseguire.
\end{itemize}
\subsubsection{UC9 - Esecuzione funzione}
\begin{itemize}
	\item \textbf{Attori primari:} Utente autenticato;
	\item \textbf{Descrizione:} l'utente autenticato esegue una delle funzioni disponibili sulla piattaforma \textit{Etherless} con l'apposito comando; 
	\item \textbf{Pre-condizioni:} l'utente ha effettuato l'accesso ad \textit{Etherless} e vuole eseguire una tra le funzioni messe a disposizione dagli utenti sviluppatori pagando la somma richiesta;
	\item \textbf{Post-condizioni:} il sistema visualizzerà a schermo gli output della funzione eseguita dall'utente;
	\item \textbf{Scenario principale:} 
	\begin{enumerate}
		\item L'utente inserisce il comando per l'esecuzione della funzione;
		\item Il sistema mostrerà sulla \textit{CLI\glo} l'output della funzione richiesta in base al valore dei parametri inseriti.
	\end{enumerate}
	\item \textbf{Inclusioni:} 
	\begin{itemize}
		\item \textbf{UC10:} Ogni qualvolta l'utente esegua una funzione, deve corrispondere il pagamento della cifra imposta dall'utente sviluppatore.
	\end{itemize}
\end{itemize}
\subsubsection{UC10 - Esecuzione funzione}
\begin{figure}[h]
	\centering
	\includegraphics[width=\linewidth]{res/img/UC9.jpg}
	\caption{Diagramma UC10 - Esecuzione funzione}
\end{figure}
\begin{itemize}
	\item \textbf{Attori primari:} Utente autenticato;
	\item \textbf{Descrizione:} l'utente autenticato esegue una delle funzioni disponibili sulla piattaforma \textit{Etherless} tramite il comando "run"; 
	\item \textbf{Pre-condizioni:} l'utente ha effettuato l'accesso ad \textit{Etherless} e vuole eseguire una tra le funzioni messe a disposizione dagli utenti sviluppatori pagando la somma richiesta;
	\item \textbf{Post-condizioni:} il sistema visualizzerà a schermo l'output della funzione eseguita dall'utente;
	\item \textbf{Scenario principale:} 
	\begin{enumerate}
		\item L'utente scriverà un comando da \textit{CLI\glo} composto nel seguente modo:
		\begin{itemize}
			\item nome del comando "run";
			\item nome della funzione;
			\item valore dei parametri.
		\end{itemize}
		\item Se il sistema non rileva errori durante dopo l'invio del comando, viene eseguito il pagamento per l'esecuzione della funzione;
		\item Il sistema, dopo aver eseguito la funzione richiesta, mostrerà sulla \textit{CLI\glo} l'output in base al valore dei parametri inseriti.
	\end{enumerate}
	\item \textbf{Inclusioni:} 
	\begin{itemize}
		\item \textbf{UC11:} Ogni qualvolta l'utente esegua una funzione, deve necessariamente corrispondere il pagamento della cifra imposta dall'utente sviluppatore.
	\end{itemize}
\end{itemize}
\subsubsection{UC11 - Visualizzazione errore "Costo della funzione non specificato"}
\begin{itemize}
	\item \textbf{Attori primari:} Utente autenticato;
	\item \textbf{Descrizione:} il sistema, in seguito al tentativo di esecuzione e pagamento di una delle funzioni disponibili da parte di un utente, notifica all'utente che il costo della stessa non è ancora stato impostato dallo sviluppatore; 
	\item \textbf{Pre-condizioni:} l'utente ha utlizzato il comando per l'esecuzione di una specifica funzione;
	\item \textbf{Post-condizioni:} il sistema notifica un messaggio di errore riguardo l'assenza del costo di esecuzione della funzione;
	\item \textbf{Scenario principale:} 
	\begin{enumerate}
		\item L'utente tenta l'esecuzione di una funzione con l'apposito comando;
		\item Il sistema interrompe l'esecuzione notificando all'utente che non è stato specificato alcun costo per l'esecuzione della funzione.
	\end{enumerate}
\end{itemize}
\subsubsection{UC12 - Visualizzazione errore "Fondi non sufficienti"}
\begin{itemize}
	\item \textbf{Attori primari:} Utente autenticato;
	\item \textbf{Descrizione:} il sistema, in seguito al tentativo di esecuzione e pagamento di una delle funzioni disponibili da parte di un utente, gli notifica che non dispone di crediti \textit{Ether\glo} a sufficienza; 
	\item \textbf{Pre-condizioni:} l'utente ha utilizzato il comando "run" per l'esecuzione di una specifica funzione;
	\item \textbf{Post-condizioni:} il sistema notifica un messaggio di errore riguardo l'assenza di crediti \textit{Ether\glos};
	\item \textbf{Scenario principale:} 
	\begin{enumerate}
		\item L'utente tenta l'esecuzione di una funzione con il comando "run";
		\item Il sistema interrompe l'esecuzione notificando all'utente che l'importo da corrispondere supera i crediti \textit{Ether\glo} presenti nel suo wallet.
	\end{enumerate}
\end{itemize}
\subsubsection{UC13 - Log funzioni eseguite}
\begin{itemize}
	\item \textbf{Attori primari:} Utente autenticato;
	\item \textbf{Descrizione:} l'utente visualizza il log delle funzioni da lui eseguite mediante l'apposito comando; 
	\item \textbf{Pre-condizioni:} l'utente ha utilizzato il comando per la visualizzazione del log;
	\item \textbf{Post-condizioni:} il sistema visualizzerà sulla \textit{CLI\glo} il log delle funzioni eseguite dall'utente;
	\item \textbf{Scenario principale:} 
	\begin{enumerate}
		\item L'utente richiede il log mediante l'apposito comando;
		\item Il sistema visualizzerà su schermo i log con le seguenti informazioni:
		\begin{itemize}
			\item data e ora di esecuzione della funzione;
			\item nome della funzione eseguita;
			\item firma della funzione;
			\item costo della funzione.
		\end{itemize}
	\end{enumerate}
\end{itemize}
\subsubsection{UC14 - Elenco funzioni disponibili}
\begin{itemize}
	\item \textbf{Attori primari:} Utente autenticato;
	\item \textbf{Descrizione:} l'utente visualizza la lista delle funzioni disponibili mediante il comando "list"; 
	\item \textbf{Pre-condizioni:} l'utente ha utilizzato il comando per la visualizzazione della lista delle funzioni rese disponibili dagli sviluppatori;
	\item \textbf{Post-condizioni:} il sistema visualizzerà sulla \textit{CLI\glo} l'elenco delle funzioni;
	\item \textbf{Scenario principale:} 
	\begin{enumerate}
		\item L'utente richiede l'elenco mediante comando "list";
		\item Il sistema visualizzerà su schermo la lista delle funzioni specificandone:
		\begin{itemize}
			\item nome;
			\item firma;
			\item costo;
			\item descrizione.
		\end{itemize}
	\end{enumerate}
\end{itemize}
\subsubsection{UC15 - Creazione funzione DA RIVEDERE}
\begin{itemize}
	\item \textbf{Attori primari:} Utente sviluppatore;
	\item \textbf{Descrizione:} l'utente, utilizzando lo specifico comando, crea l'istanza di una nuova funzione da rendere disponibile a tutti gli utenti; 
	\item \textbf{Pre-condizioni:} l'utente ha utilizzato il comando per la creazione di una nuova funzione;
	\item \textbf{Post-condizioni:} il sistema si occuperà di creare una nuova istanza su AWS Lambda con il nome specificato in fase di creazione e rendere disponibile la sua esecuzione a tutti gli utenti di \textit{Etherless}. L'utente visualizzerà a schermo l'esito del comando;
	\item \textbf{Scenario principale:} 
	\begin{enumerate}
		\item L'utente crea una nuova funzione mediante l'apposito comando;
		\item Il sistema visualizzerà su schermo l'esito della creazione della funzione.
	\end{enumerate}
\end{itemize}
\subsubsection{UC15.1 - Inserimento comando "list"}
\begin{itemize}
	\item \textbf{Attori primari:} Utente autenticato;
	\item \textbf{Descrizione:} l'utente potrà inserire tramite \textit{CLI\glo} il comando per visualizzare la lista delle funzioni disponibili su \textit{Etherless} e le informazioni riguardanti ciascuna di esse;
	\item \textbf{Pre-condizioni:} l'utente ha inserito il comando "list" per ottenere l'elenco delle funzioni disponibili;
	\item \textbf{Post-condizioni:} il sistema restituisce l'elenco delle funzioni disponibili;
	\item \textbf{Scenario principale:} l'utente inserisce il comando "list" per visualizzare tutte le funzioni presenti su \textit{Etherless} e visualizzarne i dettagli;
	\item \textbf{Estensioni:}
	\begin{itemize}
		\item \textbf{UC15.2}: l'inserimento del comando list provoca la visualizzazione dei risultati;
		\item \textbf{UC15.3}: se non è presente in \textit{Etherless} alcuna funzione con il nome inserito, verrà visualizzato un messaggio di errore.
	\end{itemize}
\end{itemize}
\subsubsection{UC15.2 - Visualizzazione errore "Percorso del file errato"}
\begin{itemize}
	\item \textbf{Attori primari:} Utente sviluppatore;
	\item \textbf{Descrizione:} il sistema, a seguito dell'invio del comando contenente il percorso di un file non corretto, restituirà un errore. Il file della funzione deve avere l'estensione .js;
	\item \textbf{Pre-condizioni:}  l'utente invia il comando "create" seguito dal percorso del file Javascript sul proprio dispositivo e dal nome della funzione da caricare su \textit{Etherless};
	\item \textbf{Post-condizioni:} il sistema notificherà un errore tramite \textit{CLI\glo} che segnala la presenza di un percorso errato;
	\item \textbf{Scenario principale:} il sistema segnalerà all'utente un messaggio "Percorso del file errato".
\end{itemize}
\subsubsection{UC15.3 - Visualizzazione errore "Nessuna funzione presente su Etherless\glos"}
\begin{itemize}
	\item \textbf{Attori primari:} Utente autenticato;
	\item \textbf{Descrizione:} il sistema, a seguito dell'inserimento del comando "list", restituirà un errore per la mancanza di funzioni presenti nel sistema;
	\item \textbf{Pre-condizioni:} l'utente invia il comando "list";
	\item \textbf{Post-condizioni:} il sistema restituisce un errore sulla \textit{CLI\glo} dopo non aver trovato alcuna funzione presente nella piattaforma;
	\item \textbf{Scenario principale:} il sistema notifica all'utente l'assenza di funzioni all'interno di \textit{Etherless\glos}.
\end{itemize}

\subsubsection{UC15.4 - Visualizzazione errore "Funzione già esistente"}
\begin{itemize}
	\item \textbf{Attori primari:} Utente sviluppatore;
	\item \textbf{Descrizione:} il sistema, a seguito dell'invio del comando, restituirà un errore "Funzione già esistente";
	\item \textbf{Pre-condizioni:}  l'utente invia il comando "create" seguito dal percorso del file Javascript sul proprio dispositivo e dal nome della funzione da caricare su \textit{Etherless};
	\item \textbf{Post-condizioni:} il sistema notificherà un errore tramite \textit{CLI\glo} che segnala la presenza di una funzione con lo stesso nome su \textit{Etherless};
	\item \textbf{Scenario principale:} il sistema segnalerà all'utente un messaggio "Funzione già esistente".
\end{itemize}
\newpage
\subsubsection{UC16 - Visione ad alto livello - utente sviluppatore}
\begin{figure}[h]
	\centering
	\includegraphics[width=\linewidth]{res/img/UC16.jpg}
	\caption{Diagramma UC16 - visione ad alto livello - utente sviluppatore}
\end{figure}
\begin{itemize}
	\item \textbf{Attori primari:} Utente sviluppatore;
	\item \textbf{Descrizione:} l'utente che ha eseguito l'autenticazione a \textit{Etherless} mediante l'utenza \textit{Ethereum\glo} e, disponendo di una funzione JavaScript da rendere utilizzabile sulla piattaforma, ha accesso ai comandi dedicati agli utenti sviluppatori;
	\item \textbf{Pre-condizioni:} l'utente ha eseguito l'accesso al sistema e e possiede una funzione JavaScript da caricare sulla piattaforma;
	\item \textbf{Post-condizioni:} l'utente potrà eseguire i comandi messi a disposizione per gli utenti sviluppatori;
	\item \textbf{Scenario principale:}
	\begin{enumerate}
		\item L'utente può eseguire il deploy di una propria funzione JavaScript (UC17);
		\item L'utente può modificare le impostazione di una sua funzione presente nel sistema (UC18);
		\item L'utente può rimuovere una sua funzione presente nel sistema (UC19);
		\item L'utente può aggiornare il codice di una sua funzione presente nel sistema (UC20);
		\item L'utente può eseguire le operazioni descritte da UC18, UC19, UC20 solo se la funzione esiste all'interno della piattaforma, quindi quando non si presenta UC21.
	\end{enumerate}
\end{itemize}

\subsubsection{UC16.1 - Inserimento costo della funzione}
\begin{itemize}
	\item \textbf{Attori primari:} Utente sviluppatore;
	\item \textbf{Descrizione:} l'utente potrà inserire tramite \textit{CLI} il costo da applicare ad una sua funzione. Un utente, per ciascuna esecuzione della funzionalità, dovrà pagare il prezzo impostato in questa fase. Una percentuale della somma dovuta all'utente sviluppatore verrà trattenuta da \textit{Ethereum} per il mantenimento della piattaforma; 
	\item \textbf{Pre-condizioni:} l'utente ha inserito tramite \textit{CLI\glo} il comando "set" seguito dal nome della funzione da lui caricata su \textit{Ethereum};
	\item \textbf{Post-condizioni:} l'utente ha inserito, a seguito del nome della funzione di riferimento, il relativo costo;
	\item \textbf{Scenario principale:} l'utente inserisce il costo che vuole assegnare a una sua funzione presente su \textit{Etherless};
	\item \textbf{Estensioni:} 
	\begin{itemize}
		\item \textbf{UC16.2}: il costo per l'esecuzione di una funzione non può essere 0 o inferiore.
	\end{itemize}
\end{itemize}
\subsubsection{UC16 - Visione ad alto livello - utente sviluppatore}
\begin{figure}[h]
	\centering
	\includegraphics[width=\linewidth]{res/img/UC16.jpg}
	\caption{Diagramma UC16 - visione ad alto livello - utente sviluppatore}
\end{figure}
\begin{itemize}
	\item \textbf{Attori primari:} Utente sviluppatore;
	\item \textbf{Descrizione:} l'utente che ha eseguito l'autenticazione a \textit{Etherless} mediante l'utenza \textit{Ethereum\glo} e, disponendo di una funzione JavaScript da rendere utilizzabile sulla piattaforma, ha accesso ai comandi dedicati agli utenti sviluppatori;
	\item \textbf{Pre-condizioni:} l'utente ha eseguito l'accesso al sistema e e possiede una funzione JavaScript da caricare sulla piattaforma;
	\item \textbf{Post-condizioni:} l'utente potrà eseguire i comandi messi a disposizione per gli utenti sviluppatori;
	\item \textbf{Scenario principale:}
	\begin{enumerate}
		\item L'utente può eseguire il deploy di una propria funzione JavaScript (UC17);
		\item L'utente può modificare le impostazione di una sua funzione presente nel sistema (UC18);
		\item L'utente può rimuovere una sua funzione presente nel sistema (UC19);
		\item L'utente può aggiornare il codice di una sua funzione presente nel sistema (UC20);
		\item L'utente può eseguire le operazioni descritte da UC18, UC19, UC20 solo se la funzione esiste all'interno della piattaforma, quindi quando non si presenta UC21.
	\end{enumerate}
\end{itemize}

\subsubsection{UC16.3 - Inserimento descrizione della funzione}
\begin{itemize}
	\item \textbf{Attori primari:} Utente sviluppatore;
	\item \textbf{Descrizione:} l'utente potrà inserire tramite \textit{CLI\glo} una descrizione per una funzione da lui caricata. Ciò consentirà ad un utente di conoscere dettagli della funzione non intuibili dalla sola segnatura; 
	\item \textbf{Pre-condizioni:} l'utente ha inserito tramite \textit{CLI\glo} il comando "set" seguito dal nome della funzione da lui caricata su \textit{Ethereum\glo} e dal costo;
	\item \textbf{Post-condizioni:} l'utente ha inserito, a seguito del costo della funzione di riferimento, la relativa descrizione;
	\item \textbf{Scenario principale:} l'utente inserisce una breve descrizione per una sua funzione presente su \textit{Etherless};
	\item \textbf{Estensioni:} 
	\begin{itemize}
		\item \textbf{UC16.4}: la descrizione non può superare una lunghezza massima in caratteri.
	\end{itemize}
\end{itemize}
\subsubsection{UC16.4 - Visualizzazione errore "Descrizione troppo lunga"}
\begin{itemize}
	\item \textbf{Attori primari:} Utente sviluppatore;
	\item \textbf{Descrizione:} il sistema, a seguito dell'invio del comando, restituirà un errore "Descrizione troppo lunga" se la descrizione della funzione eccede il numero massimo di caratteri concessi;
	\item \textbf{Pre-condizioni:}  l'utente invia il comando "set" seguito dal nome della funzione da lui caricata su \textit{Ethereum\glos}, dal costo e dalla descrizione;
	\item \textbf{Post-condizioni:} il sistema notificherà un errore tramite \textit{CLI\glo} che segnala l'inserimento della descrizione non accettabile;
	\item \textbf{Scenario principale:} il sistema segnalerà all'utente un messaggio "Descrizione troppo lunga" per avere impostato una descrizione che supera il numero consentito di caratteri.
\end{itemize}
\subsubsection{UC16.5 - Inserimento segnatura della funzione}
\begin{itemize}
	\item \textbf{Attori primari:} Utente sviluppatore;
	\item \textbf{Descrizione:} l'utente potrà inserire tramite \textit{CLI\glo} la segnatura di una propria funzione in formato stringa, in modo che un utente utilizzatore possa essere a conoscenza dei parametri richiesti; 
	\item \textbf{Pre-condizioni:} l'utente ha inserito tramite \textit{CLI\glo} il comando "set" seguito dal nome della funzione da lui caricata su \textit{Ethereum\glos}, dal costo e dalla descrizione;
	\item \textbf{Post-condizioni:} l'utente ha inserito, a seguito della descrizione della funzione di riferimento, la relativa firma specificandone tipo di ritorno, numero e tipo dei parametri;
	\item \textbf{Scenario principale:} l'utente inserisce la firma di una sua funzione presente su \textit{Etherless}.
\end{itemize}
\subsubsection{UC17 - Rimozione funzione}
\begin{itemize}
	\item \textbf{Attori primari:} Utente sviluppatore;
	\item \textbf{Descrizione:} l'utente potrà rimuovere da \textit{Etherless} una tra le funzioni da lui caricate;
	\item \textbf{Pre-condizioni:} l'utente ha eseguito il deploy di almeno una funzione su \textit{Etherless};
	\item \textbf{Post-condizioni:} la funzione inserita specificata dall'utente verrà rimossa;
	\item \textbf{Scenario principale:}
	\begin{enumerate}
		\item L'utente rimuove una sua funzione mediante il comando "delete";
		\item Il sistema visualizzerà su schermo l'esito dell'operazione effettuata.
	\end{enumerate}
\end{itemize}
\begin{itemize}
  \item \textbf{UC19}: il nome della funzione non è tra quelle caricate dall'utente
\end{itemize}

\newpage
\subsubsection{UC18 - Verifica esistenza funzione}
\begin{figure}[h]
	\centering
	\includegraphics[width=\linewidth]{res/img/UC18.jpg}
	\caption{Diagramma UC18 - Verifica esistenza funzione}
\end{figure}
\begin{itemize}
	\item \textbf{Attori primari:} Utente sviluppatore;
	\item \textbf{Descrizione:} l'utente potrà verificare la presenza di una delle sue funzioni su \textit{Etherless} durante la sua rimozione o la modifica delle impostazioni; 
	\item \textbf{Pre-condizioni:} l'utente ha eseguito il deploy di almeno una funzione su \textit{Etherless};
	\item \textbf{Post-condizioni:} viene cercata una funzione con nome corrispondente per poter procedere alla sua rimozione o alla modifica dei suoi settaggi;
	\item \textbf{Scenario principale:} 
	\begin{enumerate}
		\item L'utente verifica la presenza di una sua funzione in \textit{Etherless};
		\item Il sistema procederà con l'eliminazione, la modifica dei settaggi o l'update del codice di una funzione a seconda del comando scelto dall'utente.
	\end{enumerate}
\end{itemize}
\subsubsection{UC18.1 - Inserimento nome funzione}
\begin{itemize}
	\item \textbf{Attori primari:} Utente sviluppatore;
	\item \textbf{Descrizione:} l'utente potrà inserire tramite \textit{CLI}, a seguito di un comando di "set", "update" o "delete", il nome di una su funzione presente su \textit{Etherless}; 
	\item \textbf{Pre-condizioni:} l'utente ha inserito un comando tra "set", "update" e "delete";
	\item \textbf{Post-condizioni:} l'utente ha inserito il nome della funzione da lui caricata alla quale applicare il comando scelto;
	\item \textbf{Scenario principale:} l'utente inserisce il nome della funzione su cui compiere una rimozione, una modifica delle impostazioni o un aggiornamento del codice;
	\item \textbf{Estensioni:} 
	\begin{itemize}
		\item \textbf{UC18.2}: se non è presente in \textit{Etherless} alcuna funzione con il nome inserito, verrà visualizzato un messaggio di errore.
	\end{itemize}
\end{itemize}
\subsubsection{UC18.2 - Visualizzazione errore "Costo non valido"}
\begin{itemize}
	\item \textbf{Attori primari:} Utente sviluppatore;
	\item \textbf{Descrizione:} il sistema, a seguito dell'invio del comando, restituirà un errore "Costo non valido" se il costo inserito risulta minore o uguale a 0;
	\item \textbf{Pre-condizioni:}  l'utente invia il comando "set" seguito dal nome della funzione da lui caricata su \textit{Ethereum\glos}, dal costo e dalla descrizione;
	\item \textbf{Post-condizioni:} il sistema notificherà un errore tramite \textit{CLI\glo} che segnala l'inserimento di un costo non accettabile;
	\item \textbf{Scenario principale:} il sistema segnalerà all'utente un messaggio "Costo non valido" per avere impostato un costo con valore negativo o nullo.
\end{itemize}
\subsubsection{UC19 - Aggiornamento codice funzione}
\begin{figure}[h]
	\centering
	\includegraphics[width=\linewidth]{res/img/UC19.jpg}
	\caption{Diagramma UC19 - Aggiornamento codice funzione}
\end{figure}
\begin{itemize}
	\item \textbf{Attori primari:} Utente sviluppatore;
	\item \textbf{Descrizione:} l'utente potrà eseguire nuovamente il deploy di una sua funzione preesistente, aggiornandone il codice; 
	\item \textbf{Pre-condizioni:} l'utente possiede una funzione Javascript sul proprio dispositivo;
	\item \textbf{Post-condizioni:} il sistema si occuperà di aggiornare il codice della funzione da lui già condivisa su \textit{Etherless}. L'utente visualizzerà a schermo l'esito del comando;
	\item \textbf{Scenario principale:} 
	\begin{enumerate}
		\item L'utente tramite il comando "update" aggiorna il codice di una sua funzione;
		\item Il sistema visualizzerà su schermo l'esito dell'operazione.
	\end{enumerate}
\end{itemize}
\subsubsection{UC19.1 - Inserimento percorso del file}
\begin{itemize}
	\item \textbf{Attori primari:} Utente sviluppatore;
	\item \textbf{Descrizione:} l'utente potrà inserire tramite \textit{CLI\glo} il percorso del file Javascript contenente la funzione da modificare; 
	\item \textbf{Pre-condizioni:} l'utente ha inserito tramite \textit{CLI\glo} il comando "update" seguito dal nome della funzione di cui aggiornare il codice;
	\item \textbf{Post-condizioni:} l'utente ha inserito, a seguito del nome della funzione, il nuovo percorso del file.js;
	\item \textbf{Scenario principale:} l'utente inserisce il percorso del file per aggiornare il codice di una sua funzione presente su \textit{Etherless};
	\item \textbf{Estensioni:} 
	\begin{itemize}
		\item \textbf{UC19.2}: il percorso inserito non corrisponde ad alcun file con estensione .js.
	\end{itemize}
\end{itemize}
\subsubsection{UC19.2 - Visualizzazione errore "Percorso del file errato"}
\begin{itemize}
	\item \textbf{Attori primari:} Utente sviluppatore;
	\item \textbf{Descrizione:} il sistema, a seguito dell'inserimento dell'invio del comando contenente il percorso di un file non corretto, restituirà un errore. Il file della funzione deve avere l'estensione .js;
	\item \textbf{Pre-condizioni:}  l'utente invia il comando "update" seguito dal nome della funzione da aggiornare e dal percorso del file JavaScript sul proprio dispositivo;
	\item \textbf{Post-condizioni:} il sistema notificherà un errore tramite \textit{CLI\glo} che segnala la presenza di un percorso errato;
	\item \textbf{Scenario principale:} il sistema segnalerà all'utente un messaggio "Percorso del file errato".
\end{itemize}
