\paragraph{Qualità di prodotto}
Per quanto riguarda la qualità del prodotto, si è deciso di adottare lo
\textit{standard ISO/IEC 9126\glos} estrapolandone le proprietà ritenute necessarie
all'ottenimento della qualità del prodotto. Le restanti proprietà sono descritte in maniera dettagliata in appendice.
\subparagraph*{Funzionalità}
%Rappresenta la capacità del prodotto di soddisfare tutti i requisiti che il sistema deve avere.
%\paragraph{Obiettivi}
%\begin{itemize}
%	\item \textbf{Accuratezza:} il prodotto deve fornire i risultati attesi nel modo più preciso possibile;
%	\item \textbf{Appropriatezza:} il prodotto deve fornire il corretto numero di funzionalità per le specifiche concordate;
%	\item \textbf{Interoperabilità:} il prodotto deve interagire e operare nella maniera corretta con altri sistemi.
%\end{itemize}
%\paragraph{Metriche}
\begin{itemize}
	\item \textbf{Copertura funzionale:} indica la percentuale di implementazione delle funzionalità prestabilite. Il calcolo avviene secondo la formula:\\\\
	\centerline{
		\begin{math}
		PERC_{FS}=100*\frac{N_{FS}}{N_FR}
		\end{math}
	}
	\\\\
	Dove:
	\begin{itemize}
		\item \textbf{N\textsubscript{FS}:} numero di funzionalità soddisfatte;
		\item \textbf{N\textsubscript{FR}:} numero di funzionalità richieste.
	\end{itemize}
\end{itemize}

\subparagraph*{Manutenibilità}
%Rappresenta la capacità del prodotto nel subire modifiche o miglioramenti a seguito di cambiamenti dell'ambiente, delle specifiche o dei requisiti del sistema.
%\paragraph{Obiettivi}
%\begin{itemize}
%	\item \textbf{Analizzabilità:} il prodotto, per come è realizzato, deve permette di rilevare velocemente le cause di possibili errori o malfunzionamenti;
%	\item \textbf{Modificabilità:} una modifica correttiva del prodotto deve causare cambiamenti minimi al resto del sistema;
%	\item \textbf{Stabilità:} apportando una modifica al prodotto, devono essere ridotti al minimo effetti inaspettati.
%\end{itemize}
%\paragraph{Metriche}
\begin{itemize}
	\item \textbf{Indice di complessità ciclomatica:} si propone di calcolare con un valore numerico la complessità del flusso di controllo di un programma:\\\\
	\centerline{
		\begin{math}
		v(G) = e-n+p
		\end{math}
	}
	\\\\Dove:
	\begin{itemize}
		\item \textbf{e:} numero degli archi (nel flusso di controllo);
		\item \textbf{n:} numero dei nodi (istruzioni o espressioni);
		\item \textbf{p:} numero delle componenti collegate da un arco.
	\end{itemize}
	\item \textbf{Percentuale commenti:} indica, con un valore percentuale, la quantità di documentazione sotto forma di codice commentato nel prodotto software. Il calcolo avviene secondo la formula:\\\\
	\centerline{
		\begin{math}
		PERC_{COD}=100*\frac{NR_{COD}}{NR}
		\end{math}
	}
	\\\\Dove:
	\begin{itemize}
		\item \textbf{NR\textsubscript{COD}:} indica il numero di righe commentate nel codice del prodotto;
		\item \textbf{NR:} indica il numero di righe totali del codice del prodotto.
	\end{itemize}
\end{itemize}
