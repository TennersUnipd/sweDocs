\section{Modello di sviluppo}
Per lo svolgimento del progetto \textit{Etherless} è stato scelto il modello incrementale.

\subsection{Modello incrementale}
Il modello incrementale consiste nella realizzazione del prodotto finale per incrementi, ognuno dei quali introduce una funzionalità. \\Questa modalità si combina bene anche con il versionamento del codice che traccia e raggruppa le modifiche. L' aggiunta, modifica e cancellazione di requisiti sono consentite previa discussione con il proponente e sua approvazione; tuttavia non sono permesse durante la fase di sviluppo dell'incremento corrente.
\subsubsection{Vantaggi}
\begin{itemize}
	\item Le funzionalità principali vengono sviluppate prima e il committente può subito valutarle e comunicare il suo feedback;
	\item Sviluppare per incrementi successivi limita l'introduzione di errori;
	\item Le modifiche, l’individuazione e la correzione degli errori sono più economiche;
	\item Fase di test più semplice e mirata;
\end{itemize}
\subsubsection{Svantaggi}
\begin{itemize}
	\item È necessario prevedere delle attività di refactoring del codice: ad ogni incremento avviene un degradamento del sistema causato dall'aumentare del codice e della sua complessità che rende gli incrementi successivi più difficili.
\end{itemize}