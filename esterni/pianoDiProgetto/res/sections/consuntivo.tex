\section{Consuntivo di periodo}
A seguito di ogni periodo definito nella sezione precedente viene fatto un confronto tra quanto preventivato e le ore effettivamente impiegate per ricoprire i vari ruoli. In particolare, ci interessa definire una metrica che possa riassumere la differenza di costo preventivata ed effettiva, derivante dal residuo di ore impiegate per ruolo. \\ A tale metrica si possono assegnare tre valori distinti:
\begin{itemize}
	\item \textbf{Positivo:} il quantitativo di tempo impiegato risulta inferiore rispetto a quanto preventivato;
	\item \textbf{Pari:} il quantitativo di tempo impiegato è in linea a quanto preventivato (differenza entro il 2\%);
	\item \textbf{Negativo:} il quantitativo di tempo impiegato risulta superiore rispetto a quanto preventivato.
\end{itemize}
La metrica è anche indice delle capacità del gruppo di prevedere i tempi necessari per le varie attività.
\\Inoltre, per ogni periodo verrà specificato se la data di scadenza per il raggiungimento dell'obbiettivo di tale fase è stato rispettato.
\subsection{Analisi dei requisiti}
\subsubsection{Impiego risorse}
\begin{table}[h]
\caption{Tabella consuntivo, Analisi dei requisiti}  
\begin{center}
\begin{tabular}{ |c|c|c|  }
 \hline
 Ruolo 		& Ore & Costo (EUR)\\
 \hline\hline
	Responsabile	& 31 (+0) & 930 (+0)\\
	Amministratore	& 35 (+12) & 700 (+240)\\
	Analista		& 84 (+15) & 2100 (+375)\\
	Progettista		& 10 (+4) & 220 (+88)\\
	Programmatore	& 0 (+0) & 0 (+0)\\
	Verificatore	& 50 (-4) & 750 (-60)\\
	\hline\hline
	Totale prev.	& 210 & 4700 \\
	Totale cons.	& 237 & 5343 \\
	Differenza		& +27 & +643 \\
 \hline
\end{tabular}
\end{center}
\end{table}
Si ricorda che il lavoro sostenuto durante questo periodo non viene rendicontato nei confronti del costo per il cliente.
\subsubsection{Conclusioni}
E' risultato necessario un numero di ore maggiore rispetto a quanto preventivato per raggiungere l'obbiettivo richiesto da questo periodo. La discrepanza oraria maggiore è stata rilevata nei ruoli di Amministratore e Analista. Il team ha riscontrato difficoltà più di quanto previsto nella parte decisionale e di configurazione degli strumenti e procedure da utilizzare durante la realizzazione del progetto.
\\Il bilancio di questa periodo è \textbf{Negativo}, con un surplus di 27 ore e uno scarto economico di +643 EUR. La scadenza temporale prefissata per questo periodo è comunque stata rispettata.
\subsection{Progettazione architetturale DA FARE}
\subsubsection{Impiego risorse}
\begin{table}[h]
\caption{Tabella consuntivo, Progettazione architetturale}  
\begin{center}
\begin{tabular}{ |c|c|c|  }
 \hline
 Ruolo 		& Ore & Costo (EUR)\\
 \hline\hline
	Responsabile	& 31 (+0) & 930 (+0)\\
	Amministratore	& 35 (+12) & 700 (+240)\\
	Analista		& 84 (+15) & 2100 (+375)\\
	Progettista		& 10 (+4) & 220 (+88)\\
	Programmatore	& 0 (+0) & 0 (+0)\\
	Verificatore	& 50 (-4) & 750 (-60)\\
	\hline\hline
	Totale prev.	& 210 & 4700 \\
	Totale cons.	& 237 & 5343 \\
	Differenza		& +27 & +643 \\
 \hline
\end{tabular}
\end{center}
\end{table}
Si ricorda che il lavoro sostenuto durante questo periodo non viene rendicontato nei confronti del costo per il cliente.
\subsubsection{Conclusioni}
E' risultato necessario un numero di ore maggiore rispetto a quanto preventivato per raggiungere l'obbiettivo richiesto da questo periodo. La discrepanza oraria maggiore è stata rilevata nei ruoli di Amministratore e Analista. Il team ha riscontrato difficoltà più di quanto previsto nella parte decisionale e di configurazione degli strumenti e procedure da utilizzare durante la realizzazione del progetto.
\\Il bilancio di questa periodo è \textbf{Negativo}, con un surplus di 27 ore e uno scarto economico di +643 EUR. La scadenza temporale prefissata per questo periodo è comunque stata rispettata.
