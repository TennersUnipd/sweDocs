\subsection{Validazione}
\subsubsection{Obiettivi}
La validazione del prodotto garantisce il prodotto sia conforme alle attese e, in particolare, a tutti i bisogni del cliente. Obiettivi del processo sono:
\begin{itemize}
	\item consentire il rilascio finale del prodotto sottostando i vincoli contrattuali concordati con il cliente;
	\item dimostrare la conformità del prodotto rispetto alle attese.
\end{itemize}

\subsubsection{Attività}
Fanno parte del processo di validazione i test di sistema e di accettazione (collaudo) descritti nel modello a V del capitolo precedente. Le attività comprendono il processo di validazione e che sono propedeutiche per quest'ultimo, sono le seguenti:
\begin{itemize}
	\item definizione chiara degli input e output del sistema;
	\item definizione dei valori accettabili dal sistema;
	\item definizione dell'ambiente (hardware e software) sul quale deve essere eseguito il prodotto;
	\item definizione degli errori che possono essere sollevati e di come essi vengono gestiti dal sistema;
	\item definizione della tipologia di validazione utilizzata e dei criteri di accettazione;
	\item riepilogo dei risultati ottenuti per l'accettazione del prodotto.
\end{itemize}

\subsubsection{Procedure}
\paragraph{Procedura di validazione del prodotto}
\begin{enumerate}
	\item identificazione degli elementi sui quali effettuare la validazione;
	\item pianificazione del metodo di validazione;
	\item ottenimento dei risultati per l'elemento validato;
	\item comparazione dei risultati ottenuti con quelli attesi e deduzione del grado di accettabilità.
\end{enumerate}