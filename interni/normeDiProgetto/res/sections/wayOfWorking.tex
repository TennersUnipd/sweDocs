\section{Way of working}
\subsection{Organizzazione del valoro}
Prima di inziare una qualsiasi attività, che sia software development o stesura
di documenti, si inzia facendo un'analisi dei requisiti.
Dopo aver revisionato e approvato tale analisi, si decisono quali sono gli obbiettivi
primari da realizzare e quelli secondari.

\paragraph{Numero di versione}
Per tenere traccia dei progessi si utilizza un sistema di numerazione delle versioni
semantico.
Il numero di versione è nel formato X.Y.Z-TAG e l'incremento dei numeri è così gestito:
\begin{itemize}
  \item \textbf{Cambio di TAG}: i TAG possono essere TBR (To Be Reviewed) e TBA (To Be Approved)
  in caso di versione approvata il TAG viene rimosso.
  \item \textbf{incremento in Z}: gli incrementi in Z avvengono per modifiche minori,
  ad esempio \textit{code refactoring\glos}, correzione di errori di digitazione
  \item \textbf{l'incremento in X}: il primo incremento in X avviene quando vengono
  soddisfatti tutti gli abbiettivi primari dell'attività, un successivo incremento
  avviene se le differenza tra la versione corrente del
\end{itemize}
