\section{Verbale della riunione}
	\begin{itemize}
		\item \textbf{Cambiamenti da effettuare sul registro delle modifiche}: durante la discussione con il Prof. Vardanega abbiamo individuato una modifica sostanziale da effettuare sul registro delle modifiche. Nei report, infatti, non devono essere inserite modifiche o aggiunte non ancora approvate. Per tale motivo il team ha discusso riguardo la soluzione di inserire due colonne all'interno del registro delle modifiche, già precedentemente rilevata ricevendo riscontro positivo da parte del docente;%. Bisogna, quindi, sistemare i report di ogni documento inserendo una colonna dedicata alla data di verifica e da chi è stata svolta;
		\item \textbf{Licenze d'uso da utilizzare per il progetto}: la scelta delle licenze d'uso non deve essere limitata soltanto da quelle indicate nel \textit{capitolato}\glos. Eventualmente si può inserire un disclaimer sulle licenze;
		\item \textbf{Modifiche da effettuare nelle \textit{Norme di Progetto}\docs}: durante la discussione è emersa la necessità di rivedere la forma delle \textit{Norme di progetto}\doc considerando i documenti come componenti del prodotto finale stesso. Tuttavia è necessario che non siano i documenti o i prodotti delle attività a guidare la stesura dei documenti stessi, com'è stato erroneamente fatto in precedenza;
		\item \textbf{Modifiche da effettuare nel \textit{Piano di Progetto}\docs}: come già precedentemente individuato dal gruppo e dalle precedenti riunioni interne, il committente ha sottolineato la necessità di avere una pianificazione maggiormente dettagliata per avere piena adesione al modello di sviluppo adottato dal gruppo;
		\item \textbf{Inserimenti da effettuare nel \textit{Piano di qualifica}\docs}: è emersa la necessità di mostrare l'andamento della qualità mediante l'utilizzo di diagrammi informativi. In questo modo sarà possibile mettere in evidenza la qualità del lavoro svolto e del prodotto finale. Risulta, quindi utile l'inserimento di un cruscotto per capire l'andamento del progetto;
		%\item \textbf{Come effettuare l'approccio incrementale}: per la realizzazione del PoC è importante la collaborazione con il proponente e l'individuazione dei problemi più importanti e come risolverli. Ogni aggiunta che .
	\end{itemize}