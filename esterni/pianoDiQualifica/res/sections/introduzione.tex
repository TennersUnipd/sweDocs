\section{Introduzione}

\subsection{Scopo del Documento}
Questo documento si prefigge di dichiarare le strategie atte a garantire la qualit\'a \glo, definendo quindi i processi di verifica \glo e validazione \glo. Saranno presentate qui non solo le politiche trasversali all'intera organizzazione Tenners, ma anche gli obiettivi da perseguire sul progetto intero. Sar\'a necessario quindi rendere note le procedure e gli strumenti di controllo attraverso le quali ci prefissiamo di raggiungere la qualit'a \glo.
\'E necessario specificare che il documento non \'e statico, ma cambier\'a durante tutta la durata del progetto. Infatti, molte delle politiche o dei processi selezionati inizialmente potranno rivelarsi insufficienti o inadeguati, costringendoci a modificare o aggiungerli. Il piano di qualifica \glo sar\'a prodotto in maniera incrementale e i cambiamenti saranno tracciati nella sezione del registro delle modifiche.

\subsection{Scopo del prodotto}
Il capitolato C2 ha come obbiettivo una piattaforma per svilippatori che metta
in comunicazione un fornitore di funzionalit\`a e chi \`e interessato ad utilizzarle
all'interno del proprio progetto.
La piattaforma prender\`a nome di etherless permettendo a chi sviluppa software e
lo condivide sulla piattaforma di essere pagato, per ogni esecuzione, attraverso
la criptovaluta etherium.


\subsection{Glossario}
Come supporto alla documentazione, viene fornito un \textit{Glossario v.1.0.0},
contenente le definizioni di termini specifici che necessitano di un chiarimento.
Ognuno di questi verr\`a contrassegnato con un pedice \glo nel documento e la sua
spiegazione verr\`a riportata sotto la corrispondente lettera del glossario. Ci\`o
consentir\`a di avere un linguaggio comune ed evitare ambiguit\`a.

\subsection{Rifermenti}
\subsubsection{Rifermenti normativi}
\begin{enumerate}
  \item Piano di qualifica in particola la sezione legata al capitolato C2.
\end{enumerate}
\subsubsection{Rifermenti informativi}
\begin{enumerate}
  \item ISO Standards
	\url{https://www.praxiom.com/}
\end{enumerate}
\subsection{Strategia di gestione della qualità nel dettaglio}
\subsubsection{Risorse}
Le risorse utilizzate per effettuare il controllo di qualità sono sia umane che tecnologiche.

I ruoli principali nel team che hanno un forte impatto nella qualità del prodotto sono il Responsabile e il Verificatore. Tali ruoli sono stati descritti nelle \textit{Norme di progetto 1.0.0}\docs. Il compito del primo è di monitorare la qualità del prodotto, mentre quello del Verificatore è collegato alla qualità di processo.

Le risorse tecnologiche comprendono tutti gli strumenti software e hardware usati per effettuare le procedure di verifica. 
\subsubsection{Misure e metriche}
\`{E} importante per raggiungere degli obiettivi di qualità è far si che il processo di verifica produca dei risultati quantificabili che si possono confrontare con obbiettivi fissati a priori. Per questo vengono scelti le metriche ed i valori di sufficienza minimi necessari, che indicano se i livelli qualitativi di processo e di prodotto sono conformi agli obiettivi prefissati.

Ogni metrica considerata verrà presentata in tre elementi principali:
\begin{itemize}
	\item \textbf{Nome e descrizione}: presenta una breve descrizione della metrica utilizzata e indica cosa rappresenta;
	\item \textbf{Valore minimo}: indica il valore minimo da raggiungere in modo che il processo o il prodotto siano considerati accettabili;
	\item \textbf{Valore ottimale}: indica il valore nel quale si può collocare il risultato della verifica.
\end{itemize}

\paragraph{Metriche per il processo}
\paragraph{Metriche per i documenti}
Per verificare la validità dei documenti, dato che viene usata la lingua italiana, abbiamo deciso di usare l'indice Gulplease. 
\subparagraph*{Indice Gulplease}
Tale indice è stato definito nel 1988 dal GULP (Gruppo Universitario Linguistico Pedagogico) presso l'Istituto Filosofia dell'Università degli studi di Roma "La Sapienza".

Il vantaggio di tale metodo è dato dal fatto che utilizza la lunghezza delle parole e non sulle sillabe, semplificando di molto il calcolo. L'indice si basa su due variabili linguistiche: la lunghezza della parola e la lunghezza della frase rispetta al numero di lettere.

La formula è la seguente:
\begin{center}
	$89+\dfrac{300\times(numerodellefrasi)-10\times(numero delle lettere)}{numero di parole}$
\end{center}
Il valore che si ottiene è un numero compreso tra 0 e 100 ed è possibile suddividerlo in tre indici distinti:
\begin{itemize}
	\item inferiore a 80: indica che il testo risulta difficile da leggere per tutti coloro che hanno una licenza elementare;
	\item inferiore a 60: indica che il testo risulta difficile da leggere per tutti coloro che hanno una licenza media;
	\item inferiore a 40: indica che il testo risulta difficile da leggere per tutti coloro che hanno un diploma superiore.
\end{itemize}
\textbf{Valori}
\begin{itemize}
	\item \textbf{Valore minimo}: $\geq$50;
	\item \textbf{Valore ottimale}: 100.
\end{itemize}
\paragraph{Metriche per il software}
\subparagraph*{Code coverage}
%\subparagraph*{Valori}
\subparagraph*{Halstead complexity measures}
%\subparagraph{Valori}
%indice dipendenza
