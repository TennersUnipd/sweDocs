\subsection{Linguaggi e Strumenti}
\subsection{Scopo}
Questa sezione del documento si prefigge di spiegare quali \textbf{Linguaggi} e
\textbf{Strumenti} sono stati selezionati per semplificare lo sviluppo di documenti
e di codice anche in relazione alle richieste del proponente.
\subsubsection{Linguaggi}
\paragraph{Linguaggi per la stesura dei documenti}
  Per la stesura dei documenti si è deciso di utilizzare principalmente due Linguaggi
\LaTeX e CSV\glos.
\subparagraph{\LaTeX}
Il linguaggio scelto per la stesura dei documenti è \LaTeX \space in quanto permette
la suddivisione di un documento in parti diverse e quindi permette a più persone
di lavorare in contemporanea ad un documento.
Questo linguaggio ha anche il vantaggio di avere un numero cospicuo di plug-in
che permettono una facile estensione delle funzionalità.
\subparagraph{CSV}
Il linguaggio CSV è necessario per la creazione del glossario e viene utilizzato
per alleggerire la stesura del documento automatizzandola.
\paragraph{Linguaggi di programmazione}
Per il supporto dello sviluppo software e di documenti sono emersi diversi linguaggi
di programmazione e Scripting.
\subparagraph{TypeScript}
TypeScript è un linguaggio di programmazione derivante dal JavaScript(JS) ed è il
linguaggio scelto per lo sviluppo del progetto per quanto concerne l'uso in
\textit{CLI}\glo e per le applicazioni in ambiente \textit{AWS}\glo è verranno utilizzati
per scrivere gli unit test per la relativa applicazione.
\subparagraph{Solidity}
Solidity è il linguaggio di programmazione per applicazioni compatibili con la
piattaforma etherium, e verrà utilizzato per creare l'applicazione \textit{smart contract}\glo
per testare le applicazioni verrà utilizzato \textit{truffle}\glo come emerso
da \textit{verbaleEsterno\_2019-12-20\_01}
\subparagraph{Bash script}
Questo linguaggio è stato utilizzato per la creazione del compilatore automatico
dei documenti \LaTeX e per fare il controllo sulla digitazione.
\subparagraph{Python}
Il linguaggio Python è stato utilizzato per creare uno strumento per la generazione
automatica del glossario partendo da un CSV e utilizzando i commit relativi
al repository per creare una tabella dei cambiamenti.

\subsubsection{Strumenti}
Per supportare lo sviluppo e la stesura dei documenti sono stati utilizzati diversi
strumenti.
gli strumenti usati fin ora sono i seguenti

\paragraph{Strumenti per la stesura dei documenti}
Per facilitare la scrittura di documenti e velocizzarne la verificato sono stati
messi a punto degli strumenti per la verifica e costruzione automatica.
Sfruttando le GitHub Actions sono stati creati degli script per la compilazione
automatica dei documenti \LaTeX \space e per il conrtollo ortografico delle parole.
Successivamente è stato scritto uno script per facilitare la stesura dei termini
di glossario.

\paragraph{GitHub Actions}
GitHub Actions è un ambiente per l'esecuzione di script e programmi quando viene
fatto un'operazione sul repository del progetto ed automatizza i alcuni processi.

\paragraph{Compilatore}
Essendo \LaTeX \space un linguaggio di markup da compilare per ottenere un documento
leggibile si è scelto l'uso di un
\href{https://GitHub.com/dante-ev/docker-texlive}{compilatore} in ambiente Docker\glo
per assicurarci di utilizzare tutti la stessa versione con gli stessi plug-in.
Il compilatore sopracitato è stato utilizzato per creare una GitHub Action utile
alla verifica e creazione dei documenti, tale Action è stata resa pubblica sotto licenza MIT
ed è possibile visionarla \href{https://GitHub.com/Jatus93/Latex-multicompiler}qui

\paragraph{Spell checker}
Per facilitare e velocizzare le operazioni di verifica e validazione dei documenti è
stato scritto, ed inserito in un docker container apposito, uno script per lo
spell checking, ovvero il controllo di errori di digitazione.
Anch'esso è stato rilasciato pubblicamente ed è reperibile \href{https://GitHub.com/Jatus93/spellCheck}qui

\paragraph{Script Generazione di glossario}
Come accennato è stato scritto un script in python per velocizzare la stesura
del glossario, esso prende in input un file di estensione csv che usa come
separatore il carattere ',', e ne restituisce un documento che contiene tutti i
termini in ordine alfabetico e il registro delle modifiche realizzato automaticamente
in base ai commit del sistema di versionamento utilizzato.
