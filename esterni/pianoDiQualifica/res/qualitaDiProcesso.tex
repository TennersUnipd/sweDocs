\section{Qualità di prodotto}
\subsection{Processo di Sviluppo}
Lo Sviluppo è un processo che raggruppa le attività relative alla realizzazione effettiva del prodotto software. Garantendo la qualità di ogni attività che lo compongono mediante le apposite metriche, il Processo di Sviluppo può considerarsi a sua volta qualitativamente corretto. Si procederà dunque con la stesura di un Piano per la Qualità per:
\begin{itemize}
	\item Analisi dei Requisiti;
	\item Progettazione;
	\item Implementazione.
\end{itemize}
\subsubsection{Analisi dei Requisiti}
\paragraph{Obiettivi}
Per assicurare la qualità dell'Analisi dei Requisiti, occorre che:
\begin{itemize}
	\item i requisiti siano chiari e non contraddittori;
	\item subiscano modifiche di raffinamento con l'avanzamento del progetto;
	\item siano completi per ogni \textit{stakeholder\glo} coinvolto nel progetto.
\end{itemize}
\paragraph{Metriche}
\begin{itemize}
	\item \textbf{Requisiti obbligatori soddisfatti:} indica la percentuale di requisiti obbligatori soddisfatti. Il calcolo avviene secondo la formula:\\\\
	\centerline{
		\begin{math}
		PERC_{OS}=100*\frac{R_{OS}}{R_O}
		\end{math}
	}
	\\\\Dove:
	\begin{itemize}
		\item \textbf{R\textsubscript{OS}:} numero di requisiti obbligatori soddisfatti;
		\item \textbf{R\textsubscript{O}:} numero di requisiti obbligatori
	\end{itemize}
	\item \textbf{Requsiiti desiderabili soddisfatti:} indica la percentuale di requisiti desiderabili soddisfatti. Il calcolo avviene secondo la formula:\\\\
		\centerline{
		\begin{math}
		PERC_{DS}=100*\frac{R_{DS}}{R_D}
		\end{math}
	}
	\\\\Dove:
	\begin{itemize}
		\item \textbf{R\textsubscript{DS}:} numero di requisiti obbligatori soddisfatti;
		\item \textbf{R\textsubscript{D}:} numero di requisiti obbligatori
	\end{itemize}	
\end{itemize}

\renewcommand{\arraystretch}{2.2}
\rowcolors{2}{pari}{dispari}
\begin{longtable}{|C{4.5cm}|C{2.25cm}|C{2.25cm}|C{3cm}|}
	\arrayrulecolor{white}
	
	\caption{Metriche per la funzionalità del prodotto}\\
	\hline
	\rowcolor{header}
	
	\textbf{Metrica} & \textbf{Intervallo di accettazione}  & \textbf{Intervallo ottimale} & \textbf{Unità di misura}
	\tabularnewline
	\endfirsthead
	
	Requisiti obbligatori soddisfatti &  100 & 100 & Percentuale \\ 
	Requisiti desiderabili soddisfatti &  > 50 & 100 & Percentuale \\ 
\end{longtable}



\subsubsection{Progettazione}
\paragraph{Obiettivi}
Per assicurare una corretta Progettazione assicurandone qualità, occorre che:
\begin{itemize}
	\item DA FARE
	\item DA FARE
\end{itemize}
\paragraph{Metriche}
\begin{itemize}
	\item \textbf{Structural Fan-out (SFOUT):} indica il numero di moduli che dipendono dal modulo corrente;
	\item \textbf{Structural Fan-in (SFIN):} indica il numero di moduli dai quali dipende il modulo corrente;
	\item \textbf{Accoppiamento tra classi:} 	
\end{itemize}