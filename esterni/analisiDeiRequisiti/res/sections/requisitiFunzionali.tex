\subsection{Requisiti funzionali}

\renewcommand{\arraystretch}{2.2}

\rowcolors{2}{pari}{dispari}
\begin{longtable}{|c|p{8cm}|c|}
	\arrayrulecolor{white}
	
	\caption{Tabella riassuntiva dei requisiti funzionali}\\
	
	\rowcolor{header}
	
	\textbf{Requisito} & \centering{\textbf{Descrizione}} & \textbf{Fonte}\\
	
	\endhead
	
	R1F1 & Un utente ha la possibilità di visualizzare una guida introduttiva contenente la lista dei comandi eseguibili e il loro funzionamento. & Interna \\
	
	R1F2 & Un utente ha la possibilità di autenticarsi a \textit{Etherless} mediante un utenza \textit{Ethereum\glos}. & Capitolato \\
	
	R1F2.1 & L'utente si può autenticare mediante login manuale.  & Interna \\
	
	R2F2.2 & L'utente si può autenticare mediante login automatico. & Interna \\
	
	R1F2.1.1 &  L'utente viene autenticato al sistema mediante l'inserimento di una coppia di address e \textit{private key\glo} che identificano un utente univocamente all'interno della rete \textit{Ethereum\glos}. & Interna \\
	
	R1F2.1.1.1 &  Il sistema dovrà restituire un errore in caso di inserimento di credenziali errate. & Interna \\
	
	R1F2.1.1.2 &  Il sistema dovrà restituire un messaggio di autenticazione riuscita nel caso in cui esista nel network \textit{Ethereum\glo} un utente con le credenziali inserite. & Interna \\
	
	R2F2.2.1 &  L'utente può eseguire automaticamente l'accesso tramite un file salvato sul suo dispositivo durante la prima autenticazione o registrazione mediante \textit{Etherless}.  & Interna \\
	
	R1F3 &  L'utente ha la possibilità di registrarsi su \textit{Etherless}  & Capitolato \\

	R1F3.1 &  La registrazione su \textit{Etherless} avverrà richiedendo una coppia univoca di address - \textit{private key\glo} alla rete \textit{Ethereum\glos}.  & Capitolato \\	
	
	R2F4 &  Ogni qualvolta l'utente effettuerà l'accesso manuale o la registrazione al sistema, verrà salvato un file contenente le credenziali di accesso per permettere l'autenticazione automatica durante le successive esecuzioni della applicazione.  & Interna \\
	
	R2F5 &  Un utente autenticato può eseguire il logout richiedendo al sistema di eliminare il file di autenticazione salvato sulla propria macchina.  & Interna \\
	
	R2F5.1 &  Dovrà essere restituito all'utente un messaggio in caso di logout eseguito con successo.  & Interna \\
	
	R1F6 &  Un utente autenticato può ricercare una funzione caricata da un altro utente sviluppatore sull'\textit{Etherless-server}.  & Interna \\
	
	R1F6.1 &  La ricerca della funzione viene effettuata inserendo il comando seguito dal nome della funzione stessa.  & Interna \\
	
	R1F6.2 &  La ricerca della funzione permetterà di visualizzare sull'\textit{Etherless-cli} il nome della funzione, la sua firma, una descrizione riassuntiva delle sue funzionalità, il costo e l'autore.  & Interna \\
	
	R1F6.3 &  Verrà segnalato un errore su \textit{Etherless-cli} se il nome della funzione cercata non fosse disponibile su \textit{Etherless-server}.  & Interna \\
	
	R1F7 &  Un utente autenticato può eseguire una funzione caricata da uno sviluppatore sull'\textit{Etherless-server}.  & Capitolato \\
	
	R1F7.1 &  L'esecuzione della funzione deve avvenire inserendo il comando apposito seguito dal nome e dal valore dei parametri.  & Capitolato \\

	R1F7.1.1 &  Il sistema deve restituire un messaggio di errore nel caso in cui il nome inserito per il comando "run" non corrisponda ad alcuna funzione presente su \textit{Etherless}.  & Interna \\
	
	R1F7.1.2 &  Il sistema deve restituire un messaggio di errore nel caso in cui il numero dei valori dei parametri inseriti per il comando "run" non corrispondano con quelli richiesti dalla funzione.  & Interna \\
		
	R1F7.2 &  L'esecuzione di una funzione può avvenire solo previo pagamento al suo sviluppatore e alla piattaforma \textit{Etherless}. & Capitolato \\
	
	R1F7.2.1 &  Il pagamento dovrà avvenire con valuta \textit{Ether\glos}. & Capitolato \\
	
	R1F7.2.2 &  La transazione avverrà in maniera diretta. & Interna \\
	
	R3F7.2.3 &  La transazione avverrà tramite \textit{escrow\glos}. & Capitolato \\
	
	R1F7.2.4 &  Il sistema deve restituire un messaggio di errore all'utente nel caso non avesse fondi  \textit{Ether\glos} sufficienti per coprire il costo dell'esecuzione. & Interna \\
	
	R1F7.2.5 &  Il sistema deve restituire un messaggio di errore all'utente nel caso lo sviluppatore della funzione non dovesse averne ancora associato un costo. & Interna \\
	
	R2F8 &  Un utente autenticato può richiedere un log di tutte le funzioni eseguite. & Interna \\	
	
	R2F8.1 &  Il sistema dovrà restituire un log delle funzioni eseguite composto da: data e ora di esecuzione, nome, firma e costo della funzione. & Interna \\
	
	R1F9 &  Un utente autenticato può visualizzare un elenco di tutte le funzioni disponibili su \textit{Etherless}. & Capitolato \\	
	
	R1F9.1 &  Il sistema dovrà restituire un elenco delle funzioni disponibili composto da: nome, firma, costo e descrizione della funzione. & Interna \\
	
	R1F10 &  Un utente autenticato ha la possibilità di creare una funzione Javascript e condividerla su \textit{Etherless-server}, diventando un utente sviluppatore. La funzione dovrà essere disponibile al resto degli utenti di \textit{Etherless}. & Capitolato \\
	
	R1F10.1 &  Il deploy della funzione su \textit{Etherless-server} deve avvenire inserendo a seguito del comando preposto, il percorso del file contente la funzione e il nome con il quale la si vuole rendere disponibile a tutti gli utenti. & Capitolato \\
	
	R1F10.1.1 &  Il sistema dovrà restituire un messaggio di errore nel caso in cui il nome della funzione che si sta tentando di condividere sia già in uso. & Interna \\
	
	R1F10.1.2 &  Il sistema dovrà restituire un messaggio di errore nel caso in cui il percorso specificato non fosse corretto (il file di riferimento dovrà avere un'estensione .js). & Interna \\
	
	R1F10.1.3 &  Il sistema dovrà, in mancanza di errori nell'esecuzione del comando "deploy", restituire un messaggio di avvenuta condivisione. & Interna \\
	
	R1F11 &  Un utente sviluppatore ha la possibilità di modificare delle informazioni riguardanti una sua funzione. & Interna \\
	
	R1F11.1 &  È possibile modificare il costo di esecuzione. & Interna \\
	
	R3F11.2 &  È possibile definire una descrizione sommaria che ne riepiloghi le funzionalità. & Interna \\
	
	R1F11.3 &  È possibile inserire la firma della funzione da rendere disponibile agli utenti. & Interna \\
	
	R1F11.1.1 &  Nel caso in cui il costo della funzione inserito sia minore o uguale a 0, è necessario ritornare all'utente un errore e non procedere con la modifica. & Interna \\
	
	R3F11.2.1 &  Nel caso in cui la descrizione della funzione superi il numero massimo di caratteri consentiti, è necessario ritornare all'utente un errore e non procedere con la modifica. & Interna \\
	
	R1F11.5 &  Il sistema dovrà, in mancanza di errori nell'esecuzione del comando "set", restituire un messaggio di avvenuta modifica dei parametri richiesti. & Interna \\
	
	R1F12 &  Un utente sviluppatore ha la possibilità di eliminare una sua funzione. & Capitolato \\
	
	R1F12.1 & Il sistema dovrà, in mancanza di errori nell'esecuzione del comando "delete", restituire un messaggio di avvenuta rimozione della funzione. & Capitolato\\
	
	R1F13 &  Un utente sviluppatore ha la possibilità di aggiornare il codice di una sua funzione. & Interna \\
	
	R1F13.1 &  L'aggiornamento del codice deve avvenire inserendo il percorso del nuovo file.js contenente la funzione. & Interna \\
	
	R1F13.1.1 & Il sistema dovrà restituire un errore nel caso in cui il nuovo percorso del file inserito per il comando "update" sia errato. & Interna\\
	
	R1F13.1.2 & Il sistema dovrà, in mancanza di errori nell'esecuzione del comando "update", restituire un messaggio di avvenuta aggiornamento del codice. & Interna\\
	
	R1F14 &  Per le operazioni di aggiornamento del codice, rimozione di una funzione, modifica delle informazioni aggiuntive, sarà necessario che l'utente abbia condiviso su \textit{Etherless} almeno una funzione. & Interna \\
	
	R1F14.1 &  In ognuno dei comandi "set", "update", "delete", dovrà essere specificata la funzione di riferimento. & Interna \\
	
	R1F14.1.1 &  Dovrà essere restituito un errore nel caso in cui la funzione di riferimento non sia presente tra quelle rese disponibili su \textit{Etherless} dallo sviluppatore & Interna \\
	\hline
	
\end{longtable}