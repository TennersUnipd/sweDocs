\section{Preventivo iniziale}
A seguito della scorretta pianificazione iniziale segnalata in RR, è stata rivista la suddivisione di ore per ruolo per ciascun componente. Tuttavia, il costo del preventivo finale previsto per il committente, è rimasto inalterato senza alcun costo aggiuntivo.\\\\
Verranno utilizzate le seguenti abbreviazioni per indicare i ruoli:
\begin{itemize}
	\item \textbf{RE:} Responsabile;
	\item \textbf{AM:} Amministratore;
	\item \textbf{AN:} Analista;
	\item \textbf{PT:} Progettista;
	\item \textbf{PR:} Programmatore;
	\item \textbf{VE:} Verificatore.
\end{itemize}
Tenners applica le seguenti tariffe per tipo di ruolo coinvolto nell'attività di sviluppo:\\
\begin{table}[h]
\caption{Tabella tariffe per ruolo}
\begin{center}
\begin{tabular}{ |c|c|  }
 \hline
 Ruolo 		& Tariffa (EUR)\\
 \hline\hline
	Responsabile	& 30,00\\
	Amministratore	& 20,00\\
	Analista		& 25,00\\
	Progettista		& 22,00\\
	Programmatore	& 15,00\\
	Verificatore	& 15,00\\
 \hline
\end{tabular}
\end{center}
\end{table}
\newpage
\subsection{Analisi dei requisiti}
\subsubsection{Prospetto orario}
Il team ha previsto la seguente suddivisione di ruoli durante questo periodo:\\
\begin{table}[h]
\caption{Tabella suddivisione per ruolo, Analisi dei requisiti}
\begin{center}
\begin{tabular}{ |c|c|c|c|c|c|c|c|  }
 \hline
 Membro 		& RE 	& AM 	& AN 	& PT 	& PR 	& VE 	& Tot.\\
 \hline\hline
 Simone	Franconetti		& 4 		& 4 		& 14 	& 2 		& 0 		& 6 		& 30\\
 Gabriel Ciulei			& 3 		& 2 		& 20 	& 2 		& 0 		& 3 		& 30\\
 Nicola	Salvadore		& 6 		& 4 		& 10 	& 0 		& 0 		& 10 	& 30\\
 Gianmarco	Pettinato	& 0 		& 2 		& 16 	& 2 		& 0 		& 10 	& 30\\
 Gezim	Cikaqi			& 6 		& 10 	 	& 6 		& 2 		& 0 		& 6	 	& 30\\
 Paola	Trevisan		& 6 		& 2 		& 10 	& 2 		& 0 		& 10 	& 30\\
 Giovanni	Incalza		& 6 		& 11 		& 8 		& 0 		& 0 		& 5  	& 30\\
 \hline\hline
 Ore totali		& 31		& 35		& 84 	& 10 	& 0 		& 50 	& 210\\
  \hline
\end{tabular}
\end{center}
\end{table}
\begin{figure}[h!]
	\includegraphics[width=\textwidth]{res/img/hi17}
	\caption{Diagramma della suddivisione dei ruoli, Analisi dei requisiti}
\end{figure}

\noindent Viene riportato in seguito, a solo scopo informativo, il valore economico complessivo che il team Tenners ha deciso di investire in questo periodo.
\noindent \textbf{Si ricorda infatti che il lavoro sostenuto durante questo periodo non viene rendicontato nei confronti del cliente.} \\
Il valore economico totale investito è di 4.700,00 EUR.
\begin{table}[h]
	\caption{Tabella prospetto economico, Analisi dei requisiti}
\begin{center}
\begin{tabular}{ |c|c|c|  }
 \hline
 Ruolo 		& Ammontare ore 	& Totale (EUR)\\
 	\hline
 \hline
 	Responsabile	& 31 	& 930,00\\
	Amministratore	& 35		& 700,00\\
	Analista		& 84 	& 2.100,00\\
	Progettista		& 10		& 220,00\\
	Programmatore	& 0		& 0\\
	Verificatore	& 50		& 750,00\\
 \hline\hline
 TOTALE		& 210		& 4.700,00\\
  \hline
\end{tabular}
\end{center}
\end{table}
\newpage
\subsection{Progettazione architetturale}
\subsubsection{Prospetto orario}
Il team ha previsto la seguente suddivisione di ruoli per questa periodo:
\\
\begin{table}[h]
\caption{Tabella suddivisione per ruolo, Progettazione architetturale}
\begin{center}
\begin{tabular}{ |c|c|c|c|c|c|c|c|  }
 \hline
 Membro 		& RE 	& AM 	& AN 	& PT 	& PR 	& VE 	& Tot.\\
 \hline\hline
 Simone	Franconetti		& 5 		& 4		& 3 	& 13 	& 5 		& 0 		& 30\\
 Gabriel Ciulei		& 0 		& 5 		& 2 	& 15		& 8 		& 0 		& 30\\
 Nicola	Salvadore		& 0 		& 2 		& 4 	& 8		& 8 		& 8 		& 30\\
 Gianmarco	Pettinato	& 5 		& 8 		& 3 	& 10 	& 2 		& 2 		& 30\\
 Gezim	Cikaqi		& 0 		& 0 		& 10 	& 10		& 4 		& 6	 	& 30\\
 Paola	Trevisan		& 0 		& 4 		& 6 	& 8 		& 6 		& 6 		& 30\\
 Giovanni Incalza		& 0 		& 3	 	& 7 	& 8		& 4 		& 8  	& 30\\
 \hline\hline
 Ore totali		& 10		& 26		& 35 	& 72	 	& 37 	& 30 	& 210\\
  \hline
\end{tabular}
\end{center}
\end{table}
\begin{figure}[h!]
	\includegraphics[width=0.9\textwidth]{res/img/hi336}
	\caption{Diagramma della suddivisione dei ruoli, Progettazione architetturale}
\end{figure}

\noindent Vengono inoltre dettagliate in seguito le ore previste per portare a termine ciascun incremento del prodotto durante il periodo di progettazione architetturale, insieme al costo totale previsto per ogni incremento:

\begin{table}[h]
	\caption{Tabella ore e costo totale previsti per incremento, progettazione architetturale}
	\begin{center}
		\begin{tabular}{ |c|c|c|c|c|c|c|c|c|  }
			\hline
			Incremento 		& RE 	& AM 	& AN 	& PT 	& PR 	& VE 	& Tot. (h) & Tot. (EUR) \\
			\hline\hline
			I		& 2 		& 3			& 10 	& 2 	& 0 		& 4 		& 21	 & 474,00\\
			II		& 2 		& 11 		& 11 	& 26	& 15 		& 9 		& 74	&1.487,00\\
			III		& 2 		& 5 		& 7 	& 23	& 12 		& 7 		& 56	&1.126,00\\
			IV		& 2 		& 5 		& 6 	& 20 	& 10 		& 7 		& 50 	&1.005,00\\
			V		& 1 		& 1 		& 1 	& 1		& 0 		& 2	 		& 6			&127,00\\
			VI		& 1 		& 1 		& 0 	& 0 	& 0 		& 1 		& 3			&65,00\\
			\hline\hline
			Totale		& 10		& 26		& 35 	& 72	 	& 37 	& 30 	& 210 	&4.284,00\\
			\hline
		\end{tabular}
	\end{center}
\end{table}

\subsubsection{Prospetto economico}
In base alle ore necessarie per il completamento di questo periodo, il valore economico totale è di 4.284,00 EUR.
\begin{table}[h]
	\caption{Tabella prospetto economico, Progettazione architetturale}
\begin{center}
\begin{tabular}{ |c|c|c|  }
 \hline
 Ruolo 		& Ammontare ore 	& Totale (EUR)\\
 	\hline
 \hline
 	Responsabile	& 10 	& 300,00\\
	Amministratore	& 26		& 520,00\\
	Analista		& 35 	& 875,00\\
	Progettista		& 72		& 1.584,00\\
	Programmatore	& 37		& 555,00\\
	Verificatore	& 30 	& 450,00\\
 \hline\hline
 TOTALE		& 210		& 4.284,00\\
  \hline
\end{tabular}
\end{center}
\end{table}
\newpage
\subsection{Progettazione di dettaglio e codifica}
\subsubsection{Prospetto orario}
Il team ha previsto la seguente suddivisione di ruoli per questo periodo:
\FloatBarrier
\begin{table}[h]
\caption{Tabella suddivisione per ruolo, Progettazione di dettaglio e codifica}
\begin{center}
\begin{tabular}{ |c|c|c|c|c|c|c|c|  }
 \hline
 Membro 		& RE 	& AM 	& AN 	& PT 	& PR 	& VE 	& Tot.\\
 \hline\hline
 Simone	Franconetti		& 0 		& 4		& 5 	& 10 	& 26 		& 10 		& 55\\
 Gabriel Ciulei		& 5 		& 0 		& 0 	& 10		& 30 		& 10 		& 55\\
 Nicola	Salvadore		& 5 		& 8 		& 3 	& 9 		& 20 		& 10 		& 55\\
 Gianmarco	Pettinato	& 0 		& 0 		& 0 	& 15 	& 30 		& 10 		& 55\\
 Gezim	Cikaqi		& 5 		& 6 		& 0 	& 12 	& 24 		& 8	 		& 55\\
 Paola	Trevisan		& 5 		& 2 		& 3 	& 15 	& 20 		& 10 		& 55\\
 Giovanni	Incalza	& 5 		& 3	 	& 1 	& 12 	& 24 		& 10  		& 55\\
 \hline\hline
 Ore totali		& 25		& 23		& 12 	& 83	 	& 174 	& 68 	& 385\\
  \hline
\end{tabular}
\end{center}
\end{table}
\FloatBarrier
\FloatBarrier
\begin{figure}[h!]
	\centering
	\includegraphics[width=0.9\textwidth]{res/img/hi33}
	\caption{Diagramma della suddivisione dei ruoli, Progettazione di dettaglio e codifica}
\end{figure}
\FloatBarrier
\noindent In seguito vengono dettagliate le ore previste per ciascun incremento del prodotto durante il periodo di progettazione di dettaglio e codifica, insieme al costo totale previsto per ogni incremento:
\begin{table}[h]
	\caption{Tabella ore e costo totale previsti per incremento, progettazione di dettaglio e codifica}
	\begin{center}
		\begin{tabular}{ |c|c|c|c|c|c|c|c|c|  }
			\hline
			Incremento 		& RE 	& AM 	& AN 	& PT 	& PR 	& VE 	& Tot. (h) & Tot. (EUR) \\
			\hline\hline
			I		& 2 		& 7			& 7 	& 11 	& 20 		& 12 		& 59	 & 1.097,00\\
			II		& 5 		& 4 		& 1 	& 24	& 58 		& 18 		&110	&1.923,00\\
			III		& 6 		& 2 		& 1 	& 12	& 46 		& 11 		& 78	&1.364,00\\
			IV		& 6 		& 4 		& 2 	& 34 	& 49 		& 25 		& 120 	&2.168,00\\
			V		& 5 		& 5 		& 0 	& 0		& 0 		& 1	 		& 11			&265,00\\
			VI		& 1 		& 1 		& 1 	& 2 	& 1 		& 1 		& 7			&149,00\\
			\hline\hline
			Totale		& 25		& 23		& 12 	& 83	 	& 174 	& 68 	& 385 	&6.966,00\\
			\hline
		\end{tabular}
	\end{center}
\end{table}

\subsubsection{Prospetto economico}
In base alle ore necessarie per il completamento di questo periodo, il valore economico totale è di 6.966,00 EUR.
\begin{table}[h]
\caption{Tabella prospetto economico, Progettazione di dettaglio e codifica}
\begin{center}
\begin{tabular}{ |c|c|c|  }
 \hline
 Ruolo 		& Ammontare ore 	& Totale (EUR)\\
 	\hline
 \hline
 	Responsabile	& 25 		& 750,00\\
	Amministratore	& 23		& 460,00\\
	Analista		& 12 	& 300,00\\
	Progettista		& 83		& 1.826,00\\
	Programmatore	& 174		& 2.610,00 \\
	Verificatore	& 68 	& 1.020,00\\
 \hline\hline
 TOTALE		& 385		& 6.966,00\\
  \hline
\end{tabular}
\end{center}
\end{table}
\newpage
\subsection{Validazione e collaudo}
\subsubsection{Prospetto orario}
Il team ha previsto la seguente suddivisione di ruoli per questo periodo:
\begin{table}[h]
\caption{Tabella suddivisione per ruolo, Validazione e Collaudo}
\begin{center}
\begin{tabular}{ |c|c|c|c|c|c|c|c|  }
 \hline
 Membro 		& RE 	& AM 	& AN 	& PT 	& PR 	& VE 	& Tot.\\
 \hline\hline
 Simone Franconetti			& 4 		& 0		& 0 	& 2 		& 4 		& 10 		& 20\\
 Gabriel Ciulei		& 0 		& 6 		& 0 	& 4		& 2 		& 8 		& 20\\
 Nicola	Salvadore		& 0 		& 0 		& 0 	& 6 		& 4 		& 10 		& 20\\
 Gianmarco Pettinato		& 4 		& 6 		& 0 	& 0	 	& 4 		& 6 		& 20\\
 Gezim Cikaqi			& 0 		& 2 		& 0 	& 0 		& 8 		& 10	 	& 20\\
 Paola Trevisan			& 4 		& 4 		& 0 	& 0 		& 8 		& 4 		& 20\\
 Giovanni Incalza		& 2 		& 0	 	& 0 	& 2 		& 12 	& 4  	& 20\\
 \hline\hline
 Ore totali		& 14		& 18		& 0 	& 14	 	& 42 	& 52 	& 140\\
  \hline
\end{tabular}
\end{center}
\end{table}
\begin{figure}[h!]
	\centering
	\includegraphics[width=0.9\textwidth]{res/img/hi5}
	\caption{Diagramma della suddivisione dei ruoli, Validazione e collaudo}
\end{figure}

\subsubsection{Prospetto economico}
In base alle ore necessarie per il completamento di questo periodo, il valore economico totale è di 2.498,00 EUR.
\begin{table}[h]
\caption{Tabella prospetto economico, Verifica e collaudo}
\begin{center}
\begin{tabular}{ |c|c|c|  }
 \hline
 Ruolo 		& Ammontare ore 	& Totale (EUR)\\
 \hline
 \hline
 	Responsabile	& 14 	& 420\\
	Amministratore	& 18		& 360\\
	Analista		& 0 		& 0\\
	Progettista		& 14		& 308\\
	Programmatore	& 42		& 630\\
	Verificatore	& 52 	& 780\\
 \hline\hline
 TOTALE		& 140		& 2498\\
  \hline
\end{tabular}
\end{center}
\end{table}
\newpage
\subsection{Totale ore investite}
\subsubsection{Prospetto orario}
Il team ha previsto la seguente suddivisione di ruoli per il completamento del progetto, comprensivo delle ore investite prima dell'ingresso in RR:
\begin{table}[h]
\caption{Tabella di riepilogo del prospetto orario}
\begin{center}
\begin{tabular}{ |c|c|c|c|c|c|c|c|  }
 \hline
 Membro 		& RE 		& AM 		& AN 	& PT 	& PR 	& VE 	& Tot.\\
 \hline\hline
 Simone	Franconetti		& 13 		& 16			& 23 		& 24 		& 33 		& 26 		& 140\\
 Gabriel Ciulei		& 8 			& 13 		& 22 		& 28		& 40 		& 24 		& 140\\
 Nicola	Salvadore		& 11 		& 14 		& 16 		& 22 		& 32 		& 40 		& 140\\
 Gianmarco Pettinato		& 12 		& 16 		& 18 		& 27	 	& 34 		& 28 		& 140\\
 Gezim Cikaqi		& 11 		& 18 		& 18 		& 19 		& 36 		& 33	 	& 140\\
 Paola Trevisan		& 15 		& 12 		& 24 		& 26 		& 28 		& 30 		& 140\\
 Giovanni	Incalza	& 13 		& 17	 		& 16 		& 17 		& 41	 	& 31  		& 140\\
 \hline\hline
 Ore totali		& 83 	& 106		& 149 	& 179 	& 253 	& 210 	& 980\\
  \hline
\end{tabular}
\end{center}
\end{table}
\begin{figure}[h!]
	\includegraphics[width=\textwidth]{res/img/hip3}
	\caption{Diagramma della suddivisione dei ruoli durante l'intero il progetto}
\end{figure}

\subsubsection{Prospetto economico}
In base alle ore necessarie per il completamento di questo progetto, il valore economico totale è di 19.218,00 EUR.
\begin{table}[h]
\caption{Tabella di riepilogo del prospetto economico}
\begin{center}
\begin{tabular}{ |c|c|c|  }
 \hline
 Ruolo 		& Ammontare ore 	& Totale (EUR)\\
 \hline
 \hline
 	Responsabile	& 83 		& 2.490,00\\
	Amministratore	& 106		& 2.120,00\\
	Analista		& 149 		& 3.725,00\\
	Progettista		& 179		& 3.938,00\\
	Programmatore	& 253		& 3.795,00\\
	Verificatore	& 210 		& 3.150,00\\
 \hline\hline
 TOTALE		& 980		& 19.218,00\\
  \hline
\end{tabular}
\end{center}
\end{table}

\newpage
\subsection{Totale ore rendicontate}
\subsubsection{Prospetto orario}
Nello specifico, la programmazione della suddivisione del lavoro prevede la seguente divisione dei ruoli per i singoli membri del team:
\begin{table}[h]
	\caption{Tabella di riepilogo del prospetto orario (escluso il periodo iniziale di \textit{Analisi dei requisiti})}
\begin{center}
\begin{tabular}{ |c|c|c|c|c|c|c|c|  }
 \hline
 Membro 		& RE 		& AM 		& AN 	& PT 	& PR 	& VE 	& Tot.\\
 \hline\hline
 Simone	Franconetti		& 9  	 	& 8			& 2 		& 25 		& 35 		& 20 		& 105\\
 Gabriel Ciulei		& 5 			& 11 		& 2 		& 29			& 40 		& 18 		& 105\\
 Nicola	Salvadore		& 5  		& 10 		& 7 		& 23 		& 32 		& 28 		& 105\\
 Gianmarco	Pettinato	& 9   		& 14 		& 3 		& 25		 	& 36 		& 18 		& 105\\
 Gezim	Cikaqi		& 5  		& 8  		& 10		& 22 		& 36 		& 24	 	& 105\\
 Paola	Trevisan		& 9  		& 10 		& 9 		& 23 		& 34 		& 20 		& 105\\
 Giovanni	Incalza	& 7  		& 6	 		& 8 		& 22 		& 40		 	& 22  		& 105\\
 \hline\hline
 Ore totali		& 49 	& 67		& 47 	& 169 	& 253 	& 150 	& 735\\
  \hline
\end{tabular}
\end{center}
\end{table}

\newpage
\subsubsection{Prospetto economico}
\begin{table}[h]
	\caption{Tabella di riepilogo del prospetto economico (escluso il periodo iniziale di \textit{Analisi dei requisiti})}
	\begin{center}
		\begin{tabular}{ |c|c|c|  }
			\hline
			Ruolo 		& Ammontare ore 	& Totale (EUR)\\
			\hline
			\hline
			Responsabile	& 49 	& 1.470,00\\
			Amministratore	& 67		& 1.340,00\\
			Analista		& 47 	& 1.175,00\\
			Progettista		& 169	& 3.718,00\\
			Programmatore	& 253	& 3.795,00\\
			Verificatore	& 150 	& 2.250,00\\
			\hline\hline
			TOTALE		& 735		& 13.748,00\\
			\hline
		\end{tabular}
	\end{center}
\end{table}

\subsubsection{Conclusioni}
In conclusione, il totale preventivato per la realizzazione del progetto \textit{Etherless} è di\\ \textbf{13.748,00 EUR}, valore che rispecchia il numero di ore rendicontabili per ogni figura professionale che verrà coinvolta durante la realizzazione del progetto.
