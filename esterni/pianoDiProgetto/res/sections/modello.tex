\section{Modello di sviluppo}
Per lo svolgimento del progetto \textit{Etherless} è stato scelto l'utilizzo di un modello incrementale.
\subsection{Modello incrementale}
Il modello incrementale consiste nella realizzazione del prodotto finale per incrementi, ognuno dei quali introduce una funzionalità. Questo modello è particolarmente adatto per rilasci multipli in cui, con l'aggiunta progressiva di nuove funzionalità, si approssima sempre più il prodotto parziale a quello finale.\\\\
 I requisiti sono da trattare in ordine di importanza con l'aggiunta delle funzionalità che necessitano di essere realizzate con più urgenza. Questo approccio di tipo adattivo, differentemente da quello predittivo adottato nei modelli sequenziali, consente una maggiore flessibilità e si adatta maggiormente a situazioni in cui i requisiti possono non essere chiari e ben definiti da qualcuno degli \textit{stakeholders\glos}. \\\\
  L'aggiunta, modifica e cancellazione di requisiti sono consentite previa discussione con il proponente e sua approvazione. \\\\
   La specifica riguardante gli incrementi attuati per i diversi periodi del progetto software vengono descritti nella successiva sezione.
\subsubsection{Vantaggi}
\begin{itemize}
	\item Le funzionalità principali vengono subito sviluppate e dunque il committente può rilasciare un feedback velocemente;
	\item Ogni rilascio risulta utile per il miglioramento dei successivi;
	\item Sviluppare per incrementi limita l'introduzione di errori;
	\item Le modifiche, l'individuazione e la correzione degli errori sono più economiche;
	\item Le funzionalità introdotte inizialmente subiranno più volte il processo di verifica poiché eseguito per ogni ciclo.
\end{itemize}
\subsubsection{Svantaggi}
\begin{itemize}
	\item Un approccio predittivo, differentemente da quello adattivo attuato dal modello incrementale, consentirebbe una stima più precisa di tempi e costi;
	\item Necessita che l'intero sistema sia preventivamente definito e scomposto in modo tale da poterlo costruire incrementalmente.
\end{itemize}
\subsection{Incrementi previsti}
Di seguito tutti gli incrementi previsti. Per ogni incremento e sotto-incremento viene effettuata una verifica. Inoltre al termine di ciascun incremento viene eseguita una misurazione e i risultati vengono inseriti nel relativo cruscotto del \textit{Piano di qualifica}\docs.
\subsubsection{Progettazione architetturale}
\begin{itemize}
  \item \textbf{\RomanNumeralCaps{1} Incremento:} correzione e integrazione documenti in base alle conoscenze acquisite durante l'\textit{Analisi dei requisiti}\docs;
  \item \textbf{\RomanNumeralCaps{2} Incremento:} realizzazione del primo \textit{Proof Of Concept}\glo (\textit{smart-contract}\glos);
  \item \textbf{\RomanNumeralCaps{3} Incremento:} realizzazione del secondo \textit{Proof Of Concept}\glo (comunicazione tra client e server);
  \item \textbf{\RomanNumeralCaps{4} Incremento:} realizzazione del terzo \textit{Proof Of Concept}\glo (\textit{serverless}\glos);
  \item \textbf{\RomanNumeralCaps{5} Incremento:} aggiornamento dei documenti;
  \item \textbf{\RomanNumeralCaps{6} Incremento:} presentazione RP.
\end{itemize}
\subsubsection{Progettazione di dettaglio e codifica}
\begin{itemize}
	\item \textbf{\RomanNumeralCaps{1} Incremento:} correzione e integrazione documenti in base alle conoscenze acquisite durante la progettazione architetturale. Nel seguente ordine:
  \begin{itemize}
    \item \textit{Norme di progetto}\docs;
    \item \textit{Analisi dei requisiti}\docs;
    \item \textit{Piano di qualifica}\docs;
    \item \textit{Piano di progetto}\docs;
    \item \textit{Glossario}\docs.
  \end{itemize}
	\item \textbf{\RomanNumeralCaps{2} Incremento:} progettazione del sistema in vista della \textit{Product Baseline}\glo e stesura dell'allegato tecnico comprensivo dei diagrammi adottati.
	\begin{itemize}
		\item progettazione componente etherless-cli tramite design pattern Facade;
		\item progettazione componente etherless-smart;
		\item progettazione componente etherless-server.
	\end{itemize}
	\item \textbf{\RomanNumeralCaps{3} Incremento:} implementazione delle funzionalità della componente etherless-cli.
  Vengono implementati in particolare:
  \begin{itemize}
  	\item signup: implementazione del comando "signup" collegato al caso d'uso UC4;
    \item login: implementazione del comando "login" legato ai caso d'uso UC5;
    \item logout: implementazione del comando "logout" collegato al caso d'uso UC8;
    \item list: implementazione del comando "list" collegato al caso d'uso UC15;
    \item run: implementazione del comando "run" collegato al caso d'uso UC10;
    \item create: implementazione del comando "create" collegato al caso d'uso UC17;
    %\item set: implementazione del comando "set" collegato al caso d'uso UC18;
    %\item log: comando che permette all'utente di visualizzare la lista delle funzioni da lui eseguite;
    \item find: implementazione del comando "find" collegato al caso d'uso UC9.
  \end{itemize}
	\item \textbf{\RomanNumeralCaps{4} Incremento:} sviluppo componenti etherless-smart e etherless-server, comprensivi di test di unità e di integrazione:
	\begin{itemize}
		\item inserimento comando create \textit{(deploy}\glo) su etherless-smart;
		\item inserimento comando find su etherless-smart;
		\item inserimento comando list su etherless-smart;
		\item inserimento comando run su etherless-smart.
	\end{itemize}
	\item \textbf{\RomanNumeralCaps{5} Incremento:} aggiornamento documenti e consuntivo di periodo:
  \begin{itemize}
  	\item stesura del manuale sviluppatore.
    \item vedazione manuale utente;
    \item valutazione dei rischi nel consuntivo di periodo del \textit{Piano di progetto}\docs.
  \end{itemize}
	\item \textbf{\RomanNumeralCaps{6} Incremento:} presentazione RQ.
\end{itemize}
\subsubsection{Validazione e collaudo }
\begin{itemize}
  \item \textbf{\RomanNumeralCaps{1} Incremento:} correzione e integrazione documenti in base alle conoscenze acquisite durante la progettazione di dettaglio e codifica. Nel seguente ordine:
  \begin{itemize}
    \item \textit{Norme di progetto}\docs;
    \item \textit{Product Baseline}\glo e allegato tecnico;
    \item \textit{Analisi dei requisiti}\docs;
    \item \textit{Piano di qualifica}\docs;
    \item \textit{Piano di progetto}\docs;
    \item \textit{Glossario}\docs.
  \end{itemize}
	\item \textbf{\RomanNumeralCaps{2} Incremento:} implementazione requisiti desiderabili o opzionali:
	\begin{itemize}
		\item init: implementazione del comando "init" collegato al caso d'uso UC1;
		\item log: implementazione del commando "log" collegato al caso d'uso UC14;
		\item set: implementazione del comando "set" collegato al caso d'uso UC18;
		\item delete: implementazione del commando "delete" collegato al caso d'uso UC19;
		\item update: implementazione del commando "update" collegato al caso d'uso UC20.
	\end{itemize}
	\item \textbf{\RomanNumeralCaps{3} Incremento:} implementazione dei test di sistema e successiva codifica delle correzioni;
	\item \textbf{\RomanNumeralCaps{4} Incremento:} aggiornamento dei documenti da presentare in ingresso alla revisione di accettazione;
	\item \textbf{\RomanNumeralCaps{5} Incremento:} presentazione RA.
\end{itemize}
