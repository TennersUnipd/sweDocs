\section{Processi organizzativi}

	\subsection{Gestione}
	\subsubsection{Obiettivi}
	Il processo di gestione contiene tutte le attività atte a gestire i processi in atto durante il ciclo di vita del prodotto e le infrastrutture coinvolte.
	In particolar modo questo processo si prefigge di:
	\begin{itemize}
		\item definire una corretta pianificazione mediante la suddivisione in ruoli tra i membri del gruppo;
		\item stabilire una definizione di "done" per i task assegnati a ciascun membro del team;
		\item regolamentare le comunicazioni interne ed esterne al team;
		\item stabilire delle norme per la pianificazione degli incontri esterni ed interni
	\end{itemize}
	
	\subsubsection{Attività}
	\paragraph{Gestione della pianificazione}
	La gestione della pianificazione ha l'obiettivo di definire i ruoli che i membri del team devono ricoprire all'interno del progetto didattico. Tale suddivisione verrà stabilita in modo da garantire che ognuno dei componenti del team ricopra almeno una volta ciascuna delle posizioni. Sarà dovere del Responsabile di progetto stabilire il cambio dei ruoli in modo che non vi siano periodi di discontinuità o situazioni contraddittorie. È compito di questo processo anche la gestione dei compiti da assegnare ad ciascun componente, definendone modalità e strumenti.
	
   		\subparagraph*{Suddivisione ruoli di progetto}
   			Ciascun membro del gruppo, a rotazione, dovrà ricoprire il ruolo che gli viene affidato per un periodo significativo. Nel \textit{Piano di Progetto 2.2.1\docs} vengono organizzate e pianificate le attività e assegnate a ciascun ruolo. Ogni membro dovrà rispettare tali norme.
   			Nell'assegnamento dei ruoli si verranno evitate situazioni di conflitto, per garantire la qualità, la verifica e l'approvazione dei documenti.

   			\noindent I ruoli di progetto sono:
   			\begin{itemize}
   				\item \textbf{Analista:} l'Analista è colui che effettua l'analisi dei problemi e del dominio applicativo. È una figura molto importante ai fini della riuscita del progetto, anche se non sarà sempre presente durante tutto il ciclo di vita di quest'ultimo. Sarà suo compito redigere l'\textit{Analisi dei Requisiti 2.4.2\doc} e lo \textit{Studio di Fattibilità 1.1.0\doc}. Errori durante l'\textit{Analisi dei Requisiti 2.4.2\doc} possono portare a gravi problemi nelle fasi successive. Per evitare ciò, sarà necessario che l'Analista lavori a stretto contatto con il proponente capendo nel dettaglio tutti i bisogni impliciti ed espliciti e trasformandoli in requisiti per il sistema;
   				\item \textbf{Progettista:} il Progettista lavora in stretto contatto con l'Analista: il suo compito sarà capire come realizzare in maniera ottimale il sistema partendo dai requisiti forniti. È responsabile di scelte architetturali che portino ad una soluzione ottimale del problema iniziale in termini di efficacia ed efficienza. Per garantire ciò, è necessario che una architettura di qualità che aderisca a best practice consolidate, che risulti manutenibile nel tempo tenendo bassi i costi e le risorse consumate, sicura, modulare e capace di soddisfare tutti i requisiti;
   				\item \textbf{Programmatore:} lo Sviluppatore si occupa della codifica dell'applicativo software. Il grado di libertà nelle scelte e decisioni sull'architettura finale del prodotto è nulla. Il loro compito si limita all'implementazione dell'architettura elaborata dal Progettista rendendo il codice manutenibile nel tempo. Si occupa inoltre di predisporre le componenti di supporto alle prove di verifica e validazione del codice;
   				\item \textbf{Verificatore:} il Verificatore svolge un'operazione di controllo sulle attività svolte dagli altri componenti del progetto. Egli fa riferimento alle norme stabilite per sancire la correttezza e la conformità del prodotto rispetto alle attese. Facendo riferimento alle \textit{Norme di Progetto 2.4.0\doc}, il verificatore dovrà segnalare alla figura competente la presenza o l'assenza di difformità. La correzione effettiva degli errori non è compito del Verificatore ma dell'autore della modifica in oggetto. Egli partecipa in maniera continuativa per tutto il ciclo di vita del prodotto e si occuperà della stesura del \textit{Piano di Qualifica 2.1.1\docs};
   				\item \textbf{Responsabile:} il Responsabile è il punto di riferimento sia per il fornitore che per il committente. Inoltre è colui che raccoglie su di sé le responsabilità decisionali di scelta e approvazione e costituisce il centro di coordinamento per l'intero progetto. Nello specifico, ha il compito di gestire le risorse e le attività, coordinare il gruppo, approvare i documenti redatti;
   				\item \textbf{Amministratore:} l'Amministratore è colui che controlla e amministra l'ambiente di lavoro supportando il Responsabile. Non compie alcun tipo di scelta gestionale ed è incaricato di gestire  l'ambiente di lavoro compreso il versionamento e le configurazioni del prodotto e amministrare le infrastrutture di supporto.
   			\end{itemize}
   		
   		\subparagraph*{Tickecting}
   		Il tickecting permette ai membri del gruppo di monitorare, in qualsiasi momento, le attività in corso e al Responsabile di monitorare l'andamento delle attività.\\   		
   		\noindent Come strumento di tickecting è stato deciso di usare i servizi di Project Board offerti da \textit{GitHub\glos}.\\   		
   		Su \textit{GitHub\glo} è stata creata un'organizzazione, contenente tutti i repository del prodotto, in modo da separare le singole componenti.
   		La Project Board principale del gruppo è collegata alle Project Board di ciascun repository.
   		Ogni project board è suddivisa in cinque sezioni:
   		\begin{itemize}
   			\item \textbf{To do}: indica le attività da svolgere;
   			\item \textbf{In progress}: indica tutte le attività che sono in fase di svolgimento;
   			\item \textbf{Review in progress}: indica le verifiche in fase di svolgimento;
   			\item \textbf{Review approved}: indica tutte le verifiche approvate;
   			\item \textbf{Done}: indica tutte le attività completate.
   		\end{itemize}
   		Per ciascun task presente nella project board sarà possibile definire un assegnatario e averne un tracciamento facile e veloce dello stato di compimento.

	
   	\paragraph{Gestione delle comunicazioni}
   
   		\subparagraph*{Comunicazioni interne}
   		Per le comunicazioni interne è stato deciso di utilizzare \textit{ Slack\glos}, un'applicazione di messaggistica \textit{multipiattaforma}\glo con funzionalità specifiche per gruppi di lavoro. 
   		
   		\noindent All'interno di \textit{Slack\glo} sono stati aggiunti i seguenti canali:
   		\begin{itemize}
   			\item \textbf{github}: per la gestione del repository \textit{Github\glos};
   			\item \textbf{pianificazione-incontri}: dove vengono decise le date per riunioni;
   			\item \textbf{swe}: canale per discutere di ciò che riguarda l'organizzazione generale;
   			\item \textbf{logo}: per la scelta del logo del gruppo.
   			
   		\end{itemize}
   		Altri canali telematici, sono stati aggiunti per gestire la corretta stesura dei vari documenti quali:
   		\begin{itemize}
   			\item \textbf{norme\_di\_progetto}: per discutere e definire le \textit{Norme di Progetto 2.4.0\doc} alle quali attenersi durante le varie fasi di progetto;
   			\item \textbf{studio\_di\_fattibilità}: per redigere il relativo documento seguendo i criteri indicati;
   			\item \textbf{analisi\_requisiti}: per discutere e decidere i casi d'uso, i requisiti, i vincoli e tutto ciò che verrà inserito nel documento \textit{Analisi dei Requisiti 2.4.2\docs};
   			\item \textbf{piano\_di\_progetto}: per decidere e discutere la ripartizione dei ruoli, le ore totali ad essi collegati nelle varie versioni del \textit{Piano di Progetto 2.2.1\docs};
   			\item \textbf{piano\_di\_qualifica}: per decidere, discutere e fissare le strategie e gli obiettivi riguardanti la verifica, la validazione e la qualità.
   		\end{itemize}
   		Oltre all'applicazione \textit{Slack\glo} viene anche utilizzato un apposito gruppo Telegram e \textit{Google Hangouts\glo}, quando serve una comunicazione immediata, ma non è possibile trovarsi di persona.
   		
   		\subparagraph*{Comunicazioni esterne}
   		Le comunicazioni esterne sono affidate al Responsabile, il quale si avvale dell'utilizzo della posta elettronica, attraverso l'indirizzo \href{mailto:tenners.unipd@gmail.com}{tenners.unipd@gmail.com}.
   		Per comunicare con il proponente \textit{Red Babel}, invece, viene utilizzato un canale \textit{Slack\glo} per la chat testuale e il servizio \textit{Google Hangouts\glos} o \textit{Skype\glo} per le chiamate. 
   		
   		
   		
   	\paragraph{Gestione degli incontri}
   	
   		\subparagraph*{Incontri interni}
   		Le riunioni interne del gruppo sono organizzate dal Responsabile in accordo con gli altri componenti del team, attraverso gli appositi canali.
   		
   		\noindent Per ogni incontro viene redatto un Verbale. Questo sarà dato in incarico ad una persona nominata dal responsabile, che dovrà tenere traccia delle discussioni affrontate e delle decisioni prese. La struttura dei verbali è descritta in un paragrafo seguente.
   		
   		\subparagraph*{Incontri esterni}
   		Il Responsabile ha il compito organizzare e comunicare gli incontri con il proponente o il committente. Se uno o più membri del gruppo, il proponente o il committente volessero organizzare un incontro, sarà compito del Responsabile decidere una data, in accordo tra le parti, e successivamente di comunicarla tramite i canali appositi.
   		
   		\noindent Come per gli incontri interni, anche per quelli esterni viene redatto un Verbale.
   		\subparagraph*{Stesura dei verbali}
   		La prima parte del documento contiene le informazioni generali:
   		\begin{itemize}
   			\item \textbf{Luogo}: indica il luogo dove è avvenuto l'incontro;
   			\item \textbf{Data}: indica la data dell'incontro;
   			\item \textbf{Ora di inizio};
   			\item \textbf{Ora di fine};
   			\item \textbf{Partecipanti}: indica le persone presenti alla riunione, sotto forma di elenco puntato.
   		\end{itemize}
   		La seconda parte contiene, sotto forma di elenco puntato, l'ordine del giorno con relativo resoconto per ogni punto.
   		
   		\noindent È inoltre possibile tenere traccia, oltre che degli incontri avvenuti fisicamente, anche delle conversazioni avvenute nei canali di comunicazione telematici quali, ad esempio, Slack. In questo caso il verbale avrà un riassunto della conversazione effettuata riprendendone i punti salienti e di maggior rilievo.
   	
   
%   	\subsection{Gestione degli strumenti di coordinamento}
%   		\subsubsection{Gestione rischi}
%   		Il Responsabile ha il compito di individuare i rischi descritti nel \textit{Piano di Progetto 1.0.0\docs}, e una volta trovati, inserirli nell'analisi dei rischi. Per avere una corretta gestione dei rischi occorre:
%   		\begin{itemize}
%   			\item individuare problemi non calcolati e monitorare i rischi già previsti;
%   			\item registrare ogni riscontro previsto dei rischi nel \textit{Piano di Progetto 1.0.0\docs};
%   			\item aggiungere i nuovi rischi individuati nel \textit{Piano di Progetto 1.0.0\docs};
%   			\item se necessario, ridefinire le strategie di gestione dei rischi.
%   		\end{itemize}  			
%   			
%   		\subsubsection{Strumenti}
%   			Il gruppo durante lo svolgimento del progetto ha utilizzato o utilizzerà i seguenti strumenti:
%   			\begin{itemize}
%   				\item \textbf{Telegram}: come strumento di messaggistica.
%   				\item \textbf{Slack\glos}: come strumento di comunicazione interna e con il proponente;
%   				\item \textbf{Git\glos}: come sistema di controllo di versionamento;
%   				\item \textbf{Gitflow\glos}: come estensione \textit{Git\glo} per il flusso di sviluppo;
%   				\item \textbf{GitHub\glos}: come strumento per il versionamento e il salvataggio in remoto di tutti i file legati al progetto;
%   				\item \textbf{Hangouts\glos}: come strumento per effettuare videoconferenze e chiamate VoIP;
%   				\item \textbf{Sistemi Operativi}: i requisiti non prevedono di utilizzare un sistema operativo specifico, per questo verranno usati, dai membri del team, Windows, MacOS, sistemi di distribuzione Linux.
%   			\end{itemize}



	\subsubsection{Procedure}
 	\paragraph{Gestione project board}
	\begin{enumerate}
		\item definire una issue su GitHub corrispondente ad un determinato task da portare a termine;
		\item completare i task secondo le modalità e i tempi prefissati;
		\item spostare nella sezione "Review in progress" se il task è stato completato ed è in stato di verifica;
		\item spostare uno o più task nella sezione "Review approved" nel caso in cui siano stati verificati e corretti con successo ed approvati dal Responsabile di progetto;
		\item spostare i corrispondenti task nella sezione "Done" se i task sono stati completati con successo.
	\end{enumerate}

	\paragraph{Gestione incontri con il proponente Red Babel}
	\begin{enumerate}
		\item utilizzare l'apposito canale Slack per concordare i giorni di disponibilità di ciascun membro del gruppo;
		\item il Responsabile stabilirà una data utile in cui saranno disponibili tutti o la maggior parte dei membri del team;
		\item il Responsabile contatterà Red Babel mediante l'apposito canale Slack per verificare la disponibilità nel giorno precedentemente stabilito.
		\begin{itemize}
			\item \textbf{risposta positiva:} il Responsabile fissa un promemoria all'interno dei canali di comunicazione del team per il giorno e l'orario concordati per l'incontro;
			\item \textbf{risposta negativa:} il Responsabile contatterà nuovamente i membri del gruppo per concordare un'altra disponibilità da parte di tutti i membri del gruppo e dare priorità alle disponibilità del proponente.
		\end{itemize}
		\item il giorno dell'incontro verrà nominata una persona con ruolo di Amministratore, che avrà il compito di redigere un verbale completo contenente tutte le informazioni relative all'incontro.
	\end{enumerate}

	\paragraph{Gestione incontri con il committente}
	\begin{enumerate}
		\item utilizzare l'apposito canale Slack per concordare i giorni di disponibilità di ciascun membro del gruppo;
		\item il Responsabile stabilirà una data utile in cui saranno disponibili tutti o la maggior parte dei membri del team;
		\item il Responsabile si occuperà di inviare una email tramite l'indirizzo di posta elettronica del team, richiedendo un colloquio nel giorno preferibile. La email dovrà contenere la sigla "[SWE]" all'interno dell'oggetto, per permetterne una facile individuazione al destinatario;
		\begin{itemize}
			\item \textbf{risposta positiva:} il Responsabile fissa un promemoria all'interno dei canali di comunicazione del team per il giorno e l'orario concordati per l'incontro;
			\item \textbf{risposta negativa:} il Responsabile contatterà nuovamente i membri del gruppo per concordare un'altra disponibilità da parte di tutti i membri del gruppo e dare priorità alle disponibilità del proponente.
		\end{itemize}
		\item il giorno dell'incontro verrà nominata una persona con ruolo di Amministratore, che avrà il compito di redigere un verbale completo contenente tutte le informazioni relative all'incontro.
	\end{enumerate}

	\paragraph{Gestione rischi}
	Il Responsabile ha il compito di individuare i rischi descritti nel \textit{Piano di Progetto 2.2.1\docs}, e una volta trovati, inserirli nell'analisi dei rischi. Per avere una corretta gestione dei rischi occorre:
	\begin{enumerate}
		\item individuare problemi non calcolati e monitorare i rischi già previsti;
		\item registrare ogni riscontro previsto dei rischi nel \textit{Piano di Progetto 2.2.1\docs};
		\item aggiungere i nuovi rischi individuati nel \textit{Piano di Progetto 2.2.1\docs};
		\item se necessario, ridefinire le strategie di gestione dei rischi.
	\end{enumerate}  	


	\subsubsection{Strumenti}
   	\paragraph{Project board di GitHub}
	È stata scelta la project board di GitHub come strumento per il ticketing poiché già inclusa all'interno della piattaforma GitHub, già utilizzato per l'hosting dei sorgenti del team. La possibilità inoltre di chiudere una issue mediante riferimento.
	
	\paragraph{Slack}
	Come software per le comunicazioni sia esterne che interne è stato deciso di utilizzare Slack in quanto considerato lo strumento più consono per il progetto da intraprendere.
	Tra i tanti vantaggi, è possibile:
	\begin{itemize}
		\item creare canali dedicati per differenti argomenti;
		\item integrare bot che si interfacciano a piattaforme o applicativi di terze parti (Es.: GitHub, Google Hangouts);
		\item eseguire ricerche per tag;
		\item mantenere la condivisione di messaggi, file, documenti in un unico ambiente.
	\end{itemize}
	
	\paragraph{Google Hangouts}
	Software utilizzato per effettuare videochiamate sia tra i membri del team in caso di incontri interni, sia in caso di incontri con il proponente e committente. Google Hangouts ha facilitato molto la comunicazione soprattutto perché l'azienda proponente non risiede in Italia.
	Tra i diversi vantaggi, permette di:
	\begin{itemize}
		\item avere una chat interna alla chiamata per condividere link, file o messaggi testuali in maniera rapida;
		\item effettuare una condivisione dello schermo tra i partecipanti se necessario.
	\end{itemize}

	\paragraph{Youtube}
	Piattaforma per la condivisione video utilizzata per eseguire l'upload privato dei registrazioni di riunione interne o esterne. In questo modo viene facilitato il lavoro e migliorata la precisione del membro del team che, nel ruolo di amministratore, è incaricato per la stesura del verbale.


   	\subsection{Formazione}
	\subsubsection{Obiettivi}
	Il processo di formazione garantisce che tutti i membri del gruppo si formino autonomamente e condividano, in caso di necessità, informazioni in proprio possesso con la parte restante del gruppo. 
	
	\subsubsection{Attività}
	Il processo di formazione risulta fondamentale per portare a compimento tutte le attività e i task che le compongono con un equa suddivisione tra i membri del gruppo. Infatti, una scarsa formazione riguardo una determinata tecnologia o linguaggio, porterà inevitabilmente ad una impossibilità nell'affrontare le attività ad esso legate.\\\\
	Ogni singolo membro dovrà studiare in maniera autonoma le tecnologie da impiegare per la realizzazione del progetto. I membri del gruppo potranno creare delle guide, in modo del tutto informale relative ad un singolo argomento, per facilitare l'apprendimento ai restanti componenti del gruppo. La loro condivisione avverrà mediante gli appositi canali Slack.
	
	\subsubsection{Procedure}
	\paragraph{Piano per il training}
	\begin{enumerate}
		\item Identificazione delle tecnologie e linguaggi di utilizzare;
		\item Studio della documentazione delle tecnologie e dei linguaggi;
		\item Ideazione di un PoC il cui fine sia dimostrare la comprensione delle tecnologie coinvolte nella realizzazione del prodotto e costituire una baseline per il prodotto finale da realizzare;
		\item Realizzazione effettiva del PoC.   		
	\end{enumerate}
	
	\subsubsection{Strumenti (guide per la formazione)}
	Le guide utilizzate sono:
	\begin{itemize}
		\item \textbf{GitHub\glos}: \href{https://help.github.com}{https://help.github.com};
		\item \textbf{\LaTeX}: \href{https://www.latex-project.org/}{https://www.latex-project.org/};
		\item \textbf{AWS}: \href{https://docs.aws.amazon.com/}{https://docs.aws.amazon.com/};
		\item \textbf{Ethereum\glos}: \href{https://ethereum.org/}{https://ethereum.org/};
		\item \textbf{Solidity}: \href{https://solidity.readthedocs.io/en/develop/}{https://solidity.readthedocs.io/en/develop/};
		\item \textbf{Truffle suite}: \href{https://www.trufflesuite.com/tutorials/ethereum-overview}{https://www.trufflesuite.com/tutorials/ethereum-overview};
		\item \textbf{Web3}: \href{https://web3js.readthedocs.io/en/v1.2.4/}{https://web3js.readthedocs.io/en/v1.2.4/};
		\item \textbf{Node.js}: \href{https://nodejs.org/it/docs/}{https://nodejs.org/it/docs/};
		\item \textbf{ESLint}: \href{https://github.com/eslint/eslint}{https://github.com/eslint/eslint}.
\end{itemize}
