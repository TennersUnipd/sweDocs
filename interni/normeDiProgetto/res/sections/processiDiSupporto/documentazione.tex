\section{Processi di supporto}
  \subsection{Documentazione}
  \subsubsection{Obiettivi}
  La Documentazione è un processo che aiuta a tener traccia di tutte le informazioni relative ai processi e alle attività significative coinvolte nel ciclo di vita di un prodotto. La Documentazione aiuta a monitorare lo stato di avanzamento del progetto secondo il metodo di sviluppo scelto e disciplinare lo svolgimento dei processi. Obiettivi principali di questo processo sono dunque:
  \begin{itemize}
  	\item facilitare la collaborazione tra gli \textit{stakeholders\glo} coinvolti nel progetto software;
  	\item documentare le attività coinvolte nel ciclo di vita del prodotto software;
  	\item individuare delle convenzioni comuni per la stesura dei documenti;
  	\item garantire la stesura di documenti chiari, completi, ordinati e facilmente aggiornabili.
  \end{itemize}
  \subsubsection{Ciclo di vita}
  Ogni documento è caratterizzato delle seguenti fasi che ne caratterizzano il ciclo di vita:
	\begin{itemize}
		\item \textbf{Creazione del documento:} i redattori creano la struttura base del documento mediante l'utilizzo di un template \LaTeX;
		\item \textbf{Stesura:} i redattori producono il documento apportando delle aggiunte o modifiche in maniera incrementale e migliorativa;
		\item \textbf{Verifica:} i verificatori avranno il compito di controllare ciò che è stato redatto per controllarne la correttezza, la completezza e la qualità. Essi potranno utilizzare strumenti di controllo automatico per velocizzare l'attività di verifica. Al termine di ciascuna revisione, sarà compito del verificatore segnalare ai redattori eventuali problematiche relative al documento. I redattori avranno il compito di correggere tali segnalazioni;
		\item \textbf{Validazione:} il responsabile di progetto, solo al termine di opportune verifiche, procederà all'approvazione del documento con conseguente rilascio di una versione stabile.
	\end{itemize}
  \subsubsection{Attività}
  \paragraph{Organizzazione}
  \subparagraph*{Template}
  \`E stato creato un template per velocizzare l'attività di stesura, verifica e
  approvazione, unificando carattere, logo e stile di formattazione
  in un documento \LaTeX \space valido.

  \subparagraph*{Frontespizio}
  Il frontespizio di un documento di progetto deve essere così strutturato:
  \begin{itemize}
    \item nome del gruppo di progetto e il \textit{capitolato\glo} che è stato scelto;
    \item logo del gruppo di progetto;
    \item titolo del documento;
    \item tabella di gestione del documento che comprenda i seguenti elementi:
    \begin{itemize}
      \item versione;
      \item redattori;
      \item verificatori;
      \item responsabile;
      \item stato (da revisionare, da approvare, approvato);
      \item data di creazione;
      \item data ultima modifica;
      \item destinatari del documento.
    \end{itemize}
    \item descrizione sintetica del contenuto del documento;
    \item recapito email del gruppo di progetto.
  \end{itemize}
  
  \subparagraph*{Indici}
  L'indice dei contenuti dell'intero documento viene generato con l'apposito comando al momento della compilazione
  del documento \LaTeX. Esso è necessario per facilitare la navigazione all'interno del
  documento. L'indice viene posto dopo la tabella delle modifiche.\\\\
  Allo stesso modo, a seguito dell'indice dei contenuti, verranno inseriti gli indici delle immagini e delle tabelle presenti nel documento, ove necessario.

  \subparagraph*{Intestazione e piè di pagina}
	Ogni documento deve avere l'intestazione e il piè di pagina così costituiti:
	\begin{itemize}
		\item Intestazione:
		\begin{itemize}
			\item \textbf{Logo}: deve trovarsi sull'angolo superiore sinistro;
			\item \textbf{Nome del documento}: deve riportare il nome del documento sull'angolo superiore destro.
		\end{itemize}
		\item Piè di pagina:
		\begin{itemize}
			\item \textbf{E-mail}: deve riportare l'indirizzo e-mail del gruppo di progetto
			nell'angolo inferiore destro;
			\item \textbf{Numero di pagina}: deve riportare il numero di pagina corrente e totali
			nell'angolo inferiore destro.
		\end{itemize}
	\end{itemize}

  \paragraph{Produzione}
  \subparagraph*{Struttura e Gestione delle tabelle}
  Tutte le tabelle adottano il medesimo stile:
  \begin{itemize}
  	\item ad eccezione del registro delle modifiche, tutte le tabelle devono essere numerate;
  	\item ad eccezione del registro delle modifiche, ogni tabella deve avere una didascalia significativa associata;
  	\item le celle di intestazione di ogni tabella devono avere il contenuto centrato e in grassetto;
  	\item le celle relative alla descrizione nel registro delle modifiche hanno il contenuto in corsivo.
  \end{itemize}

  \subparagraph*{Uso e gestione delle immagini}
  Nei documenti e consentito l'uso di immagini che portino un valore esplicativo
  aggiuntivo alle informazioni in formato testuale. Il redattore del documento è tenuto a
  \begin{itemize}
  	\item inserire un indice delle figure presenti nel documento;
  	\item inserire una breve didascalia significativa;
  	\item posizionare al centro della pagina le immagini;
  	\item assicurarsi della corretta visibilità del loro contenuto.
  \end{itemize}  

  \subparagraph*{Uso del corsivo e dei pedici}
  Nella stesura dei documenti è stato adottato l'utilizzo del corsivo per la scrittura delle seguenti parole:
  \begin{itemize}
  	\item parole riportate nel \textit{Glossario 1.0.0\docs};
  	\item riferimenti a documenti prodotti dal gruppo;
  	\item termini propri del dominio applicativo di riferimento (Es.: \textit{Etherless}, \textit{Etherless-cli}, \textit{Etherless-smart}, \textit{Etherless-server}).
  \end{itemize} 
  Sarà evitato l'utilizzo della scrittura in corsivo per i termini elencati in precedenza se già marcati in grassetto.\\\\
  Per evitare fraintendimenti nell'utilizzo di questa notazione, sono stati utilizzati dei pedici con il seguente significato:
    \begin{itemize}
      \item \textbf{Pedice "D":} utilizzato per riferirsi ad un documento redatto dal team;
      \item \textbf{Pedice "G":} utilizzato quando una parola ha un significato
      di non immediata comprensione. Viene quindi riportata la definizione nel
      \textit{Glossario 1.0.0}\doc permettendo a tutti i lettori di capirne il significato.
    \end{itemize}

  \subparagraph*{Elenchi puntati e numerati}
  Gli elenchi puntati e numerati dovranno seguire le seguenti regole di scrittura:
	\begin{itemize}
		\item ogni riga dell'elenco deve terminare con il carattere ";" ad eccezione dell'ultima che termina con il carattere ".";
		\item se l'elenco consiste in una insieme ordinato di definizioni, la parola descritta dovrà esser scritta con la prima lettera in maiuscolo e in grassetto;
		\item se l'elenco puntato è introdotto da una frase, le parole di ciascuna riga dell'elenco dovranno iniziare con la lettera minuscola. Viceversa se non è presente alcuna frase introduttiva. 
	\end{itemize}

  \subparagraph*{Formato date e orari}
  Le date sono espresse nel seguente formato:\\\\ 
  \centerline{\textbf{YYYY-MM-DD}} 
  \begin{itemize}
  	\item \textbf{Anno (YYYY):} utilizzando quattro cifre;
  	\item \textbf{Mese (MM):} utilizzando due cifre (con valori compresi tra 0-12);
  	\item \textbf{Giorno (DD):} utilizzando due cifre (con valori compresi tra 1-31).
  \end{itemize}
  L'utilizzo di questo formato permette un corretto ordinamento lessicografico.\\\\
  Gli orari sono indicati utilizzando il seguente formato:\\\\
  \centerline{\textbf{hh:mm:ss}} 
  \begin{itemize}
  	\item \textbf{Ore (hh):} utilizzando due cifre (con valori compresi tra 0-23);
  	\item \textbf{Minuti (mm):} utilizzando due cifre (con valori compresi tra 0-59);
  	\item \textbf{Secondi (ss):} utilizzando due cifre (con valori compresi tra 0-59).
  \end{itemize}
  Talvolta, per comodità, potrà essere utilizzato un formato orario ridotto evitando la scrittura dei secondi.
  
  \subparagraph*{Abbreviazioni}
  Nella stesura dei documenti può essere talvolta utile l'utilizzo di sigle per riferirsi ad alcuni termini. Vengono elencate di seguito le principali:
  \begin{itemize}
  	\item \textbf{Documenti:}
  	\begin{itemize}
  		\item \textbf{NdP:} Norme di Progetto;
  		\item \textbf{PdQ:} Piano di Qualifica;
  		\item \textbf{PdP:} Piano di Progetto;
  		\item \textbf{SdF:} Studio di Fattibilità;
  		\item \textbf{AdR:} Analisi dei Requisiti.
  	\end{itemize}
  	\item \textbf{Ruoli:}
  	\begin{itemize}
  		\item \textbf{RE:} Responsabile di progetto;
  		\item \textbf{AM:} Amministratore;
  		\item \textbf{AN:} Analista;
  		\item \textbf{PT:} Progettista;
  		\item \textbf{PR:} Programmatore;
  		\item \textbf{VE:} Verificatore. 
  	\end{itemize}
  	\item \textbf{Revisioni di avanzamento:}
  	\begin{itemize}
  		\item \textbf{RR:} Revisione dei Requisiti;
  		\item \textbf{RP:} Revisione di Progettazione;
  		\item \textbf{RQ:} Revisione di Qualifica;
  		\item \textbf{RA:} Revisione di Accettazione.
  	\end{itemize}
  \end{itemize}

  \paragraph{Mantenimento}
	  \subparagraph*{Registro delle modifiche}
	  Ogni documento basato sul template fornito, deve avere una sezione dedicata al registro delle modifiche apportate.
	  La tabella delle modifiche deve essere composta nelle seguenti parti:
	  \begin{itemize}
	  	\item \textbf{Versione:} versione a cui fa riferimento la modifica apportata;
	  	\item \textbf{Nominativo:} nome di chi ha effettuato, verificato o approvato la modifica;
	  	\item \textbf{Ruolo:} ruolo di chi ha apportato la modifica;
	  	\item \textbf{Descrizione:} breve descrizione delle modifiche effettuate;
	  	\item \textbf{Data:} data della modifica;
	  	\item \textbf{Verificatore:} nominativo di colui che ha approvato la modifica;
	  	\item \textbf{Data di verifica:} data in cui è stata verificata la modifica.
	  \end{itemize}

    \subparagraph*{Tracciabilità delle decisioni}
      Ogni decisione presa all'interno di una riunione deve essere numerabile, tracciabile e menzionabile negli altri documenti al di fuori del verbale. Quando in un documento si fa rifermento ad una decisione durante una riunione
      occorre specificare un codice identificativo come mostrato in seguito:\\\\
      \centerline{\textbf{V[Destinazione]\_YYYY-MM-DD\_ND}} 
      \begin{itemize}
      	\item \textbf{V:} sigla per "verbale";
      	\item \textbf{Destinazione:} a seconda della destinazione del documento verranno inserite le sigle:
      	\begin{itemize}
      		\item \textbf{I:} verbale interno;
      		\item \textbf{E:} verbale esterno.
      	\end{itemize}
      	\item \textbf{YYYY-MM-DD:} data del verbale;
      	\item \textbf{ND:} numero della decisione utilizzando due cifre e iniziando la numerazione da 00.
      \end{itemize}
  
  
  	

