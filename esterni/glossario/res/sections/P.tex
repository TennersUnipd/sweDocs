\subsection*{\textbf{\hfill \Huge{P} \hfill}} 
\addcontentsline{toc}{subsection}{P}
\subsubsection*{PAAS}
Platform as a service (PaaS) è un'attività economica che consiste nel servizio di messa a disposizione di piattaforme di elaborazione (Computing platform) e di solution stack. Gli elementi del PaaS permettono di sviluppare, sottoporre a test, implementare e gestire le applicazioni aziendali senza i costi e la complessità associati all'acquisto, alla configurazione, all'ottimizzazione e alla gestione dell'hardware e del software di base.
\subsubsection*{PDCA}
Il ciclo di Deming (o ciclo di PDCA, acronimo dall'inglese Plan–Do–Check–Act, in italiano Pianificare - Fare - Verificare - Agire") è un metodo di gestione iterativo in quattro fasi utilizzato per il controllo e il miglioramento continuo dei processi e dei prodotti."
\subsubsection*{Plug-in}
Il plugin in campo informatico è un programma non autonomo che interagisce con un altro programma per ampliarne o estenderne le funzionalità originarie.
\subsubsection*{Porting}
Il termine porting indica un processo di trasposizione, a volte anche con modifiche, di un componente software, volto a consentirne l'uso in un ambiente di esecuzione diverso da quello originale.
\subsubsection*{Private key}
In una blockchain la private key è una chiave che garantisce la proprietà dei fondi associati ad un determinato address.
\subsubsection*{Product Baseline}
Documentazione tecnica che descrive la configurazione di una integrazione continua durante la produzione, ricerca/rilascio e fasi di supporto del suo ciclo di vita.
\subsubsection*{Project Board}
Come il nome suggerisce, una project board è una bacheca virtuale che contiene tutte le attività e il loro stato di completamento.
\subsubsection*{Promise}
Un promise è un oggetto che controlla il flusso di una chiamata asincrona, ci dice quando è terminata la chiamata o se la chiamata è fallita.
\subsubsection*{Proof of Concept (PoC)}
Prototipo software che dimostra la fattibilità 
\subsubsection*{Pull-request}
La Pull Request è una nostra richiesta, fatta all’autore originale di un sofware o di un documento, di includere le modifiche apportate al suo progetto
\subsubsection*{Python}
Linguaggio di programmazione ad alto livello, orientato agli oggetti, interpretato.
