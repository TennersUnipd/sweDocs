\appendix \section{Standard e modelli di riferimento}
\subsection{ISO/IEC 9126}
Lo standard ISO/IEC 9126 fornisce l'insieme delle caratteristiche rilevati che concorrono nel garantire la qualità di un prodotto. Esse sono suddivise in tre categorie:
\begin{itemize}
	\item \textbf{Qualità interne:} riguardanti le proprietà statiche e strutturali del software;
	\item \textbf{Qualità esterne:} riguardanti le proprietà dinamiche e comportamentali del prodotto;
	\item \textbf{Qualità d'uso:} relative alla percezione dell'utente finale.
\end{itemize}
Le tipologie di Qualità appena descritte sono fortemente influenzate l'una con l'altra. Le Qualità esterne ed interne descritte, sono ulteriormente specificate da altre sotto-caratteristiche, come verrà elencato in seguito:
\begin{itemize}
	\item \textbf{Funzionalità:} capacità di un prodotto software di fornire funzioni che soddisfino le esigenze stabilite. È caratterizzata da:
	\begin{itemize}
		\item \textbf{Appropriatezza:} capacità del prodotto di fornire un appropriato insieme di funzioni per i bisogni dell'utente;
		\item \textbf{Accuratezza:} capacità del prodotto di fornire i risultati attesi;
		\item \textbf{Interoperabilità:} capacità del prodotto di interagire con uno o più sistemi;
		\item \textbf{Conformità:} capacità del prodotto di aderire a degli standard consolidati;
		\item \textbf{Sicurezza:} capacità del prodotto di proteggere informazioni e dati sensibili.
	\end{itemize}
	\item \textbf{Affidabilità:} capacità del prodotto di mantenere un livello di prestazioni se soggetto a input differenti. È caratterizzata da:
	\begin{itemize}
		\item \textbf{Maturità:} capacità di un prodotto di evitare che si verificano errori o malfunzionamenti;
		\item \textbf{Tolleranza:} capacità di mantenere il livello prestativo anche in presenza di malfunzionamenti o usi scorretti del prodotto;
		\item \textbf{Interoperabilità:} capacità di un prodotto di ripristinare il livello appropriato di prestazioni in seguito a un malfunzionamento;
		\item \textbf{Recuperabilità:} capacità del prodotto di aderire a degli standard consolidati;
		\item \textbf{Aderenza:} capacità del prodotto di aderire a standard consolidati inerenti all'affidabilità.
	\end{itemize}
	\item \textbf{Usabilità:} capacità del prodotto software di essere compreso ed appreso dall'utente. È caratterizzata da:
	\begin{itemize}
		\item \textbf{Comprensibilità:} capacità del prodotto di essere utilizzata adeguatamente;
		\item \textbf{Apprendibilità:} esprime la velocità di apprendimento delle funzionalità del prodotto da parte di un utente;
		\item \textbf{Operabilità:} esprime l'indipendenza nell'utilizzo delle funzionalità data agli utenti;
		\item \textbf{Attrattivà:} esprime il grado di piacevolezza nell'utilizzo del prodotto;
		\item \textbf{Conformità:} capacità del prodotto di aderire a standard consolidati inerenti all'usabilità.
	\end{itemize}
	\item \textbf{Manutenibilità:} capacità del prodotto di essere posto sotto manutenzione e di essere modificato. È caratterizzata da:
	\begin{itemize}
		\item \textbf{Analizzabilità:} capacità che il prodotto ha di essere studiato e compreso per rilevare eventuali malfunzionamenti;
		\item \textbf{Modificabilità:} capacità del prodotto di essere modificato in seguito a malfunzionamenti senza la necessità di cambiare parti al di fuori di quella interessata dall'errore;
		\item \textbf{Stabilità:} capacità del software di evitare effetti inaspettati derivanti da modifiche errate;
		\item \textbf{Testabilità:} capacità del software di essere testato.
	\end{itemize}
	\item \textbf{Portabilità:} capacità del prodotto di non dipendere dal software, hardware o ambiente di esecuzione. È caratterizzata da:
	\begin{itemize}
		\item \textbf{Adattabilità:} capacità di cambiare ambiente di esecuzione senza dover apportare modifiche al prodotto;
		\item \textbf{Installabilità:} capacità di essere correttamente e facilmente installato indipendentemente dall'ambiente di utilizzo;
		\item \textbf{Conformità:} capacità del prodotto di aderire a standard consolidati inerenti alla portabilità;
		\item \textbf{Sostituibilità:} capacità del prodotto di compiere le stesse funzionalità di un sistema costruito per gli stessi scopi.
	\end{itemize}
\end{itemize}

\subsection{CMMI}
Il CMMI (Capability Maturity Model Integration) è un modello che fornisce le linee guida per migliorare stato di maturazione dei processi all'interno di un progetto software.
Questo modello descrive una strategia basata su 5 differenti livelli, ognuno dei quali ci fornisce delle informazioni sullo stato di maturazione del processo:
\begin{itemize}
	\item \textbf{Initial:} rappresenta il livello iniziale in cui i processi non hanno ancora regole e procedure stabilite e risultano non documentati adeguatamente;
	\item \textbf{Managed:} i processi principali sono organizzati, documentati e dunque anche ripetibili;
	\item \textbf{Defined:} i processi sono standardizzati e mantenuti efficaci ed aggiornati;
	\item \textbf{Quantitatively managed:} i processi, già ben organizzati e standardizzati, sono oggetto di processo di misurazione;
	\item \textbf{Optimizing:} i risultati ottenuti permettono il confronto e il miglioramento continuo. 
\end{itemize}