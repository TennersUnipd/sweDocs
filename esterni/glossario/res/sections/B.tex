\subsection*{\textbf{\hfill \Huge{B} \hfill}} 
\addcontentsline{toc}{subsection}{B}

\subsubsection*{Back end}
\index{Back end}
Back End è un termine largamente utilizzato per caratterizzare le interfacce che hanno come destinatario un programma. Una applicazione back end è un programma con il quale l'utente interagisce indirettamente, in generale attraverso l'utilizzo di una applicazione front-end.

\subsubsection*{BDD}
\index{BDD}
Il Behaviour Driven Development (BDD) fa parte degli approcci agile e ha lo scopo di migliorare la comunicazione all’interno dell’intero team di un progetto. Lo scopo del BDD è quello di fare in modo che il team di sviluppo comprenda appieno le richieste del cliente o dell’utilizzatore finale e che quest’ultimo sia a conoscenza di ciò che il team di sviluppo ha compreso. Durante il processo di sviluppo BDD le funzionalità vengono descritte con quello che si definiscono user stories\glo e si definiscono i criteri di accettazione\glo. Una volta che una user story\glo è definita, ci si concentra sui possibili scenari in cui la funzionalità viene eseguita.

\subsubsection*{Blockchain}
\index{Blockchain}
È definita come un registro digitale le cui voci sono raggruppate in blocchi, concatenati in ordine cronologico, e la cui integrità è garantita dall'uso della crittografia. È immutabile in quanto, di norma, il suo contenuto una volta scritto non è più né modificabile né eliminabile, a meno di non invalidare l'intera struttura.