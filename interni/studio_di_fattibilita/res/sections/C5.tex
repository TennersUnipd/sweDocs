\section{Capitolato C5 - Stalker}
\subsection{Descrizione Generale}
Lo scopo di questo progetto è la creazione di un'applicazione mobile che permette di identificare l'entrata e l'uscita da una determinata area di interesse.
\subsection{Finalità di progetto}% accorciare
L'obiettivo è quello di creare un'applicazione che permette di tracciare e monitorare lo spostamento di una persona in una determinata area. L'applicazione deve permettere all'utente di effettuare la registrazione, e quindi essere identificato nei luoghi dove richiesto (per esempio nel luogo di lavoro), oppure permettere all'utente di registrarsi in maniera anonima, nelle strutture dove non è prevista la registrazione (per esempio in una fiera).
\subsection{Tecnologie interessate}
Le tecnologie consigliate sono:
%• utilizzo dell’IAAS Kubernetes o di un PAAS, Openshift o Rancher, per il rilascio delle componenti del Server nonché per la gestione della scalabilità orizzontale.  
\begin{itemize} %Da segnare sul glossario?
	\item \textbf{Java\glo(versione 8 o superiori)}/\textbf{Python\glo}/\textbf{Nodejs\glo}: come linguaggi da utilizzare per la sviluppazione del server back-end;
	\item \textbf{IAAS Kubernetes\glo}/\textbf{PAAS\glo}/\textbf{Openshift\glo}/\textbf{Rancher\glo}:
	\item \textbf{Android\glo}: sistema operativo dove sviluppare l'applicazione;
	\item \textbf{GPS\glo}: per la localizzazione.
\end{itemize}
\subsection{Aspetti positivi}
Gli aspetti positivi sono:
\begin{itemize}
	\item l'apprendimento di tecnologie che non si studiano all'università;
	\item possibilità di comprendere nel dettaglio il funzionamento e le metodologie utilizzate per la localizzazione.
	
\end{itemize}
\subsection{Aspetti negativi}
Gli aspetti negativi sono:
\begin{itemize}
	\item difficoltà di tracciamento all'interno di un ambiente chiuso. Quindi presenza di un margine di errore;
	\item limite dato dalla batteria del dispositivo.
	
\end{itemize}
\subsection{Valutazione conclusiva}
