\section{Introduzione}

\subsection{Scopo del documento}
Nel presente documento viene dettagliata la pianificazione delle varie fasi del progetto \textit{Etherless}. Inoltre, verranno identificati e descritti i rischi del progetto, l'assegnazione di ruoli e compiti, stima dei costi e delle risorse necessarie allo sviluppo.

\subsection{Obiettivo del prodotto}
Lo scopo di \textit{Etherless} è quello di permettere a sviluppatori JavaScript\glo di rendere disponibili ad altri utenti del servizio (utilizzatori) le loro funzioni. Gli utilizzatori possono richiamare le funzionalità rese pubbliche pagando un compenso economico che verrà distribuito tra l'autore della funzione e il servizio stesso.
	
\subsection{Glossario}
Come supporto alla documentazione, viene fornito un \textit{Glossario 1.0.0}\docs, contenente delle definizioni per termini specifici che possono richiedere chiarimento. Ognuno di questi verrà contrassegnato con un pedice \glo nel documento e la sua spiegazione verrà riportata sotto la corrispondente lettera del glossario. Ciò consentirà di avere un linguaggio comune ed evitare ambiguità. 
	
\subsection{Riferimenti}
\subsubsection{Normativi}
	\begin{itemize}
		\item \textbf{Norme di Progetto}: \textit{Norme di Progetto 1.0.0}\docs;
		\item \textbf{Regolamento organigramma e offerta tecnico-economica}: Etherless\\ 
			\url{https://www.math.unipd.it/\textasciitilde tullio/IS-1/2019/Progetto/RO.html};
	\end{itemize}
\subsubsection{Informativi}
\begin{itemize}
	\item \textbf{Analisi dei requisiti}: \textit{Analisi dei requisiti 1.0.0}\docs;
    \item \textbf{Capitolato\glo d'appalto 2}:\\ 
			\url{https://www.math.unipd.it/\textasciitilde tullio/IS-1/2019/Progetto/C2.pdf};
	\item \textbf{Slide L05 del corso Ingegneria del Software - Ciclo di vita del software:} \\
			\url{https://www.math.unipd.it/\textasciitilde tullio/IS-1/2019/Dispense/L05.pdf};
	\item \textbf{Slide L06 del corso Ingegneria del Software - Gestione di Progetto:} \\
			\url{https://www.math.unipd.it/\textasciitilde tullio/IS-1/2019/Dispense/L06.pdf};
\end{itemize}
\subsection{Scadenze}
Le scadenze per il progetto \textit{Etherless} sono le seguenti:
\begin{itemize}
	\item \textbf{Revisione dei Requisiti:} 2020-01-14;
	\item \textbf{Presentazione:} 2020-01-21;
	\item \textbf{Revisione di Progettazione:} 2020-03-16;
	\item \textbf{Revisione di Qualifica:} 2020-04-20;
	\item \textbf{Revisione di Accettazione:} 2020-05-18;
\end{itemize}

	
	
	