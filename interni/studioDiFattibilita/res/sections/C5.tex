\section{Capitolato C5 - Stalker}
\subsection{Descrizione Generale}
Le nuove normative in merito alla sicurezza dei locali pubblici, impongono una corretta gestione del numero di individui presenti al loro interno. In questo contesto, Imola Informatica propone la realizzazione di un prodotto software che riesca a tenere traccia del numero di persone presenti in una struttura e la loro corretta posizione all'interno di essa.
\subsection{Finalità di progetto}
Il prodotto finale dovrà prevedere la realizzazione delle seguenti componenti:
\begin{itemize}
	\item un applicativo mobile che consenta all'utente finale di registrare la propria presenza. Essa verrà segnalata in maniera anonima o trasmettendo al server il proprio identificativo in base alla tipologia del luogo di interesse. L'utente avrà la possibilità di visualizzare in tempo reale la propria presenza all'interno del luogo e da quanto tempo si trova al suo interno;
	\item una sezione dedicata agli amministratori, i quali avranno la possibilità, attraverso un'apposita interfaccia grafica, di eseguire il login all'interno del server e di gestire varie informazioni come ad esempio la lista degli eventi di interesse e lo storico degli ingressi di ciascuna persona.
\end{itemize}
\subsection{Tecnologie interessate}
\begin{itemize}
	\item \textbf{Java(versione 8 o superiori)}/\textbf{Python\glos}/\textbf{Node.js}: come linguaggi da utilizzare per lo sviluppo del server \textit{back-end\glos};
	\item \textbf{IAAS Kubernetes\glos}/\textbf{PAAS\glos}/\textbf{Openshift\glos}/\textbf{Rancher\glos}: piattaforme per applicazioni per il rilascio delle componenti del Server e la gestione della scalabilità orizzontale.
	\item \textbf{Android}/\textbf{iOS}: sistemi operativi su cui sviluppare l'applicazione;
\end{itemize}

\subsection{Aspetti positivi}
Possibilità di comprendere nel dettaglio il funzionamento e le metodologie utilizzate per la localizzazione.

\subsection{Criticità}
Come specificato durante la presentazione del \textit{capitolato\glo} potrebbero esserci difficoltà di tracciamento all'interno di un ambiente chiuso e problematiche derivanti dalla batteria limitata del dispositivo.

\subsection{Valutazione conclusiva}
Il \textit{capitolato\glo} è stato scartato dal gruppo sin dall'inizio per il poco interesse nello sviluppo di un applicativo mobile, oltre che per la considerevole mole di lavoro preventivata nello sviluppo delle varie parti del prodotto. Oltretutto, la criticità relativa alla localizzazione in ambienti chiusi, ha portato l'interesse verso altri capitolati.
