\section{Capitolato C2 - Etherless}

\subsection{Descrizione generale}
\textit{Etherless} è una piattaforma che permette agli sviluppatori di caricare funzioni JavaScript\glo sul cloud\glo per poi renderle disponibili a terzi. Il servizio si presenta come un CaaS\glo (Computation-as-a-Service) in cui l'utente finale paga per l'esecuzione di una singola funzione in modo automatico mediante l'utilizzo della rete \textit{Ethereum}\glo e degli \textit{smart contract}\glo. 
\subsection{Finalità del progetto}
L'obbiettivo finale è quello di avere un ambiente in cui lo sviluppatore \textit{Bob}, dopo aver sviluppato una funzionalità che potrebbe essere di interesse per altri sviluppatori (es. \textit{Alice}), carica il suo codice JavaScript su \textit{Etherless} mediante la sua utenza e imposta un costo di esecuzione per quella funzione. \textit{Alice}, abile programmatrice, siccome ha bisogno di tale funzionalità, decide che conviene pagare la quota di esecuzione piuttosto che riscrivere la stessa procedura; attraverso la sua utenza \textit{Etherless}, \textit{Alice} è quindi capace di usufruire della funzione in cloud e sostenere il costo di esecuzione.
\\\\
La piattaforma deve essere conforme a quanto segue:
\begin{itemize}
	\item gli utenti avranno la possibilità elencare le funzioni disponibili,  caricarne di nuove, aggiornarle, eseguirle oppure eliminare attraverso una interfaccia CLI\glo;
	\item utilizzo della rete \textit{Ethereum}\glo per la comunicazione tra i vari componenti di \textit{Etherless}, per la definizione della logica di interazione tra le varie parti (smart contract\glo) e per lo storage\glo di dati;
	\item i ricavato monetario ricavato dall'esecuzione di una funzione deve essere distribuito all'autore del codice e al sostenimento dell'infrastruttura cloud su cui avviene l'esecuzione
	\item la granularità del pagamento deve corrispondere alla singola esecuzione  di una funzione;
	\item per la realizzazione dell backend deve essere prevista una infrastruttura Serverless\glo;
\end{itemize}

\subsection{Tecnologie interessate}
\begin{itemize}
	\item \textbf{AWS (Amazon Web Services)}: piattaforma che si occupa di fornire servizi di cloud computing\glo;
	\item \textbf{ASL - Lambda}: servizio che consente di eseguire codice nel cloud;
	\item \textbf{Ethereum}: una rete globale per il trasferimento di criptovalute\glo e per la realizzazione di applicativi decentralizzati
	\item \textbf{Solidity}: linguaggio OOP\glo per la definizione di smart contract\glo;
	\item \textbf{Truffle}: framework per lo sviluppo di smart contract su rete Ethereum\glo;
	\item \textbf{Web3}: API\glo JavaScript per l'interazione con un noto Ethereum\glo locale o remote;
	\item \textbf{Ropsten}: Rete Ethereum\glo pubblica usato per il testing di applicativi Ethereum\glo prima del porting\glo in produzione sulla \textit{MainNet};
	\item \textbf{MainNet}: Rete Ethereum\glo principale
	\item \textbf{Ganache}: Ambiente di sviluppo Ethereum\glo utilizzato per la simulazione locale si una rete Ethereum\glo e per l'analisi delle transazioni e log;
	\item \textbf{TypeScript 3.6}: versione standardizzata da Microsoft del linguaggio JavaScript;
	\item \textbf{Node.js}: ambiente di runtime open-source per JavaScript;
	\item \textbf{The Serverless Framework}: framework per la costruzione e deploy\glo di ambienti serverless\glo
	\item \textbf{Smart Contract}: protocolli informatico che facilita, verifica, fa rispettare ed esegue un contratto (insieme di regole);
	\item \textbf{ESLint}: strumento di analisi del codice utilizzato per identificare pattern\glo problematici nel codice JavaScript;
\end{itemize}

\subsection{Aspetti positivi}
\begin{itemize}
	\item Possibilità di apprendere e utilizzare le tecnologie AWS;
	\item Progetto basato su concetti di apprendimento automatico\glos, mai affrontati durante il percorso di studi e ritenuti interessanti da molti componenti del gruppo.
\end{itemize}

\subsection{Criticità}
\begin{itemize}
	\item Seppur stimolante, l'apprendimento e il corretto utilizzo della piattaforma AWS e i suoi servizi può essere molto dispendioso in termini di tempo;
	
\end{itemize}

\subsection{Valutazione conclusiva}
Il gruppo fin da subito ha ritenuto molto accattivante il capitolato. Tuttavia, dopo un'attenta analisi dei rischi e vantaggi, si è deciso di focalizzare le attenzioni verso il C2. Sebbene siano consigliati linguaggi per lo sviluppo della UI\glo già noti, addestrare il modello per identificare i momenti significativi dell'evento sportivo e apprendere le nuove tecnologie AWS mai utilizzate sino ad ora, hanno demotivato il team nell'intraprendere questo progetto.