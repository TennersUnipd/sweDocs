\subsection*{\textbf{\hfill \Huge{O} \hfill}} 
\addcontentsline{toc}{subsection}{O}
\subsubsection*{OOP}
\index*{OOP}
La programmazione orientata agli oggetti (OOP, Object Oriented Programming) è un paradigma di programmazione che permette di definire oggetti software in grado di interagire gli uni con gli altri attraverso lo scambio di messaggi. È particolarmente adatta nei contesti in cui si possono definire delle relazioni di interdipendenza tra i concetti da modellare (contenimento, uso, specializzazione).
\subsubsection*{Open-source}
\index*{Open-source}
In informatica il termine inglese open source viene utilizzato per riferirsi ad un tipo di software o al suo modello di sviluppo o distribuzione. Un software open source è reso tale per mezzo di una licenza attraverso cui i detentori dei diritti ne favoriscono la modifica, lo studio, l'utilizzo e la redistribuzione.
\subsubsection*{Openshift}
\index*{Openshift}
OpenShift è un platform as a service (PaaS) prodotto da Red Hat ed è una piattaforma per applicazioni cloud che rende semplice lo sviluppo, il deploy e la scalabilità di applicazioni cloud.
