\section{Capitolato C6 - Things Relationship Management (ThiReMa)}

\subsection{Descrizione generale}
\textit{Sanmarco Informatica} propone lo sviluppo di un software in grado di ricevere misurazioni da sensori eterogenei, divisibili nelle macro-categorie dati operativi e fattori influenzanti, ed accumularli in uno o più database. L'applicazione quindi deve poter fornire un servizio di dispatching\glos, basato su Telegram, per inoltrare le informazioni più utili o urgenti per la gestione dei dispositivi installati.

\subsection{Finalità del progetto}
Il software \textit{ThiReMa} deve provvedere all'interazione tra i sensori e il database sviluppando le seguenti componenti:
\begin{itemize}
	\item uno o piú "Producer" con il compito di convertire i dati inviati dai sensori in messaggi utilizzabili da Kafka (in formato .JSON);
	\item un "Connect" atto alla scrittura dei dati nel database;
	\item una serie di componenti "Stream" per la modifica e trasformazione dei dati presenti nel database;
	\item dispatching\glos delle informazioni basato su Telegram.
\end{itemize}

\subsection{Tecnologie interessate}
\begin{itemize}
	\item \textbf{Apache Kafka}: cluster\glo che si interfaccia fra i sensori ed il database;
	\item \textbf{Java}: linguaggio di programmazione utilizzato per sviluppare le componenti;
	\item \textbf{PostgreSQL}: database relazionale utilizzabile;
	\item \textbf{TimescaleDB}: database improntato alla gestione temporale dei dati;
	\item \textbf{ClickHouse}: database improntato alla gestione dei dati per colonna;
	\item \textbf{Bootstrap}: libreria per la creazione della parte front-end\glos;
	\item \textbf{Docker}: container\glo nel quale istanziare le componenti custom di Apache Kafka;
	\item \textbf{Github}: sistema di versionamento\glos;
	\item \textbf{Telegram}: servizio tramite il quale effettuare il dispatching\glos.
\end{itemize}

\subsection{Aspetti positivi}
Stimolante la possibilità di lavorare per un progetto che richiede competenze in ambito IoT\glo e di Big Data\glos.


\subsection{Criticità}
Grande complessità di esecuzione dovuta alla mole di applicativi differenti con il quale lavorare.

\subsection{Valutazione conclusiva}
Il team ha ritenuto interessante la possibilità di realizzare una web application implicata nel mondo dell'IoT\glo e dei Big Data\glos. Nonostante sia stato preso in considerazione per il gran numero di competenze che si sarebbero potute affinare, lo sforzo per imparare così tante tecnologie e la possibilità che il lavoro finale non potesse essere all'altezza delle aspettative, lo hanno reso meno appetibile.