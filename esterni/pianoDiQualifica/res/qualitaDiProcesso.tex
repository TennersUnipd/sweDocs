\section{Qualità di processo}
Il gruppo ha deciso d'istanziare, con le dovute modifiche, i processi descritti
nell'ISO/IEC 12207 adattandoli secondo esigenze, stabilendone per ciascuno
obiettivi e metriche per la qualità.
I 5 livelli di maturità di un processo descritti nel CMMI
(descritti in appendice delle \textit{Norme di progetto 2.x.x}\doc),
aiutano a comprendere quali metriche e obiettivi porre per la misurazione della
qualità e il suo miglioramento continuo.
\subsection{Sviluppo}
Lo Sviluppo è un processo che raggruppa le attività relative alla realizzazione
effettiva del prodotto software. Garantendo la qualità delle attività che lo
compongono mediante le apposite metriche, il processo di Sviluppo può
considerarsi a sua volta qualitativamente definito. Viene dunque stabilito un piano per la qualità per alcune di esse.

\subsubsection{Analisi dei requisiti}
%\paragraph{Obbiettivi}
%\begin{itemize}
%	\item l'analisi deve essere formare requisiti chiari e non contraddittori;
%	\item con l'avanzamento del progetto, i requisiti possono esser raffinati;
%	\item i requisiti devono essere completi e approvati da ogni \textit{stakeholder\glos} coinvolto nel progetto.
%\end{itemize}
\paragraph{Metriche}
\begin{itemize}
	\item Requisiti obbligatori soddisfatti;
	\item Requisiti desiderabili soddisfatti.
\end{itemize}
\renewcommand{\arraystretch}{2.2}
\rowcolors{2}{pari}{dispari}
\begin{longtable}{|C{4.5cm}|C{2.25cm}|C{2.25cm}|C{3cm}|}
	\arrayrulecolor{white}

	\caption{Metriche per l'Analisi dei requisiti }\\
	\hline
	\rowcolor{header}

	\textbf{Metrica} & \textbf{Intervallo preferibile}  & \textbf{Intervallo ottimale} & \textbf{Unità di misura}
	\tabularnewline
	\endfirsthead

	Requisiti obbligatori soddisfatti &  100 & 100 & Percentuale \\
	Requisiti desiderabili soddisfatti &  > 50 & 100 & Percentuale \\
\end{longtable}
\pagebreak


\subsection{Progettazione}
%\paragraph{Obiettivi}
%\begin{itemize}
%	\item utilizzo delle best practice derivanti da design pattern strutturali noti e consolidati;
%	\item in caso di modifiche dei requisiti, l'architettura deve permettere la modifica delle sole componenti interessate;
%	\item l'architettura realizzata deve ridurre la complessità di realizzazione del sistema.
%\end{itemize}

\subsubsection{Metriche}
\begin{itemize}
	\item Structural Fan-out (SFOUT);
	\item Structural Fan-in (SFIN).
\end{itemize}

\renewcommand{\arraystretch}{2.2}
\rowcolors{2}{pari}{dispari}
\begin{longtable}{|C{4.5cm}|C{2.25cm}|C{2.25cm}|C{3cm}|}
	\arrayrulecolor{white}

	\caption{Metriche per la progettazione}\\
	\hline
	\rowcolor{header}

	\textbf{Metrica} & \textbf{Intervallo preferibile}  & \textbf{Intervallo ottimale} & \textbf{Unità di misura}
	\tabularnewline
	\endfirsthead

	SFOUT & <= 6  & 0 & Valore numerico intero positivo \\
	SFIN &  >= 0 & > 0 & Valore numerico intero positivo \\
\end{longtable}




\subsection{Documentazione}
La Documentazione è un processo che supporta le attività svolte durante
il progetto software.
Il compito di questo processo è documentare e
gestire in forma scritta le informazioni necessarie per le diverse parti del
ciclo di vita del prodotto.
%\subsubsection{Obiettivi}
%\begin{itemize}
%	\item semplicità e immediata comprensione da parte di tutti gli \textit{stakeholders\glo};
%	\item correttezza grammaticale e ortografica;
%	\item contenuti completi e coerenti per l'attività documentata.
%\end{itemize}
\subsubsection{Metriche}
\begin{itemize}
	\item Indice Gulpease.
\end{itemize}

\renewcommand{\arraystretch}{2.2}
\rowcolors{2}{pari}{dispari}
\begin{longtable}{|C{2.5cm}|C{3.25cm}|C{3.25cm}|C{3cm}|}
	\arrayrulecolor{white}

	\caption{Metriche per la documentazione}\\
	\hline
	\rowcolor{header}

	\textbf{Metrica} & \textbf{Intervallo preferibile}  & \textbf{Intervallo ottimale} & \textbf{Unità di misura}
	\tabularnewline
	\endfirsthead

	Indice Gulpease & >= 40 e <=100  & >=60 e <=100 & Valore numerico intero positivo (da 0 a 100) \\

\end{longtable}




\subsection{Verifica}
Il processo di Verifica si attua mediante controlli basati su norme e metriche che consentono l'accertamento della correttezza del prodotto software realizzato.
%\subsubsection{Obiettivi}
%\begin{itemize}
%	\item provare la correttezza rispetto a norme e metriche;
%	\item correggere errori dei processi implicati nel ciclo di vita software.
%\end{itemize}
\subsubsection{Metriche}
\begin{itemize}
	\item Code Coverage (CC)
\end{itemize}

\renewcommand{\arraystretch}{2.2}
\rowcolors{2}{pari}{dispari}
\begin{longtable}{|C{2.5cm}|C{3.25cm}|C{3.25cm}|C{3cm}|}
	\arrayrulecolor{white}

	\caption{Metriche per la verifica}\\
	\hline
	\rowcolor{header}

	\textbf{Metrica} & \textbf{Intervallo preferibile}  & \textbf{Intervallo ottimale} & \textbf{Unità di misura}
	\tabularnewline
	\endfirsthead

	CC & >= 70  & = 100 & Percentuale \\

\end{longtable}


\subsection{Gestione}
\subsubsection{Pianificazione}
La pianificazione è un'attività del processo di Gestione che si occupa dell'allocazione delle risorse e delle attività disponibili nel tempo, rientrando nei costi e nelle risorse prestabiliti.
%\paragraph{Obiettivi}
%\begin{itemize}
%	\item definizione di un modello di sviluppo opportuno per lo scopo e la complessità del progetto;
%	\item definizione degli obiettivi da perseguire nel tempo;
%	\item allocazione delle risorse disponibili in base al budget a disposizione e alle attività da portare a termine.
%\end{itemize}
\paragraph{Metriche}
\begin{itemize}
	\item \textbf{Planned Value (PV)}
	\item \textbf{Actual Cost (AC)}
	\item \textbf{Budget At Completion (BAC)}
	\item \textbf{Earned Value (EV)}
	\item \textbf{Estimated At Completion (EAC)}
	\item \textbf{Estimate To Complete (ETC)}
	\item \textbf{Cost Variance	(CV)}
	\item \textbf{Schedule Variance	(SV)}
\end{itemize}

\renewcommand{\arraystretch}{2.2}
\rowcolors{2}{pari}{dispari}
\begin{longtable}{|C{2cm}|C{3.5cm}|C{3.5cm}|C{3cm}|}
	\arrayrulecolor{white}

	\caption{Metriche per la pianificazione }\\
	\hline
	\rowcolor{header}

	\textbf{Metrica} & \textbf{Intervallo sufficiente}  & \textbf{Intervallo ottimale} & \textbf{Unità di misura}
	\tabularnewline
	\endfirsthead

	PV & >= 0 & >= 0 e >=AC & Euro \\
	AC & >= 0 e <= BAC & >= 0 e <= PV & Euro \\
	BAC & = Preventivo  & = Preventivo & Euro \\
	EV & >0  & >0 & Euro \\
	EAC &  >= 0 e <= BAC& >= (BAC*95\% ) e <= (BAC*105\%)& Euro \\
	ETC & >= 0 & >= 0 e <= EV & Euro \\
	CV & > 0 & >= 0 & Euro \\
	SV & > 0 & = 0 & Euro \\
\end{longtable}
