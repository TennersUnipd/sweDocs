\subsection{Accertamento della qualit\` a}
\subsubsection{Scopo}
Questa sezione del documento si prefiggie di descrivere i processi che verranno
messi in moto per assicurarci di produrre un prodotto di qualità.
Per mettere in atto il piano di \textit{Quality assurance\glo} è necessario il
\textit{Continuous Testing\glo} e \textit{Continuous Integration\glo}

\paragraph{\textit{Continuous Testing\glo}}
Per assicurarci di testare ogni parte unitaria del codice, si segue un approccio
di tipo \textit{B.D.D.\glo}, ovvero prima d'implementare l'architettura del software
individuata devono essere forniti i \textit{Test di unità\glo} che l'implementazione dovrà soddisfare.
I \textit{Test di unità\glo} seguono il ciclo di vita del software, ovvero dopo la
scrittura devono essere anchessi verificati e approvati.

\paragraph{\textit{Continuous Integration\glo}}
Per accertarci che tutte le componenti del software funzionino correttamente, si
utlizzano tecniche di Continuous Integration(C.I.), ovvero tutti i componenti che
superano i \textit{Test di unità\glo} verranno sottoposti a test d'integrazione
cioè verrà verificata la capacità di ogni unità di operare con le altre.
