\section{Introduzione}

\subsection{Scopo del documento}
Il presente documento esemplifica in maniera dettagliata quali sono le specifiche del prodotto finale e i requisiti tecnici. Tali informazioni sono raccolte analizzando i documenti messi a disposizione dal cliente (capitolato\glo) e interrogando il cliente nel caso di casistiche per cui non si sa precisamente come il prodotto si dovrebbe comportare.\\
Il cliente in questione è l'azienda \textit{Red Babel}.
\\Il prodotto viene presentato al cliente finale sotto forma di servizio.

\subsection{Scopo del prodotto}
Lo scopo del prodotto è quello di permettere a sviluppatori JavaScript\glo di rendere disponibili ad altri utenti del servizio (utilizzatori) le loro funzioni. Gli utilizzatori possono richiamare le funzioni rese pubbliche in cambio di un compenso economico che verrà distribuito tra l'autore della funzione e il servizio stesso (\textit{Etherless}).
	
\subsection{Glossario}
Come supporto alla documentazione, viene fornito un \textit{Glossario v.1.0.0}, contenente delle definizioni per termini specifici che possono richiedere chiarimento. Ognuno di questi verrà contrassegnato con un pedice \glo nel documento e la sua spiegazione verrà riportata sotto la corrispondente lettera del glossario. Ciò consentirà di avere un linguaggio comune ed evitare ambiguità. 
	
\subsection{Riferimenti}
\subsubsection{Normativi}
	\begin{itemize}
		\item \textbf{Norme di Progetto}: \textit{Norme di Progetto v1.0.0};
		\item \textbf{Capitolato\glo d'appalto 2}: Etherless\\ 
			\url{https://www.math.unipd.it/\textasciitilde tullio/IS-1/2019/Progetto/C2.pdf}
		\item \textbf{VERBALE ESTERNI TODO}
	\end{itemize}
\subsubsection{Informativi}
\begin{itemize}
	\item \textbf{Studio di fattibilità}: \textit{Studio di Fattibilità v1.0.0}
    \item \textbf{Capitolato\glo d'appalto 2}: Etherless\\ 
			\url{https://www.math.unipd.it/\textasciitilde tullio/IS-1/2019/Progetto/C2.pdf}
	\item \textbf{Truffle}:  Ambiente di sviluppo e testing in ambiente Ethereum\glo. Materiale informativo su Smart contract\glo e rete Ethereum\glo
		\\ \url{https://www.trufflesuite.com/}
	\item \textbf{Serverless}:  Framework di sviluppo per applicativi serverless		\\ \url{https://serverless.com/}
	\item \textbf{AWS}: Servizi computazionali in cloud \\ \url{https://aws.amazon.com/}
	\item \textbf{Sito ufficiale Ethereum\glo}: Descrizione dei concetti base, funzionamento e procedure per la realizzazione di applicativi Ethereum\glo.
\\ \url {https://www.ethereum.org/}
\end{itemize}

	
	
	