\section{Processi Organizzativi}
	\subsection{Gestione}
		\subsubsection{Scopo}
  			%Questa parte del documento \`e necessaria per la creazione di un way of working ben organizzato permette di avere un flusso dei dati per definito dall'inizio alla fine
			Lo scopo di questo processo è quello di creare un way of working, utile ai membri del gruppo per organizzare e gestire i ruoli di ogni componente e le attività da svolgere.
  		\subsubsection{Aspettative}
  			Le aspettative per questo processo sono:
  			\begin{itemize}
  				\item definizione dei ruoli dei membri del gruppo;
  				\item pianificare le attività da seguire;
  				\item adoperare processi per regolare le attività e renderle economiche.
			\end{itemize}
   		\subsubsection{Descrizione}
   			Le attività che compongono questo processo sono:
   			\begin{itemize}
   				\item istanziazione dei processi;
   				\item assegnazione dei ruoli di progetto e dei compiti;
   				\item pianificazione e stima dei costi, tempi e risorse;
	   			\item esecuzione e controllo dei compiti;
   				\item verifica periodica dei compiti.
		   	\end{itemize}
   		\subsubsection{Ruoli di progetto}
   			Ciascun membro del gruppo, a rotazione, dovrà ricoprire il ruolo che gli viene affidato e che corrisponderà all'omonima figura aziendale. Nel \textit{Piano di Progetto} vengono organizzate e pianificate le attività assegnate a ciascun ruolo e ogni membro dovrà rispettare tali norme.
   			
   			Nell'assegnamento dei ruoli si cercherà di evitare situazioni di conflitto, per garantire la qualità dei documenti e della verifica di quest'ultimi. 
   			
   			I ruoli sono descritti qui di seguito.
   			\paragraph{Analista} \mbox{}\\ \mbox{}\\
   				L'Analista è colui che effettua l'analisi dei problemi e del dominio applicativo. \`{E} figura molto importante ai fini della riuscita del progetto, anche se non sarà sempre presente nello sviluppo di quest'ultimo, in quanto, se l'analista effettua degli errori nell'\textit{Analisi dei Requisiti}, questi possono portare a gravi problemi di progettazione.
   				
   				I compiti principali dell'Analista sono:
   				\begin{itemize}
   					\item studiare e definire il problema da risolvere, e la sua complessità;
   					\item analizzare il dominio applicativo: utenti, ambiente d'uso;
   					%\item effettuare un'analisi dei bisogni descritti dal proponente e dal committente 
   					\item produzione dello \textit{Studio di Fattibilità} e  dell'\textit{Analisi dei Requisiti}.
   				\end{itemize}
   			\paragraph{Progettista} \mbox{}\\ \mbox{}\\
   			Il Progettista è colui che si occupa degli aspetti tecnici e tecnologici del progetto. \`{E} inoltre responsabile delle scelte architetturali.
   			
   			Partendo dall'analisi effettuata dall'\textit{Analista}, il suo compito è quello di trovare una soluzione ottimale ai problemi e ai requisiti individuati.
   			
   			I suoi compiti principali sono:
   			\begin{itemize}
   				\item sviluppo dell'architettura secondo un insieme di best practice per garantire coerenza e consistenza;
   				\item effettuare scelte che portano ad una soluzione efficacie, efficiente e comprensibile rispetto ai requisiti, rientrando nei preventivi di costo e risorse;
   				\item rendere facilmente mantenibile il progetto;
   				\item sviluppo di un'architettura robusta , flessibile e sicura, che permetta al prodotto finale di rimanere funzionante e di effettuare modifiche a basso costo, in caso malfunzionamento;
   				\item applicazioni di soluzioni note e ottimizzate.
   			\end{itemize}
   			\paragraph{Programmatore} \mbox{}\\ \mbox{}\\
   			Il Programmatore è colui che si occupa della codifica del progetto e delle componenti di supporto, utili per effettuare le prove di verifica e validazione del prodotto.
   			
   			Il suo compito principale è quello di realizzare l'architettura ideata dal \textit{Progettista}.
   			
   			Tra i suoi compiti ci sono:
   			\begin{itemize}
   				\item implementare le decisioni del \textit{Progettista};
   				\item creazione e gestione delle componenti di supporto per verifica e validazione del codice.
   				\item stesura del \textit{Manuale Utente}.
   			\end{itemize}
   			\paragraph{Verificatore} \mbox{}\\ \mbox{}\\
   			Il Verificatore è colui che effettua il controllo sulle attività svolte, assicurandosi che esse siano conformi alle attese e alle norme stabilite.
   			
   			I suoi compiti sono:
   			\begin{itemize}
   				\item verificare che le \textit{Norme di progetto} siano rispettate;
   				\item verificare la presenza di errori, e in caso segnalarne la presenza all'autore dell'oggetto preso in esame;
   				\item segnalare al \textit{Responsabile} eventuali situazioni di conflitto che violano ciò che è indicato nel \textit{Piano di Progetto}.
   			\end{itemize}
   			\paragraph{Responsabile} \mbox{}\\ \mbox{}\\
   			Il Responsabile è il punto di riferimento sia per il fornitore che per il committente. Inoltre è colui che raccoglie su di sè le responsabilità decisionali di scelta e approvazione e costituisce il centro di coordinamento per l'intero progetto.
   			
   			I suoi compiti principali sono:
   			\begin{itemize}
   				\item approvazione della documentazione;
   				\item gestione, controllo e coordinamento delle risorse e dell'attività del gruppo;
   				\item studio e gestione dei rischi.
   			\end{itemize}
   			\paragraph{Amministratore} \mbox{}\\ \mbox{}\\
   			L'Amministratore è colui che controlla e amministra l'ambiente di lavoro.
   			
   			L'Amministratore non deve operare scelte gestionale, ma aiutare il\textit{Responsabile} ad organizzare e gestire l'ambiente di lavoro.
   			
   			I suoi compiti sono:
   			\begin{itemize}
   				\item gestione del versionamento e delle configurazioni del progetto;
   				\item risolvere problemi legati alla gestione dei processi;
   				\item amministrare le infrastrutture di supporto;
   				\item gestione della documentazione;
   				\item individuazione degli strumenti atti all'automazione dei processi;
   			\end{itemize}
   		\subsubsection{Procedure}
   			\paragraph{Gestione delle comunicazioni}  \mbox{}\\ \mbox{}\\
   				\textbf{Comunicazioni interne}\newline \newline
   					Per le comunicazione interne è stato deciso di utilizzare  Slack\glos, un'applicazione di messaggistica multi piattaforma con funzionalità specifiche per gruppi di lavoro. Ogni canale è specifico per una determinata attività.
   				
   					All'interno di Slack\glo sono stati aggiunti i seguenti canali:
   					\begin{itemize}
   						\item \textbf{github}: per tutto ciò che 	riguarda github;
   						\item \textbf{pianificazione-incontri}: dove i membri del gruppo decidono quando effettuare i vari incontri.
   					\end{itemize}
   					Altri canali telematici, sono stati aggiunti per gestire la stesura dei vari documenti quali:
   					\begin{itemize}
   						\item \textbf{norme\_di\_progetto}: per discutere e definire le Norme di Progetto alle quali attenersi durante le varie fasi di progetto;
   						\item \textbf{studio\_di\_fattibilità}: 
   					\end{itemize}
   					Oltre all'applicazione Slack\glo viene anche utilizzato un apposito gruppo Telegram\glos e Google Hangouts\glos, quando serve una comunicazione immediata, ma non è possibile trovarsi di persona.
   				\newline \newline
   				\textbf{Comunicazioni esterne} \newline \newline
   					Le comunicazioni esterne sono affidate al \textit{Responsabile di Progetto}, il quale si avvale dell'utilizzo della posta elettronica, attraverso l'indirizzo \href{mailto:tenners.unipd@gmail.com}{tenners.unipd@gmail.com}.
   					
   					Per comunicare con Red Babel, invece, si utilizza un canale Slack\glo per la chat testuale e il servizio Google Hangouts\glo o Skype\glo per le chiamate.
   					
   					Il \textit{Responsabile} ha il compito di tenere informato i vari componenti del gruppo, in caso di loro assenza.
   			\paragraph{Gestione degli incontri}  \mbox{}\\ \mbox{}\\
   				\textbf{Incontri interni}\newline \newline
   					Le riunioni interne del gruppo sono organizzate dal \textit{Responsabile} in accordo con gli altri componenti del team, attraverso gli appositi canali.
   					
   					Per ogni incontro viene esteso un \textit{Verbale}.Questo sarà redatto da un segretario, persona nominata dal responsabile, che dovrà tenere nota delle discussioni fatte e delle decisioni prese. LA struttura dei verbali è la seguente.
   					
   				\textbf{Verbali}
   				La prima parte contiene le informazioni generali quali:
   				\begin{itemize}
   					\item \textbf{luogo}: indica il luogo dove è avvenuto l'incontro;
   					\item \textbf{data}: indica la data dell'incontro, nel formato AA-MM-DD;
   					\item \textbf{ora di inizio};
   					\item \textbf{ora di fine};
   					\item \textbf{partecipanti}: indica chi era presente alla riunione, sotto forma di elenco puntato;
   				\end{itemize}
	   			\textbf{Incontri esterni}\newline \newline
   					Il \textit{Responsabile} ha il compito organizzare e comunicare gli incontri con il proponente o il committente. Se uno o più membri del gruppo, il proponente o il committente volessero organizzare un incontro, sarà compito del \textit{Responsabile} decidere una data, in accordo tra le parti, e successivamente di comunicarla tramite i canali appositi.
   					
   					Come per gli incontri interni, anche per quelli esterni viene redatto un \textit{Verbale}.
   			\paragraph{Gestione degli strumenti di coordinamento} \mbox{}\\ \mbox{}\\
   				\textbf{Tickecting} \newline \newline
   					Il tickecting permette ai vari membri del gruppo di monitorare, in qualsiasi momento, le attività in corso, permette al \textit{Responsabile} di monitorare l'andamento delle attività.
   			\paragraph{Gestione rischi}\mbox{}\\ \mbox{}\\
   				Il \textit{Responsabile} ha il compito di individuare i rischi descritti nel \textit{Piano di Progetto}, e una volta trovati, inserirli nell'analisi dei rischi.
   				
   				La procedura da seguire per la gestione dei rischi è la seguente:
   				\begin{itemize}
   					\item individuare problemi non calcolati e monitorare i rischi già previsti;
   					\item registrare ogni riscontro previsto dei rischi nel \textit{Piano di Progetto};
   					\item aggiungere i nuovi rischi individuati nel \textit{Piano di Progetto};
   					\item  se necessario, ridefinire le strategie di gestione dei rischi.
   				\end{itemize}
   		\subsubsection{Strumenti}
   			Il gruppo durante lo svolgimento del progetto ha utilizzato o utilizzerà i seguenti strumenti:
   			\begin{itemize}
   				\item \textbf{Telegram\glos}: come strumento di messaggistica.
   				\item \textbf{Slack\glos}: come strumento di comunicazione interna ed eventualmente con il proponente;
   				\item \textbf{Git\glos}: come sistema di controllo di versionamento;
   				\item \textbf{Gitflow\glos}: come interfaccia per usare, più comodamente, Git\glo da Desktop;
   				\item \textbf{GitHub\glos}: come strumento per il versionamento e il salvataggio in remoto di tutti i file legati al progetto;
   				\item \textbf{Hangouts\glos}: come strumento per effettuare videoconferenze e chiamate VoIP;
   				\item \textbf{Sistemi Operativi}: i requisiti non prevedono di utilizzare un sistema operativo specifico, per questo verranno usati, dai membri del team, Windows, Linux.
   			\end{itemize}
   		\subsubsection{Formazione}
   			\paragraph{Formazione dei membri del gruppo} \mbox{}\\ \mbox{}\\
   				
   			\paragraph{Guide e materiale utilizzato}\mbox{}\\ \mbox{}\\