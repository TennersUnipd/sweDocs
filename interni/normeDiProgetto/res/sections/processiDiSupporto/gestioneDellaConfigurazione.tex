\subsection{Gestione della configurazione}
\subsubsection{Scopo}
La gestione della configurazione consiste nel controllo degli oggetti che compongono sistema da realizzare. Questa sezione descrive dove e in che modo verranno conservati i file che compongono il prodotto finale.

\subsubsection{Condivisione e sistema di versionamento}
Per il versionamento del prodotto è stato scelto il software \textit{Git\glos}. Per condividere la repository in remoto, i membri del gruppo hanno optato per \textit{Github\glos}, la piattaforma per lo sviluppo collaborativo più rinomata. \\

\noindent La scelta nell'utilizzo di questi due sistemi per versionamento e condivisione, è dettata dalla precedente esperienza da parte di alcuni membri del team. I componenti del gruppo che non hanno mai avuto esperienza, avranno supporto da parte dei restanti membri e saranno tenuti a documentarsi mediante le apposite guide condivise sui canali di comunicazione del gruppo. \\

\noindent Non è stato imposto alcun vincolo sulla scelta del client \textit{Git\glos}. Ogni componente del gruppo avrà libertà di scelta nell'utilizzo di un client Git dotato di \textit{GUI\glo} o meno. 

\subsubsection{Numero di versione}
Per tenere traccia dei progressi si utilizza un sistema di numerazione delle versioni
semantico (\href{https://semver.org/lang/it/}{semver.org}) con un'aggiunta di un TAG.
Il numero di versione è rappresentato nel formato:\\\\

\centerline{\textbf{X.Y.Z-TAG}}


\begin{itemize}
  \item \textbf{Incremento in X}: l'incremento in X avviene quando le modifiche del componente soggetto all'attività, sono molto significative e non retrocompatibili;
  \item \textbf{Incremento in Y}: l'incremento in Y avviene quando c'è un'aggiunta
  di una funzionalità o, in caso di un documento, l'aggiunta di un capitolo. In questo caso l'aggiunta di funzionalità avviene in maniera retrocompatibile;
  \item \textbf{Incremento in Z}: l'incremento in Z avviene per modifiche minori,
  ad esempio \textit{code refactoring\glos}, correzioni di forme verbali o errori di digitazione.
  \item \textbf{Cambio di TAG}: i TAG possono essere:
  \begin{itemize}
  	\item \textbf{TBR (To Be Reviewed):} se dev'essere ancora soggetto a revisione;
  	\item \textbf{TBA (To Be Approved):} se sottoposto a opportuna verifica ma mancante di approvazione.
  \end{itemize}
  In caso di versione approvata il TAG viene rimosso.
\end{itemize}


\subsubsection{Gestione dei repository}
Stando all'\textit{Analisi dei Requisiti 1.0.0}\doc sono stati individuati
tre componenti principali del prodotto software oltre ai documenti.
Il gruppo ha dunque deciso di impostare la struttura dei repository in questo modo:\\\\

\dirtree{%
	.1 /.
	.2 etherless.
	.3 etherless-cli.
	.4 applicationConfigs.
	.3 etherless-server.
	.4 applicationConfigs.
	.3 etherless-smart.
	.4 applicationConfigs.
	.3 docs.
	.2 envConfigs.
}
\vspace{1cm}

\noindent \\Il repository "etherless" contiene il progetto in tutte le sue componenti
ad eccezione di "envConfigs" in quanto necessario solo alla gestione interna
degli strumenti di sviluppo. Questa organizzazione permette di avere repository separate per ogni parte
del prodotto mantenendole legate in una repository contenitore.

\paragraph{Docs}
Il repository Docs contiene tutta la documentazione del prodotto. I sorgenti \LaTeX \space dei documenti contenuti al suo interno dovranno essere organizzati secondo questa struttura:\\\\

\dirtree{%
	.1 /.
	.2 destinazione d'uso.
	.3 nome del documento.
	.4 main.tex.
	.4 config.
	.5 config.tex.
	.4 res.
}
\vspace{1cm}
\noindent In particolare: 
\begin{itemize}
	\item la \textbf{directory del documento} verrà rinominata con il nome del documento di riferimento scritto in lowerCamelCase;
	\item il \textbf{main.tex} conterrà la struttura principale del documento;
	\item il file \textbf{config.tex} conterrà tutte le inclusioni dei package e la ridefinizione di comandi utili alla stesura del documento;
	\item la directory \textbf{res} conterrà tutte le risorse necessarie per la compilazione corretta del documento. Al suo interno verranno poste le immagini e le sezioni che compongono il \textbf{main.tex}.
\end{itemize}
Un esempio di path corretto per un documento è:\\ 

\centerline{\code{docs/interno/normeDiProgetto/main.tex}}
\vspace{0.7cm}


\paragraph{Procedure per la gestione dei repository e del flusso di lavoro}
	
	\begin{itemize}
		\item \textbf{Avvio di una nuova attività: } all’inizio di una nuova attività ogni componente del gruppo
		deve creare un nuovo feature \textit{branch\glos}, partire dal ramo denominato "develop";
		\item \textbf{Richiesta di verifica: } al termine dell'attività prefissata e dei task associati, occorre eseguire una \textit{pull request\glo} sul \textit{branch\glo} di partenza accertandosi che tutti gli strumenti di verifica automatica siano correttamente funzionanti. 
		\item \textbf{Procedura di verifica:} dopo l'avvenuta \textit{pull request\glo} il verificatore dovrà accertarsi che il controllo degli strumenti automatici non abbia prodotto alcun errore e procederà con la verifica effettiva del prodotto. Gli esiti della verifica potranno essere di due tipi:
		\begin{itemize}
			\item \textbf{Accettazione:} viene rimosso il TAG "TBR" aggiunto durante la \textit{pull request\glo} e viene eventualmente aggiunto un TAG "TBA" per segnalare che il prodotto è in attesa di approvazione.
			\item \textbf{Rifiuto:} il verificatore dovrà scrivere nella sezione dedicata della
			\textit{pull request}\glo il motivo del rifiuto delle modifiche.
		\end{itemize} 
		\item \textbf{Approvazione e rilascio:} Alla ricezione di una \textit{pull request\glo} dal \textit{branch\glo} "develop" il responsabile deciderà se approvare o meno le modifiche. In caso di accettazione il responsabile dovrà rimuovere ogni TAG "TBR".
	\end{itemize}


