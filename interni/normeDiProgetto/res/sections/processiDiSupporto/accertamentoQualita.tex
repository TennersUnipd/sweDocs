\subsection{Accertamento della qualit\` a}
\subsubsection{Obiettivi}
La Qualità di un prodotto software indica la capacità di soddisfare le aspettative di tutti gli \textit{stakeholders\glo} coinvolti in un progetto.
La Qualità di un prodotto software dipende dalla maturità dell'organizzazione e dei processi utilizzati da chi lo produce, dall'utilizzo di metriche, strumenti e metodi come supporto.
Questo processo si prefigge di:
\begin{itemize}
	\item garantire la qualità dei processi adottati nel ciclo di vita del prodotto;
	\item ridurre il ciclo di sviluppo rendendolo maggiormente efficiente;
	\item garantire la qualità del prodotto finale;
	\item fornire metriche oggettive che permettano di rilevare il livello della qualità raggiunta;
	\item soddisfare le esigenze del cliente in maniera efficiente in termini di risorse usate e costi;
	\item permettere a tutti gli \textit{stakeholders\glo} di visualizzare in maniera chiara i risultati qualitativi raggiunti.
\end{itemize}


\subsubsection{Descrizione e attività coinvolte}
%Il controllo dei processi e la loro maturità sono essenziali nel processo di accertamento della qualità (Software Quality Assurance - SQA). 
%Perseguire e migliorare la Qualità di un prodotto, ha come principali obiettivi:
%\begin{itemize}
%	\item un aumento della qualità finale del prodotto;
%	\item una riduzione del ciclo di sviluppo;
%	\item efficienza in termini di risorse usate e costi.
%\end{itemize}
%Il raggiungimento di questi scopi è determinato da:
%\begin{itemize}
%	\item una pianificazione corretta degli obiettivi di qualità da perseguire;
%	\item il controllo del livello di Qualità raggiunto in base agli obiettivi prefissati;
%	\item il miglioramento continuo del processo di Qualità.
%\end{itemize}
Le attività appena descritte fanno parte del modello \textit{PDCA\glo} (anche chiamato "Ciclo di Deming"), un approccio concreto utilizzato per il controllo e il miglioramento continuo dei processi produttivi.\\
All'interno del \textit{Piano di Qualifica 1.0.0\doc} verranno stabilite le metriche e gli obiettivi specifici per ciascun processo e per il prodotto software in generale. Ogni metrica è stata stabilita con l'obiettivo di garantire l'oggettività e la quantificabilità della misurazione della qualità.

\subsubsection{Riferimenti}
Per la valutazione della qualità del prodotto e dei processi coinvolti nel progetto, il gruppo ha deciso di istanziare i seguenti standard e modelli:
\begin{itemize}
	\item \textbf{ISO/IEC 9126:} per la qualità del prodotto;
	\item \textbf{CMMI:} per la qualità del prodotto.
\end{itemize}
