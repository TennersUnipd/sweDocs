\section{Verbale della riunione}
	\begin{itemize}
		\item \textbf{Ideazione del PoC da consegnare per la Tecnhology Baseline:} il gruppo ha discusso riguardo lo sviluppo del PoC da consegnare per la Technology Baseline prevista con il docente Riccardo Cardin prima della consegna per la corrispondente revisione di avanzamento. È stato deciso, per tale data, di progettare e implementare 3 tra i comandi messi a disposizione all'utente e analizzati durante il periodo di analisi, che mettessero in luce tutte le tecnologie utilizzate e che potessero essere maggiormente riutilizzabili per lo sviluppo completo del prodotto. Riguardo ciò, sono state prese diverse decisioni tecnologiche e progettuali:
		\begin{itemize}
			\item \textbf{Commander:} utilizzato come modulo JavaScript da utilizzare nella componente \textit{etherless-cli} per gestire i comandi a disposizione dell'utente. Grazie ad esso sarà possibile definire eventuali parametri ed avere una gestione analoga a quella descritta nel \textit{capitolato} dal proponente Red Babel;
			\item \textbf{Approccio asincrono:} è stato visto, approfittando dell'esperienza a riguardo di un membro del gruppo, come utilizzare correttamente le chiamate asincrone, il cui utilizzo viene reso obbligatorio da capitolato;
			\item \textbf{Web3:} il team ha scelto di ricorrere all'utilizzo della libreria Web3 all'interno della componente \textit{etherless-cli} per interfacciarsi con gli \textit{smart-contract\glo} presenti nella rete \textit{Ethereum\glos};
			\item \textbf{Definizione delle funzioni da inserire nello smart-contract:} il gruppo ha scelto quali funzionalità implementare all'interno dello \textit{smart-contract\glo};
			\item \textbf{AWS:} è stata configurata l'utenza AWSEducate fornita gratuitamente in quanto studenti universitari, da utilizzare durante lo sviluppo del PoC e del prodotto in generale.
		\end{itemize}
		
		\item \textbf{Pianificazione dei tempi necessari per lo sviluppo del PoC:} in seguito alla discussione relativa i contenuti del PoC e le tecnologie da porre in risalto, sono stati pianificati i tempi e i compiti da assegnare a ciascun membro del team per il raggiungimento di tale scopo:
		\begin{itemize}
			\item \textbf{Componente etherless-smart:} entro il 2020-02-25;
			\item \textbf{Comunicazione client/server:} entro il 2020-02-28;
			\item \textbf{Parte Serverless:} entro la data stabilita per la Technology Baseline.
		\end{itemize}
		
		\item \textbf{Scelta della data per il colloquio della Technology Baseline:} coerentemente con le disponibilità di ciascun membro del gruppo, con le disponibilità orarie fornite dal committente e con la previsione per la conclusione del PoC da realizzare, sono state stabilite due possibili date per il colloquio:
		\begin{itemize}
			\item 2020-03-04;
			\item 2020-03-03.
		\end{itemize}
		Tali date potranno essere soggette a variazioni in base alle disponibilità del Prof. Cardin.
	\end{itemize}