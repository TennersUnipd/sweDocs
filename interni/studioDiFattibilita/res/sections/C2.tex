\section{Capitolato C2 - Etherless}

\subsection{Descrizione generale}
\textit{Etherless} è una piattaforma che permette agli sviluppatori di caricare funzioni JavaScript sul \textit{cloud\glo} per poi renderle disponibili a terzi. Il servizio si presenta come un CaaS (Computation-as-a-Service) in cui l'utente finale paga per l'esecuzione di una singola funzione in modo automatico mediante l'utilizzo della rete \textit{Ethereum\glo} e degli \textit{smart contract\glos}. 
\subsection{Finalità del progetto}
L'obbiettivo finale è quello di avere un ambiente in cui un ipotetico sviluppatore \textit{Bob}, dopo aver sviluppato una funzionalità che potrebbe essere di interesse per altri sviluppatori (es. \textit{Alice}), carica il suo codice JavaScript su \textit{Etherless} mediante la sua utenza e imposta un costo di esecuzione per quella funzione. \textit{Alice}, avendo bisogno di tale funzionalità, ha la possibilità di pagare una quota di esecuzione piuttosto che riscrivere la stessa procedura; attraverso la sua utenza \textit{Etherless}, \textit{Alice} è quindi capace di usufruire della funzione in \textit{cloud\glo} e sostenere il costo di esecuzione.
\\\\
La piattaforma deve essere conforme a quanto segue:
\begin{itemize}
	\item gli utenti avranno la possibilità elencare le funzioni disponibili,  caricarne di nuove, aggiornarle, eseguirle oppure eliminare attraverso una \textit{CLI\glos};
	\item utilizzo della rete \textit{Ethereum\glo} per la comunicazione tra i vari componenti di \textit{Etherless}, per la definizione della logica di interazione tra le varie parti (\textit{smart contract\glos}) e per lo storage di dati;
	\item il guadagno proveniente dall'esecuzione di una funzione deve essere distribuito tra l'autore della stessa e i proprietari della piattaforma \textit{Ethereum\glos}, per il sostenimento dell'infrastruttura \textit{cloud\glo} su cui avviene l'esecuzione;
	\item la granularità del pagamento deve corrispondere alla singola esecuzione di una funzione;
	\item per la realizzazione del \textit{back-end\glo} deve essere prevista una infrastruttura \textit{serverless\glos}.
\end{itemize}

\subsection{Tecnologie interessate}
\begin{itemize}
	\item \textbf{AWS (Amazon Web Services)}: piattaforma che si occupa di fornire servizi di \textit{cloud computing\glos};
	\item \textbf{AWS - Lambda}: servizio che consente di eseguire codice nel \textit{cloud\glos};
	\item \textbf{Ethereum\glos}: una rete globale per il trasferimento di \textit{criptovalute\glo} e per la realizzazione di applicativi decentralizzati;
	\item \textbf{Solidity}: linguaggio \textit{OOP\glo} per la definizione di \textit{smart contract\glos};
	\item \textbf{Truffle}: framework per lo sviluppo di \textit{smart contract\glo} su rete \textit{Ethereum\glos};
	\item \textbf{Web3}: \textit{API\glo} JavaScript per l'interazione con un nodo \textit{Ethereum\glos}, locale o remoto;
	\item \textbf{Ropsten}: rete \textit{Ethereum\glo} pubblica usato per il testing di applicativi \textit{Ethereum\glo} prima del \textit{porting\glo} in produzione sulla \textit{MainNet\glos};
	\item \textbf{MainNet\glos}: rete \textit{Ethereum\glo} principale;
	\item \textbf{Ganache}: ambiente di sviluppo \textit{Ethereum\glo} utilizzato per la simulazione locale si una rete \textit{Ethereum\glo} e per l'analisi delle transazioni e log;
	\item \textbf{TypeScript 3.6}: versione standardizzata da Microsoft del linguaggio JavaScript;
	\item \textbf{Node.js}: ambiente di runtime \textit{open-source\glo} per JavaScript;
	\item \textbf{The Serverless\glo Framework\glos}: \textit{framework\glo} per la costruzione di ambienti \textit{serverless\glo}
	\item \textbf{Smart Contract\glos}: protocollo informatico che facilita, verifica, fa rispettare ed esegue un contratto (insieme di regole);
	\item \textbf{ESLint}: strumento di analisi del codice utilizzato durante lo sviluppo in JavaScript.
\end{itemize}

\subsection{Aspetti positivi}
\begin{itemize}
	\item Possibilità di apprendere e utilizzare tecnologie nuove, come ad esempio:
		\begin{itemize}
			\item \textbf{Amazon Web Services}, leader nell'industria dei servizi \textit{cloud\glos};
			\item \textbf{Blockchain\glos}, tecnologia in crescita che mette a disposizione strumenti nuovi che permetto la realizzazioni di applicativi con caratteristiche innovative;
			\item \textbf{Node.js}, ambiente di sviluppo che negli ultimi anni ha avuto larga diffusione.
		\end{itemize}
	\item Si entra in contatto con una azienda estera implicata in tecnologie del futuro;
	\item Una volta compresi gli aspetti tecnici, l'implementazione non risulta complicata.
\end{itemize}

\subsection{Criticità}
\begin{itemize}
	\item Seppur stimolante, l'apprendimento e il corretto utilizzo della piattaforma AWS può risultare dispendioso;
	\item Una progettazione iniziale non adeguata del servizio potrebbe invalidare parte dello sviluppo;
	\item Alla base del progetto ci sono tecnologie prima non approfondite.
\end{itemize}

\subsection{Valutazione conclusiva}
La prima impressione percepita dopo la lettura del \textit{capitolato\glo} è stata una poco rassicurante in quanto le tecnologie risultano nuove per tutti i membri del team. In seguito ad una accurata analisi dell'obiettivo principale del servizio \textit{Etherless}, il progetto ha preso una sfumatura accattivante. Usare piattaforma \textit{Ethereum\glo} per realizzare un applicativo decentralizzato ha stimolato l'interesse per la sua diversità dagli ormai tipici standard architetturali per applicativi software.
\\\\
In seguito ad approfondimenti sull'argomento, il team è capace di portare a termine il progetto nei tempi richiesti.