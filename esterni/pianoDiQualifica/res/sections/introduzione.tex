\setlength\LTleft{0cm}
\section{Introduzione}

\subsection{Scopo del documento}
Questo documento descrive le strategie atte a garantire la qualità dei processi coinvolti nel ciclo di vita del prodotto e del prodotto stesso. Per garantire la qualità vengono utilizzate delle apposite metriche. Le metriche adottate, hanno lo scopo di quantificare in maniera oggettiva il processo di verifica e validazione. Il piano di qualifica viene prodotto in maniera incrementale e i cambiamenti saranno tracciati nella sezione del registro delle modifiche.


\subsection{Scopo del prodotto}
Il \textit{Capitolato\glo} C2 ha come obiettivo la realizzazione della piattaforma \textit{Etherless}. Essa ha lo scopo di mettere in comunicazione sviluppatori che intendono condividere funzioni proprie scritte in linguaggio JavaScript con altri utenti che desiderano avere accesso a tali funzionalità. Un utente, corrisposto un pagamento in valuta \textit{Ether\glo} allo sviluppatore e alla piattaforma, avrà la possibilità di eseguire una tra le funzioni messe a disposizione e visualizzarne l'output.

\subsection{Glossario}
Come supporto alla documentazione, viene fornito un \textit{Glossario}\docs,
contenente le definizioni di termini specifici che necessitano di un chiarimento.
Ognuno di questi è contrassegnato con un pedice \glo nel documento e la sua
spiegazione viene riportata sotto la corrispondente lettera del glossario. Ciò
consentir\`a di avere un linguaggio comune ed evitare ambiguità.

\subsection{Rifermenti}
\subsubsection{Rifermenti normativi}
\begin{itemize}
\item \textbf{Capitolato\glo d'appalto C2 - Etherless:} \url{https://www.math.unipd.it/~tullio/IS-1/2019/Progetto/C2.pdf};
\item \textbf{Norme di Progetto:} \textit{Norme di Progetto 2.0.0\docs}.
\end{itemize}
\subsubsection{Rifermenti informativi}
\begin{itemize}
\item \textbf{ISO/IEC 12207\:1995:} \url{https://www.math.unipd.it/~tullio/IS-1/2009/Approfondimenti/ISO_12207-1995.pdf};
\item \textbf{ISO/IEC 9126:} \url{https://en.wikipedia.org/wiki/ISO/IEC_9126};
\item \textbf{Metriche del Software 1.0 - Ottobre 1999:} \url{http://www.colonese.it/00-Manuali_Pubblicatii/08-Metriche\%20del\%20software_v1.0.pdf};
\item \textbf{Qualità Software 2.0 - Dicembre 2005:} \url{http://www.colonese.it/00-Manuali_Pubblicatii/06-Qualit\%C3\%A0Software_v2.pdf};
\item \textbf{Metriche per la pianificazione:} \url{https://it.wikipedia.org/wiki/Metriche_di_progetto};
\item \textbf{Indice Gulpease:} \url{https://it.wikipedia.org/wiki/Indice_Gulpease};
\item \textbf{Indice di Manutenibilità:} \url{https://docs.microsoft.com/it-it/archive/blogs/codeanalysis/maintainability-index-range-and-meaning};
\item \textbf{Qualità di prodotto - slides del corso Ingegneria del Software (2019-2020):} \url{https://www.math.unipd.it/~tullio/IS-1/2019/Dispense/L12.pdf};
\item \textbf{Qualità di processo - slides del corso Ingegneria del Software (2019-2020):} \url{https://www.math.unipd.it/~tullio/IS-1/2019/Dispense/L13.pdf}.
\end{itemize}
