\section{Capitolato C6 - Things Relationship Management}

\subsection{Descrizione generale}
\textit{Sanmarco Informatica} propone lo sviluppo di un software in grado di ricevere misurazioni da sensori eterogenei, divisibili nelle macro-categorie {dati operativi} e {fattori influenzanti}, ed accumularli in uno o piú database. L'applicazione quindi deve poter fornire un servizio di dispatching, basato su Telegram, per inoltre le informazioni piú utili o urgenti per la gestione dei dispositivi installati.

\subsection{Finalità del progetto}
Il prodotto finale deve essere conforme ai seguenti compiti:
\begin{itemize}
	\item la creazione di uno o piú componenti (Producer) atti a convertire i dati inviati dai sensori in messaggi utilizzabili da Kafka (JSON);
	\item la creazione di un componente (Connect) atto alla scrittura dei dati nel database;
	\item la creazione di una serie di componenti (Stream) atti alla modifica e trasformazione dei dati presenti nel database;
	\item dispatching delle informazioni basato su Telegram.
\end{itemize}

\subsection{Tecnologie interessate}
\begin{itemize}
	\item \textbf{Apache Kafka}: cluster che si interfaccia fra i sensori ed il database;
	\item \textbf{Java}: linguaggio di programmazione utilizzato per sviluppare le componenti;
	\item \textbf{PostgreSQL}: possibile database relazione utilizzabile;
	\item \textbf{TimescaleDB}: possibile database improntato alla gestione temporale dei dati;
	\item \textbf{ClickHouse}: database improntato alla gestione dei dati per colonna;
	\item \textbf{Bootstrap}: libreria per la creazione della parte front-end;
	\item \textbf{Docker}: container nel quale istanziare le componenti custom di Apache Kafka;
	\item \textbf{Github}: sistema di versionamento;
	\item \textbf{Telegram}: servizio tramite il quale effetturare il dispatching.
\end{itemize}

\subsection{Aspetti positivi}
\begin{itemize}
	\item Stimolante la possibilità di lavorare con cosí tante tecnologie in modo da approfondirne la sinergia.
\end{itemize}

\subsection{Criticità}
\begin{itemize}
	\item Grande complessitá di esecuzione dovuta alla mole di applicativi differenti con il quale lavorare.
\end{itemize}

\subsection{Valutazione conclusiva}
Nonostante sia stato preso in considerazione per il gran numero di competenze che si sarebbero potute affinare o sviluppare, il gruppo ha deciso di focalizzare la proprio attenzione su di un altro capitolato\glos, in quanto lo sforzo per imparare cosí tante tecnologie assieme e la possibilitá che il lavoro finale non sarebbe stato all'altezza delle aspettative, lo hanno reso meno appetibile.