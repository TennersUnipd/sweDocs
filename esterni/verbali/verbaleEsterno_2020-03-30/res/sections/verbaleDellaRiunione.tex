\section{Verbale della riunione}
	\begin{itemize}
		\item \textbf{Esito della Revisione di Progettazione}: il gruppo ha discusso con il committente i punti critici riguardanti l'esito della RP. In particolare, gli argomenti trattati sono stati:
		\begin{itemize}
			\item \textbf{Versionamento del prodotto e dei sotto-prodotti:} il team ha capito, dopo discussione e analisi con il committente, la necessità di avere un collegamento tra l'avanzamento di versione legato ai sotto prodotti (come ad esempio i documenti) e la progressione del prodotto nella sua totalità. Un corretto tracciamento può essere ottenuto ponendo, al termine di ciascun registro delle modifiche relativo al singolo documento, un allineamento verso la versione del prodotto che si sta rilasciando;
			\item \textbf{Definizione degli incrementi nel PdP:} il Prof. Vardanega ha chiarito al gruppo come la specifica degli incrementi prevista debba essere posta nella sez. 3 del relativo documento (altrimenti priva di significato) mentre, indicazioni relative alla pianificazione (tempistiche e risorse utilizzate) possono essere poste in sez. 4.
		\end{itemize}
		\item \textbf{Tracciamento decisioni verbali}: il committente ha segnalato che le decisioni prese a seguito di una discussione avvenuta tramite software di messaggistica dovrebbero essere, sotto conferma del proponente, tracciate in appositi verbali in maniera del tutto analoga a qualsiasi altra forma di comunicazione orale. 
	\end{itemize}
