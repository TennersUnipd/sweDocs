\section{Introduzione}

\subsection{Scopo del Documento}
Questo atto si propone di essere una raccolta delle linee di condotta per lo
sviluppo di software e documentazione a cui tutti i componenti del team Tenners
dovranno attenersi.
Questo documento sar\`a aggiornato in maniera incrementale in quanto i processi
di produzione sono in costante miglioramento.

\subsection{Scopo del prodotto}
Il capitolato C6 ha come obbiettivo una piattaforma per svilippatori che metta
in comunicazione un fornitore di funzionalit\`a e chi \`e interessato ad utilizzarle
all'interno del proprio progetto.
La piattaforma prender\`a nome di etherless permettendo a chi sviluppa software e
lo condivide sulla piattaforma di essere pagato, per ogni esecuzione, attraverso
la criptovaluta etherium.


\subsection{Glossario}
Come supporto alla documentazione, viene fornito un \textit{Glossario v.1.0.0},
contenente le definizioni di termini specifici che necessitano di un chiarimento.
Ognuno di questi verr\`a contrassegnato con un pedice \glo nel documento e la sua
spiegazione verr\`a riportata sotto la corrispondente lettera del glossario. Ci\`o
consentir\`a di avere un linguaggio comune ed evitare ambiguit\`a.

\subsection{Rifermenti}
\subsubsection{Rifermenti informativi}
\begin{enumerate}
  \item Studio di fattibili\`a in particola la sezione legata al capitolato C6.
\end{enumerate}
