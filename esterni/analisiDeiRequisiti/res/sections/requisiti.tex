\section{Requisiti}
I requisiti del sistema \textit{Etherless} sono classificabili secondo due macro categorie:
\begin{itemize}
	\item \textbf{Requisiti funzionali:} riguardano i servizi e funzionalità che il sistema deve offrire all'utente in modo che vengano soddisfatti tutti i bisogni, impliciti ed espliciti;
	\item \textbf{Requisiti non funzionali:} si riferiscono ai vincoli sui servizi che il sistema offre. Essi non sono inerenti alle funzionalità specifiche del prodotto bensì a come esse devono essere realizzate.
\end{itemize}

\subsection{Requisiti funzionali}

\renewcommand{\arraystretch}{2.2}

\rowcolors{2}{pari}{dispari}
\begin{longtable}{|c|p{8cm}|c|}
	
	\caption{Tabella riassuntiva dei requisiti funzionali}\\
	\hline
	\rowcolor{header}
	
	\textbf{Requisito} & \centering{\textbf{Descrizione}} & \textbf{Fonte}\\
	
	\hline
	
	R2V1 & Un utente ha la possibilità di visualizzare una guida introduttiva contenente la lista dei comandi eseguibili e il loro funzionamento. & Capitolato \\
	
	R1V2 & Un utente ha la possibilità di autenticarsi a \textit{Ethereum} mediante un utenza \textit{Ethereum\glos}. & Capitolato \\
	
	R1V2.1 & L'utente si può autenticare mediante login manuale.  & Capitolato \\
	
	R2V2.2 & L'utente si può autenticare mediante login automatico. & Capitolato \\
	
	R1V2.1.1 &  L'utente viene autenticato al sistema mediante l'inserimento di una coppia di address e private key\glo che identificano un utente univocamente all'interno della rete \textit{Ethereum\glos}. & Capitolato \\
	
	R2V2.2.1 &  L'utente viene autenticato automaticamente al sistema mediante un file salvato sul suo dispositivo durante la prima autenticazione o registrazione mediante \textit{Etherless}.  & Capitolato \\
	
	R1V3 &  L'utente ha la possibilità di registrarsi su \textit{Etherless}  & Capitolato \\

	R1V3.1 &  La registrazione su \textit{Etherless} avverrà richiedendo una coppia univoca di address - \textit{private key\glo} alla rete \textit{Ethereum\glos}.  & Capitolato \\	
	
	R2V4 &  Ogni qualvolta l'utente effettuerà un accesso manuale o la registrazione al sistema, verrà salvato un file contenente le credenziali di accesso per permettere l'autenticazione automatica per le successive esecuzioni della applicazione.  & Capitolato \\
	
	R2V5 &  Un utente autenticato può eseguire il logout richiedendo al sistema di eliminare il file di autenticazione salvato sulla propria macchina.  & Capitolato \\
	
	R1V6 &  Un utente autenticato può ricercare una funzione caricata da un altro utente sviluppatore sull \textit{Etherless-server}.  & Capitolato \\
	
	R1V6.1 &  La ricerca della funzione verrà eseguita inserendo il comando seguito dal nome della funzione stessa.  & Capitolato \\
	
	R1V6.2 &  La ricerca della funzione permetterà di visualizzare sull'\textit{Etherless-cli} il nome della funzione, i parametri, una descrizione riassuntiva delle sue funzionalità, il costo e l'autore.  & Capitolato \\
	
	R1V6.3 &  Verrà segnalato un errore mediante \textit{Etherless-cli} se il nome della funzione inserito non fosse disponibile su \textit{Etherless-server}.  & Capitolato \\
	
	R1V7 &  Un utente autenticato può eseguire una funzione caricata da un altro utente sviluppatore sull \textit{Etherless-server}.  & Capitolato \\
	
	R1V7.1 &  L'esecuzione della funzione deve avvenire inserendo il comando apposito seguito dal nome e dal valore dei parametri.  & Capitolato \\

	R1V7.2 &  Il sistema deve restituire un messaggio di errore nel caso in cui il nome della funzione, il numero dei parametri, il tipo dei parametri fossero errati.  & Capitolato \\
	
	R1V7.3 &  L'esecuzione di una funzione può avvenire previo pagamento al suo sviluppatore e alla piattaforma \textit{Etherless}. & Capitolato \\
	
	R1V7.3.1 &  Il pagamento dovrà avvenire con valuta \textit{Ether\glos}. & Capitolato \\
	
	R1V7.3.2 &  La transazione avverrà in maniera diretta. & Capitolato \\
	
	R3V7.3.3 &  La transazione potrà avvenire tramite \textit{escrow\glos}. & Capitolato \\
	
	R1V7.3.4 &  Il sistema deve restituire un messaggio di errore all'utente nel caso non avesse fondi  \textit{Ether\glos} sufficienti per coprire il costo dell'esecuzione. & Capitolato \\
	
	R1V7.3.5 &  Il sistema deve restituire un messaggio di errore all'utente nel caso lo sviluppatore della funzione non dovesse averne ancora associato un costo. & Capitolato \\
	
	R1V8 &  L'utente può richiedere un log di tutte le funzioni eseguite. & Capitolato \\	
	
	R1V8.1 &  Il sistema dovrà restituire un log delle funzioni eseguite composto da: data e ora di esecuzione, nome, parametri e costo della funzione. & Capitolato \\
	\hline
\end{longtable}
\newpage
\subsection{Requisiti non funzionali}
\subsubsection{Requisiti di qualità}
\renewcommand{\arraystretch}{2.2}
  
  \rowcolors{2}{pari}{dispari}
  \begin{longtable}{|c|p{8cm}|c|}
  	\arrayrulecolor{white}
  	
  	\caption{Tabella riassuntiva dei requisiti di vincolo}\\
  	
    \rowcolor{header}
    
    \textbf{Requisito} & \centering{\textbf{Descrizione}} & \textbf{Fonte}\\
    
    \endfirsthead
    
 	
 	R1Q1 & Dovrà essere utilizzato ESLint come software per l'analisi statica del codice. & Capitolato \\
 	
 	R1Q2 & Devono essere rispettate le regole descritte nelle \textit{Norme di Progetto 2.4.0\docs}. & Capitolato \\
   
   	R1Q3 & Deve essere rispetto ciò che è descritto nel \textit{Piano di Qualifica 2.1.1\docs}. & Capitolato \\
   	
   	R1Q4 & Il codice sorgente di \textit{Etherless} deve essere versionato e pubblicato mediante sistemi di versionamento quali \textit{Github\glo} o Gitlab. & Capitolato \\
   	
   	R1Q5 & Il sistema deve essere distribuito con la licenza \textit{MIT}\glos. & Capitolato \\
   	
   	R1Q6 & Il manuale d'uso contente la guida per l'utilizzo di \textit{Etherless} deve essere scritto in lingua inglese. & Capitolato \\
   	
    \hline
  \end{longtable}
\newpage
\subsubsection{Requisiti di vincolo}
\renewcommand{\arraystretch}{2.2}
  
  \rowcolors{2}{pari}{dispari}
  \begin{longtable}{|c|p{8cm}|c|}
  	\arrayrulecolor{white}
  	
  	\caption{Tabella riassuntiva dei requisiti di vincolo}\\
  	
    \rowcolor{header}
    
    \textbf{Requisito} & \centering{\textbf{Descrizione}} & \textbf{Fonte}\\
    
    \endhead
    
    R1V1 & \textit{Etherless} dev'essere sviluppato utilizzando il linguaggio Typescript 3.6. & Capitolato \\

    R1V2 & \textit{Etherless-server} dev'essere implementato utilizzando il framework Serverless. & Capitolato \\
    
    R1V3 & \textit{Etherless} deve utilizzare le tecnologie AWS. & Capitolato \\

    R1V3.1 & Per realizzare l'infrastruttura \textit{Serverless\glo} è richiesto l'utilizzo di AWS Lambda.  & Capitolato \\
    
    R1V4 & Gli \textit{smart-contract\glo} devono essere sviluppati utilizzando il linguaggio Solidity. & Capitolato \\
    
    R1V5 & Gli \textit{smart-contract\glo} devono essere sviluppati utilizzando il framework Truffle. & Capitolato \\
    
    R1V6 & Il progetto deve essere realizzato utilizzando alcuni ambienti specifici. & Capitolato \\
    
    R1V6.1 & Devono essere utilizzati gli ambienti di sviluppo locale, di testing e di staging. & Capitolato \\
    
    R2V6.2 & Il prodotto software può essere testato in ambiente di produzione. & Capitolato \\
    
    R1V6.1.1 & Per gli ambienti di sviluppo locale, testing e staging è desiderabile l'utilizzo di: \textit{Ethereum\glo} testrpc network e un web server locale per lo sviluppo locale e in ambiente di testing; l'\textit{Ethereum\glo} network Ropsten per lo staging. & Capitolato \\
    
    R1V6.2.1 & Per l'ambiente di produzione è necessario l'utilizzo della \textit{MainNet\glo} di \textit{Ethereum\glo} e della valuta \textit{Ether\glos}. & Capitolato \\
    
    R1V7 & Il codice sorgente di \textit{Etherless} deve essere versionato e pubblicato mediante sistemi di versionamento quali Github o Gitlab. & Capitolato \\
    
    R1V8 & Il sistema deve essere distribuito con la licenza MIT\glos. & Capitolato \\
    
    R1V9 & Il proponente Red Babel deve essere menzionato tra i crediti all'interno del file README.md. & Capitolato \\
    
    R1V10 & Il documento contente la guida per l'utilizzo di \textit{Etherless} deve essere scritto in lingua inglese & verbaleEsterno \\
    \hline
  \end{longtable}

\newpage
\subsection{Tracciamento}
\subsubsection{Fonte - Requisiti}

\renewcommand{\arraystretch}{2.2}

\rowcolors{2}{pari}{dispari}
\begin{longtable}{|C{6.5cm}|C{6.5cm}|}
	\arrayrulecolor{white}
	
	\caption{Tabella per il tracciamento fonte-requisiti}\\
	\hline
	\rowcolor{header}
	
	\textbf{Fonte} & \textbf{Requisito} 
	\tabularnewline
	\endhead
	
	Capitolato &  
	\centering
	R1F2\\
	R1F3 \\
	R1F3.1\\
	R1F7\\
	R1F7.1\\
	R1F7.2\\
	R1F7.2.1\\
	R3F7.2.3\\
	R1F9\\
	R1F10\\
	R1F10.1\\
	R1F12\\
	R1F12.1\\
	R1Q1\\
	R1Q2\\
	R1Q3\\
	R1Q4\\
	R1Q5\\
	R1V1\\
	R1V2\\
	R1V3\\
	R1V3.1\\
	R1V4\\
	R1V5\\
	R1V6\\
	R1V6.1\\
	R2V6.2\\
	R1V6.1.1\\
	R1V6.2.1\\
	R1V7\\
	R1V8\\
	R1V9\\
	
	\tabularnewline
	
	Interna & 
	\centering
	R1F1\\
	R1F2.1 \\
	R2F2.2\\
	R1F2.1.1\\
	R1F2.1.1.1\\
	R1F2.1.1.2\\
	R2F2.2.1\\
	R2F4\\
	R2F5\\
	R2F5.1\\
	R1F6\\
	R1F6.1\\
	R1F6.2\\
	R1F6.3\\
	R1F7.1.1\\
	R1F7.1.2\\
	R1F7.2.2\\
	R1F7.2.4\\
	R1F7.2.5\\
	R2F8\\
	R2F8.1\\
	R1F9.1\\
	R1F10.1.1\\
	R1F10.1.2\\
	R1F10.1.3\\
	R1F11\\
	R1F11.1\\
	R3F11.2\\
	R1F11.3\\
	R1F11.1.1\\
	R3F11.2.1\\
	R1F11.5\\
	R1F13\\
	R1F13.1\\
	R1F13.1.1\\
	R1F13.1.2\\
	R1F14\\
	R1F14.1\\
	R1F14.1.1\\

	
\end{longtable}
\newpage
\subsubsection{Tracciamento requisiti - fonti}

\renewcommand{\arraystretch}{2.2}

\rowcolors{2}{pari}{dispari}
\begin{longtable}{|C{6.5cm}|C{6.5cm}|}
	\arrayrulecolor{white}
	\caption{Tabella per il tracciamento requisiti-fonti}\\
	\rowcolor{header}
	\textbf{Requisito} & \textbf{Fonti}
	\endfirsthead

	\hline
	\rowcolor{header}

	\textbf{Requisito} & \textbf{Fonti}
	\tabularnewline
	\endhead

	R1F1 & \centering Interna \\ UC1 \tabularnewline

	R1F2 & Capitolato \\

	R1F2.1 & \centering Interna \\ UC5 \tabularnewline

	R2F2.2 & \centering Interna \\ UC6 \tabularnewline

	%R1F2.1.1 & \centering Interna \\ UC5.1 \tabularnewline

	R1F2.1.1 &   \centering Interna \\ UC5.1 \tabularnewline

	R1F2.1.2 &  \centering Interna \\ UC5.2 \tabularnewline

	R1F2.1.3 &  \centering Interna \\ UC22 \tabularnewline
	
	R1F2.1.4 &   \centering Interna \\ UC5 \tabularnewline

	R2F2.2.1 &  \centering Interna \\ UC6 \tabularnewline

	R1F3 & Capitolato \\

	R1F3.1 & Capitolato \\

	R1F3.1.1 &  \centering Interna \\ UC4.1 \tabularnewline

	R1F3.1.2 &  \centering Interna \\ UC4.2 \tabularnewline

	R2F4 &  Interna \\

	R2F5 &  \centering Interna \\ UC8 \tabularnewline

	R2F5.1 &  \centering Interna \\ UC8 \tabularnewline

	R1F6 & Interna \\

	R1F6.1 & \centering Interna \\ UC9.1 \tabularnewline

	R1F6.2 & \centering Interna \\ UC9.2 \tabularnewline

	R1F6.2.1 & \centering Interna \\ UC9.2.2 \tabularnewline

	R2F6.2.2 & \centering Interna \\ UC9.2.3 \tabularnewline

	R2F6.2.3 & \centering Interna \\ UC9.2.1 \tabularnewline

	R1F6.2.4 & \centering Interna \\ UC9.2.4 \tabularnewline

	R1F6.3 & \centering Interna \\ UC9.3 \tabularnewline

	R1F7 & Capitolato \\

	R1F7.1 & \centering Interna \\ UC10.1 \tabularnewline

	R1F7.2 &  \centering Interna \\ UC10.2 \tabularnewline

	R1F7.1.1 & \centering Interna \\ UC10.3 \tabularnewline

	R1F7.2.1 &   \centering Interna \\ UC10.4 \tabularnewline
	
	R1F7.3 & \centering Interna \\ UC10.5 \tabularnewline

	R1F7.4 &   Capitolato \\

	R1F7.4.1 &  Capitolato \\

	R1F7.4.2 &  Interna \\

	R3F7.4.3 &  Capitolato \\

	R1F7.4.4 &  \centering Interna \\ UC13 \tabularnewline

	%R1F7.4.5 & \centering Interna \\ UC12 \tabularnewline

	R2F8.1 & \centering Interna \\ UC14 \tabularnewline

	R2F8.1 & \centering Interna \\ UC14.2 \tabularnewline

	R2F8.1.1 & \centering Interna \\ UC14.2.1.1 \tabularnewline

	R2F8.1.2 & \centering Interna \\ UC14.2.1.2 \tabularnewline

	R2F8.1.3 & \centering Interna \\ UC14.2.1.3 \tabularnewline

	%R2F8.1.4 & \centering Interna \\ UC14.2.1.4 \tabularnewline

	R2F8.2 & \centering Interna \\ UC14.3 \tabularnewline

	R1F9 &  Capitolato \\

	R1F9.1 & \centering Interna \\ UC15.2 \tabularnewline

	R1F9.1.1 & \centering Interna \\ UC15.2.1.1 \tabularnewline

	R1F9.1.2 & \centering Interna \\ UC15.2.1.2 \tabularnewline

	R1F9.1.3 & \centering Interna \\ UC15.2.1.3 \tabularnewline

	R2F9.1.4 & \centering Interna \\ UC15.2.1.4 \tabularnewline

	R1F10 & Capitolato \\

	R1F10.1 &  \centering Interna \\ UC17.1 \tabularnewline

	R1F10.2 & \centering Interna \\ UC17.3 \tabularnewline

	R1F10.1.1 &  \centering Interna \\ UC17.2 \tabularnewline

	R1F10.2.1 &  \centering Interna \\ UC17.4 \tabularnewline

	R1F10.3 &  \centering Interna \\ UC17 \tabularnewline

	R1F11 & \centering Interna \\ UC18 \tabularnewline

	R1F11.1 &  \centering Interna \\ UC18.1 \tabularnewline

	R3F11.2 & \centering Interna \\ UC18.3 \tabularnewline

	R1F11.3 & \centering Interna \\ UC18.5 \tabularnewline

	R1F11.4 & \centering Interna \\ UC18 \tabularnewline

	R1F11.1.1 & \centering Interna \\ UC18.2 \tabularnewline

	R3F11.2.1 & \centering Interna \\ UC18.4 \tabularnewline

	R1F12 &   Capitolato \\

	R1F12.1 & \centering Interna \\ UC19 \tabularnewline

	R1F13 & \centering Interna \\ UC20 \tabularnewline

	R1F13.1 & \centering Interna \\ UC20.1 \tabularnewline

	R1F13.1.1 & \centering Interna \\ UC20.2 \tabularnewline

	R1F13.1.2 & \centering Interna \\ UC20 \tabularnewline

	R1F14 & \centering Interna \\ UC21 \tabularnewline

	R1F15 & Capitolato \\

	R1Q1 & Capitolato \\

	R1Q2 & Capitolato \\

	R1Q3 & Capitolato \\

	R1Q4 & Capitolato \\

	R1Q5 & Capitolato \\

	R1Q6 & Capitolato \\

	R1V1 & Capitolato \\

	R1V2 & Capitolato \\

	R1V3 & Capitolato \\

	R1V3.1 & Capitolato \\

	R1V4 & Capitolato \\

	R1V5 & Capitolato \\

	R1V6 & Capitolato \\

	R1V6.1 & Capitolato \\

	R1V6.1.1 & Capitolato \\

	R1V6.2 & Capitolato \\

	R1V7 & Capitolato \\
	R1V8 & Capitolato \\

	\hline

\end{longtable}


