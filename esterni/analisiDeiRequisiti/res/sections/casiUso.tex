\section{Casi d'uso}

\subsection{Attori}
In un diagramma dei casi d'uso, gli attori rappresentano le entità esterne che interagiscono con il prodotto \textit{Etherless}. Essi si possono distinguere in due categorie:
\begin{itemize}
	\item \textbf{Attori primari:} coinvolti nell'esecuzione dei casi d'uso, interagiscono con il servizio per soddisfare i propri bisogni;
	\item \textbf{Attori secondari:} forniscono servizio o supporto al sistema. 
\end{itemize}
Per l'applicativo non è stato individuato alcun attore secondario, verranno dunque riportati in seguito i soli attori primari.

\subsubsection{Attori primari}
\begin{figure}[h]
	\centering
	\includegraphics[width=9cm]{res/img/gerarchiaAttoriPrimari.jpg}
	\caption{Gerarchia attori primari}
\end{figure}

Sono state identificate quattro diverse tipologie di attori primari relazionati tra loro in maniera gerarchica:

\begin{itemize}
	\item \textbf{Utente generico:} utente che ha eseguito il comando per l'avvio dell'applicativo tramite \textit{Etherlesss-cli};
	\item \textbf{Utente non autenticato:} utente che non ha ancora eseguito l'accesso o la registrazione al network \textit{Ethereum\glo} e che dunque non potrà usufruire delle funzionalità dell'applicazione;
	\item \textbf{Utente autenticato:} utente che ha eseguito l'accesso al network \textit{Ethereum\glo} e che potrà eseguire i comandi messi a disposizione del servizio per gli utenti utilizzatori;
	\item \textbf{Utente sviluppatore:} utente che ha la possibilità di eseguire il deploy di funzioni Javascipt proprie, oltre che eseguire le altre funzioni messe a disposizione dagli altri utenti del servizio.
\end{itemize}

\newpage

\subsection{Elenco dei casi d'uso}

\begin{figure}[h]
	\centering
	\includegraphics[width=12.3cm]{res/img/useCaseDiagram.jpg}
	\caption{Casi d'uso principali}
\end{figure}
\newpage
\subsubsection{UC1 - Guida introduttiva}
\begin{itemize}
	\item \textbf{Attori primari:} Utente generico;
	\item \textbf{Descrizione:} l'utente generico, appena entrato nell'applicazione, visualizza una guida dei comandi utilizzabili, mediante il comando "init"; 
	\item \textbf{Pre-condizioni:} il sistema è raggiungibile e l'applicazione è stata avviata;
	\item \textbf{Post-condizioni:} nella \textit{CLI\glo} vengono visualizzati i comandi utilizzabili dall'utente ed una loro descrizione;
	\item \textbf{Scenario principale:} l'utente, mediante il comando "init", visualizza la guida introduttiva.
\end{itemize}
\subsubsection{UC2 - Login}
\begin{itemize}
	\item \textbf{Attori primari:} Utente non autenticato
	\item \textbf{Descrizione:} l'utente ha la possibilità di autenticarsi al network \textit{Ethereum\glo} mediante l'inserimento dell'address e di una private key\glos; 
	\item \textbf{Pre-condizioni:} l'utente ha visualizzato la guida introduttiva e vuole eseguire l'accesso al network \textit{Ethereum} tramite l'apposito comando;
	\item \textbf{Post-condizioni:} il sistema avrà autenticato o meno l'utente a seconda dei valori di accesso forniti;
\end{itemize}
	
\subsection{Tracciamento attori - casi d'uso}	