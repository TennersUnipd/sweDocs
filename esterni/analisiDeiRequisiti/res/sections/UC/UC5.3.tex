\subsubsection{UC5.3 - Inserimento password}
\begin{itemize}
	\item \textbf{Attori primari:} Utente non autenticato;
	\item \textbf{Descrizione:} l'utente potrà inserire tramite \textit{CLI\glo} la password per criptare in locale la coppia address-\textit{private key\glos}; 
	\item \textbf{Pre-condizioni:} l'utente ha inserito il comando "login";
	\item \textbf{Post-condizioni:} l'utente ha inserito, a seguito dela propria \textit{private-key\glo}, una password a sua scelta;
	\item \textbf{Scenario principale:} l'utente compila il comando per la richiesta di autenticazione inserendo una password a scelta per criptare i dati in locale.
\end{itemize}
%\subsubsection{UC5.3 -Visualizzazione errore "Credenziali inserite non valide"}
%\begin{itemize}
%	\item \textbf{Attori primari:} Utente non autenticato;
%    \item \textbf{Attori secondari:} \textit{Ethereum\glo} Network;
%	\item \textbf{Descrizione:} il sistema, a seguito dell'inserimento delle credenziali per l'accesso ad \textit{Ethereum\glo} da parte dell'utente, restituisce un errore per il fallimento dell'autenticazione;
%	\item \textbf{Pre-condizioni:} l'utente invia il comando "login" seguito da address e \textit{private key\glos};
%	\item \textbf{Post-condizioni:} il sistema restituisce un errore sulla \textit{CLI\glo} in seguito al fallimento dell'autenticazione;
%	\item \textbf{Scenario principale:} Il sistema notifica all'utente un errore avvenuto nella autenticazione al sistema tramite i valori da lui inseriti non corrispondenti ad alcuna utenza esistente. L'utente non sarà dunque autenticato al sistema e non potrà eseguire le funzionalità messe a disposizione da \textit{Etherless}.
%\end{itemize}
