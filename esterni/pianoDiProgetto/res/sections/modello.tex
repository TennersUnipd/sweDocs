\section{Modello di sviluppo}
Per lo svolgimento del progetto \textit{Etherless} è stato scelto l'utilizzo di un modello incrementale.
\subsection{Modello incrementale}
Il modello incrementale consiste nella realizzazione del prodotto finale per incrementi, ognuno dei quali introduce una funzionalità. Questo modello è particolarmente adatto per rilasci multipli in cui, con l'aggiunta progressiva di nuove funzionalità, si approssima sempre più il prodotto parziale a quello finale.\\\\
 I requisiti sono da trattare in ordine di importanza con l'aggiunta delle funzionalità che necessitano di essere realizzate con più urgenza. Questo approccio di tipo adattivo, differentemente da quello predittivo adottato nei modelli sequenziali, consente una maggiore flessibilità e si adatta maggiormente a situazioni in cui i requisiti possono non essere chiari e ben definiti da qualcuno degli \textit{stakeholders\glos}. \\\\
  L'aggiunta, modifica e cancellazione di requisiti sono consentite previa discussione con il proponente e sua approvazione. \\\\
   La specifica riguardante gli incrementi attuati per i diversi periodi del progetto software vengono descritti nella successiva sezione.
\subsubsection{Vantaggi}
\begin{itemize}
	\item Le funzionalità principali vengono subito sviluppate e dunque il committente può rilasciare un feedback velocemente;
	\item Ogni rilascio risulta utile per il miglioramento dei successivi;
	\item Sviluppare per incrementi limita l'introduzione di errori;
	\item Le modifiche, l'individuazione e la correzione degli errori sono più economiche;
	\item Le funzionalità introdotte inizialmente subiranno più volte il processo di verifica poiché eseguito per ogni ciclo.
\end{itemize}
\subsubsection{Svantaggi}
\begin{itemize}
	\item Un approccio predittivo, differentemente da quello adattivo attuato dal modello incrementale, consentirebbe una stima più precisa di tempi e costi;
	\item Necessita che l'intero sistema sia preventivamente definito e scomposto in modo tale da poterlo costruire incrementalmente.
\end{itemize}
\subsection{Incrementi previsti}
Di seguito tutti gli incrementi previsti. Ricordiamo che ogni singolo incremento e sotto-incremento viene verificato. Inoltre ogni incremento viene misurato e i risultati inseriti nel relativo cruscotto del \textit{Piano di qualifica}\docs.
\subsubsection{Progettazione architetturale}
\begin{itemize}
  \item \textbf{\RomanNumeralCaps{1} Incremento:} correzione e integrazione documenti in base alle conoscenze acquisite durante l'\textit{Analisi dei requisiti}\docs;
  \item \textbf{\RomanNumeralCaps{2} Incremento:} realizzazione del primo \textit{Proof Of Concept}\glo (\textit{smart-contract}\glos);
  \item \textbf{\RomanNumeralCaps{3} Incremento:} realizzazione del secondo \textit{Proof Of Concept}\glo (comunicazione tra client e server);
  \item \textbf{\RomanNumeralCaps{4} Incremento:} realizzazione del terzo \textit{Proof Of Concept}\glo (\textit{serverless}\glos);
  \item \textbf{\RomanNumeralCaps{5} Incremento:} aggiornamento dei documenti;
  \item \textbf{\RomanNumeralCaps{6} Incremento:} presentazione RP.
\end{itemize}
\subsubsection{Progettazione di dettaglio e codifica}
\begin{itemize}
	\item \textbf{\RomanNumeralCaps{1} Incremento:} correzione e integrazione documenti in base alle conoscenze acquisite durante la progettazione architetturale. Nel seguente ordine:
  \begin{itemize}
    \item \textit{Norme di Progetto}\docs;
    \item \textit{Analisi dei requisiti}\docs;
    \item \textit{Piano di qualifica}\docs;
    \item \textit{Piano di progetto}\docs;
    \item \textit{Glossario}\docs.
  \end{itemize}
	\item \textbf{\RomanNumeralCaps{2} Incremento:} progettazione del sistema in vista della \textit{Product Baseline}\glo e stesura dell'allegato tecnico comprensivo dei diagrammi adottati;
	\item \textbf{\RomanNumeralCaps{3} Incremento:} implementazione delle funzionalità della componente etherless-cli.
  Vengono implementati in particolare:
  \begin{itemize}
    \item login/logout: comando che permette di autenticarsi/disconnettersi dalla rete \textit{ethereum}\glos;
    \item registrazione: comando che permette di registrarsi sulla rete \textit{ethereum}\glo e ottenere la propria private key;
    \item list: comando che permette di visualizzare la lista delle funzioni disponibili su Etherless e le informazioni riguardanti ciascuna di esse;
    \item run: comando che permette di eseguire una delle funzioni disponibili;
    \item create: comando che permette all'utente di creare una funzione in JavaScript e caricarla nel sistema etherless;
    \item log:comando che permette all'utente di visualizzare la lista delle funzioni da lui eseguite;
    \item find: comando che permette di ricercare per nome i dettagli di una funzione desiderata.
  \end{itemize}
	\item \textbf{\RomanNumeralCaps{4} Incremento:} sviluppo componenti etherless-smart e etherless-server, comprensivi di test di unità e di integrazione;
  \begin{itemize}
    \item Codifica etherless-smart;
    \item Codifica etherless-server;
    \item Relativi test di unità;
    \item Test di integrazione.
  \end{itemize}
	\item \textbf{\RomanNumeralCaps{5} Incremento:} aggiornamento documenti e consuntivo di periodo;
  \begin{itemize}
    \item Redazione manuale utente;
    \item Valutazione dei rischi nel consuntivo di periodo del \textit{Piano di progetto}\docs.
  \end{itemize}
	\item \textbf{\RomanNumeralCaps{6} Incremento:} presentazione RQ.
\end{itemize}
\subsubsection{Validazione e collaudo }
\begin{itemize}
  \item \textbf{\RomanNumeralCaps{1} Incremento:} correzione e integrazione documenti in base alle conoscenze acquisite durante la progettazione di dettaglio e codifica;
  \begin{itemize}
    \item \textit{Norme di Progetto}\docs;
    \item \textit{Product Baseline}\glo e allegato tecnico;
    \item \textit{Analisi dei requisiti}\docs;
    \item \textit{Piano di qualifica}\docs;
    \item \textit{Piano di progetto}\docs;
    \item \textit{Glossario}\docs.
  \end{itemize}
	\item \textbf{\RomanNumeralCaps{2} Incremento:} implementazione dei test di sistema;
	\item \textbf{\RomanNumeralCaps{3} Incremento:} redazione della documentazione relativa al codice;
	\item \textbf{\RomanNumeralCaps{4} Incremento:} realizzazione dei test di accettazione;
	\item \textbf{\RomanNumeralCaps{5} Incremento:} aggiornamento dei documenti da presentare in ingresso alla revisione di accettazione;
	\item \textbf{\RomanNumeralCaps{6} Incremento:} presentazione RA.
\end{itemize}
