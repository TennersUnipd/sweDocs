\subsubsection{Linguaggi}
Per la stesura dei documenti il gruppo ha deciso di utilizzare \LaTeX \space in quanto permette una semplice suddivisione in sezioni di un documento e risulta facile da utilizzare in collaborazione. Oltre a ciò, fornisce la generazione automatica degli indici dei contenuti, delle immagini e delle tabelle a seguito della compilazione del file.tex. Questo linguaggio ha anche il vantaggio di avere un numero cospicuo di \textit{plug-in}\glo che ne permettono una facile estensione delle funzionalità.

\subsubsection{Strumenti}
\paragraph{TexStudio}
Software \textit{multipiattaforma\glo} che include un correttore ortografico interattivo.

\paragraph{Script in Python per la generazione del Glossario 1.0.0\docs}
Per automatizzare l'inserimento dei termini di glossario, è stato sviluppato uno script che, a partire da un file con estensione .csv, genererà il documento corrispondente.

\paragraph{Spell checker}
Pr facilitare e velocizzare le operazioni di verifica e validazione dei documenti è
stato scritto, ed inserito in un \textit{Docker\glo} \textit{container\glo} apposito, uno script per il controllo degli gli errori di digitazione e per il controllo degli errori ortografici.

\paragraph{Compilatore \LaTeX}
Si è scelto l'uso di un compilatore in ambiente\textit{ Docker\glo} (\url{https://GitHub.com/dante-ev/docker-texlive}) per garantire l'utilizzo della stessa versione di \LaTeX ed evitare errori derivanti dalla differente codifica dei caratteri sui diversi sistemi operativi. Tale compilatore è stato integrato in un'opportuna \textit{Github Action\glo} resa pubblica sotto licenza MIT. È possibile visionare lo script al link \url{https://GitHub.com/Jatus93/Latex-multicompiler}.


%\paragraph{Linguaggi di codifica del software di supporto}
%Per il supporto dello sviluppo software e di documenti sono emersi diversi linguaggi
%di programmazione e Scripting:
% \begin{itemize}
%   \item \textbf{\textit{Bash Script\glos}}: per la creazione del compilatore automatico dei documenti \LaTeX e per il controllo lessicale ssulla documentazione;
%   \item \textbf{\textit{Python\glos}}: per la generazione automatica del glossario partendo da un file in formato CSV;
%   \item \textbf{\textit{PHP\glos}}: per la creazione di un \textit{webhook}\glo che permetta l'aggiornamento del glossario a partire da un documento condiviso su Google Drive;
%   \item \textbf{\textit{YAML\glos}}: per la configurazione delle \textit{GitHub Actions\glos}.
% \end{itemize}
%
%\subsubsection{Strumenti}
%Per supportare lo sviluppo e la stesura dei documenti sono stati utilizzati diversi
%strumenti.
%gli strumenti usati fin ora sono i seguenti:
%\begin{itemize}
%  \item \textbf{\textit{GitHub Actions\glos}}:
%  GitHub Actions è un ambiente per l'esecuzione di script e programmi quando viene
%  fatto un'operazione su i repository del progetto ed automatizza i alcuni processi;

%  \item \textbf{Script Generazione di glossario}: è stato scritto un script in python per velocizzare la stesura
%  del glossario, esso prende in input un file di estensione csv che usa come
%  separatore il carattere ',', e ne restituisce un documento che contiene tutti i
%  termini in ordine alfabetico e il registro delle modifiche realizzato automaticamente
%  in base ai commit del sistema di versionamento utilizzato;
%  \item \textbf{Google Sheet}: Questo strumento viene utilizzato per la stesura del glossario
%  in congiunzione con gli scritp di automatzione e il webhook;
%  \item \textbf{\textit{Truffle\glos}}: come è emerso da \textit{verbaleEsterno\_2019-12-20\_01} è il framework che
%  utilizzeremo per lo sviluppo e gestione delle applicazioni in \textit{Ethereum\glos}.
%\end{itemize}
