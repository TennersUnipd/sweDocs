\subsection*{\textbf{\hfill \Huge{B} \hfill}} 
\addcontentsline{toc}{subsection}{B}
\subsubsection*{Back-end}
Con il termine backend o back-end, nell'ambito del web-publishing, si indica l'interfaccia con la quale il gestore di un sito web dinamico ne gestisce i contenuti e le funzionalità. A differenza del frontend, l'accesso al backend è riservato agli amministratori del sito che possono accedere dopo essersi autenticati.
\subsubsection*{Baseline}
La baseline di progetto costituisce il punto di riferimento rispetto al quale calcolare gli scostamenti delle principali variabili implicate nella gestione di un progetto.
\subsubsection*{Behaviour Driven Development (BDD)}
Il Behavior Driven Development (BDD) é un processo agile di sviluppo software che incoraggia la collaborazione fra tutti i partecipanti ad un progetto. Incoraggia i gruppi ad utilizzare conversazioni ed esempi concreti per formalizzare una comprensione condivisa di come l'applicazione dovrebbe funzionare.
\subsubsection*{Big Data}
Il termine Bog Data indica genericamente una raccolta di dati così estesa in termini di volume, velocità e varietà da richiedere tecnologie e metodi analitici specifici per l'estrazione di valore o conoscenza. Il termine è utilizzato in riferimento alla capacità (propria della scienza dei dati) di analizzare ovvero estrapolare e mettere in relazione un'enorme mole di dati eterogenei, strutturati e non strutturati, allo scopo di scoprire i legami tra fenomeni diversi (ad esempio correlazioni) e prevedere quelli futuri. La disciplina può essere vista come un'evoluzione dei tradizionali metodi di business intelligence, allargata al trattamento di masse di dati ancor più variegate e, soprattutto, più voluminose.
\subsubsection*{Blockchain}
È definita come un registro digitale le cui voci sono raggruppate in blocchi, concatenati in ordine cronologico, e la cui integrità è garantita dall’uso della crittografia. È immutabile in quanto, di norma, il suo contenuto una volta scritto non è più né modificabile né eliminabile, a meno di non invalidare l’intera struttura.
\subsubsection*{Branch}
Un branch è un puntatore mobile a uno dei commit.
