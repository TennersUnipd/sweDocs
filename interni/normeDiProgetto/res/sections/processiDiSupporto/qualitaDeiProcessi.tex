\paragraph{Qualità di Processo}
Il gruppo ha deciso di adeguare i processi descritti nell'ISO/IEC 12207 secondo esigenze e stabilire per ciascuno degli obiettivi e delle metriche per la qualità. I 5 livelli di maturità di un processo descritti nel CMMI (descritti in appendice), aiutano a comprendere quali metriche e obiettivi porre per la misurazione della qualità e il suo miglioramento continuo.
%\subparagraph*{Sviluppo}
%Lo Sviluppo è un processo che raggruppa le attività relative alla realizzazione effettiva del prodotto software. Garantendo la qualità delle attività che lo compongono mediante le apposite metriche, il Processo di Sviluppo può considerarsi a sua volta qualitativamente definito. Si procederà dunque con la stesura di un Piano per la Qualità per alcune di esse:
%\begin{itemize}
%	\item Analisi dei Requisiti;
%	\item Progettazione.
%\end{itemize}
\subparagraph*{Analisi dei requisiti}
%\subparagraph{Obiettivi}
%\begin{itemize}
%	\item l'analisi deve essere formare requisiti chiari e non contraddittori;
%	\item con l'avanzamento del progetto, i requisiti possono esser raffinati;
%	\item i requisiti devono essere completi e approvati da ogni \textit{stakeholder\glos} coinvolto nel progetto.
%\end{itemize}
%\subparagraph{Metriche}
\begin{itemize}
	\item \textbf{Requisiti obbligatori soddisfatti:} indica la percentuale di requisiti obbligatori soddisfatti. Il calcolo avviene secondo la formula:\\\\
	\centerline{
		\begin{math}
		PERC_{OS}=100*\frac{R_{OS}}{R_O}
		\end{math}
	}
	\\\\Dove:
	\begin{itemize}
		\item \textbf{R\textsubscript{OS}:} numero di requisiti obbligatori soddisfatti;
		\item \textbf{R\textsubscript{O}:} numero di requisiti obbligatori.
	\end{itemize}
	\item \textbf{Requisiti desiderabili soddisfatti:} indica la percentuale di requisiti desiderabili soddisfatti. Il calcolo avviene secondo la formula:\\\\
		\centerline{
		\begin{math}
		PERC_{DS}=100*\frac{R_{DS}}{R_D}
		\end{math}
	}
	\\\\Dove:
	\begin{itemize}
		\item \textbf{R\textsubscript{DS}:} numero di requisiti obbligatori soddisfatti;
		\item \textbf{R\textsubscript{D}:} numero di requisiti obbligatori.
	\end{itemize}
\end{itemize}

\subparagraph*{Progettazione}
%\subparagraph{Obiettivi}
%\begin{itemize}
%	\item utilizzo delle best practice derivanti da design pattern strutturali noti e consolidati;
%	\item in caso di modifiche dei requisiti, l'architettura deve permettere la modifica delle sole componenti interessate;
%	\item l'architettura realizzata deve ridurre la complessità di realizzazione del sistema.
%\end{itemize}
%\subparagraph{Metriche}
\begin{itemize}
	\item \textbf{Structural Fan-out (SFOUT):} indica il numero di moduli che dipendono dal modulo corrente;
	\item \textbf{Structural Fan-in (SFIN):} indica il numero di moduli dai quali dipende il modulo corrente. Indica dunque il grado di dipendenza di un modulo.
\end{itemize}

\subparagraph*{Documentazione}
%La Documentazione è un processo che supporta le attività svolte durante il progetto software. Il compito di questo processo è documentare e gestire in forma scritta le informazioni necessarie per le diverse fasi del ciclo di vita di un prodotto.
%\paragraph{Obiettivi}
%\begin{itemize}
%	\item semplicità e immediata comprensione da parte di tutti gli \textit{stakeholders\glo};
%	\item correttezza grammaticale e ortografica;
%	\item contenuti completi e coerenti per l'attività documentata.
%\end{itemize}
%\paragraph{Metriche}
\begin{itemize}
	\item \textbf{indice Gulpease:} indice di leggibilità per testi in lingua italiana. La formula per il calcolo dell'indice è:\\\\
	\centerline{
		\begin{math}
		I_{GULP}=89+(\frac{300*N_F-10*N_L}{N_P})
		\end{math}
	}
	\\\\Dove:
	\begin{itemize}
		\item \textbf{N\textsubscript{F}:} numero di frasi del testo;
		\item \textbf{N\textsubscript{L}:} numero di lettere del testo;
		\item \textbf{N\textsubscript{P}:} numero di parole del testo.
	\end{itemize}
\end{itemize}

\subparagraph*{Verifica}
%Il processo di Verifica si attua mediante controlli basati su norme e metriche che consentono l'accertamento della la correttezza del prodotto software.
%\paragraph{Obiettivi}
%\begin{itemize}
%	\item provare la correttezza rispetto a norme e metriche;
%	\item correggere errori dei processi implicati nel ciclo di vita software.
%\end{itemize}
%\paragraph{Metriche}
\begin{itemize}
	\item \textbf{Code Coverage (CC):} indica la percentuale di codice sorgente ispezionata e coperta dai test.
\end{itemize}

\subparagraph*{Gestione}
%\paragraph{Pianificazione}
%La pianificazione è un'attività del processo di Gestione che si occupa dell'allocazione delle risorse e delle attività disponibili nel tempo, rientrando nei costi prestabiliti.
%\subparagraph{Obiettivi}
%\begin{itemize}
%	\item definizione di un modello di sviluppo opportuno per lo scopo e la complessità del progetto;
%	\item definizione degli obiettivi da perseguire nel tempo;
%	\item allocazione delle risorse disponibili in base al budget a disposizione e alle attività da portare a termine.
%\end{itemize}
%\subparagraph{Metriche}
\begin{itemize}
	\item \textbf{Planned Value (PV):} indica il costo stimato, alla data corrente, per la realizzazione di tutte le attività del progetto;
	\item \textbf{Actual cost (AC):} indica il costo effettivo sostenuto sino alla data corrente;
	\item \textbf{Budget at Completion (BAC):} indica il costo totale di progetto previsto al termine della pianificazione;
	\item \textbf{Earned Value (EV):} indica il valore del prodotto alla data corrente. Il calcolo avviene secondo la formula:\\\\
	\centerline{
		\begin{math}
		EV=\% ProdottoRealizzato*BAC
		\end{math}
	}
	\item \textbf{Estimated at Completion (EAC):} indica una revisione del costo finale del progetto in base all'andamento sostenuto sino a quel momento. Il calcolo avviene secondo la formula:\\\\
	\centerline{
		\begin{math}
		EAC=AC+(BAC-EV)
		\end{math}
	}
	\item \textbf{Estimate to Complete (ETC):} indica il costo per realizzare le attività rimanenti, basato sull'andamento attuale del progetto. Il calcolo avviene secondo la formula:\\\\
	\centerline{
		\begin{math}
		ETC=EAC-AC
		\end{math}
	}
	\item \textbf{Cost Variance	(CV):} indica quanto il costo effettivo per realizzare il progetto sia minore o uguale al costo totale previsto. Il calcolo avviene secondo la formula:\\\\
	\centerline{
		\begin{math}
		CV=EV-AC
		\end{math}
	}
	\item \textbf{Schedule Variance	(SV):} indica se si è in linea, in anticipo o in ritardo rispetto alla schedulazione delle attività di progetto pianificate nella baseline. Il calcolo avviene secondo la formula:\\\\
	\centerline{
		\begin{math}
		SV=EV-PV
		\end{math}
	}
\end{itemize}
