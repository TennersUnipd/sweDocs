\section{Introduzione}

\subsection{Scopo del Documento}
Con questo documento verrà descritto ed analizzato il processo che ha portato il team Tenners alla scelta del capitolato\glo C2, motivandone la preferenza rispetto ai restanti. L'analisi di ciascun capitolato\glo terrà anche in considerazione la presentazione effettuata in data 2019-11-15 e i successivi approfondimenti tecnologici da parte delle aziende proponenti.
	
\subsection{Glossario}
Come supporto alla documentazione, viene fornito un \textit{Glossario v.1.0.0}, contenente delle definizioni per termini specifici che possono richiedere chiarimento. Ognuno di questi verr\'a contrassegnato con un pedice \glo nel documento e la sua spiegazione verr\'a riportata sotto la corrispondente lettera del glossario. Ci\`o consentir\'a di avere un linguaggio comune ed evitare ambiguit\'a. 
	
\subsection{Riferimenti}
\subsubsection{Normativi}
\textbf{Norme di progetto}: \textit{Norme di progetto v.1.0.0}

\subsubsection{Informativi}
\begin{itemize}
    \item \textbf{Capitolato\glo d'appalto 1 (Autonomous Highlights Platform):}\\ 
    \url{https://www.math.unipd.it/\textasciitilde tullio/IS-1/2019/Progetto/C1.pdf}
    \item \textbf{Capitolato\glo d'appalto 2 (Etherless):}\\ 
	\url{https://www.math.unipd.it/\textasciitilde tullio/IS-1/2019/Progetto/C2.pdf}
    \item \textbf{Capitolato\glo d'appalto 3 (NaturalAPI):}\\ 
    \url{https://www.math.unipd.it/\textasciitilde tullio/IS-1/2019/Progetto/C3.pdf}
    \item \textbf{Capitolato\glo d'appalto 4 (Predire in Grafana):}\\ 
    \url{https://www.math.unipd.it/\textasciitilde tullio/IS-1/2019/Progetto/C4.pdf}
    \item \textbf{Capitolato\glo d'appalto 5 (Stalker):}\\ 
    \url{https://www.math.unipd.it/\textasciitilde tullio/IS-1/2019/Progetto/C5.pdf}
    \item \textbf{Capitolato\glo d'appalto 6 (ThiReMa - Things Relationship Management):}\\ 
    \url{https://www.math.unipd.it/\textasciitilde tullio/IS-1/2019/Progetto/C6.pdf}
\end{itemize}

	
	

