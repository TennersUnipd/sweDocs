\section{Verbale della riunione}
	\begin{itemize}
		\item \textbf{Cambiamenti da effettuare sul registro delle modifiche}: durante la discussione con il professore abbiamo individuato una modifica sostanziale da effettuare sul registro delle modifiche. Nei report, infatti, non devono essere inserite modifiche o aggiunte non ancora approvate;%. Bisogna, quindi, sistemare i report di ogni documento inserendo una colonna dedicata alla data di verifica e da chi è stata svolta;
		\item \textbf{Licenze d'uso da utilizzare per il progetto}: la scelta delle licenze d'uso non deve essere limitata soltanto da quelle indicate nel \textit{capitolato}\glos, ma è importante essere creativi. Eventualmente si può inserire un disclamer sulle licenze;
		\item \textbf{Modifiche da effettuare nelle \textit{Norme di progetto}\doc e nel \textit{Piano di progetto}\docs}: durante la discussione con il professore è emersa la necessità di rivedere la forma delle \textit{Norme di progetto}\doc e del \textit{Piano di progetto}\docs, considerando i documenti come componenti del prodotto finale stesso. Nei documenti, inoltre, devono essere descritte le attività che vanno a creare il prodotto;
		\item \textbf{Inserimenti da effettuare nel \textit{Piano di qualifica}\docs}: durante la discussione con il professore è emersa la necessità di mostrare l'andamento della qualità del nostro modo di lavorare e del prodotto finale che creeremo con questo progetto. Risulta, quindi utile l'inserimento di un cruscotto per capire come sta andando l'andamento del progetto;
		%\item \textbf{Come effettuare l'approccio incrementale}: per la realizzazione del PoC è importante la collaborazione con il proponente e l'individuazione dei problemi più importanti e come risolverli. Ogni aggiunta che .
	\end{itemize}