\subsection*{\textbf{\hfill \Huge{B} \hfill}} 
\addcontentsline{toc}{subsection}{B}
\subsubsection*{Back-end}
\index*{Back-end}
Con il termine backend o back-end, nell'ambito del web-publishing, si indica l'interfaccia con la quale il gestore di un sito web dinamico ne gestisce i contenuti e le funzionalità. A differenza del frontend, l'accesso al backend è riservato agli amministratori del sito che possono accedere dopo essersi autenticati.
\subsubsection*{Baseline}
\index*{Baseline}
La baseline di progetto costituisce il punto di riferimento rispetto al quale calcolare gli scostamenti delle principali variabili implicate nella gestione di un progetto.
\subsubsection*{Behaviour Driven Development (BDD)}
\index*{Behaviour Driven Development (BDD)}

\subsubsection*{Big Data}
\index*{Big Data}

\subsubsection*{Blockchain}
\index*{Blockchain}
È definita come un registro digitale le cui voci sono raggruppate in blocchi, concatenati in ordine cronologico, e la cui integrità è garantita dall’uso della crittografia. È immutabile in quanto, di norma, il suo contenuto una volta scritto non è più né modificabile né eliminabile, a meno di non invalidare l’intera struttura.
\subsubsection*{Branch}
\index*{Branch}
Un branch è un puntatore mobile a uno dei commit.
