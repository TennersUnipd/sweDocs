\section{Verbale della riunione}
	\begin{itemize}
		\item \textbf{Illustrazione PoC}: all'inizio della riunione il gruppo ha presentato il il PoC, esaminando le componenti principali quali Etherless-cli, Etherless-smart e Etherless-server. La dimostrazione è avvenuta mostrando prima l'eseguibile e, successivamente, il codice sviluppato. Il proponente Red Babel si è dichiarato soddisfatto degli obiettivi raggiunti dal PoC constatando che le tecnologie sono state adottate nella maniera corretta;
		\item \textbf{Tracciamento di una richiesta da parte dell'utente}: uno dei dubbi del gruppo riguardava la  possibilità di tener traccia di una richiesta da parte dell'utente, in modo che la risposta non venga catturata da altri utenti connessi al servizio Etherless. Il proponente ha chiarito il nostro dubbio suggerendoci di utilizzare un possibile identificativo univoco associato all'utente che ha eseguito il comando, unito ad  un codice generato casualmente. In ogni caso questa feature è da implementare solo se sono soddisfatte tutte le altre funzionalità del prodotto;
		\item \textbf{Sostenimento del costo del GAS}: un altro nostro dubbio, che è stato chiarito dal proponente, riguardava al sostenimento del costo del gas. Questo infatti, a detta del proponente, è a carico dell'utente che esegue la funzione mediante il sistema;
		\item \textbf{Regole da utilizzare per lo strumento ESLint}: il proponente consiglia di utilizzare le regole Airbnb associate allo strumento ESLint, prima di far visionare il PoC al committente.
	\end{itemize}
