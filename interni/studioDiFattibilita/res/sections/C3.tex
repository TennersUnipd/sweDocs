\section{Capitolato C3 - NaturalAPI }

\subsection{Descrizione generale}
Il progetto \textit{NaturalAPI}, presentato dall’azienda di Bergamo \textit{teal.blue}, parte dall'idea di stabilire un linguaggio comune che possa descrivere problemi e soluzioni nell'ambito di un progetto software. Gli stakeholders\glo coinvolti nella progettazione di un software, hanno infatti metodi differenti di comunicare a causa del dominio di appartenenza differente e spesso si può andare incontro ad una errata comprensione dei problemi o casi di ambiguità.

\subsection{Finalità del progetto}
\textit{NaturalAPI} ha come obiettivo ridurre il divario tra le specifiche di un progetto (espresse in lingua inglese) e le API\glos. Per realizzare ciò verrà utilizzata la metodologia del Behaviour Driven Development (BDD)\glos, e il progetto si fonderà concetti chiave come le user story\glo e i criteri di accettazione\glo. Ogni nuova caratteristica da aggiungere al prodotto finale, può essere descritta mediante lo schema \textit{As a - I want - So that} e, ad ognuna di esse, corrisponderanno uno o più possibili scenari descritti mediante una struttura \textit{Given -When -Then}. Più dettagliatamente, il prodotto dovrà essere dotato delle seguenti suddivisioni a livello logico:  
\begin{itemize}
	\item un \textbf{NaturalAPI Discover} in grado di estrarre, a partire da un qualsiasi documento non strutturato appartenente ad un determinato dominio, le potenziali entità coinvolte, i loro processi e le relazioni che le legano tra loro. In altre parole, a partire da un documento testuale, ne verranno estratti verbi, nomi e predicati, con relativa frequenza;
	\item un \textbf{NaturalAPI Design} il cui compito è quello di utilizzare i diversi scenari descritti tramite Gherkin e la lista dei termini precedentemente estratti per creare un business application language (BAL) con il quale poter scrivere le API\glos;
	\item un \textbf{NaturalAPI Develop} con cui convertire il BAL e testare le API\glo in uno dei  linguaggi di programmazione/framework disponibili. 
\end{itemize}

\subsection{Tecnologie interessate}
\begin{itemize}
	\item \textbf{Gherkin:} linguaggio usato per scrivere gli scenari e le features in Cucumber, molto simile al linguaggio naturale; 
	\item \textbf{Cucumber:} software che supporta lo sviluppo basato sul BDD\glos;
	\item \textbf{HipTest}: piattaforma per il continuous testing con supporto nativo del BDD\glos;
	\item \textbf{Jbehave}: framework\glo per il BDD\glos;
\end{itemize}

\subsection{Aspetti positivi}

\begin{itemize}
	\item Interessante l'utilizzo di approcci statistici o mediante grammatiche context-free\glo per l'elaborazione del linguaggio naturale;
	\item Utilizzo di tecnologie completamente nuove per tutti i componenti del gruppo;
\end{itemize}


\subsection{Criticità}
Alcune problematiche relative all'ambiguità del linguaggio naturale, spiegate anche durante la conferenza di approfondimento del capitolato\glos, sembrano apparentemente difficili da risolvere e dunque eventualmente dispendiose in termini di tempo.


\subsection{Valutazione conclusiva}
\textit{Natural API} non ha creato molto interesse nel gruppo soprattutto se paragonato agli altri capitolati proposti. L'idea generale è stata di un progetto poco spendibile nel mondo del lavoro nonostante l'utilizzo di tecnologie e metodi di sviluppo prima sconosciuti al gruppo.
