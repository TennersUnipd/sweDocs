\section{Analisi dei rischi}
Il gruppo di lavoro ha effettuato un'attenta analisi dei rischi relativi al progetto in questione. L'identificazione e l'analisi dei principali fattori di rischio permette di evitare tali circostanze oppure, nel caso in cui ci si trova in una situazione difficoltosa, sapere come procedere.
\subsection{Metodo di lavoro}
Per l'analisi dei rischi viene utilizzata la seguente procedura di identificazione e risoluzione:
\begin{itemize}
	\item \textbf{Identificazione:} il team riunito identifica i fattori problematici che possono rallentare o bloccare il completamento del progetto;
	\item \textbf{Analisi:} per ogni rischio identificato viene calcolata la probabilità di occorrenza, la gravità del rischio e le possibile conseguenze;
	\item \textbf{Pianificazione:} viene identificato come evitare i rischi individuati e, nel caso questi avvengano, come mitigarne le conseguenze;
	\item \textbf{Controllo:} definizione dei processi di monitoraggio dei rischi e definizione delle procedure di intervento nel caso di occorrenza di un rischio.
\end{itemize}
\subsection{Tipologie di rischio}
\subsubsection{Categoria}
Per la classificazione dei rischi identificati verrà utilizzata la seguente categorizzazione:
\begin{itemize}
	\item \textbf{RT:} Rischi tecnologici;
	\item \textbf{RO:} Rischi organizzativi;
	\item \textbf{RI:} Rischi interpersonali;
	\item \textbf{RA:} Rischi analitici.
\end{itemize}
\subsubsection{Grado di rischio}
Il grado di rischio viene identificato dalla Probabilità di Occorrenza (\textbf{Alta, Media, Bassa}) e dalla Pericolosità (\textbf{Alta, Media, Bassa}).
\subsubsection{Elenco rischi}
\begin{itemize}
	\item \textbf{RO1: Calcolo Tempistiche}
	\begin{itemize}
		\item \textbf{Descrizione:} la molteplicità di attività nuove per il team può portare a una stima errata delle tempistiche;
		\item \textbf{Conseguenza:} sforamento delle tempistiche;
		\item \textbf{Rilevamento:} per ogni singola attività verrà prevista una tempistica. Chi svolge l'attività oppure il responsabile dovrà notificare le situazioni in cui le tempistiche non sono rispettate;
		\item \textbf{Probabilità di occorrenza:} Alta;
		\item \textbf{Pericolosità:} Alta;
		\item \textbf{Piano di contingenza:} Le tempistiche e le risorse assegnate ad ogni task dovranno essere ben calcolate. All'insorgere della problematica, il responsabile dovrà rivedere le tempistiche e risorse assegnate all'attività in questione.
	\end{itemize}
	\item \textbf{RO2: Calcolo Costi}
	\begin{itemize}
		\item \textbf{Descrizione:} un errato calcolo di tempistiche comporta una variazione dei costi;
		\item \textbf{Conseguenza:} sforamento dei costi inizialmente preventivati;
		\item \textbf{Rilevamento:} il responsabile terrà traccia delle ore di lavoro di ogni figura del team. Confrontando tali dati a quanto previsto inizialmente è possibile identificare variazioni temporali;
		\item \textbf{Probabilità di occorrenza:} Alta;
		\item \textbf{Pericolosità:} Alta;
		\item \textbf{Piano di contingenza:} la nuova previsione sui costi dovrà essere comunicata tempestivamente al committente.
	\end{itemize}
	\item \textbf{RO3: Impegni Accademici}
	\begin{itemize}
		\item \textbf{Descrizione:} i membri del team sono impegnati in un percorso scolastico in contemporanea al progetto;
		\item \textbf{Conseguenza:} periodi di assenza o non disponibilità dei membri;
		\item \textbf{Rilevamento:} ogni membro del team dovrà comunicare tempestivamente i suoi impegni accademici specificando periodo e disponibilità;
		\item \textbf{Probabilità di occorrenza:} Alta;
		\item \textbf{Pericolosità:} Bassa;
		\item \textbf{Piano di contingenza:} assegnazione di attività e scadenze in base agli impegni accademici e riassegnazione dei task in caso di imprevisti.
	\end{itemize}
	\item \textbf{RO4: Impegni Personali}
	\begin{itemize}
		\item \textbf{Descrizione:} periodo di non disponibilità di un membro a causa di impegni personali;
		\item \textbf{Conseguenza:} periodi di assenza o non disponibilità dei membri;
		\item \textbf{Rilevamento:} ogni membro del team dovrà comunicare tempestivamente i suoi impegni personali specificando periodo e disponibilità;
		\item \textbf{Probabilità di occorrenza:} Media;
		\item \textbf{Pericolosità:} Bassa;
		\item \textbf{Piano di contingenza:} assegnazione di attività e scadenze in base agli impegni personali e riassegnazione dei task in caso di imprevisti.
	\end{itemize}
	\item \textbf{RT1: Inesperienza tecnologica}
	\begin{itemize}
		\item \textbf{Descrizione:} la maggior parte delle tecnologie necessarie per la realizzazione del prodotto sono sconosciute ai membri del team;
		\item \textbf{Conseguenza:} sforamento delle tempistiche;
		\item \textbf{Rilevamento:} vengono identificate e monitorate le lacune del team di lavoro. I membri del progetto dovranno comunicare tempestivamente eventuali difficoltà;
		\item \textbf{Probabilità di occorrenza:} Alta;
		\item \textbf{Pericolosità:} Alta;
		\item \textbf{Piano di contingenza:} è previsto un periodo di studio delle nuove tecnologie per tutti i membri del team. I task che richiedono maggiori conoscenze verranno assegnate a più membri in modo tale da favorire l'aiuto reciproco e la collaborazione.
	\end{itemize}
	\item \textbf{RI1: Comunicazione interna}
	\begin{itemize}
		\item \textbf{Descrizione:} il team non condivide uno spazio di lavoro condiviso. Le comunicazioni via messaggistica e/o telefonicamente non sono efficaci quanto la comunicazione verbale diretta;
		\item \textbf{Conseguenza:} non tutti sono sempre reperibili, quindi la comunicazione è inefficace;
		\item \textbf{Rilevamento:} i membri sono tenuti ad avvertire il gruppo quando saranno irreperibili, e a chiarire qualsiasi dubbio per evitare fraintendimenti;
		\item \textbf{Probabilità di occorrenza:} Media;
		\item \textbf{Pericolosità:} Media;
		\item \textbf{Piano di contingenza:} sono stati predisposti diversi canali di comunicazione interna. Vengono organizzate riunioni di persona per discutere gli argomenti più importanti. %I membri sono tenuti a chiarire qualsiasi dubbio per evitare fraintendimenti.
	\end{itemize}
	\item \textbf{RI2: Comunicazione esterna}
	\begin{itemize}
		\item \textbf{Descrizione:} il proponente esterno ha sede all'estero, ciò potrebbe rendere più difficili eventuali incontri e riunioni esterne;
		\item \textbf{Conseguenza:} le comunicazioni risultano più difficili;
		\item \textbf{Rilevamento:} comunicazioni e incontri con il proponente verranno pianificati preventivamente;
		\item \textbf{Probabilità di occorrenza:} Bassa;
		\item \textbf{Pericolosità:} Media;
		\item \textbf{Piano di contingenza:} sono stati predisposti diversi canali di comunicazione con il proponente. Vengono organizzate videoconferenze per discutere gli argomenti più importanti.
	\end{itemize}
	\item \textbf{RI3: Contrasti interni}
	\begin{itemize}
		\item \textbf{Descrizione:} possono insorgere contrasti e tensioni tra i membri;
		\item \textbf{Conseguenza:} lavoro inefficace;
		\item \textbf{Rilevamento:} i membri coinvolti devono comunicare l'incomprensione a tutto il gruppo;
		\item \textbf{Probabilità di occorrenza:} Bassa;
		\item \textbf{Pericolosità:} Media;
		\item \textbf{Piano di contingenza:} il gruppo al completo discute e cerca di risolvere i problemi.
	\end{itemize}
	\item \textbf{RA1: Interpretazione della richiesta del committente}
	\begin{itemize}
		\item \textbf{Descrizione:} il committente richiede e si aspetta un certo comportamento del prodotto, ma il team lo sviluppa diversamente a causa della scarsa o imprecisa comunicazione;
		\item \textbf{Conseguenza:} allungamento tempistiche e aumento costi; %committente scontento;
		\item \textbf{Rilevamento:} a intervalli prefissati si controlla assieme al committente se il prodotto è congruo a quanto richiesto;
		\item \textbf{Probabilità di occorrenza:} Media;
		\item \textbf{Pericolosità:} Alta;
		\item \textbf{Piano di contingenza:} nel caso questo avvenga, si cerca di risolvere subito l'inconveniente in modo da non continuare a ostruire su una base invalida.
	\end{itemize}
\end{itemize}