\subsection{Sviluppo}
\subsubsection{Scopo}
Il processo di Sviluppo coinvolge tutte le attività volte alla produzione del prodotto finale: Analisi dei Requisiti,  design, codifica, integrazione, testing e installazione.
\subsubsection{Analisi dei Requisiti}
L'analisi dei requisiti comprende la descrizione dettagliata ed elaborata delle specifiche del sistema. È compito dell'analista descrivere, oltre che i requisiti funzionali del prodotto, anche specifiche prestazionali, di sicurezza, di manutenibilità e di qualità. Le proprietà fondamentali di un requisito sono le seguenti:
\begin{itemize}
	\item devono essere verificabili;
	\item devono soddisfare tutti i bisogni espressi dall'utente;
	\item non devono soddisfare caratteristiche superflue (solo quelli strettamente necessari);
	\item non devono essere contraddittori;
	\item devono essere modificabili.
\end{itemize} 
Più in generale, il documento di \textit{Analisi dei Requisiti\doc} ha come obiettivo il perseguimento delle seguenti caratteristiche:
\begin{itemize}
	\item \textbf{Chiarezza strutturale:} i requisiti devono essere classificati e organizzati in maniera precisa, distinguendo i requisiti funzionali da quelli non funzionali;
	\item \textbf{Chiarezza espressiva:} è necessario avere una chiarezza espositiva tale da permettere la definizione di requisiti non ambigui;
	\item \textbf{Atomicità dei requisiti:} occorre analizzare i requisiti fino a ché non si raggiungono requisiti elementari, non ulteriormente suddivisbili.  
\end{itemize}

\paragraph{Requisiti}
L'approccio utilizzato per la definizione dei requisiti è stato di tipo \textbf{top-down}: il prodotto finale, a partire dalla sua totalità, è stato scomposto nelle parti che lo compongono sino ad arrivare a requisiti atomici. Ogni requisito necessità tracciabilità. Verrà utilizzato un codice identificativo strutturato nel seguente modo:
\\\\
\centerline{\textbf{[Priorità][Tipologia][Codice]}}\\
\begin{itemize}
  \item \textbf{Priorità}:
  \begin{itemize}
    \item \textbf{1 (Obbligatorio)}: necessario per almeno uno degli stakeholder;
    \item \textbf{2 (Desiderabile)}: non strettamente necessario, ma di valore aggiunto;
    \item \textbf{3 (Opzionale)}: relativamente utili e contrattabili con l'avanzare del progetto.
  \end{itemize}
  \item \textbf{Tipologia}:
  \begin{itemize}
    \item \textbf{F:} Funzionali;
    \item \textbf{P:} Prestazionali;
    \item \textbf{Q:} Qualitativi;
    \item \textbf{V:} di Vincolo.
  \end{itemize}
  \item \textbf{Codice Identificativo}.
\end{itemize}
Ogni requisito è opportunamente seguito da:
\begin{itemize}
  \item \textbf{Descrizione}: breve descrizione riguardante il requisito;
  \item \textbf{Fonte}: da dove proviene il requisito, chi l'ha richiesto o dove è stato ricavato. Le fonti di un requisito potranno essere le seguenti:
  \begin{itemize}
  	\item \textbf{Capitolato:} se riportato nel capitolato\glos;
  	\item \textbf{Interna:} se concordato all'interno del team;
  	\item \textbf{Caso d'uso:} se derivato da un caso d'uso (verrà riportato il codice identificativo del caso d'uso come descritto nella sezione \textsection2.2.2.2);
  	\item \textbf{Verbale:} se concordato durante una riunione, verrà riportato il codice identificativo della decisione alla quale si fa riferimento.
  \end{itemize}
\end{itemize}

\paragraph{Casi d'Uso}
I casi d'uso descrivono l'insieme delle casistiche con cui il sistema si interfaccia verso l'esterno. Ogni caso d'uso sarà identificato univocamente da un codice formato da:
\\\\
\centerline{\textbf{UC[Codicepadre].[Codicefiglio]}}\\

\begin{itemize}
	\item \textbf{Codicepadre:} identificativo numerico dato al caso d'uso a cui questo è associato;
	\item \textbf{Codicefiglio:} identificativo numerico per il sotto-caso corrente.
\end{itemize}
Ogni caso d'uso, oltre ad avere un relativo diagramma, verrà descritto 
\begin{itemize}
  \item Codice;
  \item Nome;
  \item Diagramma del caso d'uso;
  \item Attori principali;
  \item Attori secondari (opzionale);
  \item Pre-condizioni;
  \item Post-condizioni;
  \item Scenario principale;
  \item Inclusione (opzionale);
  \item Esclusione (opzionale);
  \item Specializzazione (opzionale).
\end{itemize}

\paragraph{UML\glo}
Il gruppo Tenners utilizzerà la versione del linguaggio \textit{UML\glo} v2.0 per la realizzazione dei diagrammi dei casi d'uso.
\subsubsection{Progettazione}
La Progettazione è un processo che si attiva al termine dell'Analisi dei Requisiti. Lo scopo dei progettisti è capire come risolvere il problema di partenza e soddisfare i requisiti in modo ottimale. Diversamente da ciò che accade nell'Analisi dei Requisiti, in cui il problema viene progressivamente decomposto in parti più piccole, in questa fase si ricompone il tutto per definire la soluzione migliore. La Progettazione descrive come il prodotto debba essere organizzato nelle sue componenti (architettura del sistema) e come esse interagiscono tra loro.
\paragraph{Architettura}
È necessario definire un'architettura per comporre le varie parti nel miglior modo. Essa sarà descritta nei due docLenti:
\begin{itemize}
  \item \textbf{Technology Baseline}: Selezione delle tecnologie, \textit{framework}\glo e librerie tramite \textit{Proof of Concept (PoC)\glos}. Il \textit{PoC\glo} deve essere accedibile su GitHub.
  \item \textbf{Product Baseline}: Baseline architetturale, coerente con quanto mostrato in Technology Baseline, viene presentata tramite diagrammi delle classi e di sequenza. L'analisi è comprensiva di design pattern contestualizzati all'architettura.
\end{itemize}
\paragraph{Diagrammi per la Progettazione}
Per facilitare la descrizione dell'architettura del prodotto finale, si ricorre all'utilizzo di alcune tipologie di diagrammi \textit{UML\glo}:
\begin{itemize}
	\item \textbf{Diagrammi di sequenza:} descrivono uno scenario composto da una sequenza di azioni le cui scelte sono già state effettuate. Non ammette dunque flussi alternativi;
	\item \textbf{Diagrammi delle classi}: descrivono il tipo di oggetti che fanno parte di un sistema;
	\item \textbf{Diagrammi dei package:} documenta la dipendenza tra classi;
	\item \textbf{Diagrammi di attività:} descrivono la logica procedurale che compone un processo;
	\item \textbf{Diagrammi di oggetti:} descrivono il sistema mediante gli oggetti e le associazioni tra loro.
\end{itemize}
\subsubsection{Codifica}
La Codifica è la fase in cui viene sviluppato il codice del prodotto dai programmatori. È fondamentale che questo sia uniforme per migliorare le attività di manutenzione, verifica e validazione. È quindi necessario normare il lavoro in modo tale che il loro codice sia leggibile, attraverso uno stile di codifica.
\paragraph{Stile di Codifica}
Le regole di codifica cercano di seguire degli standard per i vari linguaggi utilizzati. In generale le norme utilizzate saranno:
\begin{itemize}
  \item \textbf{Indentazione}: per l'indentazione dei blocchi saranno utilizzato 4 caratteri spazio (e non il classico carattere tab), per uniformare la codifica dei caratteri su diversi sistemi operativi.
  \item \textbf{Posizione delle parentesi}: le parentesi vanno inserite in linea con i costrutti stessi.
  \item \textbf{Nomenclatura}: tutti i nomi devono essere in CamelCase. La distinzione ovvia tra classe in maiuscola e metodi in minuscolo va rispettata. L'unica eccezione sono le costanti che vanno scritte tutte in maiuscolo. I nomi devono essere univoci e devono rispecchiare la funzione di ciò che definiscono.
  \item \textbf{Lingua}: il codice come anche i commenti vanno scritti in lingua inglese.
\end{itemize}
%\subparagraph{TypeScript}
