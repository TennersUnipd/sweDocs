\section{Consuntivo di periodo}
A seguito di ogni periodo definito nella sezione precedente viene fatto un confronto tra quanto preventivato e le ore effettivamente impiegate per ricoprire i vari ruoli. In particolare, ci interessa definire una metrica che possa riassumere la differenza di costo preventivata ed effettiva, derivante dal residuo di ore impiegate per ruolo. \\ A tale metrica si possono assegnare tre valori distinti:
\begin{itemize}
	\item \textbf{Positivo:} il quantitativo di tempo impiegato risulta inferiore rispetto a quanto preventivato;
	\item \textbf{Pari:} il quantitativo di tempo impiegato è in linea a quanto preventivato (differenza entro il 2\%);
	\item \textbf{Negativo:} il quantitativo di tempo impiegato risulta superiore rispetto a quanto preventivato.
\end{itemize}
La metrica è anche indice delle capacità del gruppo di prevedere i tempi necessari per le varie attività.
\\Inoltre, per ogni periodo verrà specificato se la data di scadenza per il raggiungimento dell'obbiettivo di tale periodo è stato rispettato.
\newpage
\subsection{Analisi dei requisiti}
\subsubsection{Impiego risorse}
\begin{table}[h]
\caption{Tabella consuntivo di periodo, Analisi dei requisiti}  
\begin{center}
\begin{tabular}{ |c|c|c|  }
 \hline
 Ruolo 		& Ore & Costo (EUR)\\
 \hline\hline
	Responsabile	& 31 (+0) & 930,00 (+0)\\
	Amministratore	& 47 (+12) & 940,00 (+240,00)\\
	Analista		& 99 (+15) & 2.475,00 (+375,00)\\
	Progettista		& 14 (+4) & 308,00 (+88,00)\\
	Programmatore	& 0 (+0) & 0 (+0)\\
	Verificatore	& 46 (-4) & 690,00 (-60,00)\\
	\hline\hline
	Totale prev.	& 210 & 4.700,00 \\
	Totale cons.	& 237 & 5.343,00 \\
	Differenza		& +27 & +643,00 \\
 \hline
\end{tabular}
\end{center}
\end{table}
\subsubsection{Conclusioni}
E' risultato necessario un numero di ore maggiore rispetto a quanto preventivato per raggiungere l'obbiettivo richiesto da questo periodo. La discrepanza oraria maggiore è stata rilevata nei ruoli di Amministratore e Analista. Il team ha riscontrato difficoltà più di quanto previsto nella parte decisionale e di configurazione degli strumenti e procedure da utilizzare durante la realizzazione del progetto.
\\Il bilancio di questo periodo è \textbf{Negativo} rispetto a quanto preventivato, con un surplus di 27 ore e uno scarto economico di +643,00 EUR. Tuttavia, non essendo rendicontato, le ore aggiuntive sono state un investimento che porterà beneficio nel proseguo del progetto. La scadenza temporale prefissata per questo periodo è comunque stata rispettata.

\newpage
\subsection{Progettazione architetturale}
\subsubsection{Impiego risorse}
\begin{table}[h]
\caption{Tabella consuntivo di periodo, Progettazione architetturale}  
\begin{center}
\begin{tabular}{ |c|c|c|  }
 \hline
 Ruolo 		& Ore & Costo (EUR)\\
 \hline\hline
	Responsabile	& 10 (+0) & 300,00 (+0)\\
	Amministratore	& 26 (+0) & 520,00 (+0)\\
	Analista		& 34 (-1) & 850,00 (-25,00)\\
	Progettista		& 71 (-1) & 1.562,00 (-22,00)\\
	Programmatore	& 35 (-2) & 525,00 (-30,00)\\
	Verificatore	& 30 (+0) & 450,00 (+0)\\
	\hline\hline
	Totale prev.	& 210 & 4.284,00 \\
	Totale cons.	& 206 & 4.207,00 \\
	Differenza		& -4 & -77,00 \\
 \hline
\end{tabular}
\end{center}
\end{table}
\newpage
\noindent In particolare, per ciascun incremento previsto, sono state impiegate le seguenti ore con relativi costi:
\begin{table}[h]
	\caption{Tabella ore effettuate per incremento, Progettazione architetturale}  
	\begin{center}
		\begin{tabular}{ |c|c|c|c|c|c|c|c|c|c| }
			\hline
			Incremento 		& RE 	& AM 	& AN 	& PT 	& PR 	& VE 	& Tot. (h)		&Tot.(EUR)\\
			\hline\hline
			I		& 2 		& 3			& 10 	& 2 	& 0 		& 4 		& 21 	&474,00\\
			II		& 2 		& 11 		& 10 	& 26	& 15 		& 9 		& 73 	&1.462,00\\
			III		& 2 		& 5 		& 7 	& 22	& 11 		& 7 		& 54 	&1.089,00\\
			IV		& 2 		& 5 		& 6 	& 20 	& 9 		& 7 		& 49 	&990,00\\
			V		& 1 		& 1 		& 1 	& 1		& 0 		& 2	 		& 6 	&127,00\\
			VI		& 1 		& 1 		& 0 	& 0 	& 0 		& 1 		& 3 	&65,00\\
			\hline\hline
			Totale		& 10		& 26		& 34 	& 71	 	& 35 	& 30 	& 206		&4.207,00\\
			\hline
		\end{tabular}
	\end{center}
\end{table}

\subsubsection{Preventivo a finire}
La seguente tabella ha lo scopo di evidenziare le differenze tra la somma che è stata preventivata e il consuntivo effettivo.
\begin{table}[h]
	\caption{Tabella preventivo a finire}  
	\begin{center}
		\begin{tabular}{ |c|c|c|  }
			\hline
			Periodo 								& Preventivo (EUR) & Consuntivo (EUR)\\
			\hline\hline
			Progettazione architetturale			& 4.284,00 				& 4.207,00\\
			Progettazione di dettaglio e codifica	& 6.966,00 			& -\\
			Validazione e collaudo					& 2.498,00 			& -\\
			\hline\hline
			Totale									& 13.748,00 				& 4.207,00 \\
			\hline
		\end{tabular}
	\end{center}
\end{table}
 

\subsubsection{Conclusioni}
È risultato necessario un numero di ore minore rispetto a quanto preventivato per raggiungere l'obbiettivo richiesto da questo periodo. La discrepanza oraria è risultata simile nei ruoli di Analista, Progettista e Programmatore. Il team ha potuto godere dell'approfondimento risultato nella Analisi, che ha portato ad una migliore ideazione e realizzazione del PoC presentato durante la Technology Baseline.
\\Il bilancio di questa periodo è \textbf{Positivo}, con un guadagno di 4 ore e uno scarto economico di -77 EUR. Anche la scadenza temporale prefissata per questo periodo è stata rispettata.
\subsection{Progettazione di dettaglio e codifica}
\subsubsection{Impiego e risorse}
In seguito è rappresentato il consuntivo del periodo di progettazione di dettaglio e codifica:
\begin{table}[h]
	\caption{Tabella consuntivo di periodo, progettazione di dettaglio e codifica}  
	\begin{center}
		\begin{tabular}{ |c|c|c|  }
			\hline
			Ruolo 		& Ore & Costo (EUR)\\
			\hline\hline
			Responsabile	& 24 (-1) & 720,00 (-30,00)\\
			Amministratore	& 23 & 460,00\\
			Analista		& 12 & 300,00\\
			Progettista		& 76 (-7) & 1.672,00 (-66,00)\\
			Programmatore	& 180 (+6) & 2.700,00 (+60,00)\\
			Verificatore	& 70 (+2) & 1.050,00 (+30)\\
			\hline\hline
			Totale prev.	& 385 & 6.966,00 \\
			Totale cons.	& 385 & 6.902,00 \\
			Differenza		& 0 & -64,00 \\
			\hline
		\end{tabular}
	\end{center}
\end{table}

\newpage
\noindent In particolare, per ciascun incremento previsto, sono state impiegate le seguenti ore con relativi costi:

\begin{table}[h]
	\caption{Tabella ore e costo totale previsti per incremento, progettazione di dettaglio e codifica}
	\begin{center}
		\begin{tabular}{ |c|c|c|c|c|c|c|c|c|  }
			\hline
			Incremento 		& RE 	& AM 	& AN 	& PT 	& PR 	& VE 	& Tot. (h) & Tot. (EUR) \\
			\hline\hline
			I		& 2 		& 7			& 7 	& 10 	& 22 		& 12 		& 60	 & 1.105,00\\
			II		& 5 		& 4 		& 1 	& 22	& 58 		& 18 		&108	&1.879,00\\
			III		& 5 		& 2 		& 1 	& 11	& 48 		& 12 		& 79	&1.357,00\\
			IV		& 6 		& 4 		& 2 	& 31 	& 51 		& 26 		& 120 	&2.147,00\\
			V		& 5 		& 5 		& 0 	& 0		& 0 		& 1	 		& 11			&265,00\\
			VI		& 1 		& 1 		& 1 	& 2 	& 1 		& 1 		& 7			&149,00\\
			\hline\hline
			Totale		& 24		& 23		& 12 	& 76	 	& 180 	& 70 	& 385 	&6.902,00\\
			\hline
		\end{tabular}
	\end{center}
\end{table}

\subsubsection{Preventivo a finire}
La seguente tabella ha lo scopo di evidenziare le differenze tra la somma che è stata preventivata e il consuntivo effettivo.
\begin{table}[h]
	\caption{Tabella preventivo a finire}  
	\begin{center}
		\begin{tabular}{ |c|c|c|  }
			\hline
			Periodo 								& Preventivo (EUR) & Consuntivo (EUR)\\
			\hline\hline
			Progettazione architetturale			& 4.284,00 				& 4.207,00\\
			Progettazione di dettaglio e codifica	& 6.966,00 			& 6.902,00\\
			Validazione e collaudo					& 2.498,00 			& -\\
			\hline\hline
			Totale									& 13.748,00 				& 11.109,00 \\
			\hline
		\end{tabular}
	\end{center}
\end{table}
\subsubsection{Conclusioni}
Durante il periodo di progettazione di dettaglio e codifica sono state rispettate le ore preventivate però e stato necessario riorganizzare la suddivisione dei ruoli e ore. Sono state risparmiate 7 ore dal ruolo di \textit Progettista, questo principalmente dovuto al ruolo di \textit Programmatore il quale ha impiegato più ore rispetto a quelle preventivate, ci sono state 6 ore di lavoro in più necessarie ad ultimare gli ultimi dettagli dell’applicativo. Sono state necessarie 2 in più per il ruolo di \textit Verificatore per ultimare la verifica dei documenti. Il risultato totale del periodo è stato di 6.902,00 EUR.
\\Il bilancio di questa periodo è \textbf{Pari}, con con un risparmio economico di -64 EUR. Anche la scadenza temporale prefissata per questo periodo è stata rispettata.

\subsubsection{Pianificazione futura}
L'impiego di ore sino ad ora rispettato, consente di non effettuare in maniera preventiva delle correzioni sulla pianificazione precedente. 