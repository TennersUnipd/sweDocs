\section{Introduzione}

\subsection{Scopo del documento}
Il presente documento descrive approfonditamente le specifiche e i requisiti tecnici del prodotto finale. Tali informazioni sono raccolte analizzando i documenti messi a disposizione dal proponente, interagendo con il cliente e comprendendo il suo dominio mediante discussioni collaborative avvenute tramite i canali comunicativi specificati nelle \textit{Norme di Progetto 2.0.0}\docs.\\
Il cliente in questione è l'azienda \textit{Red Babel} e propone la realizzazione del software \textit{Etherless}, secondo le modalità descritte nel \textit{capitolato\glo} C2.

\subsection{Scopo del prodotto}
Il \textit{Capitolato\glo} C2 ha come obiettivo la realizzazione della piattaforma \textit{Etherless}. Essa ha lo scopo di mettere in comunicazione sviluppatori che intendono condividere funzioni proprie scritte in linguaggio JavaScript con altri utenti che desiderano avere accesso a tali funzionalità. Un utente, corrisposto un pagamento in valuta \textit{Ether\glo} allo sviluppatore e alla piattaforma, avrà la possibilità di eseguire una tra le funzioni messe a disposizione e visualizzarne l'output.

\subsection{Glossario}
Come supporto alla documentazione, viene fornito un \textit{Glossario}\docs,
contenente le definizioni di termini specifici che necessitano di un chiarimento.
Ognuno di questi è contrassegnato con un pedice \glo nel documento e la sua
spiegazione viene riportata sotto la corrispondente lettera del glossario. Ciò
consentir\`a di avere un linguaggio comune ed evitare ambiguità.
	
\subsection{Riferimenti}
\subsubsection{Normativi}
	\begin{itemize}
		\item \textbf{Norme di Progetto}: \textit{Norme di Progetto 2.0.0\docs};
		\item \textbf{Verbale Esterno}: \textit{verbaleEsterno2019-12-20\docs}.
	\end{itemize}
\subsubsection{Informativi}
\begin{itemize}
	\item \textbf{Studio di fattibilità}: \textit{Studio di Fattibilità 2.0.0\docs};
    \item \textbf{Capitolato\glo d'appalto 2}: \url{https://www.math.unipd.it/\textasciitilde tullio/IS-1/2019/Progetto/C2.pdf}
	\item \textbf{Truffle}:  Ambiente di sviluppo e testing in ambiente \textit{Ethereum\glos}. Materiale informativo su \textit{smart-contract\glo} e rete \textit{Ethereum\glos} (\url{https://www.trufflesuite.com/});
	\item \textbf{Serverless\glos}:  \textit{Framework\glo} di sviluppo per applicativi \textit{serverless\glos} (\url{https://serverless.com/});
	\item \textbf{AWS}: Servizi computazionali in cloud (\url{https://aws.amazon.com/});
	\item \textbf{Sito ufficiale Ethereum\glos}: Descrizione dei concetti base, funzionamento e procedure per la realizzazione di applicativi \textit{Ethereum\glos} (\url {https://www.ethereum.org/});
\end{itemize}

	
	
	