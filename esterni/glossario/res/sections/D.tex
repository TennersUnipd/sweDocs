\subsection*{\textbf{\hfill \Huge{D} \hfill}} 
\addcontentsline{toc}{subsection}{D}
\subsubsection*{Dashboard}
Una dashboard è un tipo di interfaccia utente grafica che spesso fornisce viste a colpo d’occhio degli indicatori chiave di prestazione relativi a un particolare obiettivo o processo aziendale.
\subsubsection*{Deploy}
Rilascio e messa in funzione di un'applicazione. Lo si può di fatto considerare come una fase del ciclo di vita del software, la quale conclude lo sviluppo e il relativo collaudo e dà inizio alla manutenzione.
\subsubsection*{Design Pattern}
Si tratta di una descrizione o modello logico da applicare per la risoluzione di un problema che può presentarsi in diverse situazioni durante le fasi di progettazione e sviluppo del softwar
\subsubsection*{Diagramma di Gantt}
Il diagramma di Gantt è uno strumento di supporto alla gestione dei progetti, costruito partendo da un asse orizzontale - a rappresentazione dell'arco temporale totale del progetto, suddiviso in fasi incrementali (ad esempio, giorni, settimane, mesi) - e da un asse verticale - a rappresentazione delle mansioni o attività che costituiscono il progetto.
\subsubsection*{Dispatching}
Servizio di invio di informazioni in un qualche luogo per un determinato scopo o argomento.
\subsubsection*{Docker}
Docker è un progetto open-source che automatizza il deployment di applicazioni all'interno di contenitori software, fornendo un'astrazione aggiuntiva grazie alla virtualizzazione a livello di sistema operativo di Linux.
\subsubsection*{Driver}
Nei test di unità, il driver funziona da main e permette l'esecuzione di unità chiamate dal main
