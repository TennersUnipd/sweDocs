\section{Verbale della riunione}
	\begin{itemize}
		\item \textbf{Progettazione componente \textit{etherless-server}:}
		è stata definita una classe per gestire la connessione ad AWS, verrà poi richiamata dai comandi createFunction, updateFunction e deleteFunction. Sono le tre funzioni che interagiscono con AWS Lambda per il caricamento, la modifica e l'eliminazione delle funzioni.\\
		AWS instance serve a definire i parametri per il collegamento con AWS.
		Si è stabilito di separare la componente "runner" dalle altre funzioni riguardanti i comandi a disposizione dell'utente.
		È stato deciso di utilizzare AWS lambda per l'esecuzione del "runner" e mantenere il processo costantemente attivo con l'utilizzo di \textit{cron-job\glos}, grazie ad esso ora possiamo stabilire la frequenza dell'esecuzione del comando automatizzato.

		\item \textbf{Stesura del manuale utente:}
		il manuale contiene le informazioni utili al corretto utilizzo della componente software, ed è destinato a chi deve utilizzare la piattaforma \textit{Etherless}, senza doverne per forza conoscere tutte le caratteristiche interne.\\
		Al suo interno l'utente troverà la guida all'installazione della piattaforma \textit{Etherless} e tutte le funzionalità messe a disposizione per l'utente finale.\\
		Come richiesto da \textit{Capitolato}\glo il manuale utente è stato redatto in lingua inglese.

	\end{itemize}
