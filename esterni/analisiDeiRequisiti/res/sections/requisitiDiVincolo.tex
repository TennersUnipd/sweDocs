\subsubsection{Requisiti di vincolo}
\renewcommand{\arraystretch}{2.2}
  
  \rowcolors{2}{pari}{dispari}
  \begin{longtable}{|c|p{8cm}|c|}
  	\arrayrulecolor{white}
  	
  	\caption{Tabella riassuntiva dei requisiti di vincolo}\\
  	
    \rowcolor{header}
    
    \textbf{Requisito} & \centering{\textbf{Descrizione}} & \textbf{Fonte}\\
    
    \endfirsthead
    
    R1V1 & \textit{Etherless} dev'essere sviluppato utilizzando il linguaggio Typescript 3.6. & Capitolato \\

    R1V2 & \textit{Etherless-server} dev'essere implementato utilizzando il framework {Serverless\glos}. & Capitolato \\
    
    R1V3 & \textit{Etherless} deve utilizzare le tecnologie AWS. & Capitolato \\

    R1V3.1 & Per realizzare l'infrastruttura \textit{Serverless\glo} è richiesto l'utilizzo di AWS Lambda.  & Capitolato \\
    
    R1V4 & Gli \textit{smart-contract\glo} devono essere sviluppati utilizzando il linguaggio Solidity. & Capitolato \\
    
    R1V5 & Gli \textit{smart-contract\glo} devono essere sviluppati utilizzando il framework Truffle. & Capitolato \\
    
    R1V6 & Il progetto deve essere realizzato utilizzando alcuni ambienti specifici. & Capitolato \\
    
    R1V6.1 & Devono essere utilizzati gli ambienti di sviluppo locale, di testing e di staging. & Capitolato \\
    
    R1V6.1.1 & Per gli ambienti di sviluppo locale, testing e staging è desiderabile l'utilizzo di: \textit{Ethereum\glo} testrpc network e un web server locale per lo sviluppo locale e in ambiente di testing; l'\textit{Ethereum\glo} network Ropsten per lo staging. & Capitolato \\
    
    R1V6.2 & Per l'ambiente di produzione è necessario l'utilizzo della \textit{MainNet\glo} di \textit{Ethereum\glo} e della valuta \textit{Ether\glos}. & Capitolato \\
    
    R1V7 & Il proponente Red Babel deve essere menzionato tra i crediti all'interno del file README.md. & Capitolato \\
    
    \hline
  \end{longtable}