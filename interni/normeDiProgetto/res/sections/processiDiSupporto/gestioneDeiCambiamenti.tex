\subsection{Gestione dei cambiamenti}
\subsubsection{Obiettivi}
La gestione dei cambiamenti ha un ruolo fondamentale per dare ordine all'insieme delle attività correttive che derivano dalla rilevazione di uno o più errori o problematiche all'interno del prodotto.

\subsubsection{Attività}
\paragraph{Gestione delle attività correttive}
Per far fronte all'insorgere di problematiche relative al prodotto, il team ha deciso di avvalersi dell'utilizzo delle issue di GitHub. Esse oltre che esser utilizzate per la definizione dei normali task da svolgere per portare a compimento una data attività, vengono adoperate anche per la risoluzione di criticità. Questo sistema permette di:
\begin{itemize}
	\item definire un titolo da associare alla issue;
	\item attribuirne una descrizione;
	\item definire uno o più assegnatari incaricati nella risoluzione della stesa;
	\item stabilire un'etichetta che identifichi la tipologia della issue;
	\item associare una milestone;
	\item la chiusura di una determinata issue inserendo il codice della stessa all'interno del commit;
	\item discutere della issue tramite commenti.
\end{itemize}

\subsubsection{Procedure}
\paragraph{Risoluzione delle problematiche}
\begin{enumerate}
	\item Definizione del problema rilevato mediante il sistema di issue;
	\item Creazione delle alternative per la risoluzione del problema;
	\item Valutazione delle alternative in base alle risorse, tempi, strumenti impiegate per adoperarle;
	\item Scelta dell'opzione risolutiva più adatta;
	\item Compimento delle attività correttive secondo le modalità scelte;
	\item Valutazione della procedura utilizzata e dei risultati ottenuti.
\end{enumerate}

\subsubsection{Strumenti}
\paragraph{Issue di GitHub}
Per la definizione delle attività e dei task correttivi sono state utilizzate le issue di GitHub, scelte in quanto viste come strumento di \textit{Issue Tracking System\glo} nel corso di Tecnologie Open-Source da alcuni membri del team. Inoltre è stato preferito ad altri strumenti poiché già integrato all'interno di GitHub.