\section{Introduzione}

\subsection{Scopo del Documento}
Con questo documento verrà descritto ed analizzato il processo che ha portato il team Tenners alla scelta del capitolato\glo CX, motivandone la preferenza rispetto ai restanti capitolati.
	
\subsection{Glossario}
Come supporto alla documentazione, viene fornito un \textit{Glossario v.1.0.0}, contenente delle definizioni per termini specifici che possono richiedere chiarimento. Ognuno di questi verrà contrassegnato con un pedice \glo nel documento e la sua spiegazione verrà riportata sotto la corrispondente lettera del glossario. Ciò consentirà di avere un linguaggio comune ed evitare ambiguità. 
	
\subsection{Riferimenti}
\subsubsection{Normativi}
\subsubsection{Informativi}
\begin{itemize}
    \item Capitolato d'appalto 1 (Autonomous Highlights Platform):\\ 
    \url{https://www.math.unipd.it/\textasciitilde tullio/IS-1/2019/Progetto/C1.pdf}
    \item Capitolato d'appalto 2 (Etherless):\\ 
	\url{https://www.math.unipd.it/\textasciitilde tullio/IS-1/2019/Progetto/C2.pdf}
    \item Capitolato d'appalto 3 (NaturalAPI):\\ 
    \url{https://www.math.unipd.it/\textasciitilde tullio/IS-1/2019/Progetto/C3.pdf}
    \item Capitolato d'appalto 4 (Predire in Grafana):\\ 
    \url{https://www.math.unipd.it/\textasciitilde tullio/IS-1/2019/Progetto/C4.pdf}
    \item Capitolato d'appalto 5 (Stalker):\\ 
    \url{https://www.math.unipd.it/\textasciitilde tullio/IS-1/2019/Progetto/C5.pdf}
    \item Capitolato d'appalto 6 (ThiReMa - Things Relationship Management):\\ 
    \url{https://www.math.unipd.it/\textasciitilde tullio/IS-1/2019/Progetto/C6.pdf}
\end{itemize}

	
	

